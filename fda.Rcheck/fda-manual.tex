\documentclass{article}
\usepackage[ae,hyper]{Rd}
\begin{document}
\HeaderA{argvalsy.swap}{Swap argvals with y if the latter is simpler.}{argvalsy.swap}
\keyword{smooth}{argvalsy.swap}
\begin{Description}\relax
Preprocess \code{argvals}, \code{y}, and \code{basisobj}.  If only one
of \code{argvals} and \code{y} is provided, use  it as \code{y} and
take \code{argvals} as a vector spanning basisobj[['rangreval']].  If
both are provided, the simpler becomes \code{argvals}.  If both have
the same dimensions but only one lies in basisobj[['rangreval']], that
becomes \code{argvals}.
\end{Description}
\begin{Usage}
\begin{verbatim}
argvalsy.swap(argvals=NULL, y=NULL, basisobj=NULL) 
\end{verbatim}
\end{Usage}
\begin{Arguments}
\begin{ldescription}
\item[\code{argvals}] a vector or array of argument values.

\item[\code{y}] an array containing sampled values of curves.  

\item[\code{basisobj}] One of the following:

\Itemize{
\item[basisfd] a functional basis object (class \code{basisfd}. 

\item[fd] a functional data object (class \code{fd}), from which its
\code{basis} component is extracted.  

\item[fdPar] a functional parameter object (class \code{fdPar}), from which
its \code{basis} component is extracted.  

\item[integer] an integer giving the order of a B-spline basis,
create.bspline.basis(argvals, norder=basisobj) 

\item[numeric vector] specifying the knots for a B-spline basis,
create.bspline.basis(basisobj) 
         
\item[NULL] Defaults to create.bspline.basis(argvals).

}

\end{ldescription}
\end{Arguments}
\begin{Details}\relax
1.  If \code{y} is NULL, replace by \code{argvals}.

2.  If \code{argvals} is NULL, replace by
seq(basisobj[['rangeval']][1], basisobj[['rangeval']][2], dim(y)[1])
with a warning.  

3.  If the dimensions of \code{argvals} and \code{y} match and only
one is contained in basisobj[['rangeval']], use that as \code{argvals}
and the other as \code{y}.

4.  if \code{y} has fewer dimensions than \code{argvals}, swap them.
\end{Details}
\begin{Value}
a list with components \code{argvals}, \code{y}, and \code{basisobj}.
\end{Value}
\begin{SeeAlso}\relax
\code{\LinkA{Data2fd}{Data2fd}} 
\code{\LinkA{smooth.basis}{smooth.basis}}, 
\code{\LinkA{smooth.basisPar}{smooth.basisPar}}
\end{SeeAlso}
\begin{Examples}
\begin{ExampleCode}
##
## one argument:  y
##
argvalsy.swap(1:5)
# warning ... 

##
## (argvals, y), same dimensions:  retain order 
##
argy1 <- argvalsy.swap(seq(0, 1, .2), 1:6)
argy1a <- argvalsy.swap(1:6, seq(0, 1, .2))


all.equal(argy1[[1]], argy1a[[2]]) &&
all.equal(argy1[[2]], argy1a[[1]])
# TRUE;  basisobj different 


# lengths do not match 
## Not run: 
argvalsy.swap(1:4, 1:5)
## End(Not run) 

##
## two numeric arguments, different dimensions:  put simplest first 
##
argy2 <- argvalsy.swap(seq(0, 1, .2), matrix(1:12, 6))


all.equal(argy2,
argvalsy.swap(matrix(1:12, 6), seq(0, 1, .2)) )
# TRUE with a warning ... 


## Not run: 
argvalsy.swap(seq(0, 1, .2), matrix(1:12, 2))
# ERROR:  first dimension does not match 
## End(Not run)

##
## one numeric, one basisobj
##
argy3 <- argvalsy.swap(1:6, b=4)
# warning:  argvals assumed seq(0, 1, .2) 

argy3. <- argvalsy.swap(1:6, b=create.bspline.basis(breaks=0:1))
# warning:  argvals assumed seq(0, 1, .2) 

argy3.6 <- argvalsy.swap(seq(0, 1, .2), b=create.bspline.basis(breaks=1:3))
# warning:  argvals assumed seq(1, 3 length=6)

##
## two numeric, one basisobj:  first matches basisobj
##
#  OK 
argy3a <- argvalsy.swap(1:6, seq(0, 1, .2),
              create.bspline.basis(breaks=c(1, 4, 8))) 

#  Swap (argvals, y) 

all.equal(argy3a,
argvalsy.swap(seq(0, 1, .2), 1:6, 
              create.bspline.basis(breaks=c(1, 4, 8))) )
# TRUE with a warning 


## Not run: 
# neither match basisobj:  error  
argvalsy.swap(seq(0, 1, .2), 1:6, 
              create.bspline.basis(breaks=1:3) ) 
## End(Not run)

\end{ExampleCode}
\end{Examples}

\HeaderA{arithmetic.basisfd}{Arithmatic on functional basis objects}{arithmetic.basisfd}
\aliasA{==.basisfd}{arithmetic.basisfd}{==.basisfd}
\keyword{smooth}{arithmetic.basisfd}
\begin{Description}\relax
Arithmatic on functional basis objects
\end{Description}
\begin{Usage}
\begin{verbatim}
"==.basisfd"(basis1, basis2)
\end{verbatim}
\end{Usage}
\begin{Arguments}
\begin{ldescription}
\item[\code{basis1, basis2}] functional basis object 

\end{ldescription}
\end{Arguments}
\begin{Value}
basisobj1 == basisobj2 returns a logical scalar.
\end{Value}
\begin{References}\relax
Ramsay, James O., and Silverman, Bernard W. (2005), \emph{Functional 
Data Analysis, 2nd ed.}, Springer, New York. 

Ramsay, James O., and Silverman, Bernard W. (2002), \emph{Applied
Functional Data Analysis}, Springer, New York.
\end{References}
\begin{SeeAlso}\relax
\code{\LinkA{basisfd}{basisfd}}, 
\code{\LinkA{basisfd.product}{basisfd.product}}
\code{\LinkA{arithmetic.fd}{arithmetic.fd}}
\end{SeeAlso}

\HeaderA{arithmetic.fd}{Arithmetic on functional data ('fd') objects}{arithmetic.fd}
\aliasA{*.fd}{arithmetic.fd}{*.fd}
\aliasA{+.fd}{arithmetic.fd}{+.fd}
\aliasA{-.fd}{arithmetic.fd}{.Rdash..fd}
\aliasA{minus.fd}{arithmetic.fd}{minus.fd}
\aliasA{plus.fd}{arithmetic.fd}{plus.fd}
\aliasA{times.fd}{arithmetic.fd}{times.fd}
\keyword{smooth}{arithmetic.fd}
\begin{Description}\relax
Arithmetic on functional data objects
\end{Description}
\begin{Usage}
\begin{verbatim}
"+.fd"(e1, e2) 
"-.fd"(e1, e2) 
"*.fd"(e1, e2) 
plus.fd(e1, e2, basisobj=basisobj1*basisobj2)
minus.fd(e1, e2, basisobj=basisobj1*basisobj2)
times.fd(e1, e2, basisobj=basisobj1*basisobj2)
\end{verbatim}
\end{Usage}
\begin{Arguments}
\begin{ldescription}
\item[\code{e1, e2}] object of class 'fd' or a numeric vector.  Note that 'e1+e2' will
dispatch to plus.fd(e1, e2) only of e1 has class 'fd'.  Similarly,
'e1-e2' or 'e1*e2' will dispatch to minus.fd(e1, e2) or time.fd(e1,
e2), respetively, only if e1 is of class 'fd'.  

\item[\code{basisobj}] reference basis  

\end{ldescription}
\end{Arguments}
\begin{Value}
A function data object corresponding to the pointwise sum, difference
or product of e1 and e2.  

If both arguments are functional data objects, the bases are the same,
and the coefficient matrices are the same dims, the indicated
operation is applied to the coefficient matrices of the two objects.
In other words, e1+e2 is obtained for this case by adding the
coefficient matrices from e1 and e2.  

If e1 or e2 is a numeric scalar,  that scalar is applied to the
coefficient matrix of the functional data object.  

If either e1 or e2 is a numeric vector, it must be the same length as
the number of replicated functional observations in the other
argument.  

When both arguments are functional data objects, they need not have
the same bases.  However, if they don't have the same number of
replicates, then one of them must have a single replicate.  In the
second case, the singleton function is replicated to match the number
of replicates of the other function. In either case, they must have
the same number of functions. When both arguments are functional data
objects, and the bases are not the same, the basis used for the sum is
constructed to be of higher dimension than the basis for either factor
according to rules described in function TIMES for two basis objects.
\end{Value}
\begin{SeeAlso}\relax
\code{\LinkA{basisfd}{basisfd}}, 
\code{\LinkA{basisfd.product}{basisfd.product}}
\end{SeeAlso}
\begin{Examples}
\begin{ExampleCode}
##
## add a parabola to itself
##
bspl4 <- create.bspline.basis(nbasis=4)
parab4.5 <- fd(c(3, -1, -1, 3)/3, bspl4)
str(parab4.5+parab4.5)
# coefs = c(6, -2, -2, 6)/3
str(parab4.5-parab4.5)
# coefs = c(0, 0, 0, 0)

##
## Same example with interior knots at 1/3 and 1/5
##
bspl5.3 <- create.bspline.basis(breaks=c(0, 1/3, 1))
plot(bspl5.3)
x. <- seq(0, 1, .1)
para4.5.3 <- smooth.basis(x., 4*(x.-0.5)^2, fdParobj=bspl5.3)[['fd']]
plot(para4.5.3)

bspl5.2 <- create.bspline.basis(breaks=c(0, 1/2, 1))
plot(bspl5.2)
para4.5.2 <- smooth.basis(x., 4*(x.-0.5)^2, fdParobj=bspl5.2)[['fd']]
plot(para4.5.2)

str(para4.5.3+para4.5.2)
str(para4.5.3-para4.5.2)
str(para4.5.3*para4.5.2)
# interior knots of the sum
# = union(interior knots of the summands);
# ditto for difference and product.  

\end{ExampleCode}
\end{Examples}

\HeaderA{as.array3}{Reshape a vector or array to have 3 dimensions.}{as.array3}
\keyword{utilities}{as.array3}
\begin{Description}\relax
Coerce a vector or array to have 3 dimensions, preserving dimnames if
feasible.  Throw an error if length(dim(x)) > 3.
\end{Description}
\begin{Usage}
\begin{verbatim}
as.array3(x) 
\end{verbatim}
\end{Usage}
\begin{Arguments}
\begin{ldescription}
\item[\code{x}] A vector or array.  

\end{ldescription}
\end{Arguments}
\begin{Details}\relax
1.  dimx <- dim(x);  ndim <- length(dimx) 

2.  if(ndim==3)return(x).

3.  if(ndim>3)stop.

4.  x2 <- as.matrix(x)

5.  dim(x2) <- c(dim(x2), 1)

6.  xnames <- dimnames(x)

7.  if(is.list(xnames))dimnames(x2) <- list(xnames[[1]], xnames[[2]],
NULL)
\end{Details}
\begin{Value}
A 3-dimensional array with names matching \code{x}
\end{Value}
\begin{Author}\relax
Spencer Graves
\end{Author}
\begin{SeeAlso}\relax
\code{\LinkA{dim}{dim}},
\code{\LinkA{dimnames}{dimnames}}
\code{\LinkA{checkDims3}{checkDims3}}
\end{SeeAlso}
\begin{Examples}
\begin{ExampleCode}
##
## vector -> array 
##
as.array3(c(a=1, b=2)) 

##
## matrix -> array 
##
as.array3(matrix(1:6, 2))
as.array3(matrix(1:6, 2, dimnames=list(letters[1:2], LETTERS[3:5]))) 

##
## array -> array 
##
as.array3(array(1:6, 1:3)) 

##
## 4-d array 
##
## Not run: 
as.array3(array(1:24, 1:4)) 
Error in as.array3(array(1:24, 1:4)) : 
  length(dim(array(1:24, 1:4)) = 4 > 3
## End(Not run)
\end{ExampleCode}
\end{Examples}

\HeaderA{as.fd}{Convert a spline object to class 'fd'}{as.fd}
\methaliasA{as.fd.dierckx}{as.fd}{as.fd.dierckx}
\methaliasA{as.fd.fdSmooth}{as.fd}{as.fd.fdSmooth}
\methaliasA{as.fd.function}{as.fd}{as.fd.function}
\aliasA{as.fd.smooth.spline}{as.fd}{as.fd.smooth.spline}
\keyword{smooth}{as.fd}
\keyword{manip}{as.fd}
\begin{Description}\relax
Translate a spline object of another class into the Functional Data
(class \code{fd}) format.
\end{Description}
\begin{Usage}
\begin{verbatim}
as.fd(x, ...)
## S3 method for class 'fdSmooth':
as.fd(x, ...)
## S3 method for class 'dierckx':
as.fd(x, ...) 
## S3 method for class 'function':
as.fd(x, ...)
## S3 method for class 'smooth.spline':
as.fd(x, ...) 
\end{verbatim}
\end{Usage}
\begin{Arguments}
\begin{ldescription}
\item[\code{x}] an object to be converted to class \code{fd}.  

\item[\code{...}] optional arguments passed to specific methods, currently unused.  

\end{ldescription}
\end{Arguments}
\begin{Details}\relax
The behavior depends on the \code{class} and nature of \code{x}.

\Itemize{
\item[as.fd.fdSmooth] extract the \code{fd} component

\item[as.fd.dierckx] The 'fda' package (as of version 2.0.0) supports B-splines with
coincident boundary knots.  For periodic phenomena, the
\code{DierckxSpline} packages uses periodic spines, while
\code{fda} recommends finite Fourier series.  Accordingly,
\code{as.fd.dierckx} if x[["periodic"]] is TRUE.  

The following describes how the components of a \code{dierckx}
object are handled by as.dierckx(as.fd(x)):   

\Itemize{
\item[x] lost.  Restored from the knots.
\item[y] lost.  Restored from spline predictions at the restored values
of 'x'.  

\item[w] lost.  Restored as rep(1, length(x)).
\item[from, to] fd[["basis"]][["rangeval"]] 
\item[k] coded indirectly as fd[["basis"]][["nbasis"]] -
length(fd[["basis"]][["params"]]) - 1.  

\item[s] lost, restored as 0.
\item[nest] lost, restored as length(x) + k + 1
\item[n] coded indirectly as 2*fd[["basis"]][["nbasis"]] -
length(fd[["basis"]][["params"]]).

\item[knots] The end knots are stored (unreplicated) in
fd[["basis"]][["rangeval"]], while the interior knots are
stored in fd[["basis"]][["params"]].

\item[fp] lost.  Restored as 0.
\item[wrk, lwrk, iwrk] lost.  Restore by refitting to the knots.  

\item[ier] lost.  Restored as 0.
\item[message] lost.  Restored as character(0).
\item[g] stored indirectly as
length(fd[["basis"]][["params"]]). 
\item[method] lost.  Restored as "ss".
\item[periodic] 'dierckx2fd' only translates 'dierckx' objects
with coincident boundary knots.  Therefore, 'periodic' is
restored as FALSE.
 
\item[routine] lost.  Restored as 'curfit.default'.
\item[xlab] fd[["fdnames"]][["args"]]
\item[ylab] fd[["fdnames"]][["funs"]]
}


\item[as.fd.function] Create an \code{fd} object from a function of the form created by
\code{splinefun}.  This will translate method = 'fmn' and
'natural' but not 'periodic':  'fmn' splines are isomorphic to
standard B-splines with coincident boundary knots, which is the
basis produced by \code{create.bspline.basis}.  'natural' splines
occupy a subspace of this space, with the restriction that the
second derivative at the end points is zero (as noted in the
Wikipedia \code{spline} article).  'periodic' splines do not use
coindicent boundary knots and are not currently supported in
\code{fda};  instead, \code{fda} uses finite Fourier bases for
periodic phenomena.  


\item[as.fd.smooth.spline] Create an \code{fd} object from a \code{smooth.spline} object.

}
\end{Details}
\begin{Value}
\code{as.fd.dierckx} converts an object of class 'dierckx' into one of
class \code{fd}.
\end{Value}
\begin{Author}\relax
Spencer Graves
\end{Author}
\begin{References}\relax
Dierckx, P. (1991) \emph{Curve and Surface Fitting with Splines},
Oxford Science Publications.

Ramsay, James O., and Silverman, Bernard W. (2005), \emph{Functional 
Data Analysis, 2nd ed.}, Springer, New York. 

Ramsay, James O., and Silverman, Bernard W. (2002), \emph{Applied
Functional Data Analysis}, Springer, New York.

\code{spline} entry in \emph{Wikipedia}
\url{http://en.wikipedia.org/wiki/Spline_(mathematics)}
\end{References}
\begin{SeeAlso}\relax
\code{\LinkA{as.dierckx}{as.dierckx}}
\code{\LinkA{curfit}{curfit}}
\code{\LinkA{fd}{fd}}
\code{\LinkA{splinefun}{splinefun}}
\end{SeeAlso}
\begin{Examples}
\begin{ExampleCode}
##
## as.fd.fdSmooth
##
girlGrowthSm <- with(growth, smooth.basisPar(argvals=age, y=hgtf))
girlGrowth.fd <- as.fd(girlGrowthSm)

##
## as.fd.dierckx
##
x <- 0:24
y <- c(1.0,1.0,1.4,1.1,1.0,1.0,4.0,9.0,13.0,
       13.4,12.8,13.1,13.0,14.0,13.0,13.5,
       10.0,2.0,3.0,2.5,2.5,2.5,3.0,4.0,3.5)
library(DierckxSpline) 
curfit.xy <- curfit(x, y, s=0)

curfit.fd <- as.fd(curfit.xy)
plot(curfit.fd) # as an 'fd' object 
points(x, y) # Curve goes through the points.  

x. <- seq(0, 24, length=241)
pred.y <- predict(curfit.xy, x.) 
lines(x., pred.y, lty="dashed", lwd=3, col="blue")
# dierckx and fd objects match.


all.equal(knots(curfit.xy, FALSE), knots(curfit.fd, FALSE))


all.equal(coef(curfit.xy), as.vector(coef(curfit.fd)))




##
## as.fd.function(splinefun(...), ...) 
## 
x2 <- 1:7
y2 <- sin((x2-0.5)*pi)
f <- splinefun(x2, y2)
fd. <- as.fd(f)
x. <- seq(1, 7, .02)
fx. <- f(x.)
fdx. <- eval.fd(x., fd.) 
plot(range(x2), range(y2, fx., fdx.), type='n')
points(x2, y2)
lines(x., sin((x.-0.5)*pi), lty='dashed') 
lines(x., f(x.), col='blue')
lines(x., eval.fd(x., fd.), col='red', lwd=3, lty='dashed')
# splinefun and as.fd(splineful(...)) are close
# but quite different from the actual function
# apart from the actual 7 points fitted,
# which are fitted exactly
# ... and there is no information in the data
# to support a better fit!

# Translate also a natural spline 
fn <- splinefun(x2, y2, method='natural')
fn. <- as.fd(fn)
lines(x., fn(x.), lty='dotted', col='blue')
lines(x., eval.fd(x., fn.), col='green', lty='dotted', lwd=3)

## Not run: 
# Will NOT translate a periodic spline
fp <- splinefun(x, y, method='periodic')
as.fd(fp)
#Error in as.fd.function(fp) : 
#  x (fp)  uses periodic B-splines, and as.fd is programmed
#   to translate only B-splines with coincident boundary knots.

## End(Not run)

##
## as.fd.smooth.spline
##
cars.spl <- with(cars, smooth.spline(speed, dist))
cars.fd <- as.fd(cars.spl)

plot(dist~speed, cars)
lines(cars.spl)
sp. <- with(cars, seq(min(speed), max(speed), len=101))
d. <- eval.fd(sp., cars.fd)
lines(sp., d., lty=2, col='red', lwd=3)
\end{ExampleCode}
\end{Examples}

\HeaderA{axisIntervals}{Mark Intervals on a Plot Axis}{axisIntervals}
\keyword{smooth}{axisIntervals}
\keyword{hplot}{axisIntervals}
\begin{Description}\relax
Adds an axis to the current plot, with tick marks delimiting interval
described by labels
\end{Description}
\begin{Usage}
\begin{verbatim}
axisIntervals(side, atTick1=monthBegin.5, atTick2=monthEnd.5,
      atLabels=monthMid, labels=month.abb, cex.axis=0.9, ...)
\end{verbatim}
\end{Usage}
\begin{Arguments}
\begin{ldescription}
\item[\code{side}] an integer specifying which side of the plot the axis is to
be drawn on.  The axis is placed as follows: 1=below, 2=left,
3=above and 4=right.

\item[\code{atTick1}] the points at which tick-marks marking the starting points of the
intervals are to be drawn.  This defaults to 'monthBegin.5' to mark
monthly periods for an annual cycle.  These are constructed by
calling axis(side, at=atTick1, labels=FALSE, ...).  For more detail
on this, see 'axis'.  

\item[\code{atTick2}] the points at which tick-marks marking the ends of the
intervals are to be drawn.  This defaults to 'monthBegin.5' to mark
monthly periods for an annual cycle.  These are constructed by
calling axis(side, at=atTick2, labels=FALSE, ...).  Use atTick2=NA
to rely only on atTick1.  For more detail
on this, see 'axis'.  

\item[\code{atLabels}] the points at which 'labels' should be typed.  These are constructed
by calling axis(side, at=atLabels, tick=FALSE, ...).  For more detail
on this, see 'axis'.  

\item[\code{labels}] Labels to be typed at locations 'atLabels'.  This is accomplished by
calling axis(side, at=atLabels, labels=labels, tick=FALSE, ...).
For more detail on this, see 'axis'.   

\item[\code{cex.axis}] Character expansion (magnification) used for axis annotations
('labels' in this function call) relative
to the current setting of 'cex'.  For more detail, see 'par'.    

\item[\code{... }] additional arguments passed to 
\code{axis}. 

\end{ldescription}
\end{Arguments}
\begin{Value}
The value from the third (labels) call to 'axis'.  This function is
usually invoked for its side effect, which is to add an axis to an
already existing plot.
\end{Value}
\begin{Section}{Side Effects}
An axis is added to the current plot.
\end{Section}
\begin{Author}\relax
Spencer Graves
\end{Author}
\begin{SeeAlso}\relax
\code{\LinkA{axis}{axis}}, 
\code{\LinkA{par}{par}}
\code{\LinkA{monthBegin.5}{monthBegin.5}}
\code{\LinkA{monthEnd.5}{monthEnd.5}}
\code{\LinkA{monthMid}{monthMid}}
\code{\LinkA{month.abb}{month.abb}}
\code{\LinkA{monthLetters}{monthLetters}}
\end{SeeAlso}
\begin{Examples}
\begin{ExampleCode}
daybasis65 <- create.fourier.basis(c(0, 365), 65)

daytempfd <- with(CanadianWeather, data2fd(
       dailyAv[,,"Temperature.C"], day.5,
       daybasis65, argnames=list("Day", "Station", "Deg C")) )
 
with(CanadianWeather, plotfit.fd(
      dailyAv[,,"Temperature.C"], argvals=day.5,
          daytempfd, index=1, titles=place, axes=FALSE) )
# Label the horizontal axis with the month names
axisIntervals(1) 
axis(2)
# Depending on the physical size of the plot,
# axis labels may not all print.
# In that case, there are 2 options:
# (1) reduce 'cex.lab'.
# (2) Use different labels as illustrated by adding
#     such an axis to the top of this plot 

axisIntervals(3, labels=monthLetters, cex.lab=1.2, line=-0.5) 
# 'line' argument here is passed to 'axis' via '...' 

\end{ExampleCode}
\end{Examples}

\HeaderA{basisfd.product}{Product of two basisfd objects}{basisfd.product}
\aliasA{*.basisfd}{basisfd.product}{*.basisfd}
\keyword{smooth}{basisfd.product}
\begin{Description}\relax
pointwise multiplication method for basisfd class
\end{Description}
\begin{Usage}
\begin{verbatim}
"*.basisfd"(basisobj1, basisobj2)
\end{verbatim}
\end{Usage}
\begin{Arguments}
\begin{ldescription}
\item[\code{basisobj1, basisobj2}] objects of class basisfd 

\end{ldescription}
\end{Arguments}
\begin{Details}\relax
TIMES for (two basis objects sets up a basis suitable for expanding
the pointwise product of two functional data objects with these
respective bases.  In the absence of a true product basis system in
this code, the rules followed are inevitably a compromise:
(1) if both bases are B-splines, the norder is the sum of the
two orders - 1, and the breaks are the union of the
two knot sequences, each knot multiplicity being the maximum
of the multiplicities of the value in the two break sequences.
That is, no knot in the product knot sequence will have a
multiplicity greater than the multiplicities of this value
in the two knot sequences.  
The rationale this rule is that order of differentiability
of the product at eachy value will be controlled  by
whichever knot sequence has the greater multiplicity.  
In the case where one of the splines is order 1, or a step
function, the problem is dealt with by replacing the
original knot values by multiple values at that location
to give a discontinuous derivative.
(2) if both bases are Fourier bases, AND the periods are the 
the same, the product is a Fourier basis with number of
basis functions the sum of the two numbers of basis fns.
(3) if only one of the bases is B-spline, the product basis
is B-spline with the same knot sequence and order two
higher.
(4) in all other cases, the product is a B-spline basis with
number of basis functions equal to the sum of the two
numbers of bases and equally spaced knots.
\end{Details}
\begin{SeeAlso}\relax
\code{\LinkA{basisfd}{basisfd}}
\end{SeeAlso}

\HeaderA{basisfd}{Define a Functional Basis Object}{basisfd}
\keyword{smooth}{basisfd}
\keyword{smooth}{basisfd}
\keyword{internal}{basisfd}
\begin{Description}\relax
This is the constructor function for objects of the \code{basisfd}
class.  Each function that sets up an object of this class must call
this function.  This includes functions \code{create.bspline.basis},
\code{create.constant.basis}, \code{create.fourier.basis}, and so
forth that set up basis objects of a specific type.  Ordinarily, user
of the functional data analysis software will not need to call this
function directly, but these notes are valuable to understanding what
the "slots" or "members" of the \code{basisfd} class are.
\end{Description}
\begin{Usage}
\begin{verbatim}
basisfd(type, rangeval, nbasis, params,
        dropind=NULL, quadvals=NULL,
        values=vector("list", 0))
\end{verbatim}
\end{Usage}
\begin{Arguments}
\begin{ldescription}
\item[\code{type}] a character string indicating the type of basis.  Currently,
there are eight possible types:

\Enumerate{
\item bspline
\item const
\item expon
\item fourier
\item monom
\item polyg
\item polynom
\item power
}


\item[\code{rangeval}] a vector of length 2 containing the lower and upper boundaries
of the range over which the basis is defined

\item[\code{nbasis}] the number of basis functions

\item[\code{params}] a vector of parameter values defining the basis

\item[\code{dropind}] a vector of integers specifiying the basis functions to
be dropped, if any.  For example, if it is required that
a function be zero at the left boundary, this is achieved
by dropping the first basis function, the only one that
is nonzero at that point. Default value NULL.

\item[\code{quadvals}] a matrix with two columns and a number of rows equal to the number of
argument values used to approximate an integral using Simpson's rule.
The first column contains these argument values.
A minimum of 5 values are required for
each inter-knot interval, and that is often enough. These
are equally spaced between two adjacent knots.
The second column contains the weights used for Simpson's
rule.  These are proportional to 1, 4, 2, 4, ..., 2, 4, 1.

\item[\code{values}] a list containing the basis functions and their derivatives
evaluated at the quadrature points contained in the first
column of \code{ quadvals }.

\end{ldescription}
\end{Arguments}
\begin{Details}\relax
Previous versions of the 'fda' software used the name \code{basis}
for this class, and the code in Matlab still does.  However, this
class name was already used elsewhere in the S languages, and there
was a potential for a clash that might produce mysterious and perhaps
disastrous consequences.

To check that an object is of this class, use function
\code{\LinkA{is.basis}{is.basis}}.

It is comparatively simple to add new basis types.  The code in
the following functions needs to be estended to allow for the new
type: \code{\LinkA{basisfd}{basisfd}}, \code{\LinkA{use.proper.basis}{use.proper.basis}},
\code{\LinkA{getbasismatrix}{getbasismatrix}} and \code{\LinkA{getbasispenalty}{getbasispenalty}}.
In addition, a new "create" function should be written for the
new type, as well as functions analogous to \code{\LinkA{fourier}{fourier}} and
\code{\LinkA{fourierpen}{fourierpen}} for evaluating basis functions for basis
penalty matrices.

The "create" function names are rather long, and users who mind
all that typing might be advised to modify these to versions with
shorter names, such as "splbas", "conbas", and etc.  However, a
principle of good programming practice is to keep the code readable,
preferably by somebody other than the programmer.

Normally only developers of new basis types will actually need
to use this function, so no examples are provided.
\end{Details}
\begin{Value}
an object of class \code{basisfd}
\end{Value}
\begin{Source}\relax
Ramsay, James O., and Silverman, Bernard W. (2006), \emph{Functional
Data Analysis, 2nd ed.}, Springer, New York.

Ramsay, James O., and Silverman, Bernard W. (2002), \emph{Applied
Functional Data Analysis}, Springer, New York
\end{Source}
\begin{SeeAlso}\relax
\code{\LinkA{is.basis}{is.basis}}, 
\code{\LinkA{is.eqbasis}{is.eqbasis}}, 
\code{\LinkA{plot.basisfd}{plot.basisfd}}, 
\code{\LinkA{getbasismatrix}{getbasismatrix}}, 
\code{\LinkA{getbasispenalty}{getbasispenalty}}, 
\code{\LinkA{create.bspline.basis}{create.bspline.basis}}, 
\code{\LinkA{create.constant.basis}{create.constant.basis}}, 
\code{\LinkA{create.exponential.basis}{create.exponential.basis}}, 
\code{\LinkA{create.fourier.basis}{create.fourier.basis}}, 
\code{\LinkA{create.monomial.basis}{create.monomial.basis}}, 
\code{\LinkA{create.polygonal.basis}{create.polygonal.basis}}, 
\code{\LinkA{create.polynomial.basis}{create.polynomial.basis}}, 
\code{\LinkA{create.power.basis}{create.power.basis}}
\end{SeeAlso}

\HeaderA{bifd}{Create a bivariate functional data object}{bifd}
\keyword{attribute}{bifd}
\begin{Description}\relax
This function creates a bivariate functional data object, which
consists of two bases for expanding a functional data object of two 
variables, s and t, and a set of coefficients defining this expansion.
The bases are contained in "basisfd" objects.
\end{Description}
\begin{Usage}
\begin{verbatim}
bifd (coef=matrix(0,2,1), sbasisobj=create.bspline.basis(),
      tbasisobj=create.bspline.basis(), fdnames=defaultnames)
\end{verbatim}
\end{Usage}
\begin{Arguments}
\begin{ldescription}
\item[\code{coef}] a two-, three-, or four-dimensional array containing
coefficient values for the expansion of each set of bivariate
function values=terms of a set of basis function values

If 'coef' is two dimensional, this implies that there is only
one variable and only one replication.  In that case, 
the first and second dimensions correspond to
the basis functions for the first and second argument,
respectively.

If 'coef' is three dimensional, this implies that there are multiple
replicates on only one variable.  In that case, 
the first and second dimensions correspond to
the basis functions for the first and second argument,
respectively, and the third dimension corresponds to
replications.

If 'coef' has four dimensions, the fourth dimension
corresponds to variables.

\item[\code{sbasisobj}] a functional data basis object
for the first argument s of the bivariate function.  

\item[\code{tbasisobj}] a functional data basis object
for the second argument t of the bivariate function.  

\item[\code{fdnames}] A list of length 4 containing dimnames for 'coefs' if it is a
4-dimensional array.  If it is only 2- or 3-dimensional, the later
components of fdnames are not applied to 'coefs'.  In any event, the
components of fdnames describe the following:

(1) The row of 'coefs' corresponding to the bases in sbasisobj.
Defaults to sbasisobj[["names"]] if non-null and of the proper
length, or to existing dimnames(coefs)[[1]] if non-null and of the
proper length, and to 's1', 's2', ...,
otherwise.  

(2) The columns of 'coefs' corresponding to the bases in tbasisobj.
Defaults to tbasisobj[["names"]] if non-null and of the proper
length, or to existing dimnames(coefs)[[2]] if non-null and of the
proper length, and to 't1', 't2', ...,  
otherwise.  

(3) The replicates.  Defaults to dimnames(coefs)[[3]] if non-null
and of the proper length, and to 'rep1', ..., otherwise.

(4) Variable names.  Defaults to dimnames(coefs)[[4]] if non-null
and of the proper length, and to 'var1', ..., otherwise.  

\end{ldescription}
\end{Arguments}
\begin{Value}
A bivariate functional data object = a list of class 'bifd' 
with the following components:

\begin{ldescription}
\item[\code{coefs}] the input 'coefs' possible with dimnames from dfnames if provided or
from sbasisobj$names and tbasisobsj$names

\item[\code{sbasisobj}] a functional data basis object
for the first argument s of the bivariate function.  

\item[\code{tbasisobj}] a functional data basis object
for the second argument t of the bivariate function.  

\item[\code{bifdnames}] a list of length 4 giving names for the dimensions of coefs, with
one or two unused lists of names if length(dim(coefs)) is only two
or one, respectively.  

\end{ldescription}
\end{Value}
\begin{Author}\relax
Spencer Graves
\end{Author}
\begin{SeeAlso}\relax
\code{\LinkA{basisfd}{basisfd}}
\code{\LinkA{data2fd}{data2fd}} 
\code{\LinkA{objAndNames}{objAndNames}}
\end{SeeAlso}
\begin{Examples}
\begin{ExampleCode}
Bspl2 <- create.bspline.basis(nbasis=2, norder=1)
Bspl3 <- create.bspline.basis(nbasis=3, norder=2)

(bBspl2.3 <- bifd(array(1:6, dim=2:3), Bspl2, Bspl3))
str(bBspl2.3)

\end{ExampleCode}
\end{Examples}

\HeaderA{bsplinepen}{B-Spline Penalty Matrix}{bsplinepen}
\keyword{smooth}{bsplinepen}
\begin{Description}\relax
Computes the matrix defining the roughness penalty for functions
expressed in terms of a B-spline basis.
\end{Description}
\begin{Usage}
\begin{verbatim}
bsplinepen(basisobj, Lfdobj=2, rng=basisobj$rangeval)
\end{verbatim}
\end{Usage}
\begin{Arguments}
\begin{ldescription}
\item[\code{basisobj}] a B-spline basis object.

\item[\code{Lfdobj}] either a nonnegative integer or a linear differential operator object.

\item[\code{rng}] a vector of length 2 defining range over which the basis penalty is to
be computed.

\end{ldescription}
\end{Arguments}
\begin{Details}\relax
A roughness penalty for a function $x(t)$ is defined by
integrating the square of either the derivative of $x(t)$ or,
more generally, the result of applying a linear differential operator
$L$ to it.  The most common roughness penalty is the integral of
the square of the second derivative, and
this is the default. To apply this roughness penalty, the matrix of
inner products of the basis functions (possibly after applying the
linear differential operator to them) defining this function
is necessary. This function just calls the roughness penalty evaluation
function specific to the basis involved.
\end{Details}
\begin{Value}
a symmetric matrix of order equal to the number of basis functions
defined by the B-spline basis object.  Each element is the inner product
of two B-spline basis functions after applying the derivative or linear
differential operator defined by \code{Lfdobj}.
\end{Value}
\begin{Examples}
\begin{ExampleCode}
##
## bsplinepen with only one basis function
##
bspl1.1 <- create.bspline.basis(nbasis=1, norder=1)
pen1.1 <- bsplinepen(bspl1.1, 0) 

##
## bspline pen for a cubic spline with knots at seq(0, 1, .1)
##
basisobj <- create.bspline.basis(c(0,1),13)
#  compute the 13 by 13 matrix of inner products of second derivatives
penmat <- bsplinepen(basisobj)
\end{ExampleCode}
\end{Examples}

\HeaderA{bsplineS}{B-spline Basis Function Values}{bsplineS}
\keyword{smooth}{bsplineS}
\begin{Description}\relax
Evaluates a set of B-spline basis functions, or a derivative of these
functions, at a set of arguments.
\end{Description}
\begin{Usage}
\begin{verbatim}
bsplineS(x, breaks, norder=4, nderiv=0)
\end{verbatim}
\end{Usage}
\begin{Arguments}
\begin{ldescription}
\item[\code{x}] A vector of argument values at which the B-spline basis functions
are to be evaluated.

\item[\code{breaks}] A strictly increasing set of break values defining the B-spline
basis.  The argument values \code{x} should be within the interval
spanned by the break values.

\item[\code{norder}] The order of the B-spline basis functions.  The order less one is
the degree of the piece-wise polynomials that make up any B-spline
function. The default is order 4, meaning piece-wise cubic.

\item[\code{nderiv}] A nonnegative integer specifying the order of derivative to be
evaluated.  The derivative must not exceed the order.  The default 
derivative is 0, meaning that the basis functions themselves are
evaluated. 

\end{ldescription}
\end{Arguments}
\begin{Value}
a matrix of function values.  The number of rows equals the number of 
arguments, and the number of columns equals the number of basis
functions.
\end{Value}
\begin{Examples}
\begin{ExampleCode}
# Minimal example:  A B-spline of order 1 (i.e., a step function)
# with 0 interior knots:
bsplineS(seq(0, 1, .2), 0:1, 1, 0)

#  set up break values at 0.0, 0.2,..., 0.8, 1.0.
breaks <- seq(0,1,0.2)
#  set up a set of 11 argument values
x <- seq(0,1,0.1)
#  the order willl be 4, and the number of basis functions
#  is equal to the number of interior break values (4 here)
#  plus the order, for a total here of 8.
norder <- 4
#  compute the 11 by 8 matrix of basis function values
basismat <- bsplineS(x, breaks, norder)
\end{ExampleCode}
\end{Examples}

\HeaderA{CanadianWeather}{Canadian average annual weather cycle}{CanadianWeather}
\aliasA{daily}{CanadianWeather}{daily}
\keyword{datasets}{CanadianWeather}
\begin{Description}\relax
Daily temperature and precipitation at 35 different locations
in Canada averaged over 1960 to 1994.
\end{Description}
\begin{Usage}
\begin{verbatim}
CanadianWeather
daily
\end{verbatim}
\end{Usage}
\begin{Format}\relax
'CanadianWeather' and 'daily' are lists containing essentially the
same data.  'CanadianWeather' may be preferred for most purposes;
'daily' is included primarily for compatability with scripts written
before the other format became available and for compatability with
the Matlab 'fda' code.  

\item[CanadianWeather] A list with the following components:

\Itemize{
\item[dailyAv] a three dimensional array c(365, 35, 3) summarizing data
collected at 35 different weather stations in Canada on the
following:

[,,1] = [,, 'Temperature.C']:  average daily temperature for
each day of the year 

[,,2] = [,, 'Precipitation.mm']:  average daily rainfall for
each day of the year rounded to 0.1 mm.  

[,,3] = [,, 'log10precip']:  base 10 logarithm of
Precipitation.mm after first replacing 27 zeros by 0.05 mm
(Ramsay and Silverman 2006, p. 248).   
    
\item[place] Names of the 35 different weather stations in Canada whose data
are summarized in 'dailyAv'.  These names vary between 6 and 11
characters in length.  By contrast, daily[["place"]] which are
all 11 characters, with names having fewer characters being
extended with trailing blanks.  

\item[province] names of the Canadian province containing each place

\item[coordinates] a numeric matrix giving 'N.longitude' and 'W.latitude' for each
place.  

\item[region] Which of 4 climate zones contain each place:  Atlantic, Pacific,
Continental, Arctic.  

\item[monthlyTemp] A matrix of dimensions (12, 35) giving the average temperature
in degrees celcius for each month of the year.  

\item[monthlyPrecip] A matrix of dimensions (12, 35) giving the average daily 
precipitation in milimeters for each month of the year.   

\item[geogindex] Order the weather stations from East to West to North 

}


\item[daily] A list with the following components:

\Itemize{
\item[place] Names of the 35 different weather stations in Canada whose data
are summarized in 'dailyAv'.  These names are all 11 characters,
with shorter names being extended with trailing blanks.  This is
different from CanadianWeather[["place"]], where trailing blanks
have been dropped.    

\item[tempav] a matrix of dimensions (365, 35) giving the average temperature
in degrees celcius for each day of the year.  This is
essentially the same as CanadianWeather[["dailyAv"]][,,
"Temperature.C"]. 

\item[precipav] a matrix of dimensions (365, 35) giving the average temperature
in degrees celcius for each day of the year.  This is
essentially the same as CanadianWeather[["dailyAv"]][,,
"Precipitation.mm"]. 

}
\end{Format}
\begin{Source}\relax
Ramsay, James O., and Silverman, Bernard W. (2006), \emph{Functional
Data Analysis, 2nd ed.}, Springer, New York.

Ramsay, James O., and Silverman, Bernard W. (2002), \emph{Applied
Functional Data Analysis}, Springer, New York
\end{Source}
\begin{SeeAlso}\relax
\code{\LinkA{monthAccessories}{monthAccessories}}
\end{SeeAlso}
\begin{Examples}
\begin{ExampleCode}
data(CanadianWeather) 
# Expand the left margin to allow space for place names 
op <- par(mar=c(5, 4, 4, 5)+.1)
# Plot
stations <- c("Pr. Rupert", "Montreal", "Edmonton", "Resolute")
matplot(day.5, CanadianWeather$dailyAv[, stations, "Temperature.C"],
        type="l", axes=FALSE, xlab="", ylab="Mean Temperature (deg C)") 
axis(2, las=1)
# Label the horizontal axis with the month names
axis(1, monthBegin.5, labels=FALSE)
axis(1, monthEnd.5, labels=FALSE)
axis(1, monthMid, monthLetters, tick=FALSE)
# Add the monthly averages 
matpoints(monthMid, CanadianWeather$monthlyTemp[, stations])
# Add the names of the weather stations
mtext(stations, side=4,
      at=CanadianWeather$dailyAv[365, stations, "Temperature.C"],
     las=1)
# clean up 
par(op)
\end{ExampleCode}
\end{Examples}

\HeaderA{cca.fd}{Functional Canonical Correlation Analysis}{cca.fd}
\keyword{smooth}{cca.fd}
\begin{Description}\relax
Carry out a functional canonical correlation analysis with
regularization or roughness penalties on the estimated
canonical variables.
\end{Description}
\begin{Usage}
\begin{verbatim}
cca.fd(fdobj1, fdobj2=fdobj1, ncan = 2,
       ccafdParobj1=fdPar(basisobj1, 2, 1e-10),
       ccafdParobj2=ccafdParobj1, centerfns=TRUE)
\end{verbatim}
\end{Usage}
\begin{Arguments}
\begin{ldescription}
\item[\code{fdobj1}] a functional data object.

\item[\code{fdobj2}] a functional data object.  By default this is \code{ fdobj1 }, in
which case the first argument must be a bivariate funnctional data
object.

\item[\code{ncan}] the number of canonical variables and weight functions to be
computed.  The default is 2.

\item[\code{ccafdParobj1}] a functional parameter object defining the first set of canonical
weight functions.  The object may contain specifications for a
roughness penalty. The default is defined using the same basis
as that used for \code{ fdobj1 } with a slight penalty on its
second derivative.

\item[\code{ccafdParobj2}] a functional parameter object defining the second set of canonical
weight functions.  The object may contain specifications for a
roughness penalty. The default is \code{ ccafdParobj1 }.

\item[\code{centerfns}] if TRUE, the functions are centered prior to analysis. This is the
default.

\end{ldescription}
\end{Arguments}
\begin{Value}
an object of class \code{cca.fd} with the 5 slots:

\begin{ldescription}
\item[\code{ccwtfd1}] a functional data object for the first
canonical variate weight function

\item[\code{ccwtfd2}] a functional data object for the second
canonical variate weight function

\item[\code{cancorr}] a vector of canonical correlations

\item[\code{ccavar1}] a matrix of scores on the first canonical variable.

\item[\code{ccavar2}] a matrix of scores on the second canonical variable.

\end{ldescription}
\end{Value}
\begin{SeeAlso}\relax
\code{\LinkA{plot.cca.fd}{plot.cca.fd}}, 
\code{\LinkA{varmx.cca.fd}{varmx.cca.fd}}, 
\code{\LinkA{pca.fd}{pca.fd}}
\end{SeeAlso}
\begin{Examples}
\begin{ExampleCode}
#  Canonical correlation analysis of knee-hip curves

gaittime  <- (1:20)/21
gaitrange <- c(0,1)
gaitbasis <- create.fourier.basis(gaitrange,21)
lambda    <- 10^(-11.5)
harmaccelLfd <- vec2Lfd(c(0, 0, (2*pi)^2, 0))

gaitfdPar <- fdPar(gaitbasis, harmaccelLfd, lambda)
gaitfd <- smooth.basis(gaittime, gait, gaitfdPar)$fd

ccafdPar <- fdPar(gaitfd, harmaccelLfd, 1e-8)
ccafd0    <- cca.fd(gaitfd[,1], gaitfd[,2], ncan=3, ccafdPar, ccafdPar)
#  compute a VARIMAX rotation of the canonical variables
ccafd <- varmx.cca.fd(ccafd0)
#  plot the canonical weight functions
op <- par(mfrow=c(2,1))
#plot.cca.fd(ccafd, cex=1.2, ask=TRUE)
#plot.cca.fd(ccafd, cex=1.2)
#  display the canonical correlations
#round(ccafd$ccacorr[1:6],3)
par(op)
\end{ExampleCode}
\end{Examples}

\HeaderA{center.fd}{Center Functional Data}{center.fd}
\keyword{smooth}{center.fd}
\begin{Description}\relax
Subtract the pointwise mean from each of the functions
in a functional data object; that is, to center them on the mean function.
\end{Description}
\begin{Usage}
\begin{verbatim}
center.fd(fdobj)
\end{verbatim}
\end{Usage}
\begin{Arguments}
\begin{ldescription}
\item[\code{fdobj}] a functional data object to be centered.

\end{ldescription}
\end{Arguments}
\begin{Value}
a functional data object whose mean is zero.
\end{Value}
\begin{SeeAlso}\relax
\code{\LinkA{mean.fd}{mean.fd}}, 
\code{\LinkA{sum.fd}{sum.fd}}, 
\code{\LinkA{stddev.fd}{stddev.fd}}, 
\code{\LinkA{std.fd}{std.fd}}
\end{SeeAlso}
\begin{Examples}
\begin{ExampleCode}
daytime    <- (1:365)-0.5
daybasis   <- create.fourier.basis(c(0,365), 365)
harmLcoef  <- c(0,(2*pi/365)^2,0)
harmLfd    <- vec2Lfd(harmLcoef, c(0,365))
templambda <- 0.01
tempfdPar  <- fdPar(daybasis, harmLfd, templambda)
tempfd     <- smooth.basis(daytime,
       CanadianWeather$dailyAv[,,"Temperature.C"], tempfdPar)$fd
tempctrfd  <- center.fd(tempfd)

plot(tempctrfd, xlab="Day", ylab="deg. C",
     main = "Centered temperature curves")
\end{ExampleCode}
\end{Examples}

\HeaderA{checkDims3}{Compare dimensions and dimnames of arrays}{checkDims3}
\aliasA{checkDim3}{checkDims3}{checkDim3}
\keyword{utilities}{checkDims3}
\begin{Description}\relax
Compare selected dimensions and dimnames of arrays, coercing objects
to 3-dimensional arrays and either give an error or force matching.
\end{Description}
\begin{Usage}
\begin{verbatim}
checkDim3(x, y=NULL, xdim=1, ydim=1, defaultNames='x',
         subset=c('xiny', 'yinx', 'neither'),
         xName=substring(deparse(substitute(x)), 1, 33),
         yName=substring(deparse(substitute(y)), 1, 33) )
checkDims3(x, y=NULL, xdim=2:3, ydim=2:3, defaultNames='x',
         subset=c('xiny', 'yinx', 'neither'),
         xName=substring(deparse(substitute(x)), 1, 33),
         yName=substring(deparse(substitute(y)), 1, 33) )
\end{verbatim}
\end{Usage}
\begin{Arguments}
\begin{ldescription}
\item[\code{x, y}] arrays to be compared.  If \code{y} is missing, \code{x} is used.

Currently, both \code{x} and \code{y} can have at most 3
dimensions.  If either has more, an error will be thrown.  If either
has fewer, it will be expanded to 3 dimensions using
\code{as.array3}. 

\item[\code{xdim, ydim}] For \code{checkDim3}, these are positive integers indicating which
dimension of \code{x} will be compared with which dimension of
\code{y}.  

For \code{checkDims3}, these are positive integer vectors of the same
length, passed one at a time to \code{checkDim3}.  The default here
is to force matching dimensions for \code{\LinkA{plotfit.fd}{plotfit.fd}}.  

\item[\code{defaultNames}] Either NULL, FALSE or a character string or vector or list.  If
NULL, no checking is done of dimnames.  If FALSE, an error is thrown
unless the corresponding dimensions of \code{x} and \code{y} match
exactly. 

If it is a character string, vector, or list, it is used as the
default names if neither \code{x} nor \code{y} have dimenames for
the compared dimensions.  If it is a character vector that is too
short, it is extended to the required length using
paste(defaultNames, 1:ni), where \code{ni} = the required length. 

If it is a list, it should have length (length(xdim)+1).  Each
component must be either a character vector or NULL.  If neither
\code{x} nor \code{y} have dimenames for the first compared
dimensions, defaultNames[[1]] will be used instead unless it is
NULL, in which case the last component of defaultNames will be
used.  If it is null, an error is thrown.  

\item[\code{subset}] If 'xiny', and any(dim(y)[ydim] < dim(x)[xdim]), an error is
thrown.  Else if any(dim(y)[ydim] > dim(x)[xdim]) the larger is
reduced to match the smaller.  If 'yinx', this procedure is
reversed.

If 'neither', any dimension mismatch generates an error.  

\item[\code{xName, yName}] names of the arguments \code{x} and \code{y}, used only to in error
messages.  

\end{ldescription}
\end{Arguments}
\begin{Details}\relax
For \code{checkDims3}, confirm that \code{xdim} and \code{ydim} have
the same length, and call \code{checkDim3} for each pair.  

For \code{checkDim3}, proceed as follows:

1.  if((xdim>3) | (ydim>3)) throw an error.

2.  ixperm <- list(1:3, c(2, 1, 3), c(3, 2, 1))[xdim];
iyperm <- list(1:3, c(2, 1, 3), c(3, 2, 1))[ydim];

3.  x3 <- aperm(as.array3(x), ixperm);
y3 <- aperm(as.array3(y), iyperm) 

4.  xNames <- dimnames(x3);  yNames <- dimnames(y3) 

5.  Check subset.  For example, for subset='xiny', use the following:
if(is.null(xNames)){
if(dim(x3)[1]>dim(y3)[1]) stop
else y. <- y3[1:dim(x3)[1],,]
dimnames(x) <- list(yNames[[1]], NULL, NULL) 
}
else {
if(is.null(xNames[[1]])){
if(dim(x3)[1]>dim(y3)[1]) stop
else y. <- y3[1:dim(x3)[1],,]
dimnames(x3)[[1]] <- yNames[[1]]
}
else {
if(any(!is.element(xNames[[1]], yNames[[1]])))stop
else y. <- y3[xNames[[1]],,]
}
}

6.  return(list(x=aperm(x3, ixperm), y=aperm(y., iyperm)))
\end{Details}
\begin{Value}
a list with components \code{x} and \code{y}.
\end{Value}
\begin{Author}\relax
Spencer Graves
\end{Author}
\begin{SeeAlso}\relax
\code{\LinkA{as.array3}{as.array3}}
\code{\LinkA{plotfit.fd}{plotfit.fd}}
\end{SeeAlso}
\begin{Examples}
\begin{ExampleCode}
# Select the first two rows of y 
stopifnot(all.equal( 
checkDim3(1:2, 3:5),
list(x=array(1:2, c(2,1,1), list(c('x1','x2'), NULL, NULL)), 
     y=array(3:4, c(2,1,1), list(c('x1','x2'), NULL, NULL)) )
)) 

# Select the first two rows of a matrix y 
stopifnot(all.equal(
checkDim3(1:2, matrix(3:8, 3)),
list(x=array(1:2,         c(2,1,1), list(c('x1','x2'), NULL, NULL)), 
     y=array(c(3:4, 6:7), c(2,2,1), list(c('x1','x2'), NULL, NULL)) )
))

# Select the first column of y
stopifnot(all.equal(
checkDim3(1:2, matrix(3:8, 3), 2, 2), 
list(x=array(1:2,         c(2,1,1), list(NULL, 'x', NULL)), 
     y=array(3:5, c(3,1,1), list(NULL, 'x', NULL)) )
))

# Select the first two rows and the first column of y
stopifnot(all.equal(
checkDims3(1:2, matrix(3:8, 3), 1:2, 1:2),
list(x=array(1:2, c(2,1,1), list(c('x1','x2'), 'x', NULL)), 
     y=array(3:4, c(2,1,1), list(c('x1','x2'), 'x', NULL)) ) 
))

# Select the first 2 rows of y 
x1 <- matrix(1:4, 2, dimnames=list(NULL, LETTERS[2:3]))
x1a <- x1. <- as.array3(x1)
dimnames(x1a)[[1]] <- c('x1', 'x2') 
y1 <- matrix(11:19, 3, dimnames=list(NULL, LETTERS[1:3]))
y1a <- y1. <- as.array3(y1) 
dimnames(y1a)[[1]] <- c('x1', 'x2', 'x3')

stopifnot(all.equal(
checkDim3(x1, y1),
list(x=x1a, y=y1a[1:2, , , drop=FALSE])
))

# Select columns 2 & 3 of y 
stopifnot(all.equal(
checkDim3(x1, y1, 2, 2),
list(x=x1., y=y1.[, 2:3, , drop=FALSE ])
))

# Select the first 2 rows and  columns 2 & 3 of y 
stopifnot(all.equal(
checkDims3(x1, y1, 1:2, 1:2),
list(x=x1a, y=y1a[1:2, 2:3, , drop=FALSE ])
)) 

# y = columns 2 and 3 of x 
x23 <- matrix(1:6, 2, dimnames=list(letters[2:3], letters[1:3]))
x23. <- as.array3(x23) 
stopifnot(all.equal(
checkDim3(x23, xdim=1, ydim=2),
list(x=x23., y=x23.[, 2:3,, drop=FALSE ])
))

# Transfer dimnames from y to x
x4a <- x4 <- matrix(1:4, 2)
y4 <- matrix(5:8, 2, dimnames=list(letters[1:2], letters[3:4]))
dimnames(x4a) <- dimnames(t(y4))
stopifnot(all.equal(
checkDims3(x4, y4, 1:2, 2:1),
list(x=as.array3(x4a), y=as.array3(y4))
))

# as used in plotfit.fd
daybasis65 <- create.fourier.basis(c(0, 365), 65)

daytempfd <- with(CanadianWeather, data2fd(
       dailyAv[,,"Temperature.C"], day.5, 
       daybasis65, argnames=list("Day", "Station", "Deg C")) )

defaultNms <- with(daytempfd, c(fdnames[2], fdnames[3], x='x'))
subset <- checkDims3(CanadianWeather$dailyAv[, , "Temperature.C"],
               daytempfd$coef, defaultNames=defaultNms)
# Problem:  dimnames(...)[[3]] = '1' 
# Fix:  
subset3 <- checkDims3(
        CanadianWeather$dailyAv[, , "Temperature.C", drop=FALSE],
               daytempfd$coef, defaultNames=defaultNms)
\end{ExampleCode}
\end{Examples}

\HeaderA{checkLogicalInteger}{Does an argument satisfy required conditions?}{checkLogicalInteger}
\aliasA{checkLogical}{checkLogicalInteger}{checkLogical}
\aliasA{checkNumeric}{checkLogicalInteger}{checkNumeric}
\keyword{attribute}{checkLogicalInteger}
\keyword{utilities}{checkLogicalInteger}
\begin{Description}\relax
Check whether an argument is a logical vector of a certain length or a
numeric vector in a certain range and issue an appropriate error or
warning if not:

\code{checkLogical} throws an error or returns FALSE with a warning
unless \code{x} is a  logical vector of exactly the required
\code{length}.

\code{checkNumeric} throws an error or returns FALSE with a warning
unless \code{x} is either NULL or a \code{numeric} vector of at most
\code{length} with \code{x} in the desired range.  

\code{checkLogicalInteger} returns a logical vector of exactly
\code{length} unless \code{x} is neither NULL nor \code{logical} of
the required \code{length} nor \code{numeric} with \code{x} in the
desired range.
\end{Description}
\begin{Usage}
\begin{verbatim}
checkLogical(x, length., warnOnly=FALSE)
checkNumeric(x, lower, upper, length., integer=TRUE, unique=TRUE,
             inclusion=c(TRUE,TRUE), warnOnly=FALSE)
checkLogicalInteger(x, length., warnOnly=FALSE)
\end{verbatim}
\end{Usage}
\begin{Arguments}
\begin{ldescription}
\item[\code{x}] an object to be checked 
\item[\code{length.}] The required length for \code{x} if \code{logical} and not NULL or
the maximum length if \code{numeric}.  

\item[\code{lower, upper}] lower and upper limits for \code{x}.  

\item[\code{integer}] logical:  If true, a \code{numeric} \code{x} must be
\code{integer}.  

\item[\code{unique}] logical:  TRUE if duplicates are NOT allowed in \code{x}.  

\item[\code{inclusion}] logical vector of length 2, similar to
\code{link[ifultools]\{checkRange\}}:  

if(inclusion[1]) (lower <= x) else (lower < x)

if(inclusion[2]) (x <= upper) else (x < upper)

\item[\code{warnOnly}] logical:  If TRUE, violations are reported as warnings, not as
errors.  

\end{ldescription}
\end{Arguments}
\begin{Details}\relax
1.  xName <- deparse(substitute(x)) to use in any required error or
warning.  

2.  if(is.null(x)) handle appropriately:  Return FALSE for
\code{checkLogical}, TRUE for \code{checkNumeric} and rep(TRUE,
length.) for \code{checkLogicalInteger}.  

3.  Check class(x).

4.  Check other conditions.
\end{Details}
\begin{Value}
\code{checkLogical} returns a logical vector of the required
\code{length.}, unless it issues an error message.

\code{checkNumeric} returns a numeric vector of at most \code{length.}
with all elements between \code{lower} and \code{upper}, and
optionally \code{unique}, unless it issues an error message.

\code{checkLogicalInteger} returns a logical vector of the required
\code{length.}, unless it issues an error message.
\end{Value}
\begin{Author}\relax
Spencer Graves
\end{Author}
\begin{SeeAlso}\relax
\code{\LinkA{checkVectorType}{checkVectorType}},
\code{\LinkA{checkRange}{checkRange}}
\code{\LinkA{checkScalarType}{checkScalarType}}
\code{\LinkA{isVectorAtomic}{isVectorAtomic}}
\end{SeeAlso}
\begin{Examples}
\begin{ExampleCode}
##
## checkLogical
##
checkLogical(NULL, length=3, warnOnly=TRUE)
checkLogical(c(FALSE, TRUE, TRUE), length=4, warnOnly=TRUE)
checkLogical(c(FALSE, TRUE, TRUE), length=3)

##
## checkNumeric
##
checkNumeric(NULL, lower=1, upper=3)
checkNumeric(1:3, 1, 3)
checkNumeric(1:3, 1, 3, inclusion=FALSE, warnOnly=TRUE)
checkNumeric(pi, 1, 4, integer=TRUE, warnOnly=TRUE)
checkNumeric(c(1, 1), 1, 4, warnOnly=TRUE)
checkNumeric(c(1, 1), 1, 4, unique=FALSE, warnOnly=TRUE)

##
## checkLogicalInteger
##
checkLogicalInteger(NULL, 3)
checkLogicalInteger(c(FALSE, TRUE), warnOnly=TRUE) 
checkLogicalInteger(1:2, 3) 
checkLogicalInteger(2, warnOnly=TRUE) 
checkLogicalInteger(c(2, 4), 3, warnOnly=TRUE)

##
## checkLogicalInteger names its calling function 
## rather than itself as the location of error detection
## if possible
##
tstFun <- function(x, length., warnOnly=FALSE){
   checkLogicalInteger(x, length., warnOnly) 
}
tstFun(NULL, 3)
tstFun(4, 3, warnOnly=TRUE)

tstFun2 <- function(x, length., warnOnly=FALSE){
   tstFun(x, length., warnOnly)
}
tstFun2(4, 3, warnOnly=TRUE)

\end{ExampleCode}
\end{Examples}

\HeaderA{coef.fd}{Extract functional coefficients}{coef.fd}
\aliasA{coef.fdPar}{coef.fd}{coef.fdPar}
\aliasA{coef.fdSmooth}{coef.fd}{coef.fdSmooth}
\aliasA{coef.Taylor}{coef.fd}{coef.Taylor}
\aliasA{coefficients.fd}{coef.fd}{coefficients.fd}
\aliasA{coefficients.fdPar}{coef.fd}{coefficients.fdPar}
\aliasA{coefficients.fdSmooth}{coef.fd}{coefficients.fdSmooth}
\aliasA{coefficients.Taylor}{coef.fd}{coefficients.Taylor}
\keyword{utilities}{coef.fd}
\begin{Description}\relax
Obtain the coefficients component from a functional object (functional
data, class \code{fd}, functional parameter, class \code{fdPar}, a
functional smooth, class \code{fdSmooth}, or a Taylor spline
representation, class \code{Taylor}.
\end{Description}
\begin{Usage}
\begin{verbatim}
## S3 method for class 'fd':
coef(object, ...)
## S3 method for class 'fdPar':
coef(object, ...)
## S3 method for class 'fdSmooth':
coef(object, ...)
## S3 method for class 'Taylor':
coef(object, ...)
## S3 method for class 'fd':
coefficients(object, ...)
## S3 method for class 'fdPar':
coefficients(object, ...)
## S3 method for class 'fdSmooth':
coefficients(object, ...)
## S3 method for class 'Taylor':
coefficients(object, ...)
\end{verbatim}
\end{Usage}
\begin{Arguments}
\begin{ldescription}
\item[\code{object}] An object whose functional coefficients are desired 

\item[\code{... }] other arguments 

\end{ldescription}
\end{Arguments}
\begin{Details}\relax
Functional representations are evaluated by multiplying a basis
function matrix times a coefficient vector, matrix or 3-dimensional
array. (The basis function matrix contains the basis functions as
columns evaluated at the \code{evalarg} values as rows.)
\end{Details}
\begin{Value}
A numeric vector or array of the coefficients.
\end{Value}
\begin{SeeAlso}\relax
\code{\LinkA{coef}{coef}}
\code{\LinkA{fd}{fd}}
\code{\LinkA{fdPar}{fdPar}}
\code{\LinkA{smooth.basisPar}{smooth.basisPar}}
\code{\LinkA{smooth.basis}{smooth.basis}}
\end{SeeAlso}
\begin{Examples}
\begin{ExampleCode}
##
## coef.fd
##
bspl1.1 <- create.bspline.basis(norder=1, breaks=0:1)
fd.bspl1.1 <- fd(0, basisobj=bspl1.1)
coef(fd.bspl1.1)


##
## coef.fdPar 
##
rangeval <- c(-3,3)
#  set up some standard normal data
x <- rnorm(50)
#  make sure values within the range
x[x < -3] <- -2.99
x[x >  3] <-  2.99
#  set up basis for W(x)
basisobj <- create.bspline.basis(rangeval, 11)
#  set up initial value for Wfdobj
Wfd0 <- fd(matrix(0,11,1), basisobj)
WfdParobj <- fdPar(Wfd0)

coef(WfdParobj)


##
## coef.fdSmooth
##

girlGrowthSm <- with(growth, smooth.basisPar(argvals=age, y=hgtf))
coef(girlGrowthSm)


##
## coef.Taylor 
##
# coming soon.

\end{ExampleCode}
\end{Examples}

\HeaderA{cor.fd}{Correlation matrix from functional data object(s)}{cor.fd}
\keyword{smooth}{cor.fd}
\begin{Description}\relax
Compute a correlation matrix for one or two functional data objects.
\end{Description}
\begin{Usage}
\begin{verbatim}
cor.fd(evalarg1, fdobj1, evalarg2=evalarg1, fdobj2=fdobj1)
\end{verbatim}
\end{Usage}
\begin{Arguments}
\begin{ldescription}
\item[\code{evalarg1}] a vector of argument values for fdobj1.   

\item[\code{evalarg2}] a vector of argument values for fdobj2.  

\item[\code{fdobj1, fdobj2}] functional data objects 

\end{ldescription}
\end{Arguments}
\begin{Details}\relax
1.  var1 <- var.fd(fdobj1) 

2.  evalVar1 <- eval.bifd(evalarg1, evalarg1, var1)


3.  if(missing(fdobj2)) Convert evalVar1 to correlations


4.  else:  

4.1.  var2 <- var.fd(fdobj2)

4.2.  evalVar2 <- eval.bifd(evalarg2, evalarg2, var2)

4.3.  var12 <- var.df(fdobj1, fdobj2)

4.4.  evalVar12 <- eval.bifd(evalarg1, evalarg2, var12)

4.5.  Convert evalVar12 to correlations
\end{Details}
\begin{Value}
A matrix or array:

With one or two functional data objects, fdobj1 and possibly fdobj2,
the value is a matrix of dimensions length(evalarg1) by length(evalarg2) giving the
correlations at those points of fdobj1 if missing(fdojb2) or of
correlations between eval.fd(evalarg1, fdobj1) and eval.fd(evalarg2,
fdobj2).

With a single multivariate data object with k variables, the value is
a 4-dimensional array of dim = c(nPts, nPts, 1, choose(k+1, 2)), where
nPts = length(evalarg1).
\end{Value}
\begin{SeeAlso}\relax
\code{\LinkA{mean.fd}{mean.fd}}, 
\code{\LinkA{sd.fd}{sd.fd}}, 
\code{\LinkA{std.fd}{std.fd}}
\code{\LinkA{stdev.fd}{stdev.fd}}
\code{\LinkA{var.fd}{var.fd}}
\end{SeeAlso}
\begin{Examples}
\begin{ExampleCode}
daybasis3 <- create.fourier.basis(c(0, 365))
daybasis5 <- create.fourier.basis(c(0, 365), 5)
tempfd3 <- with(CanadianWeather, data2fd(
       dailyAv[,,"Temperature.C"], day.5,
       daybasis3, argnames=list("Day", "Station", "Deg C")) )
precfd5 <- with(CanadianWeather, data2fd(
       dailyAv[,,"log10precip"], day.5,
       daybasis5, argnames=list("Day", "Station", "Deg C")) )

# Correlation matrix for a single functional data object
(tempCor3 <- cor.fd(seq(0, 356, length=4), tempfd3))

# Cross correlation matrix between two functional data objects 
# Compare with structure described above under 'value':
(tempPrecCor3.5 <- cor.fd(seq(0, 365, length=4), tempfd3,
                          seq(0, 356, length=6), precfd5))

# The following produces contour and perspective plots

daybasis65 <- create.fourier.basis(rangeval=c(0, 365), nbasis=65)
daytempfd <- with(CanadianWeather, data2fd(
       dailyAv[,,"Temperature.C"], day.5,
       daybasis65, argnames=list("Day", "Station", "Deg C")) )
dayprecfd <- with(CanadianWeather, data2fd(
       dailyAv[,,"log10precip"], day.5,
       daybasis65, argnames=list("Day", "Station", "log10(mm)")) )

str(tempPrecCor <- cor.fd(weeks, daytempfd, weeks, dayprecfd))
# dim(tempPrecCor)= c(53, 53)

op <- par(mfrow=c(1,2), pty="s")
contour(weeks, weeks, tempPrecCor, 
        xlab="Average Daily Temperature",
        ylab="Average Daily log10(precipitation)",
        main=paste("Correlation function across locations\n",
          "for Canadian Anual Temperature Cycle"),
        cex.main=0.8, axes=FALSE)
axisIntervals(1, atTick1=seq(0, 365, length=5), atTick2=NA, 
            atLabels=seq(1/8, 1, 1/4)*365,
            labels=paste("Q", 1:4) )
axisIntervals(2, atTick1=seq(0, 365, length=5), atTick2=NA, 
            atLabels=seq(1/8, 1, 1/4)*365,
            labels=paste("Q", 1:4) )
persp(weeks, weeks, tempPrecCor,
      xlab="Days", ylab="Days", zlab="Correlation")
mtext("Temperature-Precipitation Correlations", line=-4, outer=TRUE)
par(op)

# Correlations and cross correlations
# in a bivariate functional data object
gaitbasis5 <- create.fourier.basis(nbasis=5)
gaitfd5 <- data2fd(gait, basisobj=gaitbasis5)

gait.t3 <- (0:2)/2
(gaitCor3.5 <- cor.fd(gait.t3, gaitfd5))
# Check the answers with manual computations
gait3.5 <- eval.fd(gait.t3, gaitfd5)
all.equal(cor(t(gait3.5[,,1])), gaitCor3.5[,,,1])
# TRUE
all.equal(cor(t(gait3.5[,,2])), gaitCor3.5[,,,3])
# TRUE
all.equal(cor(t(gait3.5[,,2]), t(gait3.5[,,1])),
               gaitCor3.5[,,,2])
# TRUE

# NOTE:
dimnames(gaitCor3.5)[[4]]
# [1] Hip-Hip
# [2] Knee-Hip 
# [3] Knee-Knee
# If [2] were "Hip-Knee", then
# gaitCor3.5[,,,2] would match 
# cor(t(gait3.5[,,1]), t(gait3.5[,,2]))
# *** It does NOT.  Instead, it matches:  
# cor(t(gait3.5[,,2]), t(gait3.5[,,1]))

\end{ExampleCode}
\end{Examples}

\HeaderA{create.bspline.basis}{Create a B-spline Basis}{create.bspline.basis}
\keyword{smooth}{create.bspline.basis}
\begin{Description}\relax
Functional data objects are constructed by specifying a set of basis 
functions and a set of coefficients defining a linear combination of
these basis functions.  The B-spline basis is a system that is usually
used for non-periodic functions.  It has the advantages of very fast
computation and great flexibility.  On the other hand, if the data are
considered to be periodic, then the Fourier basis is usually
preferred.
\end{Description}
\begin{Usage}
\begin{verbatim}
create.bspline.basis(rangeval=NULL, nbasis=NULL, norder=4,
      breaks=NULL, dropind=NULL, quadvals=NULL, values=NULL,
      names="bspl") 
\end{verbatim}
\end{Usage}
\begin{Arguments}
\begin{ldescription}
\item[\code{rangeval}] a numeric vector of length 2 defining the interval over
which the functional data object can be evaluated;  default value
is if(is.null(breaks)) 0:1 else range(breaks).  

If length(rangeval) == 1 and rangeval <= 0, this is an error.
Otherwise, if length(rangeval) == 1, rangeval is replaced by c(0,
rangeval).  

If length(rangeval)>2 and neither \code{breaks} nor \code{nbasis}
are provided, this extra long \code{rangeval} argument is assigned
to \code{breaks}, and then rangeval = range(breaks).  

\item[\code{nbasis}] an integer variable specifying the number of basis functions.
Default value NULL.

This 'nbasis' argument is ignored if 'breaks' is supplied, in which
case nbasis = nbreaks + norder - 2, where nbreaks = length(breaks).

\item[\code{norder}] an integer specifying the order of b-splines, which is one higher
than their degree. The default of 4 gives cubic splines.

\item[\code{breaks}] a vector specifying the break points defining the b-spline.
Also called knots, these are a strictly increasing sequence
of junction points between piecewise polynomial segments.
They must satisfy \code{breaks[1] = rangeval[1]} and
\code{breaks[nbreaks] = rangeval[2]}, where \code{nbreaks} is the 
length of \code{breaks}.  There must be at least 2 values in
\code{breaks}.  There is a potential for inconsistency among
arguments \code{nbasis}, \code{norder}, and \code{breaks}.  It is
resolved as follows:

If \code{breaks} is supplied, \code{nbreaks = length(breaks)}, and
\code{nbasis = nbreaks + norder - 2}, no matter what value for
\code{nbasis} is supplied. 

If \code{breaks} is not supplied, but \code{nbasis} is,
\code{nbreaks = nbasis - norder + 2}, and if this turns out to be
less than 2, an error message results.

If neither \code{breaks} nor \code{nbasis} is supplied,
\code{nbreaks} is set to 21.

If \code{breaks} is not provided, it is set to seq(rangeval[1],
rangeval[2], length=nbreaks).  

\item[\code{dropind}] a vector of integers specifiying the basis functions to
be dropped, if any.  For example, if it is required that
a function be zero at the left boundary, this is achieved
by dropping the first basis function, the only one that
is nonzero at that point. Default value NULL.

\item[\code{quadvals}] a matrix with two columns and a number of rows equal to the number 
of argument values used to approximate an integral using Simpson's
rule.  The first column contains these argument values.
A minimum of 5 values are required for
each inter-knot interval, and that is often enough. These
are equally spaced between two adjacent knots.
The second column contains the weights used for Simpson's
rule.  These are proportional to 1, 4, 2, 4, ..., 2, 4, 1.

\item[\code{values}] a list containing the basis functions and their derivatives
evaluated at the quadrature points contained in the first
column of \code{ quadvals }.

\item[\code{names}] either a character vector of the same length as the number of basis 
functions or a single character string to which norder, "." and
1:nbasis are appended as paste(names, norder, ".", 1:nbreaks,
sep="").  For example, if norder = 4, this defaults to 'bspl4.1',
'bspl4.2', ... .   

\end{ldescription}
\end{Arguments}
\begin{Details}\relax
Spline functions are constructed by joining polynomials end-to-end at
argument values called \emph{break points} or \emph{knots}. First, the
interval is subdivided into a set of adjoining intervals
separated the knots.  Then a polynomial of order $m$ and
degree $m-1$ is defined for each interval.  In order to make the
resulting piece-wise polynomial smooth, two adjoining polynomials are
constrained to have their values and all their derivatives up to order
$m-2$ match at the point where they join.

Consider as an illustration the very common case where the order is 4 
for all polynomials, so that degree of each polynomials is 3.  That
is, the polynomials are \emph{cubic}.  Then at each break point or
knot, the values of adjacent polynomials must match, and so also for
their first and second derivatives.  Only their third derivatives will
differ at the point of junction.

The number of degrees of freedom of a cubic spline function of this
nature is calculated as follows.  First, for the first interval, there
are four degrees of freedom.  Then, for each additional interval, the
polynomial over that interval now has only one degree of freedom
because of the requirement for matching values and derivatives.  This
works out to the following equation:  The number of degrees of freedom
is equal to the number of interior knots (that is, not counting the
lower and upper limits) plus the order of the polynomials.

\emph{B-splines} are a set of special spline functions that can be
used to construct any such piece-wise polynomial by computing the
appropriate linear combination.  They derive their computational
convience from the fact that any B-spline basis function is nonzero
over at most m adjacent intervals.  The number of basis functions is
given by the rule above for the number of degrees of freedom.

The number of intervals controls the flexibility of the spline;  the
more knots, the more flexible the resulting spline will be. But the 
position of the knots also plays a role.  Where do we position the
knots?  There is room for judgment here, but two considerations must
be kept in mind:  (1) you usually want at least one argument value
between two adjacent knots, and (2)  there should be more knots where
the curve needs to have sharp curvatures such as a sharp peak or
valley or an abrupt change of level, but only a few knots are required
where the curve is changing very slowly.

This function automatically includes norder replicates of the end
points rangeval.  By contrast, the analogous splines' package
functions \LinkA{splineDesign}{splineDesign} and \LinkA{spline.des}{spline.des}
do NOT automatically replicate the end points.  To compare answers,
the end knots must be replicated manually when using
\LinkA{splineDesign}{splineDesign} or \LinkA{spline.des}{spline.des}.
\end{Details}
\begin{Value}
a basis object of the type \code{bspline}
\end{Value}
\begin{References}\relax
Ramsay, James O., and Silverman, Bernard W. (2005), \emph{Functional 
Data Analysis, 2nd ed.}, Springer, New York. 

Ramsay, James O., and Silverman, Bernard W. (2002), \emph{Applied
Functional Data Analysis}, Springer, New York.
\end{References}
\begin{SeeAlso}\relax
\code{\LinkA{basisfd}{basisfd}}, 
\code{\LinkA{create.constant.basis}{create.constant.basis}}, 
\code{\LinkA{create.exponential.basis}{create.exponential.basis}}, 
\code{\LinkA{create.fourier.basis}{create.fourier.basis}}, 
\code{\LinkA{create.monomial.basis}{create.monomial.basis}}, 
\code{\LinkA{create.polygonal.basis}{create.polygonal.basis}}, 
\code{\LinkA{create.polynomial.basis}{create.polynomial.basis}}, 
\code{\LinkA{create.power.basis}{create.power.basis}}
\code{\LinkA{splineDesign}{splineDesign}}
\code{\LinkA{spline.des}{spline.des}}
\end{SeeAlso}
\begin{Examples}
\begin{ExampleCode}
##
## The simplest basis currently available with this function:
##
bspl1.1 <- create.bspline.basis(norder=1, breaks=2:3)
plot(bspl1.1)
# 1 basis function, order 1 = degree 0 = step function:  
# constant 1 between 2 and 3.  

bspl1.2 <- create.bspline.basis(norder=1, breaks=c(0,.5, 1))
plot(bspl1.2)
# 2 bases, order 1 = degree 0 = step functions:  
# (1) constant 1 between 0 and 0.5 and 0 otherwise
# (2) constant 1 between 0.5 and 1 and 0 otherwise.

bspl2.3 <- create.bspline.basis(norder=2, breaks=c(0,.5, 1))
plot(bspl2.3)
# 3 bases:  order 2 = degree 1 = linear 
# (1) line from (0,1) down to (0.5, 0), 0 after
# (2) line from (0,0) up to (0.5, 1), then down to (1,0)
# (3) 0 to (0.5, 0) then up to (1,1).

bspl3.4 <- create.bspline.basis(norder=3, breaks=c(0,.5, 1))
plot(bspl3.4)
# 4 bases:  order 3 = degree 2 = parabolas.  
# (1) (x-.5)^2 from 0 to .5, 0 after
# (2) 2*(x-1)^2 from .5 to 1, and a parabola
#     from (0,0 to (.5, .5) to match
# (3 & 4) = complements to (2 & 1).  

# Default B-spline basis
bSpl4.23 <- create.bspline.basis()
# Cubic bspline (norder=4) with nbasis=23,
# so nbreaks = nbasis-norder+2 = 21, 
# 2 of which are rangeval, leaving 19 Interior knots.

bSpl4. <- create.bspline.basis(c(-1,1))
# Same as bSpl4.23 but over (-1,1) rather than (0,1).

# set up the b-spline basis for the lip data, using 23 basis functions,
#   order 4 (cubic), and equally spaced knots.
#  There will be 23 - 4 = 19 interior knots at 0.05, ..., 0.95
lipbasis <- create.bspline.basis(c(0,1), 23)
all.equal(bSpl4.23, lipbasis)
# TRUE 
# plot the basis functions
plot(lipbasis)

bSpl.growth <- create.bspline.basis(growth$age)
# cubic spline (order 4) 

bSpl.growth6 <- create.bspline.basis(growth$age,norder=6)
# quintic spline (order 6) 
\end{ExampleCode}
\end{Examples}

\HeaderA{create.constant.basis}{Create a Constant Basis}{create.constant.basis}
\keyword{smooth}{create.constant.basis}
\begin{Description}\relax
Create a constant basis object, defining a single basis function
whose value is everywhere 1.0.
\end{Description}
\begin{Usage}
\begin{verbatim}
create.constant.basis(rangeval=c(0, 1))
\end{verbatim}
\end{Usage}
\begin{Arguments}
\begin{ldescription}
\item[\code{rangeval}] a vector of length 2 containing the initial and final
values of argument t defining the interval over which the functional
data object can be evaluated.  However, this is seldom used 
since the value of the basis function does not depend on the range
or any argument values.

\end{ldescription}
\end{Arguments}
\begin{Value}
a basis object with type component \code{const}.
\end{Value}
\begin{SeeAlso}\relax
\code{\LinkA{basisfd}{basisfd}}, 
\code{\LinkA{create.bspline.basis}{create.bspline.basis}}, 
\code{\LinkA{create.exponential.basis}{create.exponential.basis}}, 
\code{\LinkA{create.fourier.basis}{create.fourier.basis}}, 
\code{\LinkA{create.monomial.basis}{create.monomial.basis}}, 
\code{\LinkA{create.polygonal.basis}{create.polygonal.basis}}, 
\code{\LinkA{create.polynomial.basis}{create.polynomial.basis}}, 
\code{\LinkA{create.power.basis}{create.power.basis}}
\end{SeeAlso}
\begin{Examples}
\begin{ExampleCode}

basisobj <- create.constant.basis(c(-1,1))

\end{ExampleCode}
\end{Examples}

\HeaderA{create.exponential.basis}{Create an Exponential Basis}{create.exponential.basis}
\keyword{smooth}{create.exponential.basis}
\begin{Description}\relax
Create an exponential basis object defining a set of exponential
functions with rate constants in argument ratevec.
\end{Description}
\begin{Usage}
\begin{verbatim}
create.exponential.basis(rangeval=c(0,1), nbasis=1,
                         ratevec=1, dropind=NULL,
                         quadvals=NULL, values=NULL)
\end{verbatim}
\end{Usage}
\begin{Arguments}
\begin{ldescription}
\item[\code{rangeval}] a vector of length 2 containing the initial and final values of the
interval over which the functional data object can be evaluated.

\item[\code{nbasis}] the number of basis functions.

\item[\code{ratevec}] a vector of length \code{nbasis} of rate constants
defining basis functions of the form \code{exp(rate*x)}.

\item[\code{dropind}] a vector of integers specifiying the basis functions to
be dropped, if any.  For example, if it is required that
a function be zero at the left boundary, this is achieved
by dropping the first basis function, the only one that
is nonzero at that point. Default value NULL.

\item[\code{quadvals}] a matrix with two columns and a number of rows equal to the number
of argument values used to approximate an integral using Simpson's
rule.  The first column contains these argument values.
A minimum of 5 values are required for
each inter-knot interval, and that is often enough. These
are equally spaced between two adjacent knots.
The second column contains the weights used for Simpson's
rule.  These are proportional to 1, 4, 2, 4, ..., 2, 4, 1.

\item[\code{values}] a list containing the basis functions and their derivatives
evaluated at the quadrature points contained in the first
column of \code{ quadvals }.

\end{ldescription}
\end{Arguments}
\begin{Details}\relax
Exponential functions are of the type $exp(bx)$ where $b$
is the rate constant.  If $b = 0$, the constant function is
defined.
\end{Details}
\begin{Value}
a basis object with the type \code{expon}.
\end{Value}
\begin{SeeAlso}\relax
\code{\LinkA{basisfd}{basisfd}}, 
\code{\LinkA{create.bspline.basis}{create.bspline.basis}}, 
\code{\LinkA{create.constant.basis}{create.constant.basis}}, 
\code{\LinkA{create.fourier.basis}{create.fourier.basis}}, 
\code{\LinkA{create.monomial.basis}{create.monomial.basis}}, 
\code{\LinkA{create.polygonal.basis}{create.polygonal.basis}}, 
\code{\LinkA{create.polynomial.basis}{create.polynomial.basis}}, 
\code{\LinkA{create.power.basis}{create.power.basis}}
\end{SeeAlso}
\begin{Examples}
\begin{ExampleCode}

#  Create an exponential basis over interval [0,5]
#  with basis functions 1, exp(-t) and exp(-5t)
basisobj <- create.exponential.basis(c(0,5),3,c(0,-1,-5))
#  plot the basis
plot(basisobj)

\end{ExampleCode}
\end{Examples}

\HeaderA{create.fourier.basis}{Create a Fourier Basis}{create.fourier.basis}
\keyword{smooth}{create.fourier.basis}
\begin{Description}\relax
Create an Fourier basis object defining a set of Fourier
functions with specified period.
\end{Description}
\begin{Usage}
\begin{verbatim}
create.fourier.basis(rangeval=c(0, 1), nbasis=3,
              period=width,  dropind=NULL, quadvals=NULL,
              values=NULL, longNames=TRUE)
\end{verbatim}
\end{Usage}
\begin{Arguments}
\begin{ldescription}
\item[\code{rangeval}] a vector of length 2 containing the initial and final
values of the interval over which the functional
data object can be evaluated.  Default value \code{c(0,1)}

\item[\code{nbasis}] the number of basis functions, rounded up to the nearest odd
integer.  The number of basis functions is always odd, even when an
even number is specified, so as to preserve the pairing of sine and
cosine functions.  Default value 3. 

\item[\code{period}] the width of any interval over which the Fourier functions repeat
themselves, or are periodic.  The default is the width of the
interval defined in rangeval.

\item[\code{dropind}] a vector of integers specifiying the basis functions to
be dropped, if any.  For example, if it is required that
a function be zero at the left boundary, this is achieved
by dropping the first basis function, the only one that
is nonzero at that point. Default value NULL.

\item[\code{quadvals}] a matrix with two columns and a number of rows equal to the number
of argument values used to approximate an integral using Simpson's
rule.  The first column contains these argument values.  A minimum
of 5 values are required for each inter-knot interval, and that is
often enough. These are equally spaced between two adjacent knots.
The second column contains the weights used for Simpson's rule.
These are proportional to 1, 4, 2, 4, ..., 2, 4, 1.  

\item[\code{values}] a list containing the basis functions and their derivatives
evaluated at the quadrature points contained in the first
column of \code{ quadvals }.

\item[\code{longNames}] if FALSE, the function value will include a component 'names', the
first of which will be 'const', followed by 'sin1', 'cos1', 'sin2',
... . 

if TRUE, the function value will include 'names' as when longNames =
FALSE, but with the 'period' rounded to 4 significant digits and
pasted onto 'sin1', etc.  For example, with period=12, the second
name would be 'sin1.12'.

if NA, no 'names' will be provided.      

\end{ldescription}
\end{Arguments}
\begin{Details}\relax
Functional data objects are constructed by specifying a set of basis
functions and a set of coefficients defining a linear combination of
these basis functions.  The Fourier basis is a system
that is usually used for periodic functions.  It has the advantages
of very fast computation and great flexibility.   If the data are
considered to be nonperiod, the Fourier basis is usually preferred.
The first Fourier basis function is the constant function.  The
remainder are sine and cosine pairs with integer multiples of the
base period. The number of basis functions generated is always odd.
\end{Details}
\begin{Value}
a basis object with the type \code{fourier}.
\end{Value}
\begin{SeeAlso}\relax
\code{\LinkA{basisfd}{basisfd}}, 
\code{\LinkA{create.bspline.basis}{create.bspline.basis}}, 
\code{\LinkA{create.constant.basis}{create.constant.basis}}, 
\code{\LinkA{create.exponential.basis}{create.exponential.basis}}, 
\code{\LinkA{create.monomial.basis}{create.monomial.basis}}, 
\code{\LinkA{create.polygonal.basis}{create.polygonal.basis}}, 
\code{\LinkA{create.polynomial.basis}{create.polynomial.basis}}, 
\code{\LinkA{create.power.basis}{create.power.basis}}
\end{SeeAlso}
\begin{Examples}
\begin{ExampleCode}
# Create a minimal Fourier basis for the monthly temperature data, 
#  using 3 basis functions with period 12 months.
monthbasis3 <- create.fourier.basis(c(0,12) )
#  plot the basis
plot(monthbasis3)

# set up the Fourier basis for the monthly temperature data,
#  using 9 basis functions with period 12 months.
monthbasis <- create.fourier.basis(c(0,12), 9, 12.0)

#  plot the basis
plot(monthbasis)

# Create a false Fourier basis using 1 basis function.  
falseFourierBasis <- create.fourier.basis(nbasis=1)
#  plot the basis:  constant 
plot(falseFourierBasis)

\end{ExampleCode}
\end{Examples}

\HeaderA{create.monomial.basis}{Create a Monomial Basis}{create.monomial.basis}
\keyword{smooth}{create.monomial.basis}
\begin{Description}\relax
Creates a set of basis functions consisting of powers
of the argument.
\end{Description}
\begin{Usage}
\begin{verbatim}
create.monomial.basis(rangeval=c(0, 1), nbasis=2,
                      exponents=NULL, dropind=NULL,
                      quadvals=NULL, values=NULL)
\end{verbatim}
\end{Usage}
\begin{Arguments}
\begin{ldescription}
\item[\code{rangeval}] a vector of length 2 containing the initial and final
values of the interval over which the functional
data object can be evaluated.

\item[\code{nbasis}] the number of basis functions. The default is 2,
which defines a basis for straight lines.

\item[\code{exponents}] the nonnegative integer powers to be used.  By default,
these are 0, 1, 2, ..., \code{ nbasis }.

\item[\code{dropind}] a vector of integers specifiying the basis functions to
be dropped, if any.  For example, if it is required that
a function be zero at the left boundary, this is achieved
by dropping the first basis function, the only one that
is nonzero at that point. Default value NULL.

\item[\code{quadvals}] a matrix with two columns and a number of rows equal to the number of
argument values used to approximate an integral using Simpson's rule.
The first column contains these argument values.
A minimum of 5 values are required for
each inter-knot interval, and that is often enough. These
are equally spaced between two adjacent knots.
The second column contains the weights used for Simpson's
rule.  These are proportional to 1, 4, 2, 4, ..., 2, 4, 1.

\item[\code{values}] a list containing the basis functions and their derivatives
evaluated at the quadrature points contained in the first
column of \code{ quadvals }.

\end{ldescription}
\end{Arguments}
\begin{Value}
a basis object with the type \code{monom}.
\end{Value}
\begin{SeeAlso}\relax
\code{\LinkA{basisfd}{basisfd}}, 
\code{\LinkA{create.bspline.basis}{create.bspline.basis}}, 
\code{\LinkA{create.constant.basis}{create.constant.basis}}, 
\code{\LinkA{create.fourier.basis}{create.fourier.basis}}, 
\code{\LinkA{create.exponential.basis}{create.exponential.basis}}, 
\code{\LinkA{create.polygonal.basis}{create.polygonal.basis}}, 
\code{\LinkA{create.polynomial.basis}{create.polynomial.basis}}, 
\code{\LinkA{create.power.basis}{create.power.basis}}
\end{SeeAlso}
\begin{Examples}
\begin{ExampleCode}
##
## simplest example: one constant 'basis function' 
##
m0 <- create.monomial.basis(nbasis=1, exponents=0)
plot(m0)

##
## Create a monomial basis over the interval [-1,1]
##  consisting of the first three powers of t
##
basisobj <- create.monomial.basis(c(-1,1), 3)
#  plot the basis
plot(basisobj)
\end{ExampleCode}
\end{Examples}

\HeaderA{create.polygonal.basis}{Create a Polygonal Basis}{create.polygonal.basis}
\keyword{smooth}{create.polygonal.basis}
\begin{Description}\relax
A basis is set up for constructing polygonal lines, consisting of 
straight line segments that join together.
\end{Description}
\begin{Usage}
\begin{verbatim}
create.polygonal.basis(argvals=NULL, dropind=NULL,
                       quadvals=NULL, values=NULL)
\end{verbatim}
\end{Usage}
\begin{Arguments}
\begin{ldescription}
\item[\code{argvals}] a strictly increasing vector of argument values at which line
segments join to form a polygonal line.

\item[\code{dropind}] a vector of integers specifiying the basis functions to
be dropped, if any.  For example, if it is required that
a function be zero at the left boundary, this is achieved
by dropping the first basis function, the only one that
is nonzero at that point. Default value NULL.

\item[\code{quadvals}] a matrix with two columns and a number of rows equal to the number
of argument values used to approximate an integral using Simpson's
rule.  The first column contains these argument values.  
A minimum of 5 values are required for
each inter-knot interval, and that is often enough. These
are equally spaced between two adjacent knots.
The second column contains the weights used for Simpson's
rule.  These are proportional to 1, 4, 2, 4, ..., 2, 4, 1.

\item[\code{values}] a list containing the basis functions and their derivatives
evaluated at the quadrature points contained in the first
column of \code{ quadvals }.

\end{ldescription}
\end{Arguments}
\begin{Details}\relax
The actual basis functions consist of triangles, each with its apex
over an argument value. Note that in effect the polygonal basis is
identical to a B-spline basis of order 2 and a knot or break value at
each argument value.  The range of the polygonal basis is set to the
interval defined by the smallest and largest argument values.
\end{Details}
\begin{Value}
a basis object with the type \code{polyg}.
\end{Value}
\begin{SeeAlso}\relax
\code{\LinkA{basisfd}{basisfd}}, 
\code{\LinkA{create.bspline.basis}{create.bspline.basis}}, 
\code{\LinkA{create.constant.basis}{create.constant.basis}}, 
\code{\LinkA{create.exponential.basis}{create.exponential.basis}}, 
\code{\LinkA{create.fourier.basis}{create.fourier.basis}}, 
\code{\LinkA{create.monomial.basis}{create.monomial.basis}}, 
\code{\LinkA{create.polynomial.basis}{create.polynomial.basis}}, 
\code{\LinkA{create.power.basis}{create.power.basis}}
\end{SeeAlso}
\begin{Examples}
\begin{ExampleCode}
#  Create a polygonal basis over the interval [0,1]
#  with break points at 0, 0.1, ..., 0.95, 1
(basisobj <- create.polygonal.basis(seq(0,1,0.1)))
#  plot the basis
plot(basisobj)
\end{ExampleCode}
\end{Examples}

\HeaderA{create.polynomial.basis}{Create a Polynomial Basis}{create.polynomial.basis}
\keyword{smooth}{create.polynomial.basis}
\begin{Description}\relax
Creates a set of basis functions consisting of powers
of the argument shifted by a constant.
\end{Description}
\begin{Usage}
\begin{verbatim}
create.polynomial.basis(rangeval=c(0, 1), nbasis=2,
                        ctr=midrange, dropind=NULL,
                        quadvals=NULL, values=NULL)
\end{verbatim}
\end{Usage}
\begin{Arguments}
\begin{ldescription}
\item[\code{rangeval}] a vector of length 2 defining the range.

\item[\code{nbasis}] the number of basis functions. The default is 2,
which defines a basis for straight lines.

\item[\code{ctr}] this value is used to shift the argument prior to taking
its power.

\item[\code{dropind}] a vector of integers specifiying the basis functions to
be dropped, if any.  For example, if it is required that
a function be zero at the left boundary, this is achieved
by dropping the first basis function, the only one that
is nonzero at that point. Default value NULL.

\item[\code{quadvals}] a matrix with two columns and a number of rows equal to the number
of argument values used to approximate an integral using Simpson's
rule.  The first column contains these argument values.
A minimum of 5 values are required for
each inter-knot interval, and that is often enough. These
are equally spaced between two adjacent knots.
The second column contains the weights used for Simpson's
rule.  These are proportional to 1, 4, 2, 4, ..., 2, 4, 1.

\item[\code{values}] a list containing the basis functions and their derivatives
evaluated at the quadrature points contained in the first
column of \code{ quadvals }.

\end{ldescription}
\end{Arguments}
\begin{Details}\relax
The only difference between a monomial and a polynomial basis
is the use of a shift value.  This helps to avoid rounding error
when the argument values are a long way from zero.
\end{Details}
\begin{Value}
a basis object with the type \code{polynom}.
\end{Value}
\begin{SeeAlso}\relax
\code{\LinkA{basisfd}{basisfd}}, 
\code{\LinkA{create.bspline.basis}{create.bspline.basis}}, 
\code{\LinkA{create.constant.basis}{create.constant.basis}}, 
\code{\LinkA{create.fourier.basis}{create.fourier.basis}}, 
\code{\LinkA{create.exponential.basis}{create.exponential.basis}}, 
\code{\LinkA{create.monomial.basis}{create.monomial.basis}}, 
\code{\LinkA{create.polygonal.basis}{create.polygonal.basis}}, 
\code{\LinkA{create.power.basis}{create.power.basis}}
\end{SeeAlso}
\begin{Examples}
\begin{ExampleCode}
#  Create a polynomial basis over the years in the 20th century
#  and center the basis functions on 1950.
basisobj <- create.polynomial.basis(c(1900, 2000), nbasis=3, ctr=1950)
#  plot the basis
# The following should work but doesn't;  2007.05.01
#plot(basisobj)
\end{ExampleCode}
\end{Examples}

\HeaderA{create.power.basis}{Create a Power Basis Object}{create.power.basis}
\keyword{smooth}{create.power.basis}
\begin{Description}\relax
The basis system is a set of powers of argument $x$.  That is, a basis
function would be \code{x\textasciicircum{}exponent}, where \code{exponent}
is a vector containing a set of powers or
exponents.  The power basis would normally only be used for positive
values of x, since the power of a negative number is only defined
for nonnegative integers, and the exponents here can be any real
numbers.
\end{Description}
\begin{Usage}
\begin{verbatim}
create.power.basis(rangeval=c(0, 1),
            nbasis=length(exponents), exponents=1,
            dropind=NULL, quadvals=NULL, values=NULL)
\end{verbatim}
\end{Usage}
\begin{Arguments}
\begin{ldescription}
\item[\code{rangeval}] a vector of length 2 with the first element being the lower limit of the
range of argument values, and the second the upper limit.  Of course the
lower limit must be less than the upper limit.

\item[\code{nbasis}] the number of exponential functions.

\item[\code{exponents}] a vector of length nbasis containing the powers.

\item[\code{dropind}] a vector of integers specifiying the basis functions to
be dropped, if any.  For example, if it is required that
a function be zero at the left boundary, this is achieved
by dropping the first basis function, the only one that
is nonzero at that point. Default value NULL.

\item[\code{quadvals}] a matrix with two columns and a number of rows equal to the number of
argument values used to approximate an integral using Simpson's rule.
The first column contains these argument values.
A minimum of 5 values are required for
each inter-knot interval, and that is often enough. These
are equally spaced between two adjacent knots.
The second column contains the weights used for Simpson's
rule.  These are proportional to 1, 4, 2, 4, ..., 2, 4, 1.

\item[\code{values}] a list containing the basis functions and their derivatives
evaluated at the quadrature points contained in the first
column of \code{ quadvals }.

\end{ldescription}
\end{Arguments}
\begin{Details}\relax
The power basis differs from the monomial and polynomial
bases in two ways.  First, the powers may be nonintegers.
Secondly, they may be negative.  Consequently, a power
basis is usually used with arguments that only take
positive values, although a zero value can be tolerated
if none of the powers are negative.
\end{Details}
\begin{Value}
a basis object of type \code{power}.
\end{Value}
\begin{SeeAlso}\relax
\code{\LinkA{basisfd}{basisfd}}, 
\code{\LinkA{create.bspline.basis}{create.bspline.basis}}, 
\code{\LinkA{create.constant.basis}{create.constant.basis}}, 
\code{\LinkA{create.exponential.basis}{create.exponential.basis}}, 
\code{\LinkA{create.fourier.basis}{create.fourier.basis}}, 
\code{\LinkA{create.monomial.basis}{create.monomial.basis}}, 
\code{\LinkA{create.polygonal.basis}{create.polygonal.basis}}, 
\code{\LinkA{create.polynomial.basis}{create.polynomial.basis}}
\end{SeeAlso}
\begin{Examples}
\begin{ExampleCode}

#  Create a power basis over the interval [1e-7,1]
#  with powers or exponents -1, -0.5, 0, 0.5 and 1
basisobj <- create.power.basis(c(1e-7,1), 5, seq(-1,1,0.5))
#  plot the basis
plot(basisobj)

\end{ExampleCode}
\end{Examples}

\HeaderA{CSTR}{Continuously Stirred Temperature Reactor}{CSTR}
\aliasA{CSTR2}{CSTR}{CSTR2}
\aliasA{CSTR2in}{CSTR}{CSTR2in}
\aliasA{CSTRfitLS}{CSTR}{CSTRfitLS}
\aliasA{CSTRfn}{CSTR}{CSTRfn}
\aliasA{CSTRres}{CSTR}{CSTRres}
\aliasA{CSTRres0}{CSTR}{CSTRres0}
\aliasA{CSTRsse}{CSTR}{CSTRsse}
\keyword{smooth}{CSTR}
\begin{Description}\relax
Functions for solving the Continuously Stirred Temperature Reactor
(CSTR) Ordinary Differential Equations (ODEs).  A solution for
observations where metrology error is assumed to be negligible can be 
obtained via lsoda(y, Time, CSTR2, parms);  CSTR2 calls CSTR2in.  When
metrology error can not be ignored, use CSTRfn (which calls
CSTRfitLS).  To estimate parameters in the CSTR differential equation
system (kref, EoverR, a, and / or b), pass either CSTRres or CSTRres0
to nls.  If nls fails to converge, first use optim or nlminb with
CSTRsse, then pass the estimates to nls.
\end{Description}
\begin{Usage}
\begin{verbatim}
CSTR2in(Time, condition =
   c('all.cool.step', 'all.hot.step', 'all.hot.ramp', 'all.cool.ramp',
     'Tc.hot.exponential', 'Tc.cool.exponential', 'Tc.hot.ramp',
     'Tc.cool.ramp', 'Tc.hot.step', 'Tc.cool.step'),
   tau=1)
CSTR2(Time, y, parms)  

CSTRfitLS(coef, datstruct, fitstruct, lambda, gradwrd=FALSE)
CSTRfn(parvec, datstruct, fitstruct, CSTRbasis, lambda, gradwrd=TRUE)
CSTRres(kref=NULL, EoverR=NULL, a=NULL, b=NULL,
        datstruct, fitstruct, CSTRbasis, lambda, gradwrd=FALSE)
CSTRres0(kref=NULL, EoverR=NULL, a=NULL, b=NULL, gradwrd=FALSE)
CSTRsse(par, datstruct, fitstruct, CSTRbasis, lambda)

\end{verbatim}
\end{Usage}
\begin{Arguments}
\begin{ldescription}
\item[\code{Time}] The time(s) for which computation(s) are desired

\item[\code{condition}] a character string with the name of one of ten preprogrammed input
scenarios.  

\item[\code{ tau }] time for exponential decay of exp(-1) under condition =
'Tc.hot.exponential' or 'Tc.cool.exponential';  ignored for other
values of 'condition'. 

\item[\code{y}] Either a vector of length 2 or a matrix with 2 columns giving the
observation(s) on Concentration and Temperature for which
computation(s) are desired  

\item[\code{parms}] a list of CSTR model parameters passed via the lsoda 'parms'
argument.  This list consists of the following 3 components:  

\Itemize{
\item[fitstruct] a list with 12 components describing the structure for fitting.
This is the same as the 'fitstruct' argument of 'CSTRfitLS' and
'CSTRfn' without the 'fit' component;  see below.


\item[condition] a character string identifying the inputs to the simulation.
Currently, any of the following are accepted:  'all.cool.step',
'all.hot.step', 'all.hot.ramp', 'all.cool.ramp',
'Tc.hot.exponential', 'Tc.cool.exponential', 'Tc.hot.ramp',
'Tc.cool.ramp', 'Tc.hot.step', or 'Tc.cool.step'.  

\item[Tlim] end time for the computations.

}

\item[\code{coef}] a matrix with one row for each basis function in fitstruct and
columns c("Conc", "Temp") or a vector form of such a matrix.  

\item[\code{datstruct}] a list describing the structure of the data.  CSTRfitLS uses the
following components:

\Itemize{
\item[basismat, Dbasismat] basis coefficent matrices with one row for each observation and
one column for each basis vector.  These are typically produced
by code something like the following:

basismat <- eval.basis(Time, CSTRbasis)

Dbasismat <- eval.basis(Time, CSTRbasis, 1)
    
\item[Cwt, Twt] scalar variances of 'fd' functional data objects for
Concentration and Temperature used to place the two series on
comparable scales.   


\item[y] a matrix with 2 columns for the observed 'Conc' and 'Temp'.

\item[quadbasismat, Dquadbasismat] basis coefficient matrices with one row for each quadrature
point and one column for each basis vector.  These are typically
produced by code something like the following:  

quadbasismat <- eval.basis(quadpts, CSTRbasis) 

Dquadbasismat <- eval.basis(quadpts, CSTRbasis, 1)


\item[Fc, F., CA0, T0, Tc] input series for CSTRfitLS and CSTRfn as the output list
produced by CSTR2in.


\item[quadpts] Quadrature points created by 'quadset' and stored in
CSTRbasis[["quadvals"]][, "quadpts"].   

\item[quadwts] Quadrature weights created by 'quadset' and stored in
CSTRbasis[["quadvals"]][, "quadpts"].  

}

\item[\code{fitstruct}] a list with 14 components:  

\Itemize{
\item[V] volume in cubic meters

\item[Cp] concentration in cal/(g.K) for computing betaTC and betaTT;  see
details below. 

\item[rho] density in grams per cubic meter 

\item[delH] cal/kmol 

\item[Cpc] concentration in cal/(g.K) used for computing alpha;  see
details below.
    
\item[Tref] reference temperature. 

\item[kref] reference value 

\item[EoverR] E/R in units of K/1e4 

\item[a] scale factor for Fco in alpha;  see details below.

\item[b] power of Fco in alpha;  see details below.

\item[Tcin] Tc input temperature vector.

\item[fit] logical vector of length 2 indicating whether Contentration or
Temperature or both are considered to be observed and used for
parameter estimation. 

\item[coef0] data.frame(Conc = Cfdsmth[["coef"]], Temp = Tfdsmth[["coef"]]),
where Cfdsmth and Tfdsmth are the objects returned by
smooth.basis when applied to the observations on Conc and Temp,
respectively.  

\item[estimate] logical vector of length 4 indicating which of kref, EoverR, a
and b are taken from 'parvec';  all others are taken from
'fitstruct'.  

}

\item[\code{lambda}] a 2-vector of rate parameters 'lambdaC' and 'lambdaT'.  

\item[\code{gradwrd}] a logical scalar TRUE if the gradient is to be returned as well as
the residuals matrix.      

\item[\code{parvec, par}] initial values for the parameters specified by fitstruct[[
"estimate"]] to be estimated. 

\item[\code{CSTRbasis}] Quadrature basis returned by 'quadset'.  

\item[\code{kref, EoverR, a, b}] the kref, EoverR, a, and b coefficients of the CSTR model as
individual arguments of CSTRres to support using 'nls' with the CSTR
model.  Those actually provided by name will be estimated;  the
others will be taken from '.fitstruct';  see details.   

\end{ldescription}
\end{Arguments}
\begin{Details}\relax
Ramsay et al. (2007) considers the following differential equation 
system for a continuously stirred temperature reactor (CSTR):

dC/dt = (-betaCC(T, F.in)*C + F.in*C.in)

dT/dt = (-betaTT(Fcvec, F.in)*T + betaTC(T, F.in)*C +
alpha(Fcvec)*T.co)

where

betaCC(T, F.in) = kref*exp(-1e4*EoverR*(1/T - 1/Tref)) + F.in

betaTT(Fcvec, F.in) = alpha(Fcvec) + F.in

betaTC(T, F.in) = (-delH/(rho*Cp))*betaCC(T, F.in)

\deqn{
alpha(Fcvec) = (a*Fcvec^(b+1) / (K1*(Fcvec + K2*Fcvec^b)))
}{}

K1 = V*rho*Cp

K2 = 1/(2*rhoc*Cpc)  

The four functions CSTR2in, CSTR2, CSTRfitLS, and CSTRfn compute
coefficients of basis vectors for two different solutions to this set
of differential equations.  Functions CSTR2in and CSTR2 work with
'lsoda' to provide a solution to this system of equations.  Functions 
CSTSRitLS and CSTRfn are used to estimate parameters to fit this
differential equation system to noisy data.  These solutions are
conditioned on specified values for kref, EoverR, a, and b.  The other
two functions CSTRres and CSTRres0 support estimation of these
parameters using 'nls'.    

CSTR2in translates a character string 'condition' into a data.frame
containing system inputs for which the reaction of the system is
desired.  CSTR2 calls CSTR2in and then computes the corresponding
predicted first derivatives of CSTR system outputs according to the
right hand side of the system equations.  CSTR2 can be called by
'lsoda' in the 'odesolve' package to actually solve the system of
equations.  To solve the CSTR equations for another set of inputs, the
easiest modification might be to change CSTR2in to return the desired
inputs.  Another alternative would be to add an argument
'input.data.frame' that would be used in place of CSTR2in when
present.

CSTRfitLS computes standardized residuals for systems outputs Conc, 
Temp or both as specified by fitstruct[["fit"]], a logical vector of 
length 2.  The standardization is sqrt(datstruct[["Cwt"]]) and / or
sqrt(datstruct[["Twt"]]) for Conc and Temp, respectively.  CSTRfitLS
also returns standardized deviations from the predicted first
derivatives for Conc and Temp.

CSTRfn uses a Gauss-Newton optimization to estimates the coefficients
of CSTRbasis to minimize the weighted sum of squares of residuals
returned by CSTRfitLS.  

CSTRres and CSTRres0 provide alternative interfaces between 'nls' and
'CSTRfn'.  Both get the parameters to be estimated via their official
function arguments, kref, EoverR, a, and / or b.  The subset of these
paramters to estimate must be specified both directly in the function
call to 'nls' and indirectly via fitstruct[["estimate"]].  CSTRres
gets the other CSTRfn arguments (datstruct, fitstruct, CSTRbasis, and
lambda) via the 'data' argument of 'nls'.  (The version of 'nls' in
the 'stats' package in R 2.5.0 required 'data' to be a data.frame, not
merely a list.  The version of 'nls' in 'fda' does not require this.)
By contrast, CSTRres0 uses 'get' to obtain these other arguments as
.datstruct, .fitstruct, .CSTRbasis and .lambda.  Thus, before calling
nls(~CSTRres0(...), ...), the values of these arguments must be
assigned to .datstruct, .fitstruct, .CSTRbasis and .lambda.  If 'nls'
is called from the global environment, it will look for these objects
in the global environment.

CSTRres0 has one feature absent from CSTRres:  If a variable
.CSTRres0.trace is available, it is assumed to be a matrix with columns
kref, EoverR, a, b, SSE, plus all residuals.  These numbers are
rbinded as an additional row of this matrix.  This is provided to help
diagnose a problem were 'nls' was terminating with "step factor
... reduced below 'minFactor'", facilitating the comparison between R
and Matlab for the precise sets of parameter values tested by 'nls'.    

CSTRsse computes sum of squares of residuals for use with optim or
nlminb.
\end{Details}
\begin{Value}
\begin{ldescription}
\item[\code{CSTR2in}] a matrix with number of rows = length(Time) and columns for F., CA0,
T0, Tcin, and Fc.  This gives the inputs to the CSTR simulation for
the chosen 'condition'. 

\item[\code{CSTR2}] a list with one component being a matrix with number of rows =
length(tobs) and 2 columns giving the first derivatives of Conc and
Temp according to the right hand side of the differential equation.
CSTR2 calls CSTR2in to get its inputs. 

\item[\code{CSTRfitLS}] a list with one or two components as follows:  

\Itemize{
\item[res] a list with two components

Sres = a matrix giving the residuals between observed and
predicted datstruct[["y"]] divided by sqrt(datstruct[[c("Cwt",
"Twt")]]) so the result is dimensionless.  dim(Sres) =  
dim(datstruct[["y"]]).  Thus, if datstruct[["y"]] has only one
column, 'Sres' has only one column. 

Lres = a matrix with two columns giving the difference between 
left and right hand sides of the CSTR differential equation at
all the quadrature points.  dim(Lres) = c(nquad, 2). 
   
\item[Dres] If gradwrd=TRUE, a list with two components:

DSres = a matrix with one row for each element of res[["Sres"]]
and two columns for each basis function.

DLres = a matrix with two rows for each quadrature point and two
columns for each basis function.

If gradwrd=FALSE, this component is not present.  

}

\item[\code{CSTRfn}] a list with five components:

\Itemize{
\item[res] the 'res' component of the final 'CSTRfitLS' object reformatted
with its component Sres first followed by Lres, using
with(CSTRfitLS(...)[["res"]], c(Sres, Lres)). 

\item[Dres] one of two very different gradient matrices depending on the
value of 'gradwrd'.

If gradwrd = TRUE, Dres is a matrix with one row for each
observation value to match and one column for each parameter
taken from 'parvec' per fitstruct[["estimate"]].  Also, if
fitstruct[["fit"]] = c(1,1), CSTRfn tries to  match both
Concentration and Temperature, and rows corresponding to
Concentration come first following by rows corresponding to
Temperature. 

If gradwrd = FALSE, this is the 'Dres' component of the final
'CSTRfitLS' object reformatted as follows:  

Dres <- with(CSTRfitLS(...)[["Dres"]], rbind(DSres, DLres))

\item[fitstruct] a list components matching the 'fitstruct' input, with
coefficients estimated replaced by their initial values from
parvec and with coef0 replace by its final estimate. 

\item[df] estimated degrees of freedom as the trace of the appropriate
matrix. 

\item[gcv] the Generalized cross validation estimate of the mean square
error, as discussed in Ramsay and Silverman (2006, sec. 5.4).   

} 

\item[\code{CSTRres, CSTRres0}] the 'res' component of CSTRfd(...) as a column vector.  This allows
us to use 'nls' with the CSTR model.  This can be especially useful
as 'nls' has several helper functions to facilitate evaluating
goodness of fit and and uncertainty in parameter estimates.  

\item[\code{CSTRsse}] sum(res*res) from CSTRfd(...).  This allows us to use 'optim' or
'nlminb' with the CSTR model.  This can also be used to obtain
starting values for 'nls' in cases where 'nls' fails to converge
from the initiall provided starting values.  Apart from 'par', the
other arguments 'datstruct', 'fitstruct', 'CSTRbasis', and 'lambda',
must be passed via '...' in 'optim' or 'nlminb'.  

\end{ldescription}
\end{Value}
\begin{References}\relax
Ramsay, J. O., Hooker, G., Cao, J. and Campbell, D. (2007) Parameter
estimation for differential equations: A generalized smoothing
approach (with discussion). Journal of the Royal Statistical Society,
Series B. To appear.

Ramsay, J. O., and Silverman, B. W. (2006) Functional Data Analysis,
2nd ed. (Springer)

Ramsay, James O., and Silverman, Bernard W. (2002), \emph{Applied
Functional Data Analysis}, Springer, New York
\end{References}
\begin{SeeAlso}\relax
\code{\LinkA{lsoda}{lsoda}}
\code{\LinkA{nls}{nls}}
\end{SeeAlso}
\begin{Examples}
\begin{ExampleCode}
###
###
### 1.  lsoda(y, times, func=CSTR2, parms=...)
###
###
#  The system of two nonlinear equations has five forcing or  
#  input functions.
#  These equations are taken from
#  Marlin, T. E. (2000) Process Control, 2nd Edition, McGraw Hill,
#  pages 899-902.
##
##  Set up the problem 
##
fitstruct <- list(V    = 1.0,#  volume in cubic meters 
                  Cp   = 1.0,#  concentration in cal/(g.K)
                  rho  = 1.0,#  density in grams per cubic meter 
                  delH = -130.0,# cal/kmol
                  Cpc  = 1.0,#  concentration in cal/(g.K)
                  rhoc = 1.0,#  cal/kmol
                  Tref = 350)#  reference temperature
#  store true values of known parameters 
EoverRtru = 0.83301#   E/R in units K/1e4
kreftru   = 0.4610 #   reference value
atru      = 1.678#     a in units (cal/min)/K/1e6
btru      = 0.5#       dimensionless exponent

#

fitstruct[["kref"]]   = kreftru#      
fitstruct[["EoverR"]] = EoverRtru#  kref = 0.4610
fitstruct[["a"]]      = atru#       a in units (cal/min)/K/1e6
fitstruct[["b"]]      = btru#       dimensionless exponent
 
Tlim  = 64#    reaction observed over interval [0, Tlim]
delta = 1/12#  observe every five seconds
tspan = seq(0, Tlim, delta)#

coolStepInput <- CSTR2in(tspan, 'all.cool.step')

#  set constants for ODE solver

#  cool condition solution
#  initial conditions

Cinit.cool = 1.5965#  initial concentration in kmol per cubic meter
Tinit.cool = 341.3754# initial temperature in deg K
yinit = c(Conc = Cinit.cool, Temp=Tinit.cool)

#  load cool input into fitstruct

fitstruct[["Tcin"]] = coolStepInput[, "Tcin"];

#  solve  differential equation with true parameter values

if (require(odesolve)) {
coolStepSoln <- lsoda(y=yinit, times=tspan, func=CSTR2,
  parms=list(fitstruct=fitstruct, condition='all.cool.step', Tlim=Tlim) )
}
###
###
### 2.  CSTRfn 
###
###

# See the script in '~R\library\fda\scripts\CSTR\CSTR_demo.R'
#  for more examples.  

\end{ExampleCode}
\end{Examples}

\HeaderA{cycleplot.fd}{Plot Cycles for a Periodic Bivariate Functional Data Object}{cycleplot.fd}
\keyword{smooth}{cycleplot.fd}
\begin{Description}\relax
A plotting function for data such as the knee-hip angles in
the gait data or temperature-precipitation curves for the
weather data.
\end{Description}
\begin{Usage}
\begin{verbatim}
cycleplot.fd(fdobj, matplt=TRUE, nx=201, ...)
\end{verbatim}
\end{Usage}
\begin{Arguments}
\begin{ldescription}
\item[\code{fdobj}] a bivariate functional data object to be plotted.

\item[\code{matplt}] if TRUE, all cycles are plotted simultaneously; otherwise
each cycle in turn is plotted.

\item[\code{nx}] the number of argument values in a fine mesh to be
plotted.  Increase the default number of 201 if
the curves have a lot of detail in them.

\item[\code{... }] additional plotting parameters such as axis labels and
etc. that are used in all plot functions.

\end{ldescription}
\end{Arguments}
\begin{Value}
None
\end{Value}
\begin{Section}{Side Effects}
A plot of the cycles
\end{Section}
\begin{SeeAlso}\relax
\code{\LinkA{plot.fd}{plot.fd}}, 
\code{\LinkA{plotfit.fd}{plotfit.fd}}, 
demo(gait)
\end{SeeAlso}

\HeaderA{data2fd.old}{Depricated:  use 'Data2fd'}{data2fd.old}
\aliasA{data2fd}{data2fd.old}{data2fd}
\keyword{smooth}{data2fd.old}
\begin{Description}\relax
This function converts an array \code{y} of function values
plus an array \code{argvals} of argument values into a
functional data object.  This a function that tries to
do as much for the user as possible.  A basis function
expansion is used to represent the curve, but no roughness
penalty is used.  The data are fit using the least squares
fitting criterion.  NOTE:  Interpolation with data2fd(...) can be 
shockingly bad, as illustrated in one of the examples.
\end{Description}
\begin{Usage}
\begin{verbatim}
data2fd(y, argvals=seq(0, 1, len = n), basisobj,
        fdnames=defaultnames,
        argnames=c("time", "reps", "values"))
\end{verbatim}
\end{Usage}
\begin{Arguments}
\begin{ldescription}
\item[\code{y}] an array containing sampled values of curves.

If \code{y} is a vector, only one replicate and variable are
assumed.

If \code{y} is a matrix, rows must correspond to argument values and
columns to replications or cases, and it will be assumed that there
is only one variable per observation.

If \code{y} is a three-dimensional array, the first dimension (rows)
corresponds to argument values, the second (columns) to
replications, and the third layers) to variables within
replications.  Missing values are permitted, and the number of
values may vary from one replication to another.  If this is the
case, the number of rows must equal the maximum number of argument
values, and columns of \code{y} having fewer values must be padded
out with NA's. 

\item[\code{argvals}] a set of argument values.

If this is a vector, the same set of argument values is used for all
columns of \code{y}.  If \code{argvals} is a matrix, the columns
correspond to the columns of \code{y}, and contain the argument
values for that replicate or case.    

\item[\code{basisobj}] either:  A \code{basisfd} object created by function
create.basis.fd(), 
or the value NULL, in which case a \code{basisfd} object is set up
by the function, using the values of the next three arguments.

\item[\code{fdnames}] A list of length 3, each member being a string vector containing
labels for the levels of the corresponding dimension of the discrete
data.  The first dimension is for argument values, and is given the
default name "time", the second is for replications, and is given
the default name "reps", and the third is for functions, and is
given the default name "values".  These default names are
assigned in function \{tt data2fd\}, which also assigns default
string vectors by using the dimnames attribute of the discrete data
array. 

\item[\code{argnames}] a character vector of length 3 containing:

\Itemize{
\item the name of the argument, e.g. "time" or "age"
\item a description of the cases, e.g. "weather stations"
\item the name of the observed function value, e.g. "temperature" 
}

These strings are used as names for the members of list \code{fdnames}.

\end{ldescription}
\end{Arguments}
\begin{Details}\relax
This function tends to be used in rather simple applications where
there is no need to control the roughness of the resulting curve
with any great finesse.  The roughness is essentially controlled
by how many basis functions are used.  In more sophisticated
applications, it would be better to use the function \code{\LinkA{smooth.basis}{smooth.basis}}
\end{Details}
\begin{Value}
an object of the \code{fd} class containing:

\begin{ldescription}
\item[\code{coefs}] the coefficient array

\item[\code{basis}] a basis object and

\item[\code{fdnames}] a list containing names for the arguments, function values
and variables

\end{ldescription}
\end{Value}
\begin{SeeAlso}\relax
\code{\bsl{}line\{Data2fd\}}
\code{\LinkA{smooth.basis}{smooth.basis}}, 
\code{\LinkA{smooth.basisPar}{smooth.basisPar}}, 
\code{\LinkA{project.basis}{project.basis}}, 
\code{\LinkA{smooth.fd}{smooth.fd}}, 
\code{\LinkA{smooth.monotone}{smooth.monotone}}, 
\code{\LinkA{smooth.pos}{smooth.pos}}
\code{\LinkA{day.5}{day.5}}
\end{SeeAlso}
\begin{Examples}
\begin{ExampleCode}
# Simplest possible example
b1.2 <- create.bspline.basis(norder=1, breaks=c(0, .5, 1))
# 2 bases, order 1 = degree 0 = step functions

str(fd1.2 <- data2fd(0:1, basisobj=b1.2))
plot(fd1.2)
# A step function:  0 to time=0.5, then 1 after 

b2.3 <- create.bspline.basis(norder=2, breaks=c(0, .5, 1))
# 3 bases, order 2 = degree 1 =
# continuous, bounded, locally linear

str(fd2.3 <- data2fd(0:1, basisobj=b2.3))
round(fd2.3$coefs, 4)
# 0, -.25, 1 
plot(fd2.3)
# Officially acceptable but crazy:
# Initial negative slope from (0,0) to (0.5, -0.25),
# then positive slope to (1,1).  

b3.4 <- create.bspline.basis(norder=3, breaks=c(0, .5, 1))
# 4 bases, order 3 = degree 2 =
# continuous, bounded, locally quadratic 

str(fd3.4 <- data2fd(0:1, basisobj=b3.4))
round(fd3.4$coefs, 4)
# 0, .25, -.5, 1 
plot(fd3.4)
# Officially acceptable but crazy:
# Initial positive then swings negative
# between 0.4 and ~0.75 before becoming positive again
# with a steep slope running to (1,1).  


#  Simple example 
gaitbasis3 <- create.fourier.basis(nbasis=3)
str(gaitbasis3) # note:  'names' for 3 bases
gaitfd3 <- data2fd(gait, basisobj=gaitbasis3)
str(gaitfd3)
# Note: dimanes for 'coefs' + basis[['names']]
# + 'fdnames'

#    set up the fourier basis
daybasis <- create.fourier.basis(c(0, 365), nbasis=65)
#  Make temperature fd object
#  Temperature data are in 12 by 365 matrix tempav
#    See analyses of weather data.

#  Convert the data to a functional data object
tempfd <- data2fd(CanadianWeather$dailyAv[,,"Temperature.C"],
                  day.5, daybasis)
#  plot the temperature curves
plot(tempfd)

# Terrifying interpolation
hgtbasis <- with(growth, create.bspline.basis(range(age), 
                                              breaks=age, norder=6))
girl.data2fd <- with(growth, data2fd(hgtf, age, hgtbasis))
age2 <- with(growth, sort(c(age, (age[-1]+age[-length(age)])/2)))
girlPred <- eval.fd(age2, girl.data2fd)
range(growth$hgtf)
range(growth$hgtf-girlPred[seq(1, by=2, length=31),])
# 5.5e-6 0.028 <
# The predictions are consistently too small
# but by less than 0.05 percent 

matplot(age2, girlPred, type="l")
with(growth, matpoints(age, hgtf))
# girl.data2fd fits the data fine but goes berzerk
# between points 

\end{ExampleCode}
\end{Examples}

\HeaderA{Data2fd}{Create a functional data object from data}{Data2fd}
\keyword{smooth}{Data2fd}
\begin{Description}\relax
This function converts an array \code{y} of function values plus an
array \code{argvals} of argument values into a functional data object.
This function tries to do as much for the user as possible.  NOTE:
Interpolation with data2fd(...) can be shockingly bad, as illustrated
in one of the examples.
\end{Description}
\begin{Usage}
\begin{verbatim}
Data2fd(argvals=NULL, y=NULL, basisobj=NULL, nderiv=NULL,
        lambda=0, fdnames=NULL)
\end{verbatim}
\end{Usage}
\begin{Arguments}
\begin{ldescription}
\item[\code{argvals}] a set of argument values.  If this is a vector, the same set of
argument values is used for all columns of \code{y}.  If
\code{argvals} is a matrix, the columns correspond to the columns of
\code{y}, and contain the argument values for that replicate or
case.

Dimensions for \code{argvals} must match the first dimensions of
\code{y}, though \code{y} can have more dimensions.  For example, if
dim(y) = c(9, 5, 2), \code{argvals} can be a vector of length 9 or a
matrix of dimenions c(9, 5) or an array of dimensions c(9, 5, 2).  

\item[\code{y}] an array containing sampled values of curves.

If \code{y} is a vector, only one replicate and variable are
assumed.  If \code{y} is a matrix, rows must correspond to argument 
values and columns to replications or cases, and it will be assumed
that there is only one variable per observation.  If \code{y} is a
three-dimensional array, the first dimension (rows) corresponds to
argument values, the second (columns) to replications, and the third
(layers) to variables within replications.  Missing values are
permitted, and the number of values may vary from one replication to
another.  If this is the case, the number of rows must equal the
maximum number of argument values, and columns of \code{y} having
fewer values must be padded out with NA's.   

\item[\code{basisobj}] One of the following:

\Itemize{
\item[basisfd] a functional basis object (class \code{basisfd}). 

\item[fd] a functional data object (class \code{fd}), from which its
\code{basis} component is extracted.  

\item[fdPar] a functional parameter object (class \code{fdPar}), from which
its \code{basis} component is extracted.  

\item[integer] an integer giving the order of a B-spline basis,
create.bspline.basis(argvals, norder=basisobj) 

\item[numeric vector] specifying the knots for a B-spline basis,
create.bspline.basis(basisobj) 
         
\item[NULL] Defaults to create.bspline.basis(argvals).

}

\item[\code{nderiv}] Smoothing typically specified as an integer order for the derivative
whose square is integrated and weighted by \code{lambda} to smooth.
By default, if basisobj[['type']] == 'bspline', the smoothing
operator is int2Lfd(max(0, norder-2)).

A general linear differential operator can also be supplied.  

\item[\code{lambda}] weight on the smoothing operator specified by \code{nderiv}.  

\item[\code{fdnames}] Either a charater vector of length 3 or a named list of length 3.
In either case, the three elements correspond to the following:  

\Itemize{
\item[argname] name of the argument, e.g. "time" or "age".  

\item[repname] a description of the cases, e.g. "reps" or "weather stations"

\item[value] the name of the observed function value, e.g. "temperature"

}

If fdnames is a list, the components provide labels for the levels
of the corresponding dimension of \code{y}.  

\end{ldescription}
\end{Arguments}
\begin{Details}\relax
This function tends to be used in rather simple applications where
there is no need to control the roughness of the resulting curve with
any great finesse.  The roughness is essentially controlled by how
many basis functions are used.  In more sophisticated applications, it
would be better to use the function \code{\LinkA{smooth.basisPar}{smooth.basisPar}}.
\end{Details}
\begin{Value}
an object of the \code{fd} class containing:

\begin{ldescription}
\item[\code{coefs}] the coefficient array

\item[\code{basis}] a basis object 

\item[\code{fdnames}] a list containing names for the arguments, function values
and variables

\end{ldescription}
\end{Value}
\begin{References}\relax
Ramsay, James O., and Silverman, Bernard W. (2005), \emph{Functional 
Data Analysis, 2nd ed.}, Springer, New York. 

Ramsay, James O., and Silverman, Bernard W. (2002), \emph{Applied
Functional Data Analysis}, Springer, New York.
\end{References}
\begin{SeeAlso}\relax
\code{\LinkA{smooth.basisPar}{smooth.basisPar}}, 
\code{\LinkA{smooth.basis}{smooth.basis}}, 
\code{\LinkA{project.basis}{project.basis}}, 
\code{\LinkA{smooth.fd}{smooth.fd}}, 
\code{\LinkA{smooth.monotone}{smooth.monotone}}, 
\code{\LinkA{smooth.pos}{smooth.pos}}
\code{\LinkA{day.5}{day.5}}
\end{SeeAlso}
\begin{Examples}
\begin{ExampleCode}
##
## Simplest possible example:  step function 
##
b1.1 <- create.bspline.basis(nbasis=1, norder=1)
# 1 basis, order 1 = degree 0 = step function

y12 <- 1:2
fd1.1 <- Data2fd(y12, basisobj=b1.1)
plot(fd1.1)
# fd1.1 = mean(y12) = 1.5 

fd1.1.5 <- Data2fd(y12, basisobj=b1.1, lambda=0.5)
eval.fd(seq(0, 1, .2), fd1.1.5)
# fd1.1.5 = sum(y12)/(n+lambda*integral(over arg=0 to 1 of 1))
#         = 3 / (2+0.5) = 1.2

##
## 3 step functions
##
b1.2 <- create.bspline.basis(nbasis=2, norder=1)
# 2 bases, order 1 = degree 0 = step functions
fd1.2 <- Data2fd(1:2, basisobj=b1.2)

op <- par(mfrow=c(2,1))
plot(b1.2, main='bases') 
plot(fd1.2, main='fit')
par(op) 
# A step function:  1 to 0.5, then 2 

##
## Simple oversmoothing
##
b1.3 <- create.bspline.basis(nbasis=3, norder=1)
fd1.3.5 <- Data2fd(y12, basisobj=b1.3, lambda=0.5)
plot(0:1, c(0, 2), type='n')
points(0:1, y12)
lines(fd1.3.5)
# Fit = penalized least squares with penalty = 
#          = lambda * integral(0:1 of basis^2),
#            which shrinks the points towards 0.
# X1.3 = matrix(c(1,0, 0,0, 0,1), 2)
# XtX = crossprod(X1.3) = diag(c(1, 0, 1))
# penmat = diag(3)/3
#        = 3x3 matrix of integral(over arg=0:1 of basis[i]*basis[j])
# Xt.y = crossprod(X1.3, y12) = c(1, 0, 2)
# XtX + lambda*penmat = diag(c(7, 1, 7)/6 
# so coef(fd1.3.5) = solve(XtX + lambda*penmat, Xt.y)
#                  = c(6/7, 0, 12/7)

##
## linear spline fit 
##
b2.3 <- create.bspline.basis(norder=2, breaks=c(0, .5, 1))
# 3 bases, order 2 = degree 1 =
# continuous, bounded, locally linear

fd2.3 <- Data2fd(0:1, basisobj=b2.3)
round(fd2.3$coefs, 4)
# (0, 0, 1), 
# though (0, a, 1) is also a solution for any 'a' 
op <- par(mfrow=c(2,1))
plot(b2.3, main='bases') 
plot(fd2.3, main='fit')
par(op)

# smoothing?  
fd2.3. <- Data2fd(0:1, basisobj=b2.3, lambda=1)

all.equal(as.vector(round(fd2.3.$coefs, 4)),
          c(0.0159, -0.2222, 0.8730) )

# The default smoothing with spline of order 2, degree 1
# has nderiv = max(0, norder-2) = 0.
# Direct computations confirm that the optimal B-spline
# weights in this case are the numbers given above.  

op <- par(mfrow=c(2,1))
plot(b2.3, main='bases') 
plot(fd2.3., main='fit')
par(op)

##
## quadratic spline fit
##
b3.4 <- create.bspline.basis(norder=3, breaks=c(0, .5, 1))
# 4 bases, order 3 = degree 2 =
# continuous, bounded, locally quadratic 

fd3.4 <- Data2fd(0:1, basisobj=b3.4)
round(fd3.4$coefs, 4)
# (0, 0, 0, 1),
# but (0, a, b, 1) is also a solution for any 'a' and 'b' 
op <- par(mfrow=c(2,1))
plot(b3.4) 
plot(fd3.4)
par(op)

#  try smoothing?  
fd3.4. <- Data2fd(0:1, basisobj=b3.4, lambda=1)
round(fd3.4.$coef, 4)

op <- par(mfrow=c(2,1))
plot(b3.4) 
plot(fd3.4.)
par(op)

##
##  A simple Fourier example 
##
gaitbasis3 <- create.fourier.basis(nbasis=3)
# note:  'names' for 3 bases
gaitfd3 <- Data2fd(gait, basisobj=gaitbasis3)
# Note: dimanes for 'coefs' + basis[['names']]
# + 'fdnames'

#    set up the fourier basis
daybasis <- create.fourier.basis(c(0, 365), nbasis=65)
#  Make temperature fd object
#  Temperature data are in 12 by 365 matrix tempav
#    See analyses of weather data.

#  Convert the data to a functional data object
tempfd <- Data2fd(CanadianWeather$dailyAv[,,"Temperature.C"],
                  day.5, daybasis)
#  plot the temperature curves
plot(tempfd)

##
## Terrifying interpolation
##
hgtbasis <- with(growth, create.bspline.basis(range(age), 
                                              breaks=age, norder=6))
girl.data2fd <- with(growth, Data2fd(hgtf, age, hgtbasis))
age2 <- with(growth, sort(c(age, (age[-1]+age[-length(age)])/2)))
girlPred <- eval.fd(age2, girl.data2fd)
range(growth$hgtf)
range(growth$hgtf-girlPred[seq(1, by=2, length=31),])
# 5.5e-6 0.028 <
# The predictions are consistently too small
# but by less than 0.05 percent 

matplot(age2, girlPred, type="l")
with(growth, matpoints(age, hgtf))
# girl.data2fd fits the data fine but goes berzerk
# between points

# Smooth 
girl.data2fd1 <- with(growth, Data2fd(age, hgtf, hgtbasis, lambda=1))
girlPred1 <- eval.fd(age2, girl.data2fd1)

matplot(age2, girlPred1, type="l")
with(growth, matpoints(age, hgtf))

# problems splikes disappear 

\end{ExampleCode}
\end{Examples}

\HeaderA{dateAccessories}{Numeric and character vectors to facilitate working with dates}{dateAccessories}
\aliasA{day.5}{dateAccessories}{day.5}
\aliasA{dayOfYear}{dateAccessories}{dayOfYear}
\aliasA{daysPerMonth}{dateAccessories}{daysPerMonth}
\aliasA{monthAccessories}{dateAccessories}{monthAccessories}
\aliasA{monthBegin.5}{dateAccessories}{monthBegin.5}
\aliasA{monthEnd}{dateAccessories}{monthEnd}
\methaliasA{monthEnd.5}{dateAccessories}{monthEnd.5}
\aliasA{monthLetters}{dateAccessories}{monthLetters}
\aliasA{monthMid}{dateAccessories}{monthMid}
\aliasA{weeks}{dateAccessories}{weeks}
\keyword{datasets}{dateAccessories}
\begin{Description}\relax
Numeric and character vectors to simplify functional data computations
and plotting involving dates.
\end{Description}
\begin{Format}\relax
\item[dayOfYear] a numeric vector = 1:365 

\item[day.5 ] a numeric vector = dayOfYear-0.5 = 0.5, 1.5, ..., 364.5 

\item[daysPerMonth] a numeric vector of the days in each month (ignoring leap years)
with names = month.abb

\item[monthEnd] a numeric vector of cumsum(daysPerMonth) with names = month.abb

\item[monthEnd.5] a numeric vector of the middle of the last day of each month with
names = month.abb = c(Jan=30.5, Feb=58.5, ..., Dec=364.5)

\item[monthBegin.5] a numeric vector of the middle of the first day of each month with
names - month.abb = c(Jan=0.5, Feb=31.5, ..., Dec=334.5) 

\item[monthMid] a numeric vector of the middle of the month = (monthBegin.5 +
monthEnd.5)/2  

\item[monthLetters] A character vector of c("j", "F", "m", "A", "M", "J", "J", "A",
"S", "O", "N", "D"), with 'month.abb' as the names.  

\item[weeks] a numeric vector of length 53 marking 52 periods of approximately 7
days each throughout the year = c(0, 365/52, ..., 365)
\end{Format}
\begin{Details}\relax
Miscellaneous vectors often used in 'fda' scripts.
\end{Details}
\begin{Source}\relax
Ramsay, James O., and Silverman, Bernard W. (2006), \emph{Functional
Data Analysis, 2nd ed.}, Springer, New York, pp. 5, 47-53.

Ramsay, James O., and Silverman, Bernard W. (2002), \emph{Applied
Functional Data Analysis}, Springer, New York
\end{Source}
\begin{SeeAlso}\relax
\code{\LinkA{axisIntervals}{axisIntervals}} 
\code{\LinkA{month.abb}{month.abb}}
\end{SeeAlso}
\begin{Examples}
\begin{ExampleCode}
daybasis65 <- create.fourier.basis(c(0, 365), 65)
daytempfd <- with(CanadianWeather, smooth.basisPar(day.5, 
    dailyAv[,,"Temperature.C"]) )
plot(daytempfd, axes=FALSE)
axisIntervals(1) 
# axisIntervals by default uses
# monthBegin.5, monthEnd.5, monthMid, and month.abb
axis(2)  
\end{ExampleCode}
\end{Examples}

\HeaderA{density.fd}{Compute a Probability Density Function}{density.fd}
\keyword{smooth}{density.fd}
\begin{Description}\relax
Like the regular S-PLUS function \code{density}, this function
computes a probability density function for a sample of values of a
random variable.  However, in this case the density function is defined
by a functional parameter object \code{WfdParobj} along with a normalizing
constant \code{C}.

The density function $p(x)$ has the form
\code{p(x) = C exp[W(x)]}
where function $W(x)$ is defined by the functional data object
\code{WfdParobj}.
\end{Description}
\begin{Usage}
\begin{verbatim}
density.fd(x, WfdParobj, conv=0.0001, iterlim=20,
           active=2:nbasis, dbglev=1, ...)
\end{verbatim}
\end{Usage}
\begin{Arguments}
\begin{ldescription}
\item[\code{x}] a strictly increasing set variable values.
These observations may be one of two forms:
\Enumerate{
\item a vector of observatons $x_i$
\item a two-column matrix, with the observations $x_i$ in the
first column, and frequencies $f_i$ in the second.
}
The first option corresponds to all $f_i = 1$.

\item[\code{WfdParobj}] a functional parameter object specifying the initial
value, basis object, roughness penalty and smoothing
parameter defining function $W(t).$

\item[\code{conv}] a positive constant defining the convergence criterion.

\item[\code{iterlim}] the maximum number of iterations allowed.

\item[\code{active}] a logical vector of length equal to the number of coefficients
defining \code{Wfdobj}. If an entry is TRUE, the corresponding
coefficient is estimated, and if FALSE, it is held at the value defining the
argument \code{Wfdobj}.  Normally the first coefficient is set to 0
and not estimated, since it is assumed that $W(0) = 0$.

\item[\code{dbglev}] either 0, 1, or 2.  This controls the amount information printed out on
each iteration, with 0 implying no output, 1 intermediate output level,
and 2 full output.  If levels 1 and 2 are used, it is helpful to
turn off the output buffering option in S-PLUS.

\item[\code{...}] Other arguments to match the generic function 'density'
\end{ldescription}
\end{Arguments}
\begin{Details}\relax
The goal of the function is provide a smooth density function
estimate that approaches some target density by an amount that is
controlled by the linear differential operator \code{Lfdobj} and
the penalty parameter. For example, if the second derivative of
$W(t)$ is penalized heavily, this will force the function to
approach a straight line, which in turn will force the density function
itself to be nearly normal or Gaussian.  Similarly, to each textbook
density function there corresponds a $W(t)$, and to each of these
in turn their corresponds a linear differential operator that will, when
apply to $W(t)$, produce zero as a result.
To plot the density function or to evaluate it, evaluate \code{Wfdobj},
exponentiate the resulting vector, and then divide by the normalizing
constant \code{C}.
\end{Details}
\begin{Value}
a named list of length 4 containing:

\begin{ldescription}
\item[\code{Wfdobj}] a functional data object defining function $W(x)$ that that
optimizes the fit to the data of the monotone function that it defines.

\item[\code{C}] the normalizing constant.

\item[\code{Flist}] a named list containing three results for the final converged solution:
(1)
\bold{f}: the optimal function value being minimized,
(2)
\bold{grad}: the gradient vector at the optimal solution,   and
(3)
\bold{norm}: the norm of the gradient vector at the optimal solution.

\item[\code{iternum}] the number of iterations.

\item[\code{iterhist}] a \code{iternum+1} by 5 matrix containing the iteration
history.

\end{ldescription}
\end{Value}
\begin{SeeAlso}\relax
\code{\LinkA{intensity.fd}{intensity.fd}}
\code{\LinkA{density}{density}}
\end{SeeAlso}
\begin{Examples}
\begin{ExampleCode}

#  set up range for density
rangeval <- c(-3,3)
#  set up some standard normal data
x <- rnorm(50)
#  make sure values within the range
x[x < -3] <- -2.99
x[x >  3] <-  2.99
#  set up basis for W(x)
basisobj <- create.bspline.basis(rangeval, 11)
#  set up initial value for Wfdobj
Wfd0 <- fd(matrix(0,11,1), basisobj)
WfdParobj <- fdPar(Wfd0)
#  estimate density
denslist <- density.fd(x, WfdParobj)
#  plot density
xval <- seq(-3,3,.2)
wval <- eval.fd(xval, denslist$Wfdobj)
pval <- exp(wval)/denslist$C
plot(xval, pval, type="l", ylim=c(0,0.4))
points(x,rep(0,50))

\end{ExampleCode}
\end{Examples}

\HeaderA{deriv.fd}{Compute a Derivative of a Functional Data Object}{deriv.fd}
\keyword{smooth}{deriv.fd}
\begin{Description}\relax
A derivative of a functional data object, or the result of applying
a linear differential operator to a functional data object, is then
converted to a functional data object. This is intended for situations
where a derivative is to be manipulated as a functional data object
rather than simply evaluated.
\end{Description}
\begin{Usage}
\begin{verbatim}
deriv.fd(expr, Lfdobj=int2Lfd(1), ...)
\end{verbatim}
\end{Usage}
\begin{Arguments}
\begin{ldescription}
\item[\code{expr}] a functional data object.  It is assumed that the basis for
representing the object can support the order of derivative
to be computed.  For B-spline bases, this means that the
order of the spline must be at least one larger than the order of
the derivative to be computed.

\item[\code{Lfdobj}] either a positive integer or a linear differential operator object.

\item[\code{...}] Other arguments to match generic for 'deriv'
\end{ldescription}
\end{Arguments}
\begin{Details}\relax
Typically, a derivative has more high frequency variation or detail
than the function itself.  The basis defining the function is used,
and therefore this must have enough basis functions to represent
the variation in the derivative satisfactorily.
\end{Details}
\begin{Value}
a functional data object for the derivative
\end{Value}
\begin{SeeAlso}\relax
\code{\LinkA{getbasismatrix}{getbasismatrix}}, 
\code{\LinkA{eval.basis}{eval.basis}}
\code{\LinkA{deriv}{deriv}}
\end{SeeAlso}
\begin{Examples}
\begin{ExampleCode}

#  Estimate the acceleration functions for growth curves
#  See the analyses of the growth data.
#  Set up the ages of height measurements for Berkeley data
age <- c( seq(1, 2, 0.25), seq(3, 8, 1), seq(8.5, 18, 0.5))
#  Range of observations
rng <- c(1,18)
#  Set up a B-spline basis of order 6 with knots at ages
knots  <- age
norder <- 6
nbasis <- length(knots) + norder - 2
hgtbasis <- create.bspline.basis(rng, nbasis, norder, knots)
#  Set up a functional parameter object for estimating
#  growth curves.  The 4th derivative is penalyzed to
#  ensure a smooth 2nd derivative or acceleration.
Lfdobj <- 4
lambda <- 10^(-0.5)   #  This value known in advance.
growfdPar <- fdPar(hgtbasis, Lfdobj, lambda)
#  Smooth the data.  The data for the boys and girls
#  are in matrices hgtm and hgtf, respectively.
hgtmfd <- smooth.basis(age, growth$hgtm, growfdPar)$fd
hgtffd <- smooth.basis(age, growth$hgtf, growfdPar)$fd
#  Compute the acceleration functions
accmfd <- deriv.fd(hgtmfd, 2)
accffd <- deriv.fd(hgtffd, 2)
#  Plot the two sets of curves
par(mfrow=c(2,1))
plot(accmfd)
plot(accffd)

\end{ExampleCode}
\end{Examples}

\HeaderA{df2lambda}{Convert Degrees of Freedom to a Smoothing Parameter Value}{df2lambda}
\keyword{smooth}{df2lambda}
\begin{Description}\relax
The degree of roughness of an estimated function is controlled by a
smoothing parameter $lambda$ that directly multiplies the penalty.
However, it can be difficult to interpret or choose this value, and it
is often easier to determine the roughness by choosing a value that is
equivalent of the degrees of freedom used by the smoothing procedure.
This function converts a degrees of freedom value into a multipler
$lambda$.
\end{Description}
\begin{Usage}
\begin{verbatim}
df2lambda(argvals, basisobj, wtvec=rep(1, n), Lfdobj=0,
          df=nbasis)
\end{verbatim}
\end{Usage}
\begin{Arguments}
\begin{ldescription}
\item[\code{argvals}] a vector containing rgument values associated with the values to
be smoothed.

\item[\code{basisobj}] a basis function object.

\item[\code{wtvec}] a vector of weights for the data to be smoothed.

\item[\code{Lfdobj}] either a nonnegative integer or a linear differential operator object.

\item[\code{df}] the degrees of freedom to be converted.

\end{ldescription}
\end{Arguments}
\begin{Details}\relax
The conversion requires a one-dimensional optimization and may be
therefore computationally intensive.
\end{Details}
\begin{Value}
a positive smoothing parameter value $lambda$
\end{Value}
\begin{SeeAlso}\relax
\code{\LinkA{lambda2df}{lambda2df}}, 
\code{\LinkA{lambda2gcv}{lambda2gcv}}
\end{SeeAlso}
\begin{Examples}
\begin{ExampleCode}

#  Smooth growth curves using a specified value of
#  degrees of freedom.
#  Set up the ages of height measurements for Berkeley data
age <- c( seq(1, 2, 0.25), seq(3, 8, 1), seq(8.5, 18, 0.5))
#  Range of observations
rng <- c(1,18)
#  Set up a B-spline basis of order 6 with knots at ages
knots  <- age
norder <- 6
nbasis <- length(knots) + norder - 2
hgtbasis <- create.bspline.basis(rng, nbasis, norder, knots)
#  Find the smoothing parameter equivalent to 12
#  degrees of freedom
lambda <- df2lambda(age, hgtbasis, df=12)
#  Set up a functional parameter object for estimating
#  growth curves.  The 4th derivative is penalyzed to
#  ensure a smooth 2nd derivative or acceleration.
Lfdobj <- 4
growfdPar <- fdPar(hgtbasis, Lfdobj, lambda)
#  Smooth the data.  The data for the girls are in matrix
#  hgtf.
hgtffd <- smooth.basis(age, growth$hgtf, growfdPar)$fd
#  Plot the curves
plot(hgtffd)

\end{ExampleCode}
\end{Examples}

\HeaderA{dirs}{Get subdirectories}{dirs}
\keyword{IO}{dirs}
\begin{Description}\relax
If you want only subfolders and no files, use \code{dirs}.  
With \code{recursive} = FALSE, \code{\LinkA{dir}{dir}} returns both folders
and files.  With \code{recursive} = TRUE, it returns only files.
\end{Description}
\begin{Usage}
\begin{verbatim}
dirs(path='.', pattern=NULL, exclude=NULL, all.files=FALSE,
     full.names=FALSE, recursive=FALSE, ignore.case=FALSE) 
\end{verbatim}
\end{Usage}
\begin{Arguments}
\begin{ldescription}
\item[\code{path, all.files, full.names, recursive, ignore.case}] as for \code{\LinkA{dir}{dir}}

\item[\code{pattern, exclude}] optional regular expressions of filenames to include or exclude,
respectively.  

\end{ldescription}
\end{Arguments}
\begin{Details}\relax
1.  mainDir <- dir(...)  without recurse 

2.  Use \code{\LinkA{file.info}{file.info}} to restrict mainDir to only
directories.

3.  If !recursive, return the restricted mainDir.  Else, if
length(mainDir) > 0, create dirList to hold the results of the
recursion and call \code{dirs} for each component of mainDir.  Then
\code{\LinkA{unlist}{unlist}} and return the result.
\end{Details}
\begin{Value}
A character vector of the desired subdirectories.
\end{Value}
\begin{Author}\relax
Spencer Graves
\end{Author}
\begin{SeeAlso}\relax
\code{\LinkA{dir}{dir}},
\code{\LinkA{file.info}{file.info}}
\code{\LinkA{package.dir}{package.dir}}
\end{SeeAlso}
\begin{Examples}
\begin{ExampleCode}
path2fdaM <- system.file('Matlab/fdaM', package='fda')
dirs(path2fdaM)
dirs(path2fdaM, full.names=TRUE)
dirs(path2fdaM, recursive=TRUE)
dirs(path2fdaM, exclude='^@|^private$', recursive=TRUE)

# Directories to add to Matlab path
# for R.matlab and fda
R.matExt <- system.file('externals', package='R.matlab')
fdaM <- dirs(path2fdaM, exclude='^@|^private$', full.names=TRUE,
              recursive=TRUE)  
add2MatlabPath <- c(R.matExt, path2fdaM, fdaM) 

\end{ExampleCode}
\end{Examples}

\HeaderA{Eigen}{Eigenanalysis preserving dimnames}{Eigen}
\keyword{array}{Eigen}
\begin{Description}\relax
Compute eigenvalues and vectors, assigning names to the eigenvalues
and dimnames to the eigenvectors.
\end{Description}
\begin{Usage}
\begin{verbatim}
Eigen(x, symmetric, only.values = FALSE, EISPACK = FALSE,
      valuenames )
\end{verbatim}
\end{Usage}
\begin{Arguments}
\begin{ldescription}
\item[\code{x}] a square matrix whose spectral decomposition is to be computed.  

\item[\code{symmetric}] logical:  If TRUE, the matrix is assumed to be symmetric (or 
Hermitian if complex) and only its lower triangle (diagonal
included) is used.  If 'symmetric' is not specified, the
matrix is inspected for symmetry.

\item[\code{only.values}] if 'TRUE', only the eigenvalues are computed and returned, otherwise
both eigenvalues and eigenvectors are returned. 

\item[\code{EISPACK}] logical. Should EISPACK be used (for compatibility with R < 1.7.0)?

\item[\code{valuenames}] character vector of length nrow(x) or a character string that can be
extended to that length by appening 1:nrow(x).

The default depends on symmetric and whether
\code{\LinkA{rownames}{rownames}} == \code{\LinkA{colnames}{colnames}}:  If
\code{\LinkA{rownames}{rownames}} == \code{\LinkA{colnames}{colnames}} and
symmetric = TRUE (either specified or determined by
inspection), the default is "paste('ev', 1:nrow(x), sep='')".
Otherwise, the default is colnames(x) unless this is NULL. 

\end{ldescription}
\end{Arguments}
\begin{Details}\relax
1.  Check 'symmetric'  

2.  ev <- eigen(x, symmetric, only.values = FALSE, EISPACK = FALSE);
see \code{\LinkA{eigen}{eigen}} for more details.  

3.  rNames = rownames(x);  if this is NULL, rNames = if(symmetric)
paste('x', 1:nrow(x), sep='') else paste('xcol', 1:nrow(x)).  

4.  Parse 'valuenames', assign to names(ev[['values']]).  

5.  dimnames(ev[['vectors']]) <- list(rNames, valuenames) 

NOTE:  This naming convention is fairly obvious if 'x' is symmetric.
Otherwise, dimensional analysis suggests problems with almost any
naming convention.  To see this, consider the following simple
example:

\deqn{
X <- matrix(1:4, 2, dimnames=list(LETTERS[1:2], letters[3:4]))
}{}
\Tabular{rrr}{
& c & d \\
A & 1 & 3 \\
B & 2 & 4 \\
}
\deqn{
X.inv <- solve(X)
}{}
\Tabular{rrr}{
& A & B \\
c & -2 & 1.5 \\
d & 1 & -0.5 \\
}

One way of interpreting this is to assume that colnames are really
reciprocals of the units.  Thus, in this example, X[1,1] is in units
of 'A/c' and X.inv[1,1] is in units of 'c/A'.  This would make any
matrix with the same row and column names potentially dimensionless.
Since eigenvalues are essentially the diagonal of a diagonal matrix,
this would mean that eigenvalues are dimensionless, and their names
are merely placeholders.
\end{Details}
\begin{Value}
a list with components values and (if only.values = FALSE)
vectors, as described in \code{\LinkA{eigen}{eigen}}.
\end{Value}
\begin{Author}\relax
Spencer Graves
\end{Author}
\begin{SeeAlso}\relax
\code{\LinkA{eigen}{eigen}},
\code{\LinkA{svd}{svd}}
\code{\LinkA{qr}{qr}}
\code{\LinkA{chol}{chol}}
\end{SeeAlso}
\begin{Examples}
\begin{ExampleCode}
X <- matrix(1:4, 2, dimnames=list(LETTERS[1:2], letters[3:4]))
eigen(X)
Eigen(X)
Eigen(X, valuenames='eigval')

Y <- matrix(1:4, 2, dimnames=list(letters[5:6], letters[5:6]))
Eigen(Y)

Eigen(Y, symmetric=TRUE)
# only the lower triangle is used;
# the upper triangle is ignored.  
\end{ExampleCode}
\end{Examples}

\HeaderA{eval.basis}{Values of Basis Functions or their Derivatives}{eval.basis}
\keyword{smooth}{eval.basis}
\begin{Description}\relax
A set of basis functions are evaluated at a vector of argument values.
If a linear differential object is provided, the  values are the
result of applying the the operator to each basis function.
\end{Description}
\begin{Usage}
\begin{verbatim}
eval.basis(evalarg, basisobj, Lfdobj=0)
\end{verbatim}
\end{Usage}
\begin{Arguments}
\begin{ldescription}
\item[\code{evalarg}] a vector of argument values.

\item[\code{basisobj}] a basis object defining basis functions whose values
are to be computed.

\item[\code{Lfdobj}] either a nonnegative integer or a linear differential.
operator object.

\end{ldescription}
\end{Arguments}
\begin{Details}\relax
If a linear differential operator object is supplied, the basis must
be such that the highest order derivative can be computed. If a
B-spline basis is used, for example, its order must be one larger than
the highest order of derivative required.
\end{Details}
\begin{Value}
a matrix of basis function values with rows corresponding
to argument values and columns to basis functions.
\end{Value}
\begin{Source}\relax
Ramsay, James O., and Silverman, Bernard W. (2006), \emph{Functional
Data Analysis, 2nd ed.}, Springer, New York.

Ramsay, James O., and Silverman, Bernard W. (2002), \emph{Applied
Functional Data Analysis}, Springer, New York
\end{Source}
\begin{SeeAlso}\relax
\code{\LinkA{getbasismatrix}{getbasismatrix}}, 
\code{\LinkA{eval.fd}{eval.fd}}, 
\code{\LinkA{plot.basisfd}{plot.basisfd}}
\end{SeeAlso}
\begin{Examples}
\begin{ExampleCode}
##
## 1.  B-splines
## 
# The simplest basis currently available:
# a single step function  
str(bspl1.1 <- create.bspline.basis(norder=1, breaks=0:1))
(eval.bspl1.1 <- eval.basis(seq(0, 1, .2), bspl1.1))

# The second simplest basis:
# 2 step functions, [0, .5], [.5, 1]
str(bspl1.2 <- create.bspline.basis(norder=1, breaks=c(0,.5, 1)))
(eval.bspl1.2 <- eval.basis(seq(0, 1, .2), bspl1.2))

# Second order B-splines (degree 1:  linear splines) 
str(bspl2.3 <- create.bspline.basis(norder=2, breaks=c(0,.5, 1)))
(eval.bspl2.3 <- eval.basis(seq(0, 1, .1), bspl2.3))
# 3 bases:  order 2 = degree 1 = linear 
# (1) line from (0,1) down to (0.5, 0), 0 after
# (2) line from (0,0) up to (0.5, 1), then down to (1,0)
# (3) 0 to (0.5, 0) then up to (1,1).

##
## 2.  Fourier 
## 
# The false Fourier series with 1 basis function
falseFourierBasis <- create.fourier.basis(nbasis=1)
(eval.fFB <- eval.basis(seq(0, 1, .2), falseFourierBasis))

# Simplest real Fourier basis with 3 basis functions
fourier3 <- create.fourier.basis()
(eval.fourier3 <- eval.basis(seq(0, 1, .2), fourier3))

# 3 basis functions on [0, 365]
fourier3.365 <- create.fourier.basis(c(0, 365))
eval.F3.365 <- eval.basis(day.5, fourier3.365)

matplot(eval.F3.365, type="l")

# The next simplest Fourier basis (5  basis functions)
fourier5 <- create.fourier.basis(nbasis=5)
(eval.F5 <- eval.basis(seq(0, 1, .1), fourier5))
matplot(eval.F5, type="l")

# A more complicated example
dayrng <- c(0, 365) 

nbasis <- 51
norder <- 6 

weatherBasis <- create.fourier.basis(dayrng, nbasis)
basisMat <- eval.basis(day.5, weatherBasis) 

matplot(basisMat[, 1:5], type="l")

\end{ExampleCode}
\end{Examples}

\HeaderA{eval.bifd}{Values a Two-argument Functional Data Object}{eval.bifd}
\keyword{smooth}{eval.bifd}
\begin{Description}\relax
A vector of argument values for the first argument \code{s} of the
functional data object to be evaluated.
\end{Description}
\begin{Usage}
\begin{verbatim}
eval.bifd(sevalarg, tevalarg, bifd, sLfdobj=0, tLfdobj=0)
\end{verbatim}
\end{Usage}
\begin{Arguments}
\begin{ldescription}
\item[\code{sevalarg}] a vector of argument values for the first argument \code{s} of the 
functional data object to be evaluated.

\item[\code{tevalarg}] a vector of argument values for the second argument \code{t} of the 
functional data object to be evaluated.

\item[\code{bifd}] a two-argument functional data object.

\item[\code{sLfdobj}] either a nonnegative integer or a linear differential operator
object.  If present, the derivative or the value of applying the
operator to the object as a function of the first argument \code{s}
is evaluated rather than the functions themselves.

\item[\code{tLfdobj}] either a nonnegative integer or a linear differential operator
object.  If present, the derivative or the value of applying the
operator to the object as a function of the second argument \code{t}
is evaluated rather than the functions themselves.

\end{ldescription}
\end{Arguments}
\begin{Value}
an array of 2, 3, or 4 dimensions containing the function values.  The
first dimension corresponds to the argument values in sevalarg, the
second to argument values in tevalarg, the third if present to
replications, and the fourth if present to functions.
\end{Value}
\begin{Examples}
\begin{ExampleCode}
daybasis   <- create.fourier.basis(c(0,365), 365)
harmLcoef  <- c(0,(2*pi/365)^2,0)
harmLfd    <- vec2Lfd(harmLcoef, c(0,365))
templambda <- 1.0
tempfdPar  <- fdPar(daybasis, harmLfd, lambda=1)
tempfd     <- smooth.basis(day.5,
          CanadianWeather$dailyAv[,,"Temperature.C"], tempfdPar)$fd
#    define the variance-covariance bivariate fd object
tempvarbifd <- var.fd(tempfd)
#    evaluate the variance-covariance surface and plot
weektime    <- seq(0,365,len=53)
tempvarmat  <- eval.bifd(weektime,weektime,tempvarbifd)
#    make a perspective plot of the variance function
persp(tempvarmat)
\end{ExampleCode}
\end{Examples}

\HeaderA{eval.fd}{Values of a Functional Data Object}{eval.fd}
\keyword{smooth}{eval.fd}
\begin{Description}\relax
Evaluate a functional data object at specified argument values, or
evaluate a derivative or the result of applying a linear differential 
operator to the functional object.
\end{Description}
\begin{Usage}
\begin{verbatim}
eval.fd(evalarg, fdobj, Lfdobj=0)
\end{verbatim}
\end{Usage}
\begin{Arguments}
\begin{ldescription}
\item[\code{evalarg}] a vector of argument values at which the functional data object is
to be evaluated.

\item[\code{fdobj}] a functional data object to be evaluated.

\item[\code{Lfdobj}] either a nonnegative integer or a linear differential operator
object.  If present, the derivative or the value of applying the
operator is evaluated rather than the functions themselves.

\end{ldescription}
\end{Arguments}
\begin{Value}
an array of 2 or 3 dimensions containing the function
values.  The first dimension corresponds to the argument values in
\code{evalarg},
the second to replications, and the third if present to functions.
\end{Value}
\begin{SeeAlso}\relax
\code{\LinkA{getbasismatrix}{getbasismatrix}}, 
\code{\LinkA{eval.bifd}{eval.bifd}}, 
\code{\LinkA{eval.penalty}{eval.penalty}}, 
\code{\LinkA{eval.monfd}{eval.monfd}}, 
\code{\LinkA{eval.posfd}{eval.posfd}}
\end{SeeAlso}
\begin{Examples}
\begin{ExampleCode}

#    set up the fourier basis
daybasis <- create.fourier.basis(c(0, 365), nbasis=65)
#  Make temperature fd object
#  Temperature data are in 12 by 365 matrix tempav
#  See analyses of weather data.
#  Set up sampling points at mid days
#  Convert the data to a functional data object
tempfd <- data2fd(CanadianWeather$dailyAv[,,"Temperature.C"],
                   day.5, daybasis)
#   set up the harmonic acceleration operator
Lbasis  <- create.constant.basis(c(0, 365))
Lcoef   <- matrix(c(0,(2*pi/365)^2,0),1,3)
bfdobj  <- fd(Lcoef,Lbasis)
bwtlist <- fd2list(bfdobj)
harmaccelLfd <- Lfd(3, bwtlist)
#   evaluate the value of the harmonic acceleration
#   operator at the sampling points
Ltempmat <- eval.fd(day.5, tempfd, harmaccelLfd)
#  Plot the values of this operator
matplot(day.5, Ltempmat, type="l")

\end{ExampleCode}
\end{Examples}

\HeaderA{eval.monfd}{Values of a Monotone Functional Data Object}{eval.monfd}
\keyword{smooth}{eval.monfd}
\begin{Description}\relax
Evaluate a monotone functional data object at specified argument values,
or evaluate a derivative of the functional object.
\end{Description}
\begin{Usage}
\begin{verbatim}
eval.monfd(evalarg, Wfd, Lfdobj=int2Lfd(0))
\end{verbatim}
\end{Usage}
\begin{Arguments}
\begin{ldescription}
\item[\code{evalarg}] a vector of argument values at which the functional data object is to be
evaluated.

\item[\code{Wfd}] a functional data object that defines the monotone function to be
evaluated.  Only univariate functions are permitted.

\item[\code{Lfdobj}] a nonnegative integer specifying a derivative to be evaluated.  AT
this time of writing, permissible derivative values are 0, 1, 2, or 3.
A linear differential operator is not allowed.

\end{ldescription}
\end{Arguments}
\begin{Details}\relax
A monotone function data object $h(t)$ is defined by 
$h(t) = [D^\{-1\} exp Wfd](t)$.  In this equation, the operator  $D^\{-1\}$ means
taking the indefinite integral of the function to which it applies.
Note that this equation implies that the monotone function has a value
of zero at the lower limit of the arguments.  To actually fit monotone
data, it will usually be necessary to estimate an intercept and a
regression coefficient to be applied to $h(t)$, usually with the
least squares regression function \code{lsfit}.
The function \code{Wfd} that defines the monotone function is
usually estimated by monotone smoothing function
\code{smooth.monotone.}
\end{Details}
\begin{Value}
a matrix containing the monotone function
values.  The first dimension corresponds to the argument values in
\code{evalarg} and
the second to replications.
\end{Value}
\begin{SeeAlso}\relax
\code{\LinkA{eval.fd}{eval.fd}}, 
\code{\LinkA{eval.posfd}{eval.posfd}}
\end{SeeAlso}
\begin{Examples}
\begin{ExampleCode}

#  Estimate the acceleration functions for growth curves
#  See the analyses of the growth data.
#  Set up the ages of height measurements for Berkeley data
age <- c( seq(1, 2, 0.25), seq(3, 8, 1), seq(8.5, 18, 0.5))
#  Range of observations
rng <- c(1,18)
#  First set up a basis for monotone smooth
#  We use b-spline basis functions of order 6
#  Knots are positioned at the ages of observation.
norder <- 6
nage   <- 31
nbasis <- nage + norder - 2
wbasis <- create.bspline.basis(rng, nbasis, norder, age)
#  starting values for coefficient
cvec0 <- matrix(0,nbasis,1)
Wfd0  <- fd(cvec0, wbasis)
#  set up functional parameter object
Lfdobj    <- 3          #  penalize curvature of acceleration
lambda    <- 10^(-0.5)  #  smoothing parameter
growfdPar <- fdPar(Wfd0, Lfdobj, lambda)
#  Set up wgt vector
wgt   <- rep(1,nage)
#  Smooth the data for the first girl
hgt1 = growth$hgtf[,1]
result <- smooth.monotone(age, hgt1, growfdPar, wgt)
#  Extract the functional data object and regression
#  coefficients
Wfd  <- result$Wfdobj
beta <- result$beta
#  Evaluate the fitted height curve over a fine mesh
agefine <- seq(1,18,len=101)
hgtfine <- beta[1] + beta[2]*eval.monfd(agefine, Wfd)
#  Plot the data and the curve
plot(age, hgt1, type="p")
lines(agefine, hgtfine)
#  Evaluate the acceleration curve
accfine <- beta[2]*eval.monfd(agefine, Wfd, 2)
#  Plot the acceleration curve
plot(agefine, accfine, type="l")
lines(c(1,18),c(0,0),lty=4)

\end{ExampleCode}
\end{Examples}

\HeaderA{eval.penalty}{Evaluate a Basis Penalty Matrix}{eval.penalty}
\keyword{smooth}{eval.penalty}
\begin{Description}\relax
A basis roughness penalty matrix is the matrix containing
the possible inner products of pairs of basis functions.
These inner products are typically defined in terms of
the value of a derivative or of a linear differential
operator applied to the basis function.  The basis penalty
matrix plays an important role in the computation of
functions whose roughness is controlled by a roughness
penalty.
\end{Description}
\begin{Usage}
\begin{verbatim}
eval.penalty(basisobj, Lfdobj=int2Lfd(0), rng=rangeval)
\end{verbatim}
\end{Usage}
\begin{Arguments}
\begin{ldescription}
\item[\code{basisobj}] a basis object.
\item[\code{Lfdobj}] either a nonnegative integer defining an order of a
derivative or a linear differential operator.

\item[\code{rng}] a vector of length 2 defining a restricted range.
Optionally, the inner products can be computed over
a range of argument values that lies within the
interval covered by the basis function definition.

\end{ldescription}
\end{Arguments}
\begin{Details}\relax
The inner product can be computed exactly for many
types of bases if $m$ is an integer.  These include
B-spline, fourier, exponential, monomial, polynomial and power bases.
In other cases, and for noninteger operators, the
inner products are computed by an iterative numerical
integration method called Richard extrapolation using the
trapezoidal rule.

If the penalty matrix must be evaluated repeatedly,
computation can be greatly speeded up by avoiding the use
of this function, and instead using quadrature points and
weights defined by Simpson's rule.
\end{Details}
\begin{Value}
a square symmetric matrix whose order is equal
to the number of basis functions defined by
the basis function object \code{ basisobj }.
If \code{Lfdobj} is $m$ or a linear
differential operator of order $m$, the rank
of the matrix should be at least approximately equal to
its order minus  $m$.
\end{Value}
\begin{SeeAlso}\relax
\code{\LinkA{getbasispenalty}{getbasispenalty}}, 
\code{\LinkA{eval.basis}{eval.basis}},
\end{SeeAlso}

\HeaderA{eval.posfd}{Evaluate a Positive Functional Data Object}{eval.posfd}
\keyword{smooth}{eval.posfd}
\begin{Description}\relax
Evaluate a positive functional data object at specified argument values,
or evaluate a derivative of the functional object.
\end{Description}
\begin{Usage}
\begin{verbatim}
eval.posfd(evalarg, Wfdobj, Lfdobj=int2Lfd(0))
\end{verbatim}
\end{Usage}
\begin{Arguments}
\begin{ldescription}
\item[\code{evalarg}] a vector of argument values at which the functional data object is to be
evaluated.

\item[\code{Wfdobj}] a functional data object that defines the positive function to be
evaluated.  Only univariate functions are permitted.

\item[\code{Lfdobj}] a nonnegative integer specifying a derivative to be evaluated.  AT
this time of writing, permissible derivative values are 0, 1 or 2.
A linear differential operator is not allowed.

\end{ldescription}
\end{Arguments}
\begin{Details}\relax
A positive function data object $h(t)$ is defined by $h(t) =[exp Wfd](t)$.
The function \code{Wfdobj} that defines the positive function is
usually estimated by positive smoothing function
\code{smooth.positive}
\end{Details}
\begin{Value}
a matrix containing the positive function
values.  The first dimension corresponds to the argument values in
\code{evalarg} and
the second to replications.
\end{Value}
\begin{SeeAlso}\relax
\code{\LinkA{eval.fd}{eval.fd}}, 
\code{\LinkA{eval.monfd}{eval.monfd}}
\end{SeeAlso}

\HeaderA{evaldiag.bifd}{Evaluate the Diagonal of a Bivariate Functional Data Object}{evaldiag.bifd}
\keyword{smooth}{evaldiag.bifd}
\begin{Description}\relax
Bivariate function data objects are functions of
two arguments, $f(s,t)$.  It can be useful to evaluate
the function for argument values satisying $s=t$, such
as evaluating the univariate variance function given the
bivariate function that defines the variance-covariance
function or surface.  A linear differential operator can
be applied to function $f(s,t)$ considered as a univariate
function of either object holding the other object fixed.
\end{Description}
\begin{Usage}
\begin{verbatim}
evaldiag.bifd(evalarg, bifdobj, sLfd=int2Lfd(0),
                                tLfd=int2Lfd(0))
\end{verbatim}
\end{Usage}
\begin{Arguments}
\begin{ldescription}
\item[\code{evalarg}] a vector of values of $s = t$.

\item[\code{bifdobj}] a bivariate functional data object of the \code{bifd} class.

\item[\code{sLfd}] either a nonnegative integer or a linear differential operator
object.

\item[\code{tLfd}] either a nonnegative integer or a linear differential operator
object.

\end{ldescription}
\end{Arguments}
\begin{Value}
a vector or matrix of diagonal function values.
\end{Value}
\begin{SeeAlso}\relax
\code{\LinkA{var.fd}{var.fd}}, 
\code{\LinkA{eval.bifd}{eval.bifd}}
\end{SeeAlso}

\HeaderA{expect.phi}{Expectatation of basis functions}{expect.phi}
\aliasA{expectden.phi}{expect.phi}{expectden.phi}
\aliasA{expectden.phiphit}{expect.phi}{expectden.phiphit}
\aliasA{normden.phi}{expect.phi}{normden.phi}
\aliasA{normint.phi}{expect.phi}{normint.phi}
\keyword{smooth}{expect.phi}
\keyword{smooth}{expect.phi}
\begin{Description}\relax
Computes expectations of basis functions with respect to a density
by numerical integration using Romberg integration
\end{Description}
\begin{Usage}
\begin{verbatim}
normint.phi(basisobj, cvec, JMAX=15, EPS=1e-7) 
normden.phi(basisobj, cvec, JMAX=15, EPS=1e-7) 
expect.phi(basisobj, cvec, nderiv=0, rng=rangeval,
                     JMAX=15, EPS=1e-7) 
expectden.phi(basisobj, cvec, Cval=1, nderiv=0, rng=rangeval,
                     JMAX=15, EPS=1e-7)
expectden.phiphit(basisobj, cvec, Cval=1, nderiv1=0,
                 nderiv2=0, rng=rangeval, JMAX=15, EPS=1e-7) 
\end{verbatim}
\end{Usage}
\begin{Arguments}
\begin{ldescription}
\item[\code{basisobj}] a basis function object 

\item[\code{cvec}] coefficient vector defining density, of length NBASIS 

\item[\code{Cval}] normalizing constant defining density 

\item[\code{nderiv, nderiv1, nderiv2}] order of derivative required for basis function expectation


\item[\code{rng}] a vector of length 2 giving the interval over which the integration is
to take place

\item[\code{JMAX}] maximum number of allowable iterations 

\item[\code{EPS}] convergence criterion for relative stop 

\end{ldescription}
\end{Arguments}
\begin{Details}\relax
normint.phi computes integrals of  
p(x) = exp phi'(x) 

normdel.phi computes integrals of
p(x) = exp phi"(x) 

expect.phi computes expectations of basis functions with respect to
intensity
p(x) <- exp t(c)*phi(x)


expectden.phi computes expectations of basis functions with respect
to density

p(x) <- exp(t(c)*phi(x))/Cval

expectden.phiphit computes expectations of cross product of basis
functions with respect to density

p(x) <- exp(t(c)*phi(x))/Cval
\end{Details}
\begin{Value}
A vector SS of length NBASIS of integrals of functions.
\end{Value}
\begin{SeeAlso}\relax
\code{\LinkA{plot.basisfd}{plot.basisfd}},
\end{SeeAlso}

\HeaderA{expon}{Exponential Basis Function Values}{expon}
\keyword{smooth}{expon}
\begin{Description}\relax
Evaluates a set of exponential basis functions, or a derivative of these
functions, at a set of arguments.
\end{Description}
\begin{Usage}
\begin{verbatim}
expon(x, ratevec=1, nderiv=0)
\end{verbatim}
\end{Usage}
\begin{Arguments}
\begin{ldescription}
\item[\code{x}] a vector of values at which the basis functions are to be evaluated.

\item[\code{ratevec}] a vector of rate or time constants defining the exponential
functions.  That is, if $a$ is the value of an
element of this vector, then the corresponding basis function
is $exp(at)$. The number of basis functions is equal
to the length of \code{ratevec}.

\item[\code{nderiv}] a nonnegative integer specifying an order of derivative to
be computed.  The default is 0, or the basis function value.

\end{ldescription}
\end{Arguments}
\begin{Details}\relax
There are no restrictions on the rate constants.
\end{Details}
\begin{Value}
a matrix of basis function values with rows corresponding
to argument values and columns to basis functions.
\end{Value}
\begin{SeeAlso}\relax
\code{\LinkA{exponpen}{exponpen}}
\end{SeeAlso}

\HeaderA{exponpen}{Exponential Penalty Matrix}{exponpen}
\keyword{smooth}{exponpen}
\begin{Description}\relax
Computes the matrix defining the roughness penalty for functions
expressed in terms of an exponential basis.
\end{Description}
\begin{Usage}
\begin{verbatim}
exponpen(basisobj, Lfdobj=int2Lfd(2))
\end{verbatim}
\end{Usage}
\begin{Arguments}
\begin{ldescription}
\item[\code{basisobj}] an exponential basis object.

\item[\code{Lfdobj}] either a nonnegative integer or a linear differential operator object.

\end{ldescription}
\end{Arguments}
\begin{Details}\relax
A roughness penalty for a function $x(t)$ is defined by
integrating the square of either the derivative of  $ x(t) $ or,
more generally, the result of applying a linear differential operator
$L$ to it.  The most common roughness penalty is the integral of
the square of the second derivative, and
this is the default. To apply this roughness penalty, the matrix of
inner products of the basis functions (possibly after applying the
linear differential operator to them) defining this function
is necessary. This function just calls the roughness penalty evaluation
function specific to the basis involved.
\end{Details}
\begin{Value}
a symmetric matrix of order equal to the number of basis functions
defined by the exponential basis object.  Each element is the inner product
of two exponential basis functions after applying the derivative or linear
differential operator defined by Lfdobj.
\end{Value}
\begin{SeeAlso}\relax
\code{\LinkA{expon}{expon}}, 
\code{\LinkA{eval.penalty}{eval.penalty}}, 
\code{\LinkA{getbasispenalty}{getbasispenalty}}
\end{SeeAlso}
\begin{Examples}
\begin{ExampleCode}

#  set up an exponential basis with 3 basis functions
ratevec  <- c(0, -1, -5)
basisobj <- create.exponential.basis(c(0,1),3,ratevec)
#  compute the 3 by 3 matrix of inner products of
#  second derivatives
penmat <- exponpen(basisobj)

\end{ExampleCode}
\end{Examples}

\HeaderA{fd}{Define a Functional Data Object}{fd}
\keyword{smooth}{fd}
\keyword{internal}{fd}
\begin{Description}\relax
This is the constructor function for objects of the \code{fd} class.
Each function that sets up an object of this class must call this
function.  This includes functions \code{data2fd},
\code{smooth.basis}, \code{density.fd}, and so forth that estimate
functional data objects that smooth or otherwise represent data.
Ordinarily, user of the functional data analysis software will not
need to call this function directly, but these notes are valuable to
understanding what the "slots" or "members" of the \code{fd} class
are.
\end{Description}
\begin{Usage}
\begin{verbatim}
fd(coef=NULL, basisobj=NULL, fdnames=defaultnames)
\end{verbatim}
\end{Usage}
\begin{Arguments}
\begin{ldescription}
\item[\code{coef}] a vector, matrix, or three-dimensional array of coefficients.  The
first dimension of matrices and arrays, as well as that of a vector,
corresponds to basis functions.  The second dimension corresponds to
the number of functions or curves, or to replicates.  The third
dimension if present corresponds to variables for multivariate
functional data objects.  A functional data object is referred to as
"univariate" if this argument is a vector or a matrix, and
"multivariate" if it is a three-dimensional array.

if(is.null(coef)) coef <- rep(0, basisobj[['nbasis']]) 

\item[\code{basisobj}] a functional basis object defining the basis

if(is.null(basisobj)){
rc <- range(coef)
if(diff(rc)==0) rc <- rc+0:1
nb <- max(4, nrow(coef)) 
basisobj <- create.bspline.basis(rc, nbasis = nb)
}

\item[\code{fdnames}] A list of length 3, each member being a string vector containing
labels for the levels of the corresponding dimension of the discrete
data.  The first dimension is for argument values, and is given the
default name "time", the second is for replications, and is given
the default name "reps", and the third is for functions, and is
given the default name "values".  

\end{ldescription}
\end{Arguments}
\begin{Details}\relax
To check that an object is of this class, use function
\code{is.fd}.

Normally only developers of new functional data analysis
functions will actually need to use this function.
\end{Details}
\begin{Value}
A functional data object (i.e., having class \code{fd}), which is a
list with components named \code{coefs}, \code{basis}, and
\code{fdnames}.
\end{Value}
\begin{Source}\relax
Ramsay, James O., and Silverman, Bernard W. (2006), \emph{Functional
Data Analysis, 2nd ed.}, Springer, New York.

Ramsay, James O., and Silverman, Bernard W. (2002), \emph{Applied
Functional Data Analysis}, Springer, New York
\end{Source}
\begin{SeeAlso}\relax
\code{\LinkA{data2fd}{data2fd}} 
\code{\LinkA{smooth.basis}{smooth.basis}}
\code{\LinkA{density.fd}{density.fd}}
\code{\LinkA{create.bspline.basis}{create.bspline.basis}}
\end{SeeAlso}
\begin{Examples}
\begin{ExampleCode}
##
## The simplest b-spline basis:  order 1, degree 0, zero interior knots:  
##       a single step function 
##
bspl1.1 <- create.bspline.basis(norder=1, breaks=0:1)
fd.bspl1.1 <- fd(0, basisobj=bspl1.1)

fd.bspl1.1a <- fd(basisobj=bspl1.1)

all.equal(fd.bspl1.1, fd.bspl1.1a)

# TRUE

## Not run: 
fd.bspl1.1b <- fd(0)
Error in fd(0) : 
  Number of coefficients does not match number of basis functions.

... because fd by default wants to create a cubic spline 
## End(Not run)
##
## Cubic spline:  4  basis functions 
##
bspl4 <- create.bspline.basis(nbasis=4)
plot(bspl4) 
parab4.5 <- fd(c(3, -1, -1, 3)/3, bspl4)
# = 4*(x-.5)^2
plot(parab4.5) 

\end{ExampleCode}
\end{Examples}

\HeaderA{fda-internal}{FDA internal functions}{fda.Rdash.internal}
\aliasA{/.fd}{fda-internal}{/.fd}
\aliasA{c.fd}{fda-internal}{c.fd}
\aliasA{derivchk}{fda-internal}{derivchk}
\aliasA{derivs}{fda-internal}{derivs}
\aliasA{eigchk}{fda-internal}{eigchk}
\aliasA{expect.phiphit}{fda-internal}{expect.phiphit}
\aliasA{fd2list}{fda-internal}{fd2list}
\aliasA{fdchk}{fda-internal}{fdchk}
\aliasA{fngrad.smooth.monotone}{fda-internal}{fngrad.smooth.monotone}
\aliasA{fngrad.smooth.morph}{fda-internal}{fngrad.smooth.morph}
\aliasA{geigen}{fda-internal}{geigen}
\aliasA{hesscal.smooth.monotone}{fda-internal}{hesscal.smooth.monotone}
\aliasA{hesscal.smooth.morph}{fda-internal}{hesscal.smooth.morph}
\aliasA{is.diag}{fda-internal}{is.diag}
\aliasA{is.eqbasis}{fda-internal}{is.eqbasis}
\aliasA{is.integerLfd}{fda-internal}{is.integerLfd}
\aliasA{isotone}{fda-internal}{isotone}
\aliasA{knotmultchk}{fda-internal}{knotmultchk}
\aliasA{linesearch}{fda-internal}{linesearch}
\aliasA{linesearch.smooth.monotone}{fda-internal}{linesearch.smooth.monotone}
\aliasA{linesearch.smooth.morph}{fda-internal}{linesearch.smooth.morph}
\aliasA{loglfnden}{fda-internal}{loglfnden}
\aliasA{loglfninten}{fda-internal}{loglfninten}
\aliasA{loglfnpos}{fda-internal}{loglfnpos}
\aliasA{loglhesspos}{fda-internal}{loglhesspos}
\aliasA{m2ij}{fda-internal}{m2ij}
\aliasA{monfn}{fda-internal}{monfn}
\aliasA{mongrad}{fda-internal}{mongrad}
\aliasA{monhess}{fda-internal}{monhess}
\aliasA{normalize.phi}{fda-internal}{normalize.phi}
\aliasA{pendiagfn}{fda-internal}{pendiagfn}
\aliasA{plot.cca.fd}{fda-internal}{plot.cca.fd}
\aliasA{plot.Lfd}{fda-internal}{plot.Lfd}
\aliasA{polintarray}{fda-internal}{polintarray}
\aliasA{polintmat}{fda-internal}{polintmat}
\aliasA{polynom}{fda-internal}{polynom}
\aliasA{polynompen}{fda-internal}{polynompen}
\aliasA{polyprod}{fda-internal}{polyprod}
\aliasA{ppBspline}{fda-internal}{ppBspline}
\aliasA{ppderiv}{fda-internal}{ppderiv}
\aliasA{print.basisfd}{fda-internal}{print.basisfd}
\aliasA{print.bifd}{fda-internal}{print.bifd}
\aliasA{print.fd}{fda-internal}{print.fd}
\aliasA{print.fdPar}{fda-internal}{print.fdPar}
\aliasA{print.Lfd}{fda-internal}{print.Lfd}
\aliasA{rangechk}{fda-internal}{rangechk}
\aliasA{regfngrad}{fda-internal}{regfngrad}
\aliasA{reghess}{fda-internal}{reghess}
\aliasA{regyfn}{fda-internal}{regyfn}
\aliasA{rkck}{fda-internal}{rkck}
\aliasA{rkqs}{fda-internal}{rkqs}
\aliasA{shifty}{fda-internal}{shifty}
\aliasA{sqrt.fd}{fda-internal}{sqrt.fd}
\aliasA{stepchk}{fda-internal}{stepchk}
\aliasA{stepit}{fda-internal}{stepit}
\aliasA{symsolve}{fda-internal}{symsolve}
\aliasA{trapzmat}{fda-internal}{trapzmat}
\aliasA{use.proper.basis}{fda-internal}{use.proper.basis}
\aliasA{Varfnden}{fda-internal}{Varfnden}
\aliasA{Varfninten}{fda-internal}{Varfninten}
\aliasA{[.fd}{fda-internal}{[.fd}
\aliasA{\textasciicircum{}.fd}{fda-internal}{\textasciicircum.Rlbrace..Rrbrace..fd}
\keyword{internal}{fda-internal}
\begin{Description}\relax
Internal undocumentation functions
\end{Description}
\begin{Usage}
\begin{verbatim}
center.fd(fdobj)
\end{verbatim}
\end{Usage}

\HeaderA{fda-package}{Functional Data Analysis in R}{fda.Rdash.package}
\aliasA{fda}{fda-package}{fda}
\keyword{smooth}{fda-package}
\begin{Description}\relax
Functions and data sets companion to Ramsay, J. O., and Silverman,
B. W. (2005) Functional Data Analysis, 2nd ed. and (2002) Applied
Functional Data Analysis (Springer).  This includes finite bases
approximations (such as splines and Fourier series) to functions fit
to data smoothing on the integral of the squared deviations from an
arbitrary differential operator.
\end{Description}
\begin{Details}\relax
\Tabular{ll}{
Package: & fda\\
Type: & Package\\
Version: & 2.0.0\\
Date: & 2008-05-05\\
License: & GPL-2\\
LazyLoad: & yes\\
}
\end{Details}
\begin{Author}\relax
J. O. Ramsay, 

Maintainer:  J. O. Ramsay <ramsay@psych.mcgill.ca>
\end{Author}
\begin{References}\relax
Ramsay, James O., and Silverman, Bernard W. (2005), \emph{Functional 
Data Analysis, 2nd ed.}, Springer, New York. 

Ramsay, James O., and Silverman, Bernard W. (2002), \emph{Applied
Functional Data Analysis}, Springer, New York.
\end{References}
\begin{Examples}
\begin{ExampleCode}
##
## Simple smoothing
##
girlGrowthSm <- with(growth, smooth.basisPar(argvals=age, y=hgtf))
plot(girlGrowthSm$fd, xlab="age", ylab="height (cm)",
         main="Girls in Berkeley Growth Study" )
plot(deriv(girlGrowthSm$fd), xlab="age", ylab="growth rate (cm / year)",
         main="Girls in Berkeley Growth Study" )
plot(deriv(girlGrowthSm$fd, 2), xlab="age",
        ylab="growth acceleration (cm / year^2)",
        main="Girls in Berkeley Growth Study" )
##
## Simple basis 
##
bspl1.2 <- create.bspline.basis(norder=1, breaks=c(0,.5, 1))
plot(bspl1.2)
# 2 bases, order 1 = degree 0 = step functions:  
# (1) constant 1 between 0 and 0.5 and 0 otherwise
# (2) constant 1 between 0.5 and 1 and 0 otherwise.

fd1.2 <- Data2fd(0:1, basisobj=bspl1.2)
op <- par(mfrow=c(2,1))
plot(bspl1.2, main='bases') 
plot(fd1.2, main='fit')
par(op) 
# A step function:  0 to time=0.5, then 1 after 

\end{ExampleCode}
\end{Examples}

\HeaderA{fdaMatlabPath}{Add 'fdaM' to the Matlab path}{fdaMatlabPath}
\keyword{programming}{fdaMatlabPath}
\begin{Description}\relax
Write a sequence of Matlab commands to \code{fdaMatlabPath.m} in the
working directory containing commands to add \code{fdaM} to the path
for Matlab.
\end{Description}
\begin{Usage}
\begin{verbatim}
fdaMatlabPath(R.matlab) 
\end{verbatim}
\end{Usage}
\begin{Arguments}
\begin{ldescription}
\item[\code{R.matlab}] logical:  If TRUE, include '~R/library/R.matlab/externals' in the
path.  If(missing(R.matlab)) include '~R/library/R.matlab/externals'
only if R.matlab is installed.   

\end{ldescription}
\end{Arguments}
\begin{Details}\relax
\Itemize{
\item[USAGE] If your Matlab installation does NOT have a \code{startup.m} file,
it might be wise to copy \code{fdaMatlabPath.m} into a directory
where Matlab would look for \code{startup.m}, then rename it to
\code{startup.m}.

If you have a \code{startup.m}, you could add the contents of
\code{fdaMatlabPath.m} to \code{startup.m}.

Alternatively, you can copy \code{fdaMatlabPath.m} into the
directory containing \code{startup.m} and add the following to the
end of \code{startup.m}:

\Tabular{lll}{
\& if exist('fdaMatlabPath') \& \bsl{}
\& \& fdaMatlabPath ; \bsl{}
\& end \& 
}

\item[ALGORITHM] 1.  path2fdaM = path to the \code{Matlab/fdaM} subdiretory of the
\code{fda} installation directory. 

2.  Find all subdirectories of path2fdaM except those beginning in
'@' or including 'private'. 

3.  if(requires(R.matlab)) add the path to \code{MatlabServer.m}
to \code{dirs2add} 

4.  d2a <- paste("addpath('", dirs2add, "');", sep='')

5.  writeLines(d2a, 'fdaMatlabPath.m')

6.  if(exists(startupFile)) append \code{d2a} to it

}
\end{Details}
\begin{Value}
A character vector of Matlab \code{addpath} commands is returned
invisibly.
\end{Value}
\begin{Author}\relax
Spencer Graves with help from Jerome Besnard
\end{Author}
\begin{References}\relax
Matlab documentation for \code{addpath} and \code{startup.m}.
\end{References}
\begin{SeeAlso}\relax
\code{\LinkA{Matlab}{Matlab}},
\code{\LinkA{dirs}{dirs}}
\end{SeeAlso}
\begin{Examples}
\begin{ExampleCode}
# Modify the Matlab startup.m only when you really want to,
# typically once per installation ... certaintly not
# every time we test this package.
fdaMatlabPath()
\end{ExampleCode}
\end{Examples}

\HeaderA{fdPar}{Define a Functional Parameter Object}{fdPar}
\keyword{smooth}{fdPar}
\begin{Description}\relax
Functional parameter objects are used as arguments to functions that
estimate functional parameters, such as smoothing functions like
\code{smooth.basis}.  A functional parameter object is a functional
data object with additional slots specifying a roughness penalty, a
smoothing parameter and whether or not the functional parameter is to
be estimated or held fixed.  Functional parameter objects are used as
arguments to functions that estimate functional parameters.
\end{Description}
\begin{Usage}
\begin{verbatim}
fdPar(fdobj=NULL, Lfdobj=NULL, lambda=0, estimate=TRUE, penmat=NULL)
\end{verbatim}
\end{Usage}
\begin{Arguments}
\begin{ldescription}
\item[\code{fdobj}] a functional data object, functional basis object, a functional
parameter object or a matrix.  If class(fdobj) == 'basisfd', it is
converted to functional data objects with the identity matrix as the
coefficient matrix.  If it a matrix, it is replaced by fd(fdobj).      

\item[\code{Lfdobj}] either a nonnegative integer or a linear differential operator
object.  If NULL and fdobj[['type']] == 'bspline', Lfdobj =
int2Lfd(max(0, norder-2)), where norder = order of fdobj.   

\item[\code{lambda}] a nonnegative real number specifying the amount of smoothing
to be applied to the estimated functional parameter.

\item[\code{estimate}] a logical value:  if \code{TRUE}, the functional parameter is
estimated, otherwise, it is held fixed.

\item[\code{penmat}] a roughness penalty matrix.  Including this can eliminate the need
to compute this matrix over and over again in some types of
calculations.

\end{ldescription}
\end{Arguments}
\begin{Details}\relax
Functional parameters are often needed to specify initial
values for iteratively refined estimates, as is the case in
functions \code{register.fd} and \code{smooth.monotone}.

Often a list of functional parameters must be supplied to a function
as an argument, and it may be that some of these parameters are
considered known and must remain fixed during the analysis.  This is
the case for functions \code{fRegress} and  \code{pda.fd}, for
example.
\end{Details}
\begin{Value}
a functional parameter object
\end{Value}
\begin{Source}\relax
Ramsay, James O., and Silverman, Bernard W. (2006), \emph{Functional
Data Analysis, 2nd ed.}, Springer, New York.

Ramsay, James O., and Silverman, Bernard W. (2002), \emph{Applied
Functional Data Analysis}, Springer, New York
\end{Source}
\begin{SeeAlso}\relax
\code{\LinkA{cca.fd}{cca.fd}}, 
\code{\LinkA{density.fd}{density.fd}}, 
\code{\LinkA{fRegress}{fRegress}}, 
\code{\LinkA{intensity.fd}{intensity.fd}}, 
\code{\LinkA{pca.fd}{pca.fd}}, 
\code{\LinkA{smooth.fdPar}{smooth.fdPar}}, 
\code{\LinkA{smooth.basis}{smooth.basis}}, 
\code{\LinkA{smooth.basisPar}{smooth.basisPar}}, 
\code{\LinkA{smooth.monotone}{smooth.monotone}},
\code{\bsl{}line\{int2Lfd\}}
\end{SeeAlso}
\begin{Examples}
\begin{ExampleCode}
##
## Simple example
##
#  set up range for density
rangeval <- c(-3,3)
#  set up some standard normal data
x <- rnorm(50)
#  make sure values within the range
x[x < -3] <- -2.99
x[x >  3] <-  2.99
#  set up basis for W(x)
basisobj <- create.bspline.basis(rangeval, 11)
#  set up initial value for Wfdobj
Wfd0 <- fd(matrix(0,11,1), basisobj)
WfdParobj <- fdPar(Wfd0)

WfdP3 <- fdPar(seq(-3, 3, length=11))

##
##  smooth the Canadian daily temperature data 
##
#    set up the fourier basis
nbasis   <- 365
dayrange <- c(0,365)
daybasis <- create.fourier.basis(dayrange, nbasis)
dayperiod <- 365
harmaccelLfd <- vec2Lfd(c(0,(2*pi/365)^2,0), dayrange)
#  Make temperature fd object
#  Temperature data are in 12 by 365 matrix tempav
#    See analyses of weather data.
#  Set up sampling points at mid days
daytime  <- (1:365)-0.5
#  Convert the data to a functional data object
daybasis65 <- create.fourier.basis(dayrange, nbasis, dayperiod)
templambda <- 1e1
tempfdPar  <- fdPar(fdobj=daybasis65, Lfdobj=harmaccelLfd, lambda=templambda)

#FIXME
#tempfd <- smooth.basis(CanadianWeather$tempav, daytime, tempfdPar)
#  Set up the harmonic acceleration operator
Lbasis  <- create.constant.basis(dayrange);
Lcoef   <- matrix(c(0,(2*pi/365)^2,0),1,3)
bfdobj  <- fd(Lcoef,Lbasis)
bwtlist <- fd2list(bfdobj)
harmaccelLfd <- Lfd(3, bwtlist)
#  Define the functional parameter object for
#  smoothing the temperature data
lambda   <- 0.01  #  minimum GCV estimate
#tempPar <- fdPar(daybasis365, harmaccelLfd, lambda)
#  smooth the data
#tempfd <- smooth.basis(daytime, CanadialWeather$tempav, tempPar)$fd
#  plot the temperature curves
#plot(tempfd)

\end{ExampleCode}
\end{Examples}

\HeaderA{file.copy2}{Copy a file with a default 'to' name}{file.copy2}
\keyword{IO}{file.copy2}
\begin{Description}\relax
Copy a file appending a number to make the \code{to} name unique, with
default \code{to} = \code{from}.
\end{Description}
\begin{Usage}
\begin{verbatim}
file.copy2(from, to) 
\end{verbatim}
\end{Usage}
\begin{Arguments}
\begin{ldescription}
\item[\code{from}] character:  name of a file to be copied 
\item[\code{to}] character:  name of copy.  Default = \code{from} with an integer
appended to the name.  

\end{ldescription}
\end{Arguments}
\begin{Details}\relax
1.  length(from) != 1:  Error:  Only one file can be copied.  

2.  file.exists(from)?  If no, If no, return FALSE.

3.  if(missing(to))to <- from;  else if(length(to)!=1) error.

4.  file.exists(to)?  If yes, Dir <- dir(dirname(to)), find all
\code{Dir} starting with \code{to}, and find the smallest integer to
append to make a unique \code{to} name.  

5.  file.copy(from, to)

6.  Return TRUE.
\end{Details}
\begin{Value}
logical:  TRUE (with a name = name of the file created);  FALSE if no
file created.
\end{Value}
\begin{Author}\relax
Spencer Graves
\end{Author}
\begin{SeeAlso}\relax
\code{\LinkA{file.copy}{file.copy}},
\end{SeeAlso}
\begin{Examples}
\begin{ExampleCode}
## Not run: 
file.copy2('startup.m')
# Used by 'fdaMatlabPath' so an existing 'startup.m' is not destroyed
## End(Not run)
\end{ExampleCode}
\end{Examples}

\HeaderA{fourier}{Fourier Basis Function Values}{fourier}
\keyword{smooth}{fourier}
\begin{Description}\relax
Evaluates a set of Fourier basis functions, or a derivative of these
functions, at a set of arguments.
\end{Description}
\begin{Usage}
\begin{verbatim}
fourier(x, nbasis=n, period=span, nderiv=0)
\end{verbatim}
\end{Usage}
\begin{Arguments}
\begin{ldescription}
\item[\code{x}] a vector of argument values at which the Fourier basis functions are
to be evaluated.

\item[\code{nbasis}] the number of basis functions in the Fourier basis.  The first basis
function is the constant function, followed by sets of  sine/cosine
pairs.  Normally the number of basis functions will be an odd.  The
default number is the number of argument values.

\item[\code{period}] the width of an interval over which all sine/cosine basis functions
repeat themselves. The default is the difference between the largest
and smallest argument values.

\item[\code{nderiv}] the derivative to be evaluated.  The derivative must not exceed the
order.  The default derivative is 0, meaning that the basis functions
themselves are evaluated.

\end{ldescription}
\end{Arguments}
\begin{Value}
a matrix of function values.  The number of rows equals the number of
arguments, and the number of columns equals the number of basis functions.
\end{Value}
\begin{SeeAlso}\relax
\code{\LinkA{fourierpen}{fourierpen}}
\end{SeeAlso}
\begin{Examples}
\begin{ExampleCode}

#  set up a set of 11 argument values
x <- seq(0,1,0.1)
names(x) <- paste("x", 0:10, sep="")
#  compute values for five Fourier basis functions
#  with the default period (1) and derivative (0)
(basismat <- fourier(x, 5))

# Create a false Fourier basis, i.e., nbasis = 1
# = a constant function
fourier(x, 1)

\end{ExampleCode}
\end{Examples}

\HeaderA{fourierpen}{Fourier Penalty Matrix}{fourierpen}
\keyword{smooth}{fourierpen}
\begin{Description}\relax
Computes the matrix defining the roughness penalty for functions
expressed in terms of a Fourier basis.
\end{Description}
\begin{Usage}
\begin{verbatim}
fourierpen(basisobj, Lfdobj=int2Lfd(2))
\end{verbatim}
\end{Usage}
\begin{Arguments}
\begin{ldescription}
\item[\code{basisobj}] a Fourier basis object.

\item[\code{Lfdobj}] either a nonnegative integer or a linear differential operator object.

\end{ldescription}
\end{Arguments}
\begin{Details}\relax
A roughness penalty for a function $x(t)$ is defined by
integrating the square of either the derivative of  $ x(t) $ or,
more generally, the result of applying a linear differential operator
$L$ to it.  The most common roughness penalty is the integral of
the square of the second derivative, and
this is the default. To apply this roughness penalty, the matrix of
inner products of the basis functions (possibly after applying the
linear differential operator to them) defining this function
is necessary. This function just calls the roughness penalty evaluation
function specific to the basis involved.
\end{Details}
\begin{Value}
a symmetric matrix of order equal to the number of basis functions
defined by the Fourier basis object.  Each element is the inner product
of two Fourier basis functions after applying the derivative or linear
differential operator defined by Lfdobj.
\end{Value}
\begin{SeeAlso}\relax
\code{\LinkA{fourier}{fourier}}, 
\code{\LinkA{eval.penalty}{eval.penalty}}, 
\code{\LinkA{getbasispenalty}{getbasispenalty}}
\end{SeeAlso}
\begin{Examples}
\begin{ExampleCode}

#  set up a Fourier basis with 13 basis functions
#  and and period 1.0.
basisobj <- create.fourier.basis(c(0,1),13)
#  compute the 13 by 13 matrix of inner products
#  of second derivatives
penmat <- fourierpen(basisobj)

\end{ExampleCode}
\end{Examples}

\HeaderA{Fperm.fd}{Permutation F-test for functional linear regression.}{Fperm.fd}
\keyword{smooth}{Fperm.fd}
\begin{Description}\relax
Fperm.fd creates a null distribution for a test of no effect in functional
linear regression. It makes generic use of \code{fRegress} and permutes the
\code{yfdPar} input.
\end{Description}
\begin{Usage}
\begin{verbatim}
Fperm.fd(yfdPar, xfdlist, betalist,wt=NULL,
            nperm=200,argvals=NULL,q=0.95,plotres=TRUE)
\end{verbatim}
\end{Usage}
\begin{Arguments}
\begin{ldescription}
\item[\code{yfdPar}] the dependent variable object.  It may be an object of
three possible classes:
\Itemize{
\item[vector] if the dependent variable is scalar.
\item[fd] a functional data object if the dependent variable is
functional.

\item[fdPar] a functional parameter object if the dependent variable is
functional, and if it is necessary to smooth the prediction of
the dependent variable.

}

\item[\code{xfdlist}] a list of length equal to the number of independent variables. Members
of this list are the independent variables.  They be objects of either
of these two classes:

\Itemize{
\item a vector if the independent dependent variable is scalar.
\item a functional data object if the dependent variable is functional.
}

In either case, the object must have the same number of replications as
the dependent variable object.  That is, if it is a scalar, it must be
of the same length as the dependent variable, and if it is functional,
it must have the same number of replications as the dependent variable.

\item[\code{betalist}] a list of length equal to the number of independent variables. Members
of this list define the regression functions to be estimated.
They are functional parameter objects.  Note that even if corresponding
independent variable is scalar, its regression coefficient will be
functional if the dependent variable is functional.  Each of these
functional parameter objects defines a single functional data object,
that is, with only one replication.

\item[\code{wt}] weights for weighted least squares, defaults to all 1.

\item[\code{nperm}] number of permutations to use in creating the null distribution.

\item[\code{argvals}] If \code{yfdPar} is a \code{fd} object, the points at which to evaluate
the point-wise F-statistic.

\item[\code{q}] Critical quantile of the null distribution to compare to the observed
F-statistic.

\item[\code{plotres}] Argument to plot a visual display of the null distribution displaying the
\code{q}th quantile and observed F-statistic.

\end{ldescription}
\end{Arguments}
\begin{Details}\relax
An F-statistic is calculated as the ratio of residual variance to predicted
variance. The observed F-statistic is returned along with the permutation
distribution.

If \code{yfdPar} is a \code{fd} object, the maximal value of the pointwise
F-statistic is calculated. The pointwise F-statistics are also returned.

The default of setting \code{q = 0.95} is, by now, fairly standard. The default
\code{nperm = 200} may be small, depending on the amount of computing time available.

If \code{argvals} is not specified and \code{yfdPar} is a \code{fd} object,
it defaults to 101 equally-spaced points on the range of \code{yfdPar}.
\end{Details}
\begin{Value}
A list with components
\begin{ldescription}
\item[\code{pval}] the observed p-value of the permutation test.
\item[\code{qval}] the \code{q}th quantile of the null distribution.
\item[\code{Fobs}] the observed maximal F-statistic.
\item[\code{Fnull}] a vector of length \code{nperm} giving the observed values of the
permutation distribution.

\item[\code{Fvals}] the pointwise values of the observed F-statistic.
\item[\code{Fnullvals}] the pointwise values of of the permutation observations.

\item[\code{pvals.pts}] pointwise p-values of the F-statistic.
\item[\code{qvals.pts}] pointwise \code{q}th quantiles of the null distribution

\item[\code{fRegressList}] the result of \code{fRegress} on the observed data

\item[\code{argvals}] argument values for evaluating the F-statistic if \code{yfdPar} is
a functional data object.

\end{ldescription}

normal-bracket70bracket-normal
\end{Value}
\begin{Source}\relax
Ramsay, James O., and Silverman, Bernard W. (2006), \emph{Functional
Data Analysis, 2nd ed.}, Springer, New York.
\end{Source}
\begin{SeeAlso}\relax
\code{\LinkA{fRegress}{fRegress}}
\code{\LinkA{Fstat.fd}{Fstat.fd}}
\end{SeeAlso}

\HeaderA{fRegress.CV}{Computes Cross-validated Error Sum of Squares for a
Functional Regression Model}{fRegress.CV}
\keyword{smooth}{fRegress.CV}
\begin{Description}\relax
For a functional regression model with a scalar dependent variable,
a cross-validated error sum of squares is computed.  This function
aids the choice of smoothing parameters in this model using the
cross-validated error sum of squares criterion.
\end{Description}
\begin{Usage}
\begin{verbatim}
fRegress.CV(yvec, xfdlist, betalist)
\end{verbatim}
\end{Usage}
\begin{Arguments}
\begin{ldescription}
\item[\code{yvec}] a vector of dependent variable values.

\item[\code{xfdlist}] a list whose members are functional parameter objects
specifying functional independent variables.  Some
of these may also be vectors specifying scalar independent
variables.

\item[\code{betalist}] a list containing functional parameter objects specifying the
regression functions and their level of smoothing.

\end{ldescription}
\end{Arguments}
\begin{Value}
the sum of squared errors in predicting \code{yvec}.
\end{Value}
\begin{SeeAlso}\relax
\code{\LinkA{fRegress}{fRegress}}, 
\code{\LinkA{fRegress.stderr}{fRegress.stderr}}
\end{SeeAlso}
\begin{Examples}
\begin{ExampleCode}
#See the analyses of the Canadian daily weather data.
\end{ExampleCode}
\end{Examples}

\HeaderA{fRegress.stderr}{Compute Standard errors of Coefficient Functions
Estimated by Functional Regression
Analysis}{fRegress.stderr}
\keyword{smooth}{fRegress.stderr}
\begin{Description}\relax
Function \code{fRegress} carries out a functional regression analysis
of the concurrent kind, and estimates a regression coefficient function
corresponding to each independent variable, whether it is scalar or
functional.  This function uses the list that is output by \code{fRegress}
to provide standard error functions for each regression function.  These
standard error functions are pointwise, meaning that sampling standard
deviation functions only are computed, and not sampling covariances.
\end{Description}
\begin{Usage}
\begin{verbatim}
fRegress.stderr(fRegressList, y2cMap, SigmaE)
\end{verbatim}
\end{Usage}
\begin{Arguments}
\begin{ldescription}
\item[\code{fRegressList}] the named list of length six that is returned from a call to function
\code{fRegress}.

\item[\code{y2cMap}] a matrix that contains the linear transformation that
takes the raw data values into the coefficients defining a smooth
functional data object. Typically, this matrix is returned from a
call to function \code{smooth.basis} that generates the
dependent variable objects.  If the dependent variable is scalar,
this matrix is an identity matrix of order equal to the length
of the vector.

\item[\code{SigmaE}] either a matrix or a bivariate functional data object
according to whether the dependent variable is scalar or functional,
respectively.
This object has a number of replications equal to
the length of the dependent variable object.  It contains an estimate
of the variance-covariance matrix or function for the residuals.

\end{ldescription}
\end{Arguments}
\begin{Value}
a named list of length 3 containing:

\begin{ldescription}
\item[\code{betastderrlist}] a list object of length
the number of independent variables. Each member contains a
functional parameter object
for the standard error of a regression function.

\item[\code{bvar}] a symmetric matrix containing sampling variances and
covariances for the matrix of regression coefficients
for the regression functions.  These are stored
column-wise in defining BVARIANCE.

\item[\code{c2bMap}] a matrix containing the mapping from response variable
coefficients to coefficients for regression coefficients.

\end{ldescription}
\end{Value}
\begin{SeeAlso}\relax
\code{\LinkA{fRegress}{fRegress}}, 
\code{\LinkA{fRegress.CV}{fRegress.CV}}
\end{SeeAlso}
\begin{Examples}
\begin{ExampleCode}
#See the weather data analyses in the file this-is-escaped-codenormal-bracket29bracket-normal for
#examples of the use of function this-is-escaped-codenormal-bracket30bracket-normal.
\end{ExampleCode}
\end{Examples}

\HeaderA{fRegress}{A Functional Regression Analysis of the Concurrent Type}{fRegress}
\keyword{smooth}{fRegress}
\begin{Description}\relax
This function carries out a functional regression analysis,
where either the dependent variable or one or more
independent variables are functional.  Non-functional variables
may be included on either side of the equation.  In a concurrent
functional linear model all function variables are all evaluated at
a common time or argument value $t$.   That is, the fit is
defined in terms of the behavior of all variables at a fixed time,
or in terms of "now" behavior.
\end{Description}
\begin{Usage}
\begin{verbatim}
fRegress(yfdPar, xfdlist, betalist, wt=rep(1,N))
\end{verbatim}
\end{Usage}
\begin{Arguments}
\begin{ldescription}
\item[\code{yfdPar}] the dependent variable object.  It may be an object of
three possible classes:
\Itemize{
\item a vector if the dependent variable is scalar.
\item a functional data object if the dependent variable is functional.
\item a functional parameter object if the dependent variable is functional,    and if it is necessary to smooth the prediction of the dependent variable.
}

\item[\code{xfdlist}] a list of length equal to the number of independent variables. Members
of this list are the independent variables.  They be objects of either
of these two classes:

\Itemize{
\item a vector if the independent dependent variable is scalar.
\item a functional data object if the dependent variable is functional.
}

In either case, the object must have the same number of replications as
the dependent variable object.  That is, if it is a scalar, it must be
of the same length as the dependent variable, and if it is functional,
it must have the same number of replications as the dependent variable.

\item[\code{betalist}] a list of length equal to the number of independent variables. Members
of this list define the regression functions to be estimated.
They are functional parameter objects.  Note that even if corresponding
independent variable is scalar, its regression coefficient will be
functional if the dependent variable is functional.  Each of these
functional parameter objects defines a single functional data object,
that is, with only one replication.

\item[\code{wt}] weights for weighted least squares 

\end{ldescription}
\end{Arguments}
\begin{Details}\relax
In the computation of regression function estimates, all
independent variables are treated as if they are functional.  If
argument \code{xfdlist} contains one or more vectors, these
are converted to functional data objects having the constant basis
with coefficients equal to the elements of the vector.

Needless to say, if all the variables in the model are scalar,
use this function, but rather either \code{ls} or \code{lsfit}.
\end{Details}
\begin{Value}
a named list of length 6 with these members:

\begin{ldescription}
\item[\code{yfdPar}] yhe first argument in the call to \code{fRegress}.

\item[\code{xfdlist}] the second argument in the call to \code{fRegress}.

\item[\code{betalist}] the third argument in the call to \code{fRegress}.

\item[\code{betaestlist}] a list of length equal to the number of independent
variables and with members having the same functional parameter
structure as the corresponding
members of \code{betalist}.  These are the estimated regression
coefficient functions.

\item[\code{yhatfdobj}] a functional data object if the dependent variable
is functional or a vector if the dependent variable is scalar.  This
is the set of predicted by the functional regression model for the
dependent variable.

\item[\code{Cmatinv}] a matrix containing the inverse of the coefficient
matrix for the linear equations that define the solution to the
regression problem.  This matrix is required for function 
\code{\LinkA{fRegress.stderr}{fRegress.stderr}} that estimates confidence regions for the
regression coefficient function estimates.

\end{ldescription}
\end{Value}
\begin{SeeAlso}\relax
\code{\LinkA{fRegress.stderr}{fRegress.stderr}}, 
\code{\LinkA{fRegress.CV}{fRegress.CV}}, 
\code{\LinkA{linmod}{linmod}}
\end{SeeAlso}
\begin{Examples}
\begin{ExampleCode}
#See the Canadian daily weather data analyses in the file
# this-is-escaped-code{ for 
#examples of all the cases covered by this-is-escaped-codenormal-bracket48bracket-normal.
\end{ExampleCode}
\end{Examples}

\HeaderA{Fstat.fd}{F-statistic for functional linear regression.}{Fstat.fd}
\keyword{smooth}{Fstat.fd}
\begin{Description}\relax
Fstat.fd calculates a pointwise F-statistic for functional linear regression.
\end{Description}
\begin{Usage}
\begin{verbatim} Fstat.fd(y,yhat,argvals=NULL)\end{verbatim}
\end{Usage}
\begin{Arguments}
\begin{ldescription}
\item[\code{y}] the dependent variable object.  It may be:
\Itemize{
\item a vector if the dependent variable is scalar.
\item a functional data object if the dependent variable is functional.
}

\item[\code{yhat}] The predicted values corresponding to \code{y}. It must be of the same class.

\item[\code{argvals}] If \code{yfdPar} is a functional data object, the points at which to evaluate
the pointwise F-statistic.

\end{ldescription}
\end{Arguments}
\begin{Details}\relax
An F-statistic is calculated as the ratio of residual variance to predicted
variance.

If \code{argvals} is not specified and \code{yfdPar} is a \code{fd} object,
it defaults to 101 equally-spaced points on the range of \code{yfdPar}.
\end{Details}
\begin{Value}
A list with components
\begin{ldescription}
\item[\code{F}] the calculated pointwise F-statistics.
\item[\code{argvals}] argument values for evaluating the F-statistic if \code{yfdPar} is
a functional data object.

\end{ldescription}

normal-bracket21bracket-normal
\end{Value}
\begin{Source}\relax
Ramsay, James O., and Silverman, Bernard W. (2006), \emph{Functional
Data Analysis, 2nd ed.}, Springer, New York.
\end{Source}
\begin{SeeAlso}\relax
\code{\LinkA{fRegress}{fRegress}}
\code{\LinkA{Fstat.fd}{Fstat.fd}}
\end{SeeAlso}

\HeaderA{gait}{Hip and knee angle while walking}{gait}
\keyword{datasets}{gait}
\begin{Description}\relax
Hip and knee angle in degrees through a 20 point movement cycle for 39
boys
\end{Description}
\begin{Format}\relax
An array of dim c(20, 39, 2) giving the "Hip Angle" and "Knee Angle"
for 39 repetitions of a 20 point gait cycle.
\end{Format}
\begin{Details}\relax
The components of dimnames(gait) are as follows:

[[1]] standardized gait time = seq(from=0.025, to=0.975, by=0.05) 

[[2]] subject ID = "boy1", "boy2", ..., "boy39"  

[[3]] gait variable = "Hip Angle" or "Knee Angle"
\end{Details}
\begin{Source}\relax
Ramsay, James O., and Silverman, Bernard W. (2006), \emph{Functional
Data Analysis, 2nd ed.}, Springer, New York.

Ramsay, James O., and Silverman, Bernard W. (2002), \emph{Applied
Functional Data Analysis}, Springer, New York.
\end{Source}
\begin{Examples}
\begin{ExampleCode}
plot(gait[,1, 1], gait[, 1, 2], type="b")
\end{ExampleCode}
\end{Examples}

\HeaderA{getbasismatrix}{Values of Basis Functions or their Derivatives}{getbasismatrix}
\keyword{smooth}{getbasismatrix}
\begin{Description}\relax
Evaluate a set of basis functions or their derivatives at
a set of argument values.
\end{Description}
\begin{Usage}
\begin{verbatim}
getbasismatrix(evalarg, basisobj, nderiv=0)
\end{verbatim}
\end{Usage}
\begin{Arguments}
\begin{ldescription}
\item[\code{evalarg}] a vector of arguments values.

\item[\code{basisobj}] a basis object.

\item[\code{nderiv}] a nonnegative integer specifying the derivative to be evaluated.

\end{ldescription}
\end{Arguments}
\begin{Value}
a matrix of basis function or derivative values.  Rows correspond
to argument values and columns to basis functions.
\end{Value}
\begin{SeeAlso}\relax
\code{\LinkA{eval.fd}{eval.fd}}
\end{SeeAlso}
\begin{Examples}
\begin{ExampleCode}
# Minimal example:  a B-spline of order 1, i.e., a step function
# with 0 interior knots:
bspl1.1 <- create.bspline.basis(norder=1, breaks=0:1)
getbasismatrix(seq(0, 1, .2), bspl1.1)

\end{ExampleCode}
\end{Examples}

\HeaderA{getbasispenalty}{Evaluate a Roughness Penalty Matrix}{getbasispenalty}
\keyword{smooth}{getbasispenalty}
\begin{Description}\relax
A basis roughness penalty matrix is the matrix containing
the possible inner products of pairs of basis functions.
These inner products are typically defined in terms of
the value of a derivative or of a linear differential
operator applied to the basis function.  The basis penalty
matrix plays an important role in the computation of
functions whose roughness is controlled by a roughness
penalty.
\end{Description}
\begin{Usage}
\begin{verbatim}
getbasispenalty(basisobj, Lfdobj=NULL)
\end{verbatim}
\end{Usage}
\begin{Arguments}
\begin{ldescription}
\item[\code{basisobj}] a basis object.

\item[\code{Lfdobj}] 
\end{ldescription}
\end{Arguments}
\begin{Details}\relax
A roughness penalty for a function $x(t)$ is defined by
integrating the square of either the derivative of  $ x(t) $ or,
more generally, the result of applying a linear differential operator
$L$ to it.  The most common roughness penalty is the integral of
the square of the second derivative, and
this is the default. To apply this roughness penalty, the matrix of
inner products of the basis functions defining this function is
necessary. This function just calls the roughness penalty evaluation
function specific to the basis involved.
\end{Details}
\begin{Value}
a symmetric matrix of order equal to the number of basis functions
defined by the B-spline basis object.  Each element is the inner product
of two B-spline basis functions after taking the derivative.
\end{Value}
\begin{SeeAlso}\relax
\code{\LinkA{eval.penalty}{eval.penalty}}
\end{SeeAlso}
\begin{Examples}
\begin{ExampleCode}

#  set up a B-spline basis of order 4 with 13 basis functions
#  and knots at 0.0, 0.1,..., 0.9, 1.0.
basisobj <- create.bspline.basis(c(0,1),13)
#  compute the 13 by 13 matrix of inner products of second derivatives
penmat <- getbasispenalty(basisobj)
#  set up a Fourier basis with 13 basis functions
#  and and period 1.0.
basisobj <- create.fourier.basis(c(0,1),13)
#  compute the 13 by 13 matrix of inner products of second derivatives
penmat <- getbasispenalty(basisobj)

\end{ExampleCode}
\end{Examples}

\HeaderA{getbasisrange}{Extract the range from a basis object}{getbasisrange}
\keyword{attribute}{getbasisrange}
\begin{Description}\relax
Extracts the 'range' component from basis object 'basisobj'.
\end{Description}
\begin{Usage}
\begin{verbatim}
getbasisrange(basisobj) 
\end{verbatim}
\end{Usage}
\begin{Arguments}
\begin{ldescription}
\item[\code{basisobj}] a functional basis object 

\end{ldescription}
\end{Arguments}
\begin{Value}
a numeric vector of length 2
\end{Value}

\HeaderA{growth}{Berkeley Growth Study data}{growth}
\keyword{datasets}{growth}
\begin{Description}\relax
A list containing the heights of 39 boys and 54 girls from age 1 to 18
and the ages at which they were collected.
\end{Description}
\begin{Format}\relax
This list contains the following components:
\describe{
\item[hgtm] a 31 by 39 numeric matrix giving the heights in centimeters of
39 boys at 31 ages.    

\item[hgtf] a 31 by 54 numeric matrix giving the heights in centimeters of
54 girls at 31 ages.  

\item[age] a numeric vector of length 31 giving the ages at which the
heights were measured.  

}
\end{Format}
\begin{Details}\relax
The ages are not equally spaced.
\end{Details}
\begin{Source}\relax
Ramsay, James O., and Silverman, Bernard W. (2006), \emph{Functional
Data Analysis, 2nd ed.}, Springer, New York. 

Ramsay, James O., and Silverman, Bernard W. (2002), \emph{Applied
Functional Data Analysis}, Springer, New York, ch. 6. 

Tuddenham, R. D., and Snyder, M. M. (1954) "Physical growth of
California boys and girls from birth to age 18", \emph{University of
California Publications in Child Development}, 1, 183-364.
\end{Source}
\begin{Examples}
\begin{ExampleCode}
with(growth, matplot(age, hgtf[, 1:10], type="b"))
\end{ExampleCode}
\end{Examples}

\HeaderA{handwrit}{Cursive handwriting samples}{handwrit}
\keyword{datasets}{handwrit}
\begin{Description}\relax
20 cursive samples of 1401 (x, y,) coordinates for writing "fda"
\end{Description}
\begin{Format}\relax
An array of dimensions (1401, 20, 2) giving 1401 pairs of (x, y)
coordinates for each of 20 replicates of cursively writing "fda"
\end{Format}
\begin{Details}\relax
These data are the X-Y coordinates of 20 replications of writing
the script "fda".  The subject was Jim Ramsay.  Each replication
is represented by 1401 coordinate values.  The scripts have been 
extensively pre-processed.  They have been adjusted to a common
length that corresponds to 2.3 seconds or 2300 milliseconds, and
they have already been registered so that important features in
each script are aligned.

This analysis is designed to illustrate techniques for working
with functional data having rather high frequency variation and
represented by thousands of data points per record.  Comments
along the way explain the choices of analysis that were made.

The final result of the analysis is a third order linear 
differential equation for each coordinate forced by a 
constant and by time.  The equations are able to reconstruct
the scripts to a fairly high level of accuracy, and are also
able to accommodate a substantial amount of the variation in
the observed scripts across replications.  by contrast, a 
second order equation was found to be completely inadequate.

An interesting suprise in the results is the role placed by
a 120 millisecond cycle such that sharp features such as cusps
correspond closely to this period.  This 110-120 msec cycle
seems is usually seen in human movement data involving rapid
movements, such as speech, juggling and so on.

These 20 records have already been normalized to a common time
interval of 2300 milliseconds and have beeen also registered so that
prominent features occur at the same times across replications.  Time
will be measured in (approximate) milliseconds and space in meters.
The data will require a small amount of smoothing, since an error of
0.5 mm is characteristic of the OPTOTRAK 3D measurement system used to
collect the data.

Milliseconds were chosen as a time scale in order to make the ratio of
the time unit to the inter-knot interval not too far from one.
Otherwise, smoothing parameter values may be extremely small or
extremely large.

The basis functions will be B-splines, with a spline placed at each
knot.  One may question whether so many basis functions are required,
but this decision is found to be essential for stable derivative
estimation up to the third order at and near the boundaries.  

Order 7 was used to get a smooth third derivative, which requires
penalizing the size of the 5th derivative, which in turn requires an
order of at least 7.  This implies norder + no. of interior knots =
1399 + 7 = 1406 basis functions.  

The smoothing parameter value 1e8 was chosen to obtain a fitting error
of about 0.5 mm, the known error level in the OPTOTRACK equipment.
\end{Details}
\begin{Source}\relax
Ramsay, James O., and Silverman, Bernard W. (2006), \emph{Functional
Data Analysis, 2nd ed.}, Springer, New York.
\end{Source}
\begin{Examples}
\begin{ExampleCode}
plot(handwrit[, 1, 1], handwrit[, 1, 2], type="l")
\end{ExampleCode}
\end{Examples}

\HeaderA{inprod.bspline}{Compute Inner Products B-spline Expansions.}{inprod.bspline}
\keyword{smooth}{inprod.bspline}
\begin{Description}\relax
Computes the matrix of inner products when both functions
are represented by B-spline expansions and when both
derivatives are integers.  This function is called by function
\code{inprod}, and is not normally used directly.
\end{Description}
\begin{Usage}
\begin{verbatim}
inprod.bspline(fdobj1, fdobj2=fdobj1, nderiv1=0, nderiv2=0)
\end{verbatim}
\end{Usage}
\begin{Arguments}
\begin{ldescription}
\item[\code{fdobj1}] a functional data object having a B-spline basis function
expansion.

\item[\code{fdobj2}] a second functional data object with a B-spline basis
function expansion.  By default, this is the same as
the first argument.

\item[\code{nderiv1}] a nonnegative integer specifying the derivative for the
first argument.

\item[\code{nderiv2}] a nonnegative integer specifying the derivative for the
second argument.

\end{ldescription}
\end{Arguments}
\begin{Value}
a matrix of inner products with number of rows equal
to the number of replications of the first argument and
number of columns equal to the number of replications
of the second object.
\end{Value}

\HeaderA{inprod}{Inner products of Functional Data Objects.}{inprod}
\keyword{smooth}{inprod}
\begin{Description}\relax
Computes a matrix of inner products for each pairing of a
replicate for the first argument with a replicate for the
second argument.  This is perhaps the most important function
in the functional data library.  Hardly any analysis fails
to use inner products in some way, and many employ multiple
inner products.  While in certain cases
these may be computed exactly, this is a more general function that
approximates the inner product approximately when required.
The inner product is defined by two derivatives or linear
differential operators that are applied to the
first two arguments.  The range used to compute the inner
product may be contained within the range over which the
functions are defined.  A weight functional data object may
also be used to define weights for the inner product.
\end{Description}
\begin{Usage}
\begin{verbatim}
inprod(fdobj1, fdobj2,
       Lfdobj1=int2Lfd(0), Lfdobj2=int2Lfd(0),
       rng = range1, wtfd = 0)
\end{verbatim}
\end{Usage}
\begin{Arguments}
\begin{ldescription}
\item[\code{fdobj1}] a functional data object or a basis object.  If the object is
of the basis class, it is converted to a functional data object
by using the identity matrix as the coefficient matrix.

\item[\code{fdobj2}] a functional data object or a basis object.  If the object is
of the basis class, it is converted to a functional data object
by using the identity matrix as the coefficient matrix.

\item[\code{Lfdobj1}] either a nonnegative integer specifying the derivative of
the first argument to be used, or a linear differential operator
object to be applied to the first argument.

\item[\code{Lfdobj2}] either a nonnegative integer specifying the derivative of
the second argument to be used, or a linear differential operator
object to be applied to the second argument.

\item[\code{rng}] a vector of length 2 defining a restricted range contained
within the range over which the arguments are defined.

\item[\code{wtfd}] a univariate functional data object with a single replicate
defining weights to be used in computing the inner product.

\end{ldescription}
\end{Arguments}
\begin{Details}\relax
The approximation method is Richardson extrapolation using numerical
integration by the trapezoidal rule.  At each iteration, the number of
values at which the functions are evaluated is doubled, and a polynomial
extrapolation method is used to estimate the converged integral values
as well as an error tolerance.  Convergence is declared when the
relative error falls below \code{EPS} for all products.  The
extrapolation method generally saves at least one and often two
iterations relative to un-extrapolated trapezoidal integration.
Functional data analyses will seldom need to use \code{inprod}
directly, but code developers should be aware of its pivotal role.
Future work may require more sophisticated and specialized numerical
integration methods.
\code{inprod} computes the definite integral, but some functions
such as \code{smooth.monotone} and \code{register.fd} also need to
compute indefinite integrals.  These use the same approximation scheme,
but usually require more accuracy, and hence more iterations.
When one or both arguments are basis objects, they are converted to
functional data objects using identity matrices as the coefficient
matrices.
\code{inprod} is only called when there is no faster or exact
method available.  In cases where there is, it has been found that the
approximation is good to about four to five significant digits, which is
sufficient for most applications.  Perhaps surprisingly, in the case of
B-splines, the exact method is not appreciably faster, but of course is
more accurate.
\code{inprod} calls function \code{eval.fd} perhaps thousands
of times, so high efficiency for this function and the functions that
it calls is important.
\end{Details}
\begin{Value}
a matrix of inner products.  The number of rows is the number of
functions or basis functions in argument \code{fd1}, and the number of
columns is the same thing for argument \code{fd2}.
\end{Value}
\begin{Section}{References}
Press, et, al, $Numerical Recipes$.
\end{Section}
\begin{SeeAlso}\relax
\code{\LinkA{eval.penalty}{eval.penalty}},
\end{SeeAlso}

\HeaderA{int2Lfd}{Convert Integer to Linear Differential Operator}{int2Lfd}
\keyword{smooth}{int2Lfd}
\begin{Description}\relax
This function turns an integer specifying an order of a derivative into the
equivalent linear differential operator object.  It is also useful for
checking that an object is of the "Lfd" class.
\end{Description}
\begin{Usage}
\begin{verbatim}
int2Lfd(m=0)
\end{verbatim}
\end{Usage}
\begin{Arguments}
\begin{ldescription}
\item[\code{m}] either a nonnegative integer or a linear differential operator object.

\end{ldescription}
\end{Arguments}
\begin{Value}
a linear differential operator object of the "Lfd" class that is
equivalent to the integer argument.
\end{Value}

\HeaderA{intensity.fd}{Intensity Function for Point Process}{intensity.fd}
\keyword{smooth}{intensity.fd}
\begin{Description}\relax
The intensity $mu$ of a series of event times that obey a
homogeneous Poisson process is the mean number of events per unit time.
When this event rate varies over time, the process is said to be
nonhomogeneous, and $mu(t)$, and is estimated by this function
\code{intensity.fd}.
\end{Description}
\begin{Usage}
\begin{verbatim}
intensity.fd(x, WfdParobj, conv=0.0001, iterlim=20,
             dbglev=1)
\end{verbatim}
\end{Usage}
\begin{Arguments}
\begin{ldescription}
\item[\code{x}] a vector containing a strictly increasing series of event times.
These event times assume that the the events begin to be observed
at time 0, and therefore are times since the beginning of
observation.

\item[\code{WfdParobj}] a functional parameter object estimating the log-intensity function
$W(t) = log[mu(t)]$ .
Because the intensity function $mu(t)$ is necessarily positive,
it is represented by \code{mu(x) = exp[W(x)]}.

\item[\code{conv}] a convergence criterion, required because the estimation
process is iterative.

\item[\code{iterlim}] maximum number of iterations that are allowed.

\item[\code{dbglev}] either 0, 1, or 2.  This controls the amount information printed out on
each iteration, with 0 implying no output, 1 intermediate output level,
and 2 full output.  If levels 1 and 2 are used, turn off the output
buffering option.

\end{ldescription}
\end{Arguments}
\begin{Details}\relax
The intensity function $I(t)$ is almost the same thing as a
probability density function $p(t)$ estimated by function
\code{densify.fd}.  The only difference is the absence of
the normalizing constant $C$ that a density function requires
in order to have a unit integral.
The goal of the function is provide a smooth intensity function
estimate that approaches some target intensity by an amount that is
controlled by the linear differential operator \code{Lfdobj} and
the penalty parameter in argument \code{WfdPar}.
For example, if the first derivative of
$W(t)$ is penalized heavily, this will force the function to
approach a constant, which in turn will force the estimated Poisson
process itself to be nearly homogeneous.
To plot the intensity function or to evaluate it,
evaluate \code{Wfdobj}, exponentiate the resulting vector.
\end{Details}
\begin{Value}
a named list of length 4 containing:

\begin{ldescription}
\item[\code{Wfdobj}] a functional data object defining function $W(x)$ that that
optimizes the fit to the data of the monotone function that it defines.

\item[\code{Flist}] a named list containing three results for the final converged solution:
(1)
\bold{f}: the optimal function value being minimized,
(2)
\bold{grad}: the gradient vector at the optimal solution,   and
(3)
\bold{norm}: the norm of the gradient vector at the optimal solution.

\item[\code{iternum}] the number of iterations.

\item[\code{iterhist}] a \code{iternum+1} by 5 matrix containing the iteration
history.

\end{ldescription}
\end{Value}
\begin{SeeAlso}\relax
\code{\LinkA{density.fd}{density.fd}}
\end{SeeAlso}
\begin{Examples}
\begin{ExampleCode}

#  Generate 101 Poisson-distributed event times with
#  intensity or rate two events per unit time
N  <- 101
mu <- 2
#  generate 101 uniform deviates
uvec <- runif(rep(0,N))
#  convert to 101 exponential waiting times
wvec <- -log(1-uvec)/mu
#  accumulate to get event times
tvec <- cumsum(wvec)
tmax <- max(tvec)
#  set up an order 4 B-spline basis over [0,tmax] with
#  21 equally spaced knots
tbasis <- create.bspline.basis(c(0,tmax), 23)
#  set up a functional parameter object for W(t),
#  the log intensity function.  The first derivative
#  is penalized in order to smooth toward a constant
lambda <- 10
WfdParobj <- fdPar(tbasis, 1, lambda)
#  estimate the intensity function
Wfdobj <- intensity.fd(tvec, WfdParobj)$Wfdobj
#  get intensity function values at 0 and event times
events <- c(0,tvec)
intenvec <- exp(eval.fd(events,Wfdobj))
#  plot intensity function
plot(events, intenvec, type="b")
lines(c(0,tmax),c(mu,mu),lty=4)

\end{ExampleCode}
\end{Examples}

\HeaderA{is.basis}{Confirm Object is Class "Basisfd"}{is.basis}
\keyword{smooth}{is.basis}
\begin{Description}\relax
Check that an argument is a basis object.
\end{Description}
\begin{Usage}
\begin{verbatim}
is.basis(basisobj)
\end{verbatim}
\end{Usage}
\begin{Arguments}
\begin{ldescription}
\item[\code{basisobj}] an object to be checked.

\end{ldescription}
\end{Arguments}
\begin{Value}
a logical value:
\code{TRUE} if the class is correct, \code{FALSE} otherwise.
\end{Value}
\begin{SeeAlso}\relax
\code{\LinkA{is.fd}{is.fd}}, 
\code{\LinkA{is.fdPar}{is.fdPar}}, 
\code{\LinkA{is.Lfd}{is.Lfd}}
\end{SeeAlso}

\HeaderA{is.fd}{Confirm Object has Class "fd"}{is.fd}
\keyword{smooth}{is.fd}
\begin{Description}\relax
Check that an argument is a functional data object.
\end{Description}
\begin{Usage}
\begin{verbatim}
is.fd(fdobj)
\end{verbatim}
\end{Usage}
\begin{Arguments}
\begin{ldescription}
\item[\code{fdobj}] an object to be checked.

\end{ldescription}
\end{Arguments}
\begin{Value}
a logical value:
\code{TRUE} if the class is correct, \code{FALSE} otherwise.
\end{Value}
\begin{SeeAlso}\relax
\code{\LinkA{is.basis}{is.basis}}, 
\code{\LinkA{is.fdPar}{is.fdPar}}, 
\code{\LinkA{is.Lfd}{is.Lfd}}
\end{SeeAlso}

\HeaderA{is.fdPar}{Confirm Object has Class "fdPar"}{is.fdPar}
\keyword{smooth}{is.fdPar}
\begin{Description}\relax
Check that an argument is a functional parameter object.
\end{Description}
\begin{Usage}
\begin{verbatim}
is.fdPar(fdParobj)
\end{verbatim}
\end{Usage}
\begin{Arguments}
\begin{ldescription}
\item[\code{fdParobj}] an object to be checked.

\end{ldescription}
\end{Arguments}
\begin{Value}
a logical value:
\code{TRUE} if the class is correct, \code{FALSE} otherwise.
\end{Value}
\begin{SeeAlso}\relax
\code{\LinkA{is.basis}{is.basis}}, 
\code{\LinkA{is.fd}{is.fd}}, 
\code{\LinkA{is.Lfd}{is.Lfd}}
\end{SeeAlso}

\HeaderA{is.Lfd}{Confirm Object has Class "Lfd"}{is.Lfd}
\keyword{smooth}{is.Lfd}
\begin{Description}\relax
Check that an argument is a linear differential operator object.
\end{Description}
\begin{Usage}
\begin{verbatim}
is.Lfd(Lfdobj)
\end{verbatim}
\end{Usage}
\begin{Arguments}
\begin{ldescription}
\item[\code{Lfdobj}] an object to be checked.

\end{ldescription}
\end{Arguments}
\begin{Value}
a logical value:
\code{TRUE} if the class is correct, \code{FALSE} otherwise.
\end{Value}
\begin{SeeAlso}\relax
\code{\LinkA{is.basis}{is.basis}}, 
\code{\LinkA{is.fd}{is.fd}}, 
\code{\LinkA{is.fdPar}{is.fdPar}}
\end{SeeAlso}

\HeaderA{knots.fd}{Extract the knots from a function basis or data object}{knots.fd}
\aliasA{knots.basisfd}{knots.fd}{knots.basisfd}
\aliasA{knots.fdSmooth}{knots.fd}{knots.fdSmooth}
\keyword{smooth}{knots.fd}
\keyword{optimize}{knots.fd}
\begin{Description}\relax
Extract either all or only the interior knots from an object of class
\code{basisfd}, \code{fd}, or \code{fdSmooth}.
\end{Description}
\begin{Usage}
\begin{verbatim}
## S3 method for class 'fd':
knots(Fn, interior=TRUE, ...)
## S3 method for class 'fdSmooth':
knots(Fn, interior=TRUE, ...)
## S3 method for class 'basisfd':
knots(Fn, interior=TRUE, ...)
\end{verbatim}
\end{Usage}
\begin{Arguments}
\begin{ldescription}
\item[\code{Fn}] an object of class \code{basisfd} or containing such an object 

\item[\code{interior}] logical:

if TRUE, the first Fn[["k"]]+1 of Fn[["knots"]] are dropped, and the
next Fn[["g"]] are returned.

Otherwise, the first Fn[["n"]] of Fn[["knots"]] are returned.  

\item[\code{...}] ignored
\end{ldescription}
\end{Arguments}
\begin{Details}\relax
The interior knots of a \code{bspline} basis are stored in the
\code{params} component.  The remaining knots are in the
\code{rangeval} component, with mulltiplicity norder(Fn).
\end{Details}
\begin{Value}
Numeric vector.  If 'interior' is TRUE, this is the \code{params}
component of the \code{bspline} basis.  Otherwise, \code{params} is
bracketed by rep(rangeval, norder(basisfd).
\end{Value}
\begin{Author}\relax
Spencer Graves
\end{Author}
\begin{References}\relax
Dierckx, P. (1991) \emph{Curve and Surface Fitting with Splines},
Oxford Science Publications.
\end{References}
\begin{SeeAlso}\relax
\code{\LinkA{fd}{fd}},
\code{\LinkA{create.bspline.basis}{create.bspline.basis}},
\code{\LinkA{knots.dierckx}{knots.dierckx}}
\end{SeeAlso}
\begin{Examples}
\begin{ExampleCode}
x <- 0:24
y <- c(1.0,1.0,1.4,1.1,1.0,1.0,4.0,9.0,13.0,
       13.4,12.8,13.1,13.0,14.0,13.0,13.5,
       10.0,2.0,3.0,2.5,2.5,2.5,3.0,4.0,3.5)
if(require(DierckxSpline)){
   z1 <- curfit(x, y, method = "ss", s = 0, k = 3)
   knots1 <- knots(z1)
   knots1All <- knots(z1, interior=FALSE) # to see all knots
#
   fda1 <- dierckx2fd(z1)
   fdaKnots <- knots(fda1)
   fdaKnotsA <- knots(fda1, interior=FALSE)
   stopifnot(all.equal(knots1, fdaKnots))
   stopifnot(all.equal(knots1All, fdaKnotsA))
}

# knots.fdSmooth 
girlGrowthSm <- with(growth, smooth.basisPar(argvals=age, y=hgtf))

girlKnots.fdSm <- knots(girlGrowthSm) 
girlKnots.fdSmA <- knots(girlGrowthSm, interior=FALSE)
stopifnot(all.equal(girlKnots.fdSm, girlKnots.fdSmA[5:33]))

girlKnots.fd <- knots(girlGrowthSm$fd) 
girlKnots.fdA <- knots(girlGrowthSm$fd, interior=FALSE)

stopifnot(all.equal(girlKnots.fdSm, girlKnots.fd))
stopifnot(all.equal(girlKnots.fdSmA, girlKnots.fdA))

\end{ExampleCode}
\end{Examples}

\HeaderA{lambda2df}{Convert Smoothing Parameter to Degrees of Freedom}{lambda2df}
\keyword{smooth}{lambda2df}
\begin{Description}\relax
The degree of roughness of an estimated function is controlled by a
smoothing parameter $lambda$ that directly multiplies the penalty.
However, it can be difficult to interpret or choose this value, and it
is often easier to determine the roughness by choosing a value that is
equivalent of the degrees of freedom used by the smoothing procedure.
This function converts a multipler $lambda$ into a degrees of freedom value.
\end{Description}
\begin{Usage}
\begin{verbatim}
lambda2df(argvals, basisobj, wtvec=rep(1, n),
          Lfdobj=NULL, lambda=0)
\end{verbatim}
\end{Usage}
\begin{Arguments}
\begin{ldescription}
\item[\code{argvals}] a vector containing the argument values used in the
smooth of the data.

\item[\code{basisobj}] the basis object used in the smoothing of the data.

\item[\code{wtvec}] the weight vector, if any, that was used in the smoothing
of the data.

\item[\code{Lfdobj}] the linear differential operator object used to defining
the roughness penalty employed in smoothing the data.

\item[\code{lambda}] the smoothing parameter to be converted.

\end{ldescription}
\end{Arguments}
\begin{Value}
the equivalent degrees of freedom value.
\end{Value}
\begin{SeeAlso}\relax
\code{\LinkA{df2lambda}{df2lambda}}
\end{SeeAlso}

\HeaderA{lambda2gcv}{Compute GCV Criterion}{lambda2gcv}
\keyword{smooth}{lambda2gcv}
\begin{Description}\relax
The generalized cross-validation or GCV criterion is often
used to select an appropriate smoothing parameter value, by
finding the smoothing parameter that minimizes GCV.  This
function locates that value.
\end{Description}
\begin{Usage}
\begin{verbatim}
lambda2gcv(log10lambda, argvals, y, fdParobj,
           wtvec=rep(1, length(y)))
\end{verbatim}
\end{Usage}
\begin{Arguments}
\begin{ldescription}
\item[\code{log10lambda}] the logarithm (base 10) of the smoothing parameter

\item[\code{argvals}] a vector of argument values.

\item[\code{y}] the data to be smoothed.

\item[\code{fdParobj}] a functional parameter object defining the smooth.

\item[\code{wtvec}] a weight vector used in the smoothing.

\end{ldescription}
\end{Arguments}
\begin{Value}
the value of the GCV criterion.
\end{Value}

\HeaderA{landmarkreg}{Landmark Registration of Functional Observations}{landmarkreg}
\keyword{smooth}{landmarkreg}
\begin{Description}\relax
It is common to see that among a set of functions certain prominent
features such peaks and valleys, called $landmarks$, do not occur
at the same times, or other
argument values.  This is called $phase variation$, and it can be
essential to align these features before proceeding with further
functional data analyses.  This function uses the timings of these
features to align or register the curves.  The registration involves
estimating a nonlinear transformation of the argument continuum for each
functional observation.  This transformation is called a warping
function. It must be strictly increasing and smooth.
\end{Description}
\begin{Usage}
\begin{verbatim}
landmarkreg(fdobj, ximarks, x0marks=xmeanmarks,
            WfdPar=fdPar(defbasis), monwrd=FALSE)
\end{verbatim}
\end{Usage}
\begin{Arguments}
\begin{ldescription}
\item[\code{fdobj}] a functional data object containing the curves to be registered.

\item[\code{ximarks}] a matrix containing the timings or argument values associated with the
landmarks for the observations in \code{fd} to be registered.  The
number of rows N
equals the number of observations, and the number of columns NL equals the
number of landmarks. These landmark times must be in the interior of the
interval over which the functions are defined.

\item[\code{x0marks}] a vector of length NL of times of landmarks for target curve.  If
not supplied, the mean of the landmark times in \code{ximarks}
is used.

\item[\code{WfdPar}] a functional parameter object defining the warping functions
that transform time in order to register the curves.

\item[\code{monwrd}] A logical value:
if \code{TRUE}, the warping function is estimated using a monotone
smoothing methhod; otherwise, a regular smoothing method is used,
which is not guaranteed to give strictly monotonic warping
functions.

\end{ldescription}
\end{Arguments}
\begin{Details}\relax
It is essential that the location of every landmark be clearly
defined
in each of the curves as well as the template function.  If this is not
the case, consider using the continuous registration function
\code{register.fd}.
Although requiring that a monotone smoother be used to estimate the
warping functions is safer, it adds considerably to the computatation
time since monotone smoothing is itself an iterative process.  It is
usually better to try an initial registration with this feature to see
if there are any failures of monotonicity.  Moreover, monotonicity
failures can usually be cured by increasing the smoothing parameter
defining \code{WfdPar.}
Not much curvature is usually required in the warping functions, so
a rather low power basis, usually B-splines, is suitable for
defining the functional paramter argument \code{WfdPar.}
A registration with a few prominent landmarks is often a good
preliminary to using the more sophisticated but more lengthy process in
\code{register.fd.}
\end{Details}
\begin{Value}
a names list of length 2 with components:

\begin{ldescription}
\item[\code{fdreg}] a functional data object for the registered curves.

\item[\code{warpfd}] a functional data object for the warping functions.

\end{ldescription}
\end{Value}
\begin{SeeAlso}\relax
\code{\LinkA{register.fd}{register.fd}}, 
\code{\LinkA{smooth.morph}{smooth.morph}}
\end{SeeAlso}
\begin{Examples}
\begin{ExampleCode}
#See the analysis for the lip data in the examples.
\end{ExampleCode}
\end{Examples}

\HeaderA{Lfd}{Define a Linear Differential Operator Object}{Lfd}
\keyword{smooth}{Lfd}
\begin{Description}\relax
A linear differential operator of order $m$ is defined,
usually to specify a roughness penalty.
\end{Description}
\begin{Usage}
\begin{verbatim}
Lfd(nderiv=0, bwtlist=vector("list", 0))
\end{verbatim}
\end{Usage}
\begin{Arguments}
\begin{ldescription}
\item[\code{nderiv}] a nonnegative integer specifying the order $m$ of the
highest order derivative in the operator

\item[\code{bwtlist}] a list of length $m$.  Each member contains a
functional data object that acts as a weight function for a
derivative.  The first member weights the function, the
second the first derivative, and so on up to order $m-1$.

\end{ldescription}
\end{Arguments}
\begin{Details}\relax
To check that an object is of this class, use functions
\code{is.Lfd} or \code{int2Lfd}.

Linear differential operator objects are often used to
define roughness penalties for smoothing towards a
"hypersmooth" function that is annihilated by the operator.
For example, the harmonic acceleration operator used in the
analysis of the Canadian daily weather data annihilates linear
combinations of $1, sin(2 pi t/365)$ and $cos(2 pi t/365)$,
and the larger the smoothing parameter, the closer the smooth
function will be to a function of this shape.

Function \code{pda.fd} estimates a linear differential
operator object that comes as close as possible to annihilating
a functional data object.

A linear differential operator of order $m$ is a
linear combination of the derivatives of a functional
data object up to order $m$.  The derivatives of
orders 0, 1, ..., $m-1$ can each be multiplied
by a weight function $b(t)$ that may or may not vary with
argument $t$.

If the notation $D^j$ is taken to
mean "take the derivative of order $j$", then a linear
differental operator $L$ applied to function $x$
has the expression

$Lx(t) = b_0(t) x(t) + b_1(t)Dx(t) + ... + b_\{m-1\}(t) D^\{m-1\} x(t)
+ D^mx(t)$
\end{Details}
\begin{Value}
a linear differential operator object
\end{Value}
\begin{SeeAlso}\relax
\code{\LinkA{int2Lfd}{int2Lfd}}, 
\code{\LinkA{vec2Lfd}{vec2Lfd}}, 
\code{\LinkA{fdPar}{fdPar}}, 
\code{\LinkA{pda.fd}{pda.fd}}
\end{SeeAlso}
\begin{Examples}
\begin{ExampleCode}
#  Set up the harmonic acceleration operator
dayrange  <- c(0,365)
Lbasis  <- create.constant.basis(dayrange)
Lcoef   <- matrix(c(0,(2*pi/365)^2,0),1,3)
bfdobj  <- fd(Lcoef,Lbasis)
bwtlist <- fd2list(bfdobj)
harmaccelLfd <- Lfd(3, bwtlist)
\end{ExampleCode}
\end{Examples}

\HeaderA{lines.fd}{Add Lines from Functional Data to a Plot}{lines.fd}
\aliasA{lines.fdSmooth}{lines.fd}{lines.fdSmooth}
\keyword{smooth}{lines.fd}
\begin{Description}\relax
Lines defined by functional observations are added to an existing plot.
\end{Description}
\begin{Usage}
\begin{verbatim}
## S3 method for class 'fd':
lines(x, Lfdobj=int2Lfd(0), nx=201, ...)
## S3 method for class 'fdSmooth':
lines(x, Lfdobj=int2Lfd(0), nx=201, ...)
\end{verbatim}
\end{Usage}
\begin{Arguments}
\begin{ldescription}
\item[\code{x}] a univariate functional data object to be evaluated at \code{nx}
points over \code{xlim} and added as a line to an existing plot.  

\item[\code{Lfdobj}] either a nonnegative integer or a linear differential operator
object.  If present, the derivative or the value of applying the
operator is evaluated rather than the functions themselves.

\item[\code{nx}] Number of points within \code{xlim} at which to evaluate \code{x}
for plotting.  

\item[\code{... }] additional arguments such as axis titles and so forth that can be
used in plotting programs called by \code{lines.fd} or
\code{lines.fdSmooth}.  

\end{ldescription}
\end{Arguments}
\begin{Section}{Side Effects}
Lines added to an existing plot.
\end{Section}
\begin{SeeAlso}\relax
\code{\LinkA{plot.fd}{plot.fd}}, 
\code{\LinkA{plotfit.fd}{plotfit.fd}}
\end{SeeAlso}
\begin{Examples}
\begin{ExampleCode}
##
## plot a fit with 3 levels of smoothing
##
x <- seq(-1,1,0.02)
y <- x + 3*exp(-6*x^2) + sin(1:101)/2
# sin not rnorm to make it easier to compare
# results across platforms 

result4. <- smooth.basisPar(argvals=x, y=y, lambda=1)
result4.4 <- smooth.basisPar(argvals=x, y=y, lambda=1e-4)
result4.0 <- smooth.basisPar(x, y, lambda=0)

plot(x, y)
lines(result4.)
lines(result4.4, col='green')
lines.fdSmooth(result4.0, col='red') 

plot(x, y, xlim=c(0.5, 1))
lines.fdSmooth(result4.)
lines.fdSmooth(result4.4, col='green')
lines.fdSmooth(result4.0, col='red')  
lines.fdSmooth(result4.0, col='red', nx=101)
# no visible difference from the default?  

lines.fdSmooth(result4.0, col='orange', nx=31)
# Clear difference, especially between 0.95 and 1  

\end{ExampleCode}
\end{Examples}

\HeaderA{linmod}{Fit Fully Functional Linear Model}{linmod}
\keyword{smooth}{linmod}
\begin{Description}\relax
A functional dependent variable is approximated by a single
functional covariate, and the
covariate can affect the dependent variable for all
values of its argument.  The regression function is a bivariate function.
\end{Description}
\begin{Usage}
\begin{verbatim}
linmod(xfdobj, yfdobj, wtvec=rep(1,nrep),
       xLfdobj=int2Lfd(2), yLfdobj=int2Lfd(2),
       xlambda=0, ylambda=0)
\end{verbatim}
\end{Usage}
\begin{Arguments}
\begin{ldescription}
\item[\code{xfdobj}] a functional data object for the covariate

\item[\code{yfdobj}] a functional data object for the dependent variable

\item[\code{wtvec}] a vector of weights for each observation.

\item[\code{xLfdobj}] either a nonnegative integer or a linear differential operator
object.  This operator is applied to the regression function's
first argument.

\item[\code{yLfdobj}] either a nonnegative integer or a linear differential operator
object.  This operator is applied to the regression function's
second argument.

\item[\code{xlambda}] a smoothing parameter for the first argument of the regression
function.

\item[\code{ylambda}] a smoothing parameter for the second argument of the regression
function.

\end{ldescription}
\end{Arguments}
\begin{Value}
a named list of length 3 with the following entries:

\begin{ldescription}
\item[\code{alphafd}] the intercept functional data object.

\item[\code{regfd}] a bivariate functional data object for the regression function.

\item[\code{yhatfd}] a functional data object for the approximation to the dependent variable
defined by the linear model, if the dependent variable is functional.
Otherwise the matrix of approximate values.

\end{ldescription}
\end{Value}
\begin{SeeAlso}\relax
\code{\LinkA{fRegress}{fRegress}}
\end{SeeAlso}
\begin{Examples}
\begin{ExampleCode}
#See the prediction of precipitation using temperature as
#the independent variable in the analysis of the daily weather
#data.
\end{ExampleCode}
\end{Examples}

\HeaderA{lip}{Lip motion}{lip}
\aliasA{lipmarks}{lip}{lipmarks}
\aliasA{liptime}{lip}{liptime}
\keyword{datasets}{lip}
\begin{Description}\relax
51 measurements of the position of the lower lip every 7 milliseconds
for 20 repitions of the syllable 'bob'.
\end{Description}
\begin{Usage}
\begin{verbatim}
lip
lipmarks 
liptime
\end{verbatim}
\end{Usage}
\begin{Format}\relax
\item[lip] a matrix of dimension c(51, 20) giving the position of the lower
lip every 7 milliseconds for 350 miliseconds.

\item[lipmarks] a matrix of dimension c(20, 2) giving the positions of the
'leftElbow' and 'rightElbow' in each of the 20 repitions of the
syllable 'bob'.  

\item[liptime] time in seconds from the start = seq(0, 0.35, 51) = every 7
milliseconds.  
\end{Format}
\begin{Details}\relax
These are rather simple data, involving the movement of the lower lip
while saying "bob".  There are 20 replications and 51 sampling points.
The data are used to illustrate two techniques:  landmark registration
and principal differental analysis.  
Principal differential analysis estimates a linear differential equation
that can be used to describe not only the observed curves, but also a 
certain number of their derivatives.  
For a rather more elaborate example of principal differential analysis, 
see the handwriting data.

See the \code{lip} \code{demo}.
\end{Details}
\begin{Source}\relax
Ramsay, James O., and Silverman, Bernard W. (2006), \emph{Functional
Data Analysis, 2nd ed.}, Springer, New York, sections 19.2 and
19.3.
\end{Source}
\begin{Examples}
\begin{ExampleCode}
#  See the this-is-escaped-codenormal-bracket21bracket-normal this-is-escaped-codenormal-bracket22bracket-normal.  
\end{ExampleCode}
\end{Examples}

\HeaderA{lmeWinsor}{Winsorized Regression with mixed effects}{lmeWinsor}
\keyword{models}{lmeWinsor}
\begin{Description}\relax
Clip inputs and mixed-effects predictions to (upper, lower) or to
selected quantiles to limit wild predictions outside the training
set.
\end{Description}
\begin{Usage}
\begin{verbatim}
  lmeWinsor(fixed, data, random, lower=NULL, upper=NULL, trim=0,
        quantileType=7, correlation, weights, subset, method,
        na.action, control, contrasts = NULL, keep.data=TRUE, 
        ...)
\end{verbatim}
\end{Usage}
\begin{Arguments}
\begin{ldescription}
\item[\code{fixed}] a two-sided linear formula object describing the fixed-effects part
of the model, with the response on the left of a '~' operator and
the terms, separated by '+' operators, on the right.  The left hand
side of 'formula' must be a single vector in 'data', untransformed.

\item[\code{data}] an optional data frame containing the variables named in 'fixed',
'random', 'correlation', 'weights', and 'subset'.  By default the
variables are taken from the environment from which \LinkA{lme}{lme}
is  called. 

\item[\code{random}] a random- / mixed-effects specification, as described with
\LinkA{lme}{lme}.

NOTE:  Unlike \LinkA{lme}{lme}, 'random' must be provided;  it can
not be inferred from 'data'.  

\item[\code{lower, upper}] optional numeric vectors with names matching columns of 'data'
giving limits on the ranges of predictors and predictions:  If
present, values below 'lower' will be increased to 'lower', and
values above 'upper' will be decreased to 'upper'.  If absent, these
limit(s) will be inferred from quantile(..., prob=c(trim, 1-trim),
na.rm=TRUE, type=quantileType).  

\item[\code{trim}] the fraction (0 to 0.5) of observations to be considered outside the
range of the data in determining limits not specified in 'lower' and
'upper'.  

NOTES:

(1) trim>0 with a singular fit may give an error.  In such cases,
fix the singularity and retry.

(2) trim = 0.5 should NOT be used except to check the algorithm,
because it trims everything to the median, thereby providing zero
leverage for estimating a regression.

(3) The current algorithm does does NOT adjust any of the variance
parameter estimates to account for predictions outside 'lower' and
'upper'.  This will have no effect for trim = 0 or trim otherwise so
small that there are not predictions outside 'lower' and 'upper'.
However, for more substantive trimming, this could be an issue.
This is different from \LinkA{lmWinsor}{lmWinsor}. 

\item[\code{quantileType}] an integer between 1 and 9 selecting one of the nine quantile
algorithms to be used with 'trim' to determine limits not provided
with 'lower' and 'upper'.  

\item[\code{correlation}] an optional correlation structure, as described with
\LinkA{lme}{lme}. 

\item[\code{weights}] an optional heteroscedasticity structure, as described with
\LinkA{lme}{lme}. 

\item[\code{ subset }] an optional vector specifying a subset of observations to be used in
the fitting process, as described with \LinkA{lme}{lme}.  

\item[\code{method}] a character string.  If '"REML"' the model is fit by maximizing the
restricted log-likelihood.  If '"ML"' the log-likelihood is
maximized.  Defaults to '"REML"'. 

\item[\code{ na.action }] a function that indicates what should happen when the data contain
'NA's.  The default action ('na.fail') causes 'lme' to print an
error message and terminate if there are any incomplete
observations. 

\item[\code{control}] a list of control values for the estimation algorithm to replace the
default values returned by the function
\LinkA{lmeControl}{lmeControl}. Defaults to an empty list.

NOTE:  Other control parameters such as 'singular.ok' as documented
in \LinkA{glsControl}{glsControl} may also work, but should be used with
caution.  

\item[\code{ contrasts }] an optional list. See the 'contrasts.arg' of
'model.matrix.default'. 

\item[\code{keep.data}] logical: should the 'data' argument (if supplied and a data frame)
be saved as part of the model object? 

\item[\code{...}] additional arguments to be passed to the low level regression
fitting functions;  see \LinkA{lm}{lm}. 

\end{ldescription}
\end{Arguments}
\begin{Details}\relax
1.  Identify inputs and outputs as follows:

1.1.  mdly <- mdlx <- fixed;  mdly[[3]] <- NULL;  mdlx[[2]] <- NULL;

1.2.  xNames <- c(all.vars(mdlx), all.vars(random)).

1.3.  yNames <- all.vars(mdly).  Give an error if
as.character(mdly[[2]]) != yNames. 

2.  Do 'lower' and 'upper' contain limits for all numeric columns of
'data?  Create limits to fill any missing.   

3.  clipData = data with all xNames clipped to (lower, upper).

4.  fit0 <- lme(...)

5.  Add components lower and upper to fit0 and convert it to class
c('lmeWinsor', 'lme').  

6.  Clip any stored predictions at the Winsor limits for 'y'.  

NOTE:  This is different from \LinkA{lmWinsor}{lmWinsor}, which uses quadratic 
programming with predictions outside limits, transferring extreme
points one at a time to constraints that force the unWinsorized
predictions for those points to be at least as extreme as the limits.
\end{Details}
\begin{Value}
an object of class c('lmeWinsor', 'lme') with 'lower', 'upper', and
'message' components in addition to the standard 'lm' components.  The
'message' is a list with its first component being either 'all
predictions inside limits' or 'predictions outside limits'.  In the
latter case, there rest of the list summarizes how many and which
points have predictions outside limits.
\end{Value}
\begin{Author}\relax
Spencer Graves
\end{Author}
\begin{SeeAlso}\relax
\code{\LinkA{lmWinsor}{lmWinsor}} 
\code{\LinkA{predict.lmeWinsor}{predict.lmeWinsor}} 
\code{\LinkA{lme}{lme}}
\code{\LinkA{quantile}{quantile}}
\end{SeeAlso}
\begin{Examples}
\begin{ExampleCode}
fm1w <- lmeWinsor(distance ~ age, data = Orthodont,
                 random=~age|Subject) 
fm1w.1 <- lmeWinsor(distance ~ age, data = Orthodont,
                 random=~age|Subject, trim=0.1) 
\end{ExampleCode}
\end{Examples}

\HeaderA{lmWinsor}{Winsorized Regression}{lmWinsor}
\keyword{models}{lmWinsor}
\begin{Description}\relax
Clip inputs and predictions to (upper, lower) or to selected quantiles
to limit wild predictions outside the training set.
\end{Description}
\begin{Usage}
\begin{verbatim}
  lmWinsor(formula, data, lower=NULL, upper=NULL, trim=0,
        quantileType=7, subset, weights=NULL, na.action,
        method = "qr", model = TRUE, x = FALSE, y = FALSE, qr = TRUE,
        singular.ok = TRUE, contrasts = NULL, offset=NULL,
        eps=sqrt(.Machine$double.eps), ...)
\end{verbatim}
\end{Usage}
\begin{Arguments}
\begin{ldescription}
\item[\code{formula}] an object of class '"formula"' (or one that can be coerced to that
class): a symbolic description of the model to be fitted.  See
\LinkA{lm}{lm}.  The left hand side of 'formula' must be a single vector
in 'data', untransformed.  

\item[\code{data}] an optional data frame, list or environment (or object coercible by
'as.data.frame' to a data frame) containing the variables in the
model.  If not found in 'data', the variables are taken from
'environment(formula)';  see \LinkA{lm}{lm}. 

\item[\code{lower, upper}] optional numeric vectors with names matching columns of 'data'
giving limits on the ranges of predictors and predictions:  If
present, values below 'lower' will be increased to 'lower', and
values above 'upper' will be decreased to 'upper'.  If absent, these
limit(s) will be inferred from quantile(..., prob=c(trim, 1-trim),
na.rm=TRUE, type=quantileType).  

\item[\code{trim}] the fraction (0 to 0.5) of observations to be considered outside the
range of the data in determining limits not specified in 'lower' and
'upper'.  

NOTES:

(1) trim = 0.5 should NOT be used except to check  the algorithm,
because it trims everything to the median, thereby providing zero
leverage for estimating a regression.

(2) trim>0 will give an error with a singular fit.  In such cases,
fix the singularity and retry.  

\item[\code{quantileType}] an integer between 1 and 9 selecting one of the nine quantile
algorithms to be used with 'trim' to determine limits not provided
with 'lower' and 'upper';  see   \code{\LinkA{quantile}{quantile}}.  

\item[\code{ subset }] an optional vector specifying a subset of observations to be used in
the fitting process. 

\item[\code{ weights }] an optional vector of weights to be used in the fitting process.
Should be 'NULL' or a numeric vector. If non-NULL, weighted least
squares is used with weights 'weights' (that is, minimizing
'sum(w*e*e)'); otherwise ordinary least squares is used.

\item[\code{ na.action }] a function which indicates what should happen when the data contain
'NA's.  The default is set by the 'na.action' setting of 'options',
and is 'na.fail' if that is unset.  The factory-fresh default is
'na.omit'.  Another possible value is 'NULL', no action.  Value
'na.exclude' can be useful. 

\item[\code{method}] the method to be used; for fitting, currently only 'method = "qr"'
is supported; 'method = "model.frame"' returns the model frame (the
same as with 'model = TRUE', see below). 

\item[\code{model, x, y, qr}] logicals.  If 'TRUE' the corresponding components of the fit (the
model frame, the model matrix, the response, the QR decomposition)
are returned. 

\item[\code{ singular.ok }] logical. If 'FALSE' (the default in S but not in R) a singular fit
is an error.

\item[\code{ contrasts }] an optional list. See the 'contrasts.arg' of
'model.matrix.default'. 

\item[\code{offset}] this can be used to specify an a priori known component to be
included in the linear predictor during fitting.  This should be
'NULL' or a numeric vector of length either one or equal to the
number of cases. One or more 'offset' terms can be included in the
formula instead or as well, and if both are specified their sum is
used.  See 'model.offset'. 

\item[\code{eps}] small positive number used in two ways:

\Itemize{
\item[limits] 'pred' is judged between 'lower' and 'upper' for 'y' as follows:
First compute mod = mean(abs(y)).  If this is 0, let Eps = eps;
otherwise let Eps = eps*mod.  Then pred is low if it is less
than (lower - Eps), high if it exceeds (upper + Eps), and
inside limits otherwise.

\item[QP] To identify singularity in the quadratic program (QP) discussed
in 'details', step 7 below, first compute the model.matrix of
the points with interior predictions.  Then compute the QR
decomposition of this reduced model.matix.  Then compute the
absolute values of the diagonal elements of R.  If the smallest
of these numbers is less than eps times the largest, terminate
the QP with the previous parameter estimates.  

}

\item[\code{...}] additional arguments to be passed to the low level regression
fitting functions;  see \LinkA{lm}{lm}. 

\end{ldescription}
\end{Arguments}
\begin{Details}\relax
1.  Identify inputs and outputs via mdly <- mdlx <- formula;
mdly[[3]] <- NULL;  mdlx[[2]] <- NULL;  xNames <- all.vars(mdlx); 
yNames <- all.vars(mdly).  Give an error if as.character(mdly[[2]]) !=
yNames.  

2.  Do 'lower' and 'upper' contain limits for all numeric columns of
'data?  Create limits to fill any missing.   

3.  clipData = data with all xNames clipped to (lower, upper).

4.  fit0 <- lm(formula, clipData, subset = subset, weights = weights,
na.action = na.action, method = method, x=x, y=y, qr=qr,
singular.ok=singular.ok, contrasts=contrasts, offset=offset, ...)

5.  Add components lower and upper to fit0 and convert it to class
c('lmWinsor', 'lm').  

6.  If all fit0[['fitted.values']] are inside (lower, upper)[yNames], 
return(fit0).

7.  Else, use quadratic programming (QP) to minimize the 'Winsorized
sum of squares of residuals' as follows:

7.1.  First find the prediction farthest outside (lower,
upper)[yNames].  Set temporary limits at the next closest point inside
that point (or at the limit if that's closer).  

7.2.  Use QP to minimize the sum of squares of residuals among all
points not outside the temporary limits while keeping the prediction
for the exceptional point away from the interior of (lower,
upper)[yNames].

7.3.  Are the predictions for all points unconstrained in QP inside
(lower, upper)[yNames]?  If yes, quit.    

7.4.  Otherwise, among the points still unconstrained, find the
prediction farthest outside (lower, upper)[yNames].  Adjust the
temporary limits to the next closest point inside that point (or at
the limit if that's closer).

7.5.  Use QP as in 7.2 but with multiple exceptional points, then
return to step 7.3.  

8.  Modify the components of fit0 as appropriate and return the
result.
\end{Details}
\begin{Value}
an object of class c('lmWinsor', 'lm') with 'lower', 'upper', and
'message' components in addition to the standard 'lm' components.  In
addition, if the initial fit produces predictions outside the limits,
this object returned will also include components 'coefIter' and
'tempLimits' containing the model coefficients and temporary limits
obtained during the iteration.  

The options for 'message' are as follows:  

\begin{ldescription}
\item[\code{1}] 'Initial fit in bounds':  All predictions were between 'lower' and
'upper' for 'y'. 

\item[\code{2}] 'QP iterations successful':  The QP iteration described in
'Details', step 7, terminated with all predictions either at or
between the 'lower' and 'upper' for 'y'.  

\item[\code{3}] 'Iteration terminated by a singular quadratic program':  The QP
iteration described in 'Details', step 7, terminated when the
model.matrix for the QP objective function became rank deficient.
(Rank deficient in this case means that the smallest singular
value is less than 'eps' times the largest.) 

\end{ldescription}

normal-bracket54bracket-normal

In addition to the coefficients, 'coefIter' also includes columns for
'SSEraw' and 'SSEclipped', containing the residual sums of squres from
the estimated linear model before and after clipping to the 'lower'
and 'upper' limits for 'y', plus 'nLoOut', 'nLo.', 'nIn', 'nHi.', and
'nHiOut', summarizing the distribtion of model predictions at each
iteration relative to the limits.
\end{Value}
\begin{Author}\relax
Spencer Graves
\end{Author}
\begin{SeeAlso}\relax
\code{\LinkA{predict.lmWinsor}{predict.lmWinsor}}
\code{\LinkA{lmeWinsor}{lmeWinsor}}
\code{\LinkA{lm}{lm}}
\code{\LinkA{quantile}{quantile}}
\code{\LinkA{solve.QP}{solve.QP}}
\end{SeeAlso}
\begin{Examples}
\begin{ExampleCode}
# example from 'anscombe' 
lm.1 <- lmWinsor(y1~x1, data=anscombe)

# no leverage to estimate the slope 
lm.1.5 <- lmWinsor(y1~x1, data=anscombe, trim=0.5)

# test nonlinear optimization  
lm.1.25 <- lmWinsor(y1~x1, data=anscombe, trim=0.25)

\end{ExampleCode}
\end{Examples}

\HeaderA{mean.fd}{Mean of Functional Data}{mean.fd}
\keyword{smooth}{mean.fd}
\begin{Description}\relax
Evaluate the mean of a set of functions in a functional data object.
\end{Description}
\begin{Usage}
\begin{verbatim}
mean.fd(x, ...)
\end{verbatim}
\end{Usage}
\begin{Arguments}
\begin{ldescription}
\item[\code{x}] a functional data object.

\item[\code{...}] Other arguments to match the generic function for 'mean'
\end{ldescription}
\end{Arguments}
\begin{Value}
a functional data object with a single replication
that contains the mean of the functions in the object \code{fd}.
\end{Value}
\begin{SeeAlso}\relax
\code{\LinkA{stddev.fd}{stddev.fd}}, 
\code{\LinkA{var.fd}{var.fd}}, 
\code{\LinkA{sum.fd}{sum.fd}}, 
\code{\LinkA{center.fd}{center.fd}}
\code{\LinkA{mean}{mean}}
\end{SeeAlso}
\begin{Examples}
\begin{ExampleCode}
##
## 1.  univeriate:  lip motion
##
liptime  <- seq(0,1,.02)
liprange <- c(0,1)

#  -------------  create the fd object -----------------
#       use 31 order 6 splines so we can look at acceleration

nbasis <- 51
norder <- 6
lipbasis <- create.bspline.basis(liprange, nbasis, norder)

#  ------------  apply some light smoothing to this object  -------

lipLfdobj   <- int2Lfd(4)
lipLambda   <- 1e-12
lipfdPar <- fdPar(lipbasis, lipLfdobj, lipLambda)

lipfd <- smooth.basis(liptime, lip, lipfdPar)$fd
names(lipfd$fdnames) = c("Normalized time", "Replications", "mm")

lipmeanfd <- mean.fd(lipfd)
plot(lipmeanfd)

##
## 2.  Trivariate:  CanadianWeather
##
dayrng <- c(0, 365) 

nbasis <- 51
norder <- 6 

weatherBasis <- create.fourier.basis(dayrng, nbasis)

weather.fd <- smooth.basis(day.5, CanadianWeather$dailyAv,
            weatherBasis)

str(weather.fd.mean <- mean.fd(weather.fd$fd))

\end{ExampleCode}
\end{Examples}

\HeaderA{melanoma}{melanoma 1936-1972}{melanoma}
\keyword{datasets}{melanoma}
\begin{Description}\relax
These data from the Connecticut Tumor Registry present
age-adjusted numbers of melanoma skin-cancer incidences per
100,000 people in Connectict for the years from 1936 to 1972.
\end{Description}
\begin{Format}\relax
A data frame with 37 observations on the following 2 variables.
\describe{
\item[year] Years 1936 to 1972.

\item[incidence] Rate of melanoma cancer per 100,000 population.

}
\end{Format}
\begin{Details}\relax
This is a copy of the 'melanoma' dataset in the 'lattice' package.  It
is unrelated to the object of the same name in the 'boot' package.
\end{Details}
\begin{Source}\relax
Ramsay, James O., and Silverman, Bernard W. (2006), \emph{Functional
Data Analysis, 2nd ed.}, Springer, New York.
\end{Source}
\begin{SeeAlso}\relax
\code{\LinkA{melanoma}{melanoma}}
\code{\LinkA{melanoma}{melanoma}}
\end{SeeAlso}
\begin{Examples}
\begin{ExampleCode}
plot(melanoma[, -1], type="b")
\end{ExampleCode}
\end{Examples}

\HeaderA{monomial}{Evaluate Monomial Basis}{monomial}
\keyword{smooth}{monomial}
\begin{Description}\relax
Computes the values of the powers of argument t.
\end{Description}
\begin{Usage}
\begin{verbatim}
monomial(evalarg, exponents, nderiv=0)
\end{verbatim}
\end{Usage}
\begin{Arguments}
\begin{ldescription}
\item[\code{evalarg}] a vector of argument values.

\item[\code{exponents}] a vector of nonnegative integer values specifying the
powers to be computed.

\item[\code{nderiv}] a nonnegative integer specifying the order of derivative to be
evaluated.

\end{ldescription}
\end{Arguments}
\begin{Value}
a matrix of values of basis functions.  Rows correspond to
argument values and columns to basis functions.
\end{Value}
\begin{SeeAlso}\relax
\code{\LinkA{polynom}{polynom}}, 
\code{\LinkA{power}{power}}, 
\code{\LinkA{expon}{expon}}, 
\code{\LinkA{fourier}{fourier}}, 
\code{\LinkA{polyg}{polyg}}, 
\code{\LinkA{bsplineS}{bsplineS}}
\end{SeeAlso}
\begin{Examples}
\begin{ExampleCode}

# set up a monomial basis for the first five powers
nbasis   <- 5
basisobj <- create.monomial.basis(c(-1,1),nbasis)
#  evaluate the basis
tval <- seq(-1,1,0.1)
basismat <- monomial(tval, 1:basisobj$nbasis)

\end{ExampleCode}
\end{Examples}

\HeaderA{monomialpen}{Evaluate Monomial Roughness Penalty Matrix}{monomialpen}
\keyword{smooth}{monomialpen}
\begin{Description}\relax
The roughness penalty matrix is the set of
inner products of all pairs of a derivative of integer powers of the
argument.
\end{Description}
\begin{Usage}
\begin{verbatim}
monomialpen(basisobj, Lfdobj=int2Lfd(2),
            rng=basisobj$rangeval)
\end{verbatim}
\end{Usage}
\begin{Arguments}
\begin{ldescription}
\item[\code{basisobj}] a monomial basis object.

\item[\code{Lfdobj}] either a nonnegative integer specifying an order of derivative
or a linear differential operator object.

\item[\code{rng}] the inner product may be computed over a range that is contained
within the range defined in the basis object.  This is a vector
or length two defining the range.

\end{ldescription}
\end{Arguments}
\begin{Value}
a symmetric matrix of order equal to the number of
monomial basis functions.
\end{Value}
\begin{SeeAlso}\relax
\code{\LinkA{polynompen}{polynompen}}, 
\code{\LinkA{exponpen}{exponpen}}, 
\code{\LinkA{fourierpen}{fourierpen}}, 
\code{\LinkA{bsplinepen}{bsplinepen}}, 
\code{\LinkA{polygpen}{polygpen}}
\end{SeeAlso}
\begin{Examples}
\begin{ExampleCode}

# set up a monomial basis for the first five powers
nbasis   <- 5
basisobj <- create.monomial.basis(c(-1,1),nbasis)
#  evaluate the rougness penalty matrix for the
#  second derivative.
penmat <- monomialpen(basisobj, 2)

\end{ExampleCode}
\end{Examples}

\HeaderA{nondurables}{Nondurable goods index}{nondurables}
\keyword{datasets}{nondurables}
\begin{Description}\relax
US nondurable goods index time series, January 1919 to January 2000.
\end{Description}
\begin{Format}\relax
An object of class 'ts'.
\end{Format}
\begin{Source}\relax
Ramsay, James O., and Silverman, Bernard W. (2006), \emph{Functional
Data Analysis, 2nd ed.}, Springer, New York. 

Ramsay, James O., and Silverman, Bernard W. (2002), \emph{Applied
Functional Data Analysis}, Springer, New York, ch. 3.
\end{Source}
\begin{Examples}
\begin{ExampleCode}
plot(nondurables, log="y")
\end{ExampleCode}
\end{Examples}

\HeaderA{norder}{Order of a B-spline}{norder}
\methaliasA{norder.basisfd}{norder}{norder.basisfd}
\methaliasA{norder.bspline}{norder}{norder.bspline}
\methaliasA{norder.default}{norder}{norder.default}
\methaliasA{norder.fd}{norder}{norder.fd}
\keyword{smooth}{norder}
\begin{Description}\relax
norder = number of basis functions minus the number of interior
knots.
\end{Description}
\begin{Usage}
\begin{verbatim}
norder(x, ...) 
## S3 method for class 'fd':
norder(x, ...)
## S3 method for class 'basisfd':
norder(x, ...)
## Default S3 method:
norder(x, ...)
norder.bspline(x, ...) 
\end{verbatim}
\end{Usage}
\begin{Arguments}
\begin{ldescription}
\item[\code{x}] Either a basisfd object or an object containing a basisfd object as
a component.  

\item[\code{...}] optional arguments currently unused
\end{ldescription}
\end{Arguments}
\begin{Details}\relax
norder throws an error of basisfd[['type']] != 'bspline'.
\end{Details}
\begin{Value}
An integer giving the order of the B-spline.
\end{Value}
\begin{Author}\relax
Spencer Graves
\end{Author}
\begin{SeeAlso}\relax
\code{\LinkA{create.bspline.basis}{create.bspline.basis}}
\end{SeeAlso}
\begin{Examples}
\begin{ExampleCode}
bspl1.1 <- create.bspline.basis(norder=1, breaks=0:1)

stopifnot(norder(bspl1.1)==1)

stopifnot(norder(fd(0, basisobj=bspl1.1))==1)

stopifnot(norder(fd(rep(0,4)))==4)

stopifnot(norder(list(fd(rep(0,4))))==4)
## Not run: 
norder(list(list(fd(rep(0,4)))))
Error in norder.default(list(list(fd(rep(0, 4))))) : 
  input is not a 'basisfd' object and does not have a 'basisfd'
component. 
## End(Not run)

stopifnot(norder(create.bspline.basis(norder=1, breaks=c(0,.5, 1))) == 1) 

stopifnot(norder(create.bspline.basis(norder=2, breaks=c(0,.5, 1))) == 2)

# Defaut B-spline basis:  Cubic spline:  degree 3, order 4,
# 21 breaks, 19 interior knots.  
stopifnot(norder(create.bspline.basis()) == 4)

## Not run: 
norder(create.fourier.basis(c(0,12) ))
Error in norder.bspline(x) : 
  object x is of type = fourier;  'norder' is only defined for type = 'bsline'
## End(Not run)

\end{ExampleCode}
\end{Examples}

\HeaderA{objAndNames}{Add names to an object}{objAndNames}
\keyword{attribute}{objAndNames}
\begin{Description}\relax
Add names to an object from 'preferred' if available and 'default' if
not.
\end{Description}
\begin{Usage}
\begin{verbatim}
objAndNames(object, preferred, default)
\end{verbatim}
\end{Usage}
\begin{Arguments}
\begin{ldescription}
\item[\code{object}] an object of some type to which names must be added.  If
length(dim(object))>0 add 'dimnames', else add 'names'.  

\item[\code{preferred}] A list to check first for names to add to 'object'.  

\item[\code{default}] A list to check for names to add to 'object' if appropriate names
are not found in 'preferred'.  

\end{ldescription}
\end{Arguments}
\begin{Details}\relax
1.  If length(dim(object))<1, names(object) are taken from 'preferred'
if they are not NULL and have the correct length, else try 'default'.

2.  Else for(lvl in 1:length(dim(object))) take dimnames[[lvl]] from
'preferred[[i]]' if they are not NULL and have the correct length,
else try 'default[[lvl]].
\end{Details}
\begin{Value}
An object of the same class and structure as 'object' but with either
names or dimnames added or changed.
\end{Value}
\begin{Author}\relax
Spencer Graves
\end{Author}
\begin{SeeAlso}\relax
\code{\LinkA{data2fd}{data2fd}}, 
\code{\LinkA{bifd}{bifd}}
\end{SeeAlso}
\begin{Examples}
\begin{ExampleCode}
# The following should NOT check 'anything' here
tst1 <- objAndNames(1:2, list(letters[1:2], LETTERS[1:2]), anything)
all.equal(tst1, c(a=1, b=2))

# The following should return 'object unchanged
tst2 <- objAndNames(1:2, NULL, list(letters))
all.equal(tst2, 1:2)

tst3 <- objAndNames(1:2, list("a", 2), list(letters[1:2]))
all.equal(tst3, c(a=1, b=2) )

# The following checks a matrix / array
tst4 <- array(1:6, dim=c(2,3))
tst4a <- tst4
dimnames(tst4a) <- list(letters[1:2], LETTERS[1:3])
tst4b <- objAndNames(tst4, 
       list(letters[1:2], LETTERS[1:3]), anything)
all.equal(tst4b, tst4a)

tst4c <- objAndNames(tst4, NULL,        
       list(letters[1:2], LETTERS[1:3]) )
all.equal(tst4c, tst4a)

\end{ExampleCode}
\end{Examples}

\HeaderA{odesolv}{Numerical Solution mth Order Differential Equation System}{odesolv}
\keyword{smooth}{odesolv}
\begin{Description}\relax
The system of differential equations is linear, with
possibly time-varying coefficient functions.
The numerical solution is computed with the Runge-Kutta method.
\end{Description}
\begin{Usage}
\begin{verbatim}
odesolv(bwtlist, ystart=diag(rep(1,norder)),
        h0=width/100, hmin=width*1e-10, hmax=width*0.5,
        EPS=1e-4, MAXSTP=1000)
\end{verbatim}
\end{Usage}
\begin{Arguments}
\begin{ldescription}
\item[\code{bwtlist}] a list whose members are functional parameter objects
defining the weight functions for the linear differential
equation.

\item[\code{ystart}] a vector of initial values for the equations.  These
are the values at time 0 of the solution and its first
m - 1 derivatives.

\item[\code{h0}] a positive initial step size.

\item[\code{hmin}] the minimum allowable step size.

\item[\code{hmax}] the maximum allowable step size.

\item[\code{EPS}] a convergence criterion.

\item[\code{MAXSTP}] the maximum number of steps allowed.

\end{ldescription}
\end{Arguments}
\begin{Details}\relax
This function is required to compute a set of solutions of an
estimated linear differential equation in order compute a fit
to the data that solves the equation.  Such a fit will be a
linear combinations of m independent solutions.
\end{Details}
\begin{Value}
a named list of length 2 containing

\begin{ldescription}
\item[\code{tp}] a vector of time values at which the system is evaluated

\item[\code{yp}] a matrix of variable values corresponding to \code{tp}.

\end{ldescription}
\end{Value}
\begin{SeeAlso}\relax
\code{\LinkA{pda.fd}{pda.fd}},
\end{SeeAlso}
\begin{Examples}
\begin{ExampleCode}
#See the analyses of the lip data.
\end{ExampleCode}
\end{Examples}

\HeaderA{onechild}{growth in height of one 10-year-old boy}{onechild}
\keyword{datasets}{onechild}
\begin{Description}\relax
Heights of a boy of age approximately 10 collected during one school
year.  The data were collected "over one school year, with gaps
corresponding to the school vacations" (AFDA, p. 84)
\end{Description}
\begin{Format}\relax
A data.frame with two columns:  
\describe{
\item[day] Integers counting the day into data collection with gaps
indicating days during which data were not collected.  

\item[height] Height of the boy measured on the indicated day.  

}
\end{Format}
\begin{Source}\relax
Ramsay, James O., and Silverman, Bernard W. (2002), \emph{Applied
Functional Data Analysis, 2nd ed.}, Springer, New York. 

Tuddenham, R. D., and Snyder, M. M. (1954) "Physical growth of
California boys and girls from birth to age 18", \emph{University of
California Publications in Child Development}, 1, 183-364.
\end{Source}
\begin{Examples}
\begin{ExampleCode}
with(onechild, plot(day, height, type="b"))

\end{ExampleCode}
\end{Examples}

\HeaderA{pca.fd}{Functional Principal Components Analysis}{pca.fd}
\keyword{smooth}{pca.fd}
\begin{Description}\relax
Functional Principal components analysis aims to display types of
variation across a sample of functions.  Principal components analysis
is an exploratory data analysis that tends to be an early part of many
projects.  These modes of variation are called $principal components$
or $harmonics.$  This function computes these harmonics, the
eigenvalues that indicate how important each mode of variation, and
harmonic scores for individual functions. If the functions are
multivariate, these harmonics are combined into a composite function
that summarizes joint variation among the several functions that make
up a multivariate functional observation.
\end{Description}
\begin{Usage}
\begin{verbatim}
pca.fd(fdobj, nharm = 2, harmfdPar=fdPar(fdobj),
       centerfns = TRUE)
\end{verbatim}
\end{Usage}
\begin{Arguments}
\begin{ldescription}
\item[\code{fdobj}] a functional data object.

\item[\code{nharm}] the number of harmonics or principal components to compute.

\item[\code{harmfdPar}] a functional parameter object that defines the
harmonic or principal component functions to be estimated.

\item[\code{centerfns}] a logical value:
if TRUE, subtract the mean function from each function before
computing principal components.

\end{ldescription}
\end{Arguments}
\begin{Value}
an object of class "pca.fd" with these named entries:

\begin{ldescription}
\item[\code{harmonics}] a functional data object for the harmonics or eigenfunctions

\item[\code{values}] the complete set of eigenvalues

\item[\code{scores}] s matrix of scores on the principal components or harmonics

\item[\code{varprop}] a vector giving the proportion of variance explained
by each eigenfunction

\item[\code{meanfd}] a functional data object giving the mean function

\end{ldescription}
\end{Value}
\begin{SeeAlso}\relax
\code{\LinkA{cca.fd}{cca.fd}}, 
\code{\LinkA{pda.fd}{pda.fd}}
\end{SeeAlso}
\begin{Examples}
\begin{ExampleCode}

#  carry out a PCA of temperature
#  penalize harmonic acceleration, use varimax rotation

daybasis65 <- create.fourier.basis(c(0, 365), nbasis=65, period=365)

harmaccelLfd <- vec2Lfd(c(0,(2*pi/365)^2,0), c(0, 365))
harmfdPar     <- fdPar(daybasis65, harmaccelLfd, lambda=1e5)
daytempfd <- data2fd(CanadianWeather$dailyAv[,,"Temperature.C"],
      day.5, daybasis65, argnames=list("Day", "Station", "Deg C"))

daytemppcaobj <- pca.fd(daytempfd, nharm=4, harmfdPar)
daytemppcaVarmx <- varmx.pca.fd(daytemppcaobj)
#  plot harmonics
op <- par(mfrow=c(2,2))
plot.pca.fd(daytemppcaobj, cex.main=0.9)

plot.pca.fd(daytemppcaVarmx, cex.main=0.9)
par(op)
\end{ExampleCode}
\end{Examples}

\HeaderA{pda.fd}{Principal Differential Analysis}{pda.fd}
\keyword{smooth}{pda.fd}
\begin{Description}\relax
Principal differential analysis (PDA) estimates a
system of linear differential equations that define functions
that fit the data and their derivatives.
\end{Description}
\begin{Usage}
\begin{verbatim}
pda.fd(xfdlist, bwtlist=NULL,
       awtlist=NULL, ufdlist=NULL, nfine=501)
\end{verbatim}
\end{Usage}
\begin{Arguments}
\begin{ldescription}
\item[\code{xfdlist}] a list whose members are functional data objects.  Each of these objects
contain one or more functions that are variables to be represented
by a differential equation.  The length of the list is the size of the
system of differential equations. The number of replications must be
the same for each member functional data object.

\item[\code{bwtlist}] a list array with the first two dimensions are equal to the number of
variables in the system (the length of list \code{xfdlist}) and
the last dimension equal to the order of each equation.  The order of
the equations is assumed to be the same for each equation.  If only a
single differential equation is involved, this argument can be an
ordinary list with length equal to the order of the equation.

Each member of \code{bwtlist} defines a weight function
in the system of linear differential equation.  For any equation,
there can be a term in the equation for each order of derivative
from 0 to m-1, where m is the order of the equation.  These derivatives
can be those of each variable in the system, moreover.  Thus, a two-variable
system of order 3 can have, for each of the two equations, contributions
from derivatives of order 0, 1 and 2 for each of the variables, or
six terms in total.

The first dimension of \code{bwtlist} corresponds to equations,
and the second dimension to contributions of variables for a fixed derivative
within the equation defined by the first index.  The third index indicates
the order of derivative of the contribution, which is one less than the
index value.

Each member of \code{bwtlist} is a either a functional
parameter object or a functional data object.  Functional data objects
are converted to functional parameters objects with no smoothing.
Functional parameter objects permit selected weight functions to be
held fixed and not estimated.  These are often constant functions with
the value 0, corresponding to variables and derivatives that make no
contribution.

For example, the harmonic acceleration differential equation
would be of the form $D^3 x(t) = -b Dx(t)$ so that
\code{bwtlist}
would be an ordinary list of length 3, each functional parameter object
being defined with a constant basis, but only the second object would
have a value of TRUE for the \code{estimate} slot, and the first
and third coefficient functions would have value 0.

\item[\code{awtlist}] a list containing weight functions for forcing functions.

In addition to terms in each of the equations involving terms corresponding
to each derivative of each variable in the system, each equation can
also have a contribution from one or more exogenous variables, often
called forcing functions.

This argument defines the weights multiplying these
forcing functions, and is a list array with first dimension equal to
the number of variables or equations, and second dimension equal to the
number of forcing functions. It is assumed that the number of forcing functions is
the same for all equations. If only one forcing function is involved,
or there is only one equation, then \code{awtlist} can be an
ordinary list.

Each member is a functional parameter object
(or a functional data object) defining a weighting function for a
forcing function.  As with
\code{bwtlist}, each of these weighting functions may be estimated
or held fixed.  If the number of forcing functions actually varies from
equation to equation, one can still use all forcing functions for all
variables, but just weight unwanted ones in a particular equation by a
fixed zero function.

Each member's functional data object has only a single replicate.

\item[\code{ufdlist}] a list containing forcing functions.
This is a list array of the same size as \code{awtlist} and each
member is a functional data object corresponding to a forcing function.
The number of replicates must be equal to that of the variables themselves,
which is assumed to be the same for all variables.

\item[\code{nfine}] a number of values for a fine mesh.
The estimation of the differential equation involves discrete
numerical quadrature estimates of integrals, and these require
that functions be evaluated at a fine mesh of values of the
argument.  This argument defines the number to use.  If not
constant, this number is set to five times the largest number of
basis functions used to represent any variable in the system.

\end{ldescription}
\end{Arguments}
\begin{Value}
a named list of length 3 with components:

\begin{ldescription}
\item[\code{bwtlist}] a list array of the same dimensions as the
corresponding argument, containing the estimated or fixed weight
functions defining the system of linear differential equations.

\item[\code{resfdlist}] a list of length equal to the number of variables
or equations.  Each members is a functional data object giving the
residual functions or forcing functions defined as the left side
of the equation (the derivative of order m of a variable) minus
the linear fit on the right side.  The number of replicates for each
residual functional data object is the same as that for the variables.

\item[\code{awtlist}] a list of the same dimensions as the corresponding
argument.  Each member is an estimated or fixed weighting function for
a forcing function.

\end{ldescription}
\end{Value}
\begin{SeeAlso}\relax
\code{\LinkA{pca.fd}{pca.fd}}, 
\code{\LinkA{cca.fd}{cca.fd}}
\end{SeeAlso}
\begin{Examples}
\begin{ExampleCode}
#See analyses of daily weather data for examples.
\end{ExampleCode}
\end{Examples}

\HeaderA{phaseplanePlot}{Phase-plane plot}{phaseplanePlot}
\keyword{smooth}{phaseplanePlot}
\keyword{hplot}{phaseplanePlot}
\begin{Description}\relax
Plot acceleration (or Ldfobj2) vs. velocity (or Lfdobj1) of a function
data object.
\end{Description}
\begin{Usage}
\begin{verbatim}
phaseplanePlot(evalarg, fdobj, Lfdobj1=1, Lfdobj2=2,
        lty=c("longdash", "solid"),  
      labels=list(evalarg=seq(evalarg[1], max(evalarg), length=13),
             labels=monthLetters),
      abline=list(h=0, v=0, lty=2), xlab="Velocity",
      ylab="Acceleration", ...)
\end{verbatim}
\end{Usage}
\begin{Arguments}
\begin{ldescription}
\item[\code{evalarg}] a vector of argument values at which the functional data object is
to be evaluated.

Defaults to a sequence of 181 points in the range
specified by fdobj[["basis"]][["rangeval"]].      

If(length(evalarg) == 1)it is replaced by seq(evalarg[1],
evalarg[1]+1, length=181).  

If(length(evalarg) == 2)it is replaced by seq(evalarg[1],
evalarg[2], length=181).      

\item[\code{fdobj}] a functional data object to be evaluated.

\item[\code{Lfdobj1}] either a nonnegative integer or a linear differential operator
object.  The points plotted on the horizontal axis are
eval.fd(evalarg, fdobj, Lfdobj1).  By default, this is the
velocity.  

\item[\code{Lfdobj2}] either a nonnegative integer or a linear differential operator
object.  The points plotted on the vertical axis are
eval.fd(evalarg, fdobj, Lfdobj2).  By default, this is the
acceleration.  

\item[\code{lty}] line types for the first and second halves of the plot.  

\item[\code{labels}] a list of length two:

evalarg = a numeric vector of 'evalarg' values to be labeled.

labels = a character vector of labels, replicated to the same length
as labels[["evalarg"]] in case it's not of the same length.  

\item[\code{abline}] arguments to a call to abline.  

\item[\code{xlab}] x axis label 

\item[\code{ylab}] y axis label 

\item[\code{...}] optional arguments passed to plot.  

\end{ldescription}
\end{Arguments}
\begin{Value}
Invisibly returns a matrix with two columns containg the points
plotted.
\end{Value}
\begin{SeeAlso}\relax
\code{\LinkA{plot}{plot}}, 
\code{\LinkA{eval.fd}{eval.fd}}
\code{\LinkA{plot.fd}{plot.fd}}
\end{SeeAlso}
\begin{Examples}
\begin{ExampleCode}
goodsbasis <- create.bspline.basis(rangeval=c(1919,2000),
                                   nbasis=979, norder=8)
LfdobjNonDur <- int2Lfd(4) 

library(zoo)
logNondurSm <- smooth.basisPar(argvals=index(nondurables),
                y=log10(coredata(nondurables)), fdobj=goodsbasis,
                Lfdobj=LfdobjNonDur, lambda=1e-11)
phaseplanePlot(1964, logNondurSm$fd)

\end{ExampleCode}
\end{Examples}

\HeaderA{pinch}{pinch force data}{pinch}
\aliasA{pinchtime}{pinch}{pinchtime}
\keyword{datasets}{pinch}
\begin{Description}\relax
151 measurements of pinch force during 20 replications, registered, with
time from start of measurement.
\end{Description}
\begin{Usage}
\begin{verbatim}
pinch
pinchtime
\end{verbatim}
\end{Usage}
\begin{Format}\relax
\describe{
\item[pinch] Matrix of dimension c(151, 20) = 20 replications of measuring
pinch force every 2 milliseconds for 300 milliseconds.  

\item[pinchtime] time in seconds from the start = seq(0, 0.3, 151) = every 2
milliseconds.

}
\end{Format}
\begin{Details}\relax
Measurements every 2 milliseconds.
\end{Details}
\begin{Source}\relax
Ramsay, James O., and Silverman, Bernard W. (2006), \emph{Functional
Data Analysis, 2nd ed.}, Springer, New York, p. 13, Figure 1.11,
pp. 22-23, Figure 2.2, and p. 144, Figure 7.13.
\end{Source}
\begin{Examples}
\begin{ExampleCode}
  plot(pinchtime, pinch[, 1], type="b")
\end{ExampleCode}
\end{Examples}

\HeaderA{plot.basisfd}{Plot a Basis Object}{plot.basisfd}
\keyword{smooth}{plot.basisfd}
\begin{Description}\relax
Plots all the basis functions.
\end{Description}
\begin{Usage}
\begin{verbatim}
plot.basisfd(x, knots=TRUE, ...)
\end{verbatim}
\end{Usage}
\begin{Arguments}
\begin{ldescription}
\item[\code{x}] a basis object

\item[\code{knots}] logical:  If TRUE and x[['type']] == 'bslpine', the knot locations
are plotted using vertical dotted, red lines.  Ignored otherwise.  

\item[\code{... }] additional plotting parameters passed to \code{matplot}.  

\end{ldescription}
\end{Arguments}
\begin{Value}
none
\end{Value}
\begin{Section}{Side Effects}
a plot of the basis functions
\end{Section}
\begin{SeeAlso}\relax
\code{\LinkA{plot.fd}{plot.fd}}
\end{SeeAlso}
\begin{Examples}
\begin{ExampleCode}

# set up the b-spline basis for the lip data, using 23 basis functions,
#   order 4 (cubic), and equally spaced knots.
#  There will be 23 - 4 = 19 interior knots at 0.05, ..., 0.95
lipbasis <- create.bspline.basis(c(0,1), 23)
# plot the basis functions
plot(lipbasis)

\end{ExampleCode}
\end{Examples}

\HeaderA{plot.fd}{Plot a Functional Data Object}{plot.fd}
\aliasA{plot.fdSmooth}{plot.fd}{plot.fdSmooth}
\keyword{smooth}{plot.fd}
\keyword{hplot}{plot.fd}
\begin{Description}\relax
Functional data observations, or a derivative of them, are plotted.
These may be either plotted simultaneously, as \code{matplot} does for
multivariate data, or one by one with a mouse click to move from one
plot to another.  The function also accepts the other plot
specification arguments that the regular \code{plot} does.  Calling
\code{plot} with an \code{fdSmooth} object is simply plots its
code{fd} component.
\end{Description}
\begin{Usage}
\begin{verbatim}
## S3 method for class 'fd':
plot(x, y, Lfdobj=0, href=TRUE, titles=NULL,
                    xlim=NULL, ylim=NULL, xlab=NULL,
                    ylab=NULL, ask=FALSE, nx=201, ...)
## S3 method for class 'fdSmooth':
plot(x, y, Lfdobj=0, href=TRUE, titles=NULL,
                    xlim=NULL, ylim=NULL, xlab=NULL,
                    ylab=NULL, ask=FALSE, nx=201, ...)
\end{verbatim}
\end{Usage}
\begin{Arguments}
\begin{ldescription}
\item[\code{x}] functional data object(s) to be plotted.

\item[\code{y}] sequence of points at which to evaluate the functions 'x' and plot
on the horizontal axis.  Defaults to seq(rangex[1], rangex[2],
length = nx).

NOTE:  This will be the values on the horizontal axis, NOT the
vertical axis.  

\item[\code{Lfdobj}] either a nonnegative integer or a linear differential operator
object. If present, the derivative or the value of applying the
operator is plotted rather than the functions themselves.

\item[\code{href}] a logical variable:  If \code{TRUE}, add a horizontal reference line
at 0.  

\item[\code{titles}] a vector of strings for identifying curves

\item[\code{xlab}] a label for the horizontal axis.

\item[\code{ylab}] a label for the vertical axis.

\item[\code{xlim}] a vector of length 2 containing axis limits for the horizontal axis.

\item[\code{ylim}] a vector of length 2 containing axis limits for the vertical axis.

\item[\code{ask}] a logical value:  If \code{TRUE}, each curve is shown separately, and
the plot advances with a mouse click

\item[\code{nx}] the number of points to use to define the plot.  The default is
usually enough, but for a highly variable function more may be
required.

\item[\code{... }] additional plotting arguments that can be used with function
\code{plot}

\end{ldescription}
\end{Arguments}
\begin{Details}\relax
Note that for multivariate data, a
suitable array must first be defined using the \code{par} function.
\end{Details}
\begin{Value}
'done'
\end{Value}
\begin{Section}{Side Effects}
a plot of the functional observations
\end{Section}
\begin{SeeAlso}\relax
\code{\LinkA{lines.fd}{lines.fd}}, \code{\LinkA{plotfit.fd}{plotfit.fd}}
\end{SeeAlso}
\begin{Examples}
\begin{ExampleCode}
##
## plot.df
##
#daytime   <- (1:365)-0.5
#dayrange  <- c(0,365)
#dayperiod <- 365
#nbasis     <- 65
#dayrange  <- c(0,365)

daybasis65 <- create.fourier.basis(c(0, 365), 65)
harmaccelLfd <- vec2Lfd(c(0,(2*pi/365)^2,0), c(0, 365))

harmfdPar     <- fdPar(daybasis65, harmaccelLfd, lambda=1e5)

daytempfd <- with(CanadianWeather, data2fd(
        dailyAv[,,"Temperature.C"], day.5, 
        daybasis65,argnames=list("Day", "Station", "Deg C")) )

#  plot all the temperature functions for the monthly weather data
plot(daytempfd, main="Temperature Functions")

## Not run: 
# To plot one at a time:  
# The following pauses to request page changes.

plot(daytempfd, main="Temperature Functions", ask=TRUE)
## End(Not run)

##
## plot.fdSmooth
##
b3.4 <- create.bspline.basis(norder=3, breaks=c(0, .5, 1))
# 4 bases, order 3 = degree 2 =
# continuous, bounded, locally quadratic 
fdPar3 <- fdPar(b3.4, lambda=1)
# Penalize excessive slope Lfdobj=1;  
# second derivative Lfdobj=2 is discontinuous.
fd3.4s0 <- smooth.basis(0:1, 0:1, fdPar3)

# using plot.fd directly 
plot(fd3.4s0$fd)

# same plot via plot.fdSmooth 
plot(fd3.4s0)

\end{ExampleCode}
\end{Examples}

\HeaderA{plot.lmWinsor}{lmWinsor plot}{plot.lmWinsor}
\keyword{hplot}{plot.lmWinsor}
\begin{Description}\relax
plot an lmWinsor model or list of models as line(s) with the data as
points
\end{Description}
\begin{Usage}
\begin{verbatim}
## S3 method for class 'lmWinsor':
plot(x, n=101, lty=1:9, col=1:9,
         lwd=c(2:4, rep(3, 6)), lty.y=c('dotted', 'dashed'),
         lty.x = lty.y, col.y=1:9, col.x= col.y, lwd.y = c(1.2, 1),
         lwd.x=lwd.y, ...)
\end{verbatim}
\end{Usage}
\begin{Arguments}
\begin{ldescription}
\item[\code{x}] an object of class 'lmWinsor', which is either a list of objects of
class c('lmWinsor', 'lm') or is a single object of that double
class.  Each object of class c('lmWinsor', 'lm') is the result of a
single 'lmWinsor' fit.  If 'x' is a list, it summarizes multiple
fits with different limits to the same data.  

\item[\code{n}] integer;  with only one explanatory variable 'xNames' in the model,
'n' is the number of values at which to evaluate the model
predictions.  This is ignored if the number of explanatory variable
'xNames' in the model is different from 1.  

\item[\code{lty, col, lwd, lty.y, lty.x, col.y, col.x, lwd.y, lwd.x}] 'lty', 'col' and 'lwd' are each replicated to a length matching the
number of fits summarized in 'x' and used with one line for each fit
in the order appearing in 'x'.  The others refer to horizontal and
vertical limit lines. 

\item[\code{...}] optional arguments for 'plot'  

\end{ldescription}
\end{Arguments}
\begin{Details}\relax
1.  One fit or several?  

2.  How many explanatory variables are involved in the model(s) in
'x'?  If only one, then the response variable is plotted vs. that one
explanatory variable.  Otherwise, the response is plotted
vs. predictions. 

3.  Plot the data.

4.  Plot one line for each fit with its limits.
\end{Details}
\begin{Value}
invisible(NULL)
\end{Value}
\begin{Author}\relax
Spencer Graves
\end{Author}
\begin{SeeAlso}\relax
\code{\LinkA{lmWinsor}{lmWinsor}}
\code{\LinkA{plot}{plot}}
\end{SeeAlso}
\begin{Examples}
\begin{ExampleCode}
lm.1 <- lmWinsor(y1~x1, data=anscombe)
plot(lm.1)
plot(lm.1, xlim=c(0, 15), main="other title")

# list example
lm.1. <- lmWinsor(y1~x1, data=anscombe, trim=c(0, 0.25, .4, .5)) 
plot(lm.1.)

\end{ExampleCode}
\end{Examples}

\HeaderA{plot.pca.fd}{Plot Functional Principal Components}{plot.pca.fd}
\keyword{smooth}{plot.pca.fd}
\begin{Description}\relax
Display the types of variation across a sample of functions.  Label
with the eigenvalues that indicate the relative importance of each
mode of variation.
\end{Description}
\begin{Usage}
\begin{verbatim}
plot.pca.fd(x, nx = 128, pointplot = TRUE, harm = 0,
                        expand = 0, cycle = FALSE, ...)
\end{verbatim}
\end{Usage}
\begin{Arguments}
\begin{ldescription}
\item[\code{x}] a functional data object.

\item[\code{nx}] Number of points to plot or vector (if length > 1) to use as
\code{evalarg} in evaluating and plotting the functional principal
components. 

\item[\code{pointplot}] logical:  If TRUE, the harmonics / principal components are plotted
as '+' and '-'.   Otherwise lines are used.

\item[\code{harm}] Harmonics / principal components to plot.  If 0, plot all.

If length(harm) > sum(par("mfrow")), the user advised, "Waiting to
confirm page change..." and / or 'Click or hit ENTER for next page'
for each page after the first.  

\item[\code{expand}] nonnegative real:  If expand == 0 then effect of +/- 2 standard
deviations of each pc are given otherwise the factor expand is
used.  

\item[\code{cycle}] logical:  If cycle=TRUE and there are 2 variables then a cycle plot
will be drawn If the number of variables is anything else, cycle
will be ignored. 

\item[\code{...}] other arguments for 'plot'.  

\end{ldescription}
\end{Arguments}
\begin{Details}\relax
Produces one plot for each principal component / harmonic to be
plotted.
\end{Details}
\begin{Value}
invisible(NULL)
\end{Value}
\begin{SeeAlso}\relax
\code{\LinkA{cca.fd}{cca.fd}}, 
\code{\LinkA{pda.fd}{pda.fd}}
\code{\LinkA{plot.pca.fd}{plot.pca.fd}}
\end{SeeAlso}
\begin{Examples}
\begin{ExampleCode}

#  carry out a PCA of temperature
#  penalize harmonic acceleration, use varimax rotation

daybasis65 <- create.fourier.basis(c(0, 365), nbasis=65, period=365)

harmaccelLfd <- vec2Lfd(c(0,(2*pi/365)^2,0), c(0, 365))
harmfdPar     <- fdPar(daybasis65, harmaccelLfd, lambda=1e5)
daytempfd <- data2fd(CanadianWeather$dailyAv[,,"Temperature.C"],
      day.5, daybasis65, argnames=list("Day", "Station", "Deg C"))

daytemppcaobj <- pca.fd(daytempfd, nharm=4, harmfdPar)
#  plot harmonics, asking before each new page after the first:  
plot.pca.fd(daytemppcaobj)

# plot 4 on 1 page
op <- par(mfrow=c(2,2))
plot.pca.fd(daytemppcaobj, cex.main=0.9)
par(op)

\end{ExampleCode}
\end{Examples}

\HeaderA{plotfit}{Plot a Functional Data Object With Data}{plotfit}
\methaliasA{plotfit.fd}{plotfit}{plotfit.fd}
\methaliasA{plotfit.fdSmooth}{plotfit}{plotfit.fdSmooth}
\keyword{smooth}{plotfit}
\keyword{hplot}{plotfit}
\begin{Description}\relax
Plot either functional data observations 'x' with a fit 'fdobj' or
residuals from the fit. 

This function is useful for assessing how well a functional data
object fits the actual discrete data.

The default is to make one plot per functional observation with fit
if residual is FALSE and superimposed lines if residual==TRUE.  

With multiple plots, the system waits to confirm a desire to move to
the next page unless ask==FALSE.
\end{Description}
\begin{Usage}
\begin{verbatim}
plotfit.fd(y, argvals, fdobj, rng = NULL, index = NULL,
      nfine = 101, residual = FALSE, sortwrd = FALSE, titles=NULL,
      ylim=NULL, ask=TRUE, type=c("p", "l")[1+residual],
      xlab=NULL, ylab=NULL, sub=NULL, col=1:9, lty=1:9, lwd=1,
      cex.pch=1, ...)
plotfit.fdSmooth(y, argvals, fdSm, rng = NULL, index = NULL,
      nfine = 101, residual = FALSE, sortwrd = FALSE, titles=NULL,
      ylim=NULL, ask=TRUE, type=c("p", "l")[1+residual],
      xlab=NULL, ylab=NULL, sub=NULL, col=1:9, lty=1:9, lwd=1,
      cex.pch=1, ...) 
\end{verbatim}
\end{Usage}
\begin{Arguments}
\begin{ldescription}
\item[\code{y}] a vector, matrix or array containing the discrete observations used
to estimate the functional data object. 

\item[\code{argvals}] a vector containing the argument values corresponding to the first
dimension of \code{y}. 

\item[\code{fdobj}] a functional data object estimated from the data.

\item[\code{fdSm}] an object of class \code{fdSmooth} 
\item[\code{rng}] a vector of length 2 specifying the limits for the horizontal axis.
This must be a subset of fdobj[['basis']][['rangeval']], which is
the default. 

\item[\code{index}] a set of indices of functions if only a subset of the observations
are to be plotted.  Subsetting can also be achieved by subsetting
\code{y};  see \code{details}, below.  

\item[\code{nfine}] the number of argument values used to define the plot of the
functional data object.  This may need to be increased if the
functions have a great deal of fine detail. 

\item[\code{residual}] a logical variable:  if \code{TRUE}, the residuals are plotted
rather than the data and functional data object.

\item[\code{sortwrd}] a logical variable:  if \code{TRUE}, the observations (i.e., second
dimension of \code{y}) are sorted for plotting by the size of the
sum of squared residuals. 

\item[\code{titles}] a vector containing strings that are titles for each observation.

\item[\code{ylim}] a numeric vector of length 2 giving the y axis limits;  see 'par'.  

\item[\code{ask}] If TRUE and if 'y' has more levels than the max length of col, lty,
lwd and cex.pch, the user must confirm page change before the next
plot will be created.

\item[\code{type}] type of plot desired, as described with \code{\LinkA{plot}{plot}}.  If
residual == FALSE, 'type' controls the representation for 'x', which
will typically be 'p' to plot points but not lines;  'fdobj' will
always plot as a line.  If residual == TRUE, the default type ==
"l";  an alternative is "b" for both.  

\item[\code{xlab}] x axis label. 
\item[\code{ylab}] Character vector of y axis labels.  If(residual), ylab defaults to
'Residuals', else to varnames derived from names(fdnames[[3]] or
fdnames[[3]] or dimnames(y)[[3]]. 

\item[\code{sub}] subtitle under the x axis label.  Defaults to the RMS residual from
the smooth.  

\item[\code{col, lty, lwd, cex.pch}] Numeric or character vectors specifying the color (col), line type
(lty), line width (lwd) and size of plotted character symbols
(cex.pch) of the data representation on the plot. 

If ask==TRUE, the length of the longest of these determines the
number of levels of the array 'x' in each plot before asking the
user to acknowledge a desire to change to the next page.  Each of
these is replicated to that length, so col[i] is used for x[,i] (if
x is 2 dimensional), with line type and width controlled by lty[i]
and lwd[i], respectively.  

If ask==FALSE, these are all replicated to length = the number of
plots to be superimposed.

For more information on alternative values for these paramters, see
'col', 'lty', 'lwd', or 'cex' with \code{\LinkA{par}{par}}.    

\item[\code{... }] additional arguments such as axis labels that may be used with other
\code{plot} functions. 

\end{ldescription}
\end{Arguments}
\begin{Details}\relax
\code{plotfit} plots discrete data along with a functional data object
for fitting the data.  It is designed to be used after something like
\code{data2fd}, \code{smooth.fd}, \code{smooth.basis} or
\code{smoothe.basisPar} to check the fit of the data offered by the
\code{fd} object.

\code{plotfit.fdSmooth} calls \code{plotfit} for its 'fd' component.

The plot can be restricted to a subset of observations (i.e., second
dimension of \code{y}) or variables (i.e., third dimension of
\code{y}) by providing \code{y} with the dimnames for its second and
third dimensions matching a subset of the dimnames of fdobj[['coef']]
(for \code{plotfit.fd} or fdSm[['fdobj']][['coef']] for
\code{plotfit.fdSmooth}).  If only one observation or variable is to
be plotted, \code{y} must include 'drop = TRUE', as, e.g., y[, 2, 3,
drop=TRUE].  If \code{y} or fdobj[['coef']] does not have dimnames on
its second or third dimension, subsetting is acheived by taking the
first few colums so the second or third dimensions match.  This is
acheived using checkDims3(y, fdobj[['coef']], defaultNames =
fdobj[['fdnames']]]).
\end{Details}
\begin{Value}
A matrix of mean square deviations from predicted is returned
invisibly.  If fdobj[["coefs"]] is a 3-dimensional array, this is a
matrix of dimensions equal to the last two dimensions of
fdobj[["coefs"]].  This will typically be the case when x is also a
3-dimensional array with the last two dimensions matching those of
fdobj[["coefs"]].  The second dimension is typically replications and
the third different variables.

If x and fobj[["coefs"]] are vectors or 2-dimensional arrays, they are
padded to three dimensions, and then MSE is computed as a matrix with
the second dimension = 1;  if x and fobj[["coefs"]] are vectors, the
first dimension of the returned matrix will also be 1.
\end{Value}
\begin{Section}{Side Effects}
a plot of the the data 'x' with the function or the deviations between
observed and predicted, depending on whether residual is FALSE or
TRUE.
\end{Section}
\begin{SeeAlso}\relax
\code{\LinkA{plot}{plot}}, 
\code{\LinkA{plot.fd}{plot.fd}}, 
\code{\LinkA{lines.fd}{lines.fd}}
\code{\LinkA{plot.fdSmooth}{plot.fdSmooth}}, 
\code{\LinkA{lines.fdSmooth}{lines.fdSmooth}}
\code{\LinkA{par}{par}}
\code{\LinkA{data2fd}{data2fd}}
\code{\LinkA{smooth.fd}{smooth.fd}}
\code{\LinkA{smooth.basis}{smooth.basis}}
\code{\LinkA{smooth.basisPar}{smooth.basisPar}}
\code{\LinkA{checkDims3}{checkDims3}}
\end{SeeAlso}
\begin{Examples}
\begin{ExampleCode}
daybasis65 <- create.fourier.basis(c(0, 365), 65)

daytempfd <- with(CanadianWeather, data2fd(
       dailyAv[,,"Temperature.C"], day.5, 
       daybasis65, argnames=list("Day", "Station", "Deg C")) )
 
with(CanadianWeather, plotfit.fd(dailyAv[, , "Temperature.C"],
     argvals= day.5, daytempfd, index=1, titles=place, axes=FALSE) )
# Default ylab = daytempfd[['fdnames']] 

with(CanadianWeather, plotfit.fd(dailyAv[, , "Temperature.C", drop=FALSE],
     argvals= day.5, daytempfd, index=1, titles=place, axes=FALSE) )
# Better:  ylab = dimnames(y)[[3]]

# Label the horizontal axis with the month names
axis(1, monthBegin.5, labels=FALSE)
axis(1, monthEnd.5, labels=FALSE)
axis(1, monthMid, monthLetters, tick=FALSE)
axis(2)

## Not run: 
# The following pauses to request page changes.
# (Without 'dontrun', the package build process
# might encounter problems with the par(ask=TRUE)
# feature.)
with(CanadianWeather, plotfit.fd(
          dailyAv[,, "Temperature.C"], day.5,
          daytempfd, ask=TRUE) )
## End(Not run)

# If you want only the fitted functions, use plot(daytempfd)

# To plot only a single fit vs. observations, use index
# to request which one you want.  

op <- par(mfrow=c(2,1), xpd=NA, bty="n")
# xpd=NA:  clip lines to the device region,
#       not the plot or figure region
# bty="n":  Do not draw boxes around the plots.  
ylim <- range(CanadianWeather$dailyAv[,,"Temperature.C"])
# Force the two plots to have the same scale 
with(CanadianWeather, plotfit.fd(dailyAv[,,"Temperature.C"], day.5, 
          daytempfd, index=2, titles=place, ylim=ylim, axes=FALSE) )
axis(1, monthBegin.5, labels=FALSE)
axis(1, monthEnd.5, labels=FALSE)
axis(1, monthMid, monthLetters, tick=FALSE)
axis(2)

with(CanadianWeather, plotfit.fd(dailyAv[,,"Temperature.C"], day.5, 
          daytempfd, index=35, titles=place, ylim=ylim) )
axis(1, monthBegin.5, labels=FALSE)
axis(1, monthEnd.5, labels=FALSE)
axis(1, monthMid, monthLetters, tick=FALSE)
axis(2)
par(op)

# plot residuals
with(CanadianWeather, plotfit.fd(dailyAv[, , "Temperature.C"], 
          day.5, daytempfd, residual=TRUE) )
# Can't read this, so try with 2 lines per page with ask=TRUE, 
# and limiting length(col), length(lty), etc. <=2
## Not run: 
with(CanadianWeather, plotfit.fd(
          dailyAv[,,"Temperature.C"], day.5, 
          daytempfd, residual=TRUE, col=1:2, lty=1, ask=TRUE) )
## End(Not run)

\end{ExampleCode}
\end{Examples}

\HeaderA{plotscores}{Plot Principal Component Scores}{plotscores}
\keyword{smooth}{plotscores}
\begin{Description}\relax
The coefficients multiplying the harmonics or principal component functions
are plotted as points.
\end{Description}
\begin{Usage}
\begin{verbatim}
plotscores(pcafd, scores=c(1, 2), xlab=NULL, ylab=NULL,
           loc=1, matplt2=FALSE, ...)
\end{verbatim}
\end{Usage}
\begin{Arguments}
\begin{ldescription}
\item[\code{pcafd}] an object of the "pca.fd" class that is output by function
\code{pca.fd}.

\item[\code{scores}] the indices of the harmonics for which coefficients are
plotted.

\item[\code{xlab}] a label for the horizontal axis.

\item[\code{ylab}] a label for the vertical axis.

\item[\code{loc}] an integer:
if loc  >0, you can then click on the plot in loc places and you'll get
plots of the functions with these values of the principal component
coefficients.

\item[\code{matplt2}] a logical value:
if \code{TRUE}, the curves are plotted on the same plot;
otherwise, they are plotted separately.

\item[\code{... }] additional plotting arguments used in function \code{plot}.

\end{ldescription}
\end{Arguments}
\begin{Section}{Side Effects}
a plot of scores
\end{Section}
\begin{SeeAlso}\relax
\code{\LinkA{pca.fd}{pca.fd}}
\end{SeeAlso}

\HeaderA{polyg}{Polygonal Basis Function Values}{polyg}
\keyword{smooth}{polyg}
\begin{Description}\relax
Evaluates a set of polygonal basis functions, or a derivative of these
functions, at a set of arguments.
\end{Description}
\begin{Usage}
\begin{verbatim}
polyg(x, argvals, nderiv=0)
\end{verbatim}
\end{Usage}
\begin{Arguments}
\begin{ldescription}
\item[\code{x}] a vector of argument values at which the polygonal basis functions are to
evaluated.

\item[\code{argvals}] a strictly increasing set of argument values containing the range of x
within it that defines the polygonal basis.  The default is x itself.

\item[\code{nderiv}] the order of derivative to be evaluated.  The derivative must not exceed
one.  The default derivative is 0, meaning that the basis functions
themselves are evaluated.

\end{ldescription}
\end{Arguments}
\begin{Value}
a matrix of function values.  The number of rows equals the number of
arguments, and the number of columns equals the number of basis
\end{Value}
\begin{SeeAlso}\relax
\code{\LinkA{create.polygonal.basis}{create.polygonal.basis}}, 
\code{\LinkA{polygpen}{polygpen}}
\end{SeeAlso}
\begin{Examples}
\begin{ExampleCode}

#  set up a set of 21 argument values
x <- seq(0,1,0.05)
#  set up a set of 11 argument values
argvals <- seq(0,1,0.1)
#  with the default period (1) and derivative (0)
basismat <- polyg(x, argvals)
#  plot the basis functions
matplot(x, basismat, type="l")

\end{ExampleCode}
\end{Examples}

\HeaderA{polygpen}{Polygonal Penalty Matrix}{polygpen}
\keyword{smooth}{polygpen}
\begin{Description}\relax
Computes the matrix defining the roughness penalty for functions
expressed in terms of a polygonal basis.
\end{Description}
\begin{Usage}
\begin{verbatim}
polygpen(basisobj, Lfdobj=int2Lfd(1))
\end{verbatim}
\end{Usage}
\begin{Arguments}
\begin{ldescription}
\item[\code{basisobj}] a polygonal functional basis object.

\item[\code{Lfdobj}] either an integer that is either 0 or 1, or a
linear differential operator object of degree 0 or 1.

\end{ldescription}
\end{Arguments}
\begin{Details}\relax
a roughness penalty for a function $ x(t) $ is defined by
integrating the square of either the derivative of  $ x(t) $ or,
more generally, the result of applying a linear differential operator
$ L $ to it.  The only roughness penalty possible aside from
penalizing the size of the function itself is the integral
of the square of the first derivative, and
this is the default. To apply this roughness penalty, the matrix of
inner products produced by this function is necessary.
\end{Details}
\begin{Value}
a symmetric matrix of order equal to the number of basis functions
defined by the polygonal basis object.  Each element is the inner product
of two polygonal basis functions after applying the derivative or linear
differential operator defined by Lfdobj.
\end{Value}
\begin{SeeAlso}\relax
\code{\LinkA{create.polygonal.basis}{create.polygonal.basis}}, 
\code{\LinkA{polyg}{polyg}}
\end{SeeAlso}
\begin{Examples}
\begin{ExampleCode}

#  set up a sequence of 11 argument values
argvals <- seq(0,1,0.1)
#  set up the polygonal basis
basisobj <- create.polygonal.basis(argvals)
#  compute the 11 by 11 penalty matrix

# The following should work but doesn't;  2007.05.01
#penmat <- polygpen(basisobj)

\end{ExampleCode}
\end{Examples}

\HeaderA{powerbasis}{Power Basis Function Values}{powerbasis}
\keyword{smooth}{powerbasis}
\begin{Description}\relax
Evaluates a set of power basis functions, or a derivative of these
functions, at a set of arguments.
\end{Description}
\begin{Usage}
\begin{verbatim}
powerbasis(x, exponents, nderiv=0)
\end{verbatim}
\end{Usage}
\begin{Arguments}
\begin{ldescription}
\item[\code{x}] a vector of argument values at which the power basis functions are to
evaluated. Since exponents may be negative, for example after
differentiation, it is required that all argument values be positive.

\item[\code{exponents}] a vector of exponents defining the power basis functions.  If
$y$ is such a rate value, the corresponding basis function is
$x$ to the power $y$.  The number of basis functions is equal to the
number of exponents.

\item[\code{nderiv}] the derivative to be evaluated.  The derivative must not exceed the
order.  The default derivative is 0, meaning that the basis functions
themselves are evaluated.

\end{ldescription}
\end{Arguments}
\begin{Value}
a matrix of function values.  The number of rows equals the number of
arguments, and the number of columns equals the number of basis
functions.
\end{Value}
\begin{SeeAlso}\relax
\code{\LinkA{create.power.basis}{create.power.basis}}, 
\code{\LinkA{powerpen}{powerpen}}
\end{SeeAlso}
\begin{Examples}
\begin{ExampleCode}

#  set up a set of 10 positive argument values.
x <- seq(0.1,1,0.1)
#  compute values for three power basis functions
exponents <- c(0, 1, 2)
#  evaluate the basis matrix
basismat <- powerbasis(x, exponents)

\end{ExampleCode}
\end{Examples}

\HeaderA{powerpen}{Power Penalty Matrix}{powerpen}
\keyword{smooth}{powerpen}
\begin{Description}\relax
Computes the matrix defining the roughness penalty for functions
expressed in terms of a power basis.
\end{Description}
\begin{Usage}
\begin{verbatim}
powerpen(basisobj, Lfdobj=int2Lfd(2))
\end{verbatim}
\end{Usage}
\begin{Arguments}
\begin{ldescription}
\item[\code{basisobj}] a power basis object.

\item[\code{Lfdobj}] either a nonnegative integer or a linear differential operator object.

\end{ldescription}
\end{Arguments}
\begin{Details}\relax
A roughness penalty for a function $ x(t) $ is defined by
integrating the square of either the derivative of  $ x(t) $ or,
more generally, the result of applying a linear differential operator
$ L $ to it.  The most common roughness penalty is the integral of
the square of the second derivative, and
this is the default. To apply this roughness penalty, the matrix of
inner products produced by this function is necessary.
\end{Details}
\begin{Value}
a symmetric matrix of order equal to the number of basis functions
defined by the power basis object.  Each element is the inner product
of two power basis functions after applying the derivative or linear
differential operator defined by \code{Lfdobj}.
\end{Value}
\begin{SeeAlso}\relax
\code{\LinkA{create.power.basis}{create.power.basis}}, 
\code{\LinkA{powerbasis}{powerbasis}}
\end{SeeAlso}
\begin{Examples}
\begin{ExampleCode}

#  set up an power basis with 3 basis functions.
#  the powers are 0, 1, and 2.
basisobj <- create.power.basis(c(0,1),3,c(0,1,2))
#  compute the 3 by 3 matrix of inner products of second derivatives
#FIXME
#penmat <- powerpen(basisobj, 2)

\end{ExampleCode}
\end{Examples}

\HeaderA{predict.lmeWinsor}{Predict method for Winsorized linear model fits with mixed effects}{predict.lmeWinsor}
\keyword{models}{predict.lmeWinsor}
\begin{Description}\relax
Model predictions for object of class 'lmeWinsor'.
\end{Description}
\begin{Usage}
\begin{verbatim}
## S3 method for class 'lmeWinsor':
predict(object, newdata, level=Q, asList=FALSE,
      na.action=na.fail, ...)
\end{verbatim}
\end{Usage}
\begin{Arguments}
\begin{ldescription}
\item[\code{ object }] Object of class inheriting from 'lmeWinsor', representing a fitted
linear mixed-effects model.  

\item[\code{ newdata }] an optional data frame to be used for obtaining the predictions. All
variables used in the fixed and random effects models, as well as
the grouping factors, must be present in the data frame. If missing,
the fitted values are returned.

\item[\code{ level }] an optional integer vector giving the level(s) of grouping to be
used in obtaining the predictions. Level values increase from
outermost to innermost grouping, with level zero corresponding to
the population predictions. Defaults to the highest or innermost
level of grouping. 

\item[\code{ asList }] an optional logical value. If 'TRUE' and a single value is given in
'level', the returned object is a list with the predictions split by
groups; else the returned value is either a vector or a data frame,
according to the length of 'level'. 

\item[\code{na.action}] a function that indicates what should happen when 'newdata' contains
'NA's.  The default action ('na.fail') causes the function to print
an error message and terminate if there are any incomplete
observations. 

\item[\code{...}] additional arguments for other methods

\end{ldescription}
\end{Arguments}
\begin{Details}\relax
1.  Identify inputs and outputs as with \LinkA{lmeWinsor}{lmeWinsor}.  

2.  If 'newdata' are provided, clip all numeric xNames to
(object[["lower"]], object[["upper"]]). 

3.  Call \LinkA{predict.lme}{predict.lme}  

4.  Clip the responses to the relevant components of
(object[["lower"]], object[["upper"]]).

5.  Done.
\end{Details}
\begin{Value}
'predict.lmeWinsor' produces a vector of predictions or a matrix of
predictions with limits or a list, as produced by
\LinkA{predict.lme}{predict.lme}
\end{Value}
\begin{Author}\relax
Spencer Graves
\end{Author}
\begin{SeeAlso}\relax
\code{\LinkA{lmeWinsor}{lmeWinsor}}
\code{\LinkA{predict.lme}{predict.lme}}
\code{\LinkA{lmWinsor}{lmWinsor}}
\code{\LinkA{predict.lm}{predict.lm}}
\end{SeeAlso}
\begin{Examples}
\begin{ExampleCode}
fm1w <- lmeWinsor(distance ~ age, data = Orthodont,
                 random=~age|Subject)
# predict with newdata 
newDat <- data.frame(age=seq(0, 30, 2),
           Subject=factor(rep("na", 16)) )
pred1w <- predict(fm1w, newDat, level=0)

# fit with 10 percent Winsorization 
fm1w.1 <- lmeWinsor(distance ~ age, data = Orthodont,
                 random=~age|Subject, trim=0.1)
pred30 <- predict(fm1w.1)
stopifnot(all.equal(as.numeric(
              quantile(Orthodont$distance, c(.1, .9))),
          range(pred30)) )

\end{ExampleCode}
\end{Examples}

\HeaderA{predict.lmWinsor}{Predict method for Winsorized linear model fits}{predict.lmWinsor}
\keyword{models}{predict.lmWinsor}
\begin{Description}\relax
Model predictions for object of class 'lmWinsor'.
\end{Description}
\begin{Usage}
\begin{verbatim}
## S3 method for class 'lmWinsor':
predict(object, newdata, se.fit = FALSE,
     scale = NULL, df = Inf, 
     interval = c("none", "confidence", "prediction"),
     level = 0.95, type = c("response", "terms"), 
     terms = NULL, na.action = na.pass,
     pred.var = res.var/weights, weights = 1, ...)
\end{verbatim}
\end{Usage}
\begin{Arguments}
\begin{ldescription}
\item[\code{ object }] Object of class inheriting from 'lmWinsor'  

\item[\code{ newdata }] An optional data frame in which to look for variables with which to
predict.  If omitted, the fitted values are used. 

\item[\code{ se.fit}] a switch indicating if standard errors of predictions are required 

\item[\code{ scale }] Scale parameter for std.err. calculation  

\item[\code{df}] degrees of freedom for scale 

\item[\code{interval}] type of prediction (response or model term) 

\item[\code{ level }] Tolerance/confidence level 
\item[\code{ type }] Type of prediction (response or model term);  see
\LinkA{predict.lm}{predict.lm} 

\item[\code{terms}] If 'type="terms"', which terms (default is all terms)

\item[\code{na.action}] function determining what should be done with missing values in
'newdata'.  The default is to predict 'NA'. 

\item[\code{ pred.var }] the variance(s) for future observations to be assumed for prediction
intervals.  See \LinkA{predict.lm}{predict.lm} 'Details'.

\item[\code{ weights }] variance weights for prediction. This can be a numeric vector or a
one-sided model formula. In the latter case, it is interpreted as an
expression evaluated in 'newdata'

\item[\code{...}] additional arguments for other methods

\end{ldescription}
\end{Arguments}
\begin{Details}\relax
1.  Identify inputs and outputs via mdly <- mdlx <- formula(object);
mdly[[3]] <- NULL;  mdlx[[2]] <- NULL;  xNames <- all.vars(mdlx);
yNames <- all.vars(mdly).  Give an error if as.character(mdly[[2]]) !=
yNames.  

2.  If 'newdata' are provided, clip all xNames to (object[["lower"]],
object[["upper"]]). 

3.  Call \LinkA{predict.lm}{predict.lm}  

4.  Clip the responses to the relevant components of
(object[["lower"]], object[["upper"]]).

5.  Done.
\end{Details}
\begin{Value}
'predict.lmWinsor' produces a vector of predictions or a matrix of
predictions with limits or a list, as produced by
\LinkA{predict.lm}{predict.lm}
\end{Value}
\begin{Author}\relax
Spencer Graves
\end{Author}
\begin{SeeAlso}\relax
\code{\LinkA{lmWinsor}{lmWinsor}}
\code{\LinkA{predict.lm}{predict.lm}}
\end{SeeAlso}
\begin{Examples}
\begin{ExampleCode}
# example from 'anscombe' 
lm.1 <- lmWinsor(y1~x1, data=anscombe)

newD <- data.frame(x1=seq(1, 22, .1))
predW <- predict(lm.1, newdata=newD) 
plot(y1~x1, anscombe, xlim=c(1, 22)) 
lines(newD[["x1"]], predW, col='blue')
abline(h=lm.1[['lower']]['y1'], col='red', lty='dashed') 
abline(h=lm.1[['upper']]['y1'], col='red', lty='dashed')
abline(v=lm.1[['lower']]['x1'], col='green', lty='dashed') 
abline(v=lm.1[['upper']]['x1'], col='green', lty='dashed') 

\end{ExampleCode}
\end{Examples}

\HeaderA{project.basis}{Approximate Functional Data Using a Basis}{project.basis}
\keyword{smooth}{project.basis}
\begin{Description}\relax
A vector or matrix of discrete data is projected into the space
spanned by the values of a set of basis functions.  This amounts to
a least squares regression of the data on to the values of the basis
functions.  A small penalty can be applied to deal with stiautions in
which the number of basis functions exceeds the number of basis points.
This function is used with function \code{data2fd}, and is not
normally used directly in a functional data analysis.
\end{Description}
\begin{Usage}
\begin{verbatim}
project.basis(y, argvals, basisobj, penalize=FALSE)
\end{verbatim}
\end{Usage}
\begin{Arguments}
\begin{ldescription}
\item[\code{y}] a vector or matrix of discrete data.

\item[\code{argvals}] a vector containing the argument values correspond to the
values in \code{y}.

\item[\code{basisobj}] a basis object.

\item[\code{penalize}] a logical variable.  If TRUE, a small roughness penalty is applied
to ensure that the linear equations defining the least squares
solution are linearly independent or nonsingular.

\end{ldescription}
\end{Arguments}
\begin{Value}
the matrix of coefficients defining the least squares approximation.
This matrix has as many rows are there are basis functions, as many
columns as there are curves, and if the data are multivariate, as many
layers as there are functions.
\end{Value}
\begin{SeeAlso}\relax
\code{\LinkA{data2fd}{data2fd}}
\end{SeeAlso}

\HeaderA{quadset}{Quadrature points and weights for Simpson's rule}{quadset}
\keyword{smooth}{quadset}
\begin{Description}\relax
Set up quadrature points and weights for Simpson's rule.
\end{Description}
\begin{Usage}
\begin{verbatim}
quadset(nquad=5, basisobj=NULL, breaks) 
\end{verbatim}
\end{Usage}
\begin{Arguments}
\begin{ldescription}
\item[\code{nquad}] an odd integer at least 5 giving the number of evenly spaced
Simpson's rule quadrature points to use over each interval
(breaks[i], breaks[i+1]). 

\item[\code{basisobj}] the basis object that will contain the quadrature points and weights

\item[\code{breaks}] optional interval boundaries.  If this is provided, the first value
must be the initial point of the interval over which the basis is
defined, and the final value must be the end point.  If this is not
supplied, and 'basisobj' is of type 'bspline', the knots are used as
these values.

\end{ldescription}
\end{Arguments}
\begin{Details}\relax
Set up quadrature points and weights for Simpson's rule and store
information in the output 'basisobj'.  Simpson's rule is used to
integrate a function between successive values in vector 'breaks'.
That is, over each interval (breaks[i], breaks[i+1]).  Simpson's rule
uses 'nquad' equally spaced quadrature points over this interval,
starting with the the left boundary and ending with the right
boundary.  The quadrature weights are the values
delta*c(1,4,2,4,2,4,..., 2,4,1) where 'delta' is the difference
between successive quadrature points, that is, delta =
(breaks[i-1]-breaks[i])/(nquad-1).
\end{Details}
\begin{Value}
If is.null(basisobj), quadset returns a 'quadvals' matrix with columns
quadpts and quadwts.  Otherwise, it returns basisobj with the
two components set as follows:   

\begin{ldescription}
\item[\code{quadvals}] cbind(quadpts=quadpts, quadwts=quadwts)    

\item[\code{value}] a list with two components containing eval.basis(quadpts, basisobj,
ival-1) for ival=1, 2.  

\end{ldescription}
\end{Value}
\begin{SeeAlso}\relax
\code{\LinkA{create.bspline.basis}{create.bspline.basis}}
\code{\LinkA{eval.basis}{eval.basis}}
\end{SeeAlso}
\begin{Examples}
\begin{ExampleCode}
(qs7.1 <- quadset(nquad=7, breaks=c(0, .3, 1)))
# cbind(quadpts=c(seq(0, 0.3, length=7),
#              seq(0.3, 1, length=7)), 
#    quadwts=c((0.3/18)*c(1, 4, 2, 4, 2, 4, 1),
#              (0.7/18)*c(1, 4, 2, 4, 2, 4, 1) ) )

# The simplest basis currently available with this function:
bspl2.2 <- create.bspline.basis(norder=2, breaks=c(0,.5, 1))

bspl2.2a <- quadset(basisobj=bspl2.2)
bspl2.2a$quadvals
# cbind(quadpts=c((0:4)/8, .5+(0:4)/8),
#       quadwts=rep(c(1,4,2,4,1)/24, 2) )
bspl2.2a$values
# a list of length 2
# [[1]] = matrix of dimension c(10, 3) with the 3 basis 
#      functions evaluated at the 10 quadrature points:
# values[[1]][, 1] = c(1, .75, .5, .25, rep(0, 6))
# values[[1]][, 2] = c(0, .25, .5, .75, 1, .75, .5, .25, 0)
# values[[1]][, 3] = values[10:1, 1]
#
# values[[2]] = matrix of dimension c(10, 3) with the
#     first derivative of values[[1]], being either 
#    -2, 0, or 2.  
\end{ExampleCode}
\end{Examples}

\HeaderA{refinery}{Reflux and tray level in a refinery}{refinery}
\keyword{datasets}{refinery}
\begin{Description}\relax
194 observations on reflux and "tray 47 level" in a distallation
column in an oil refinery.
\end{Description}
\begin{Format}\relax
A data.frame with the following components:  
\describe{
\item[Time] observation time 0:193 

\item[Reflux] reflux flow centered on the mean of the first 60 observations 

\item[Tray47] tray 47 level centered on the mean of the first 60
observations 

}
\end{Format}
\begin{Source}\relax
Ramsay, James O., and Silverman, Bernard W. (2006), \emph{Functional
Data Analysis, 2nd ed.}, Springer, New York, p. 4, Figure 1.4, and
chapter 17.
\end{Source}
\begin{Examples}
\begin{ExampleCode}
    attach(refinery)
# allow space for an axis on the right 
    op <- par(mar=c(5, 4, 4, 5)+0.1)
# plot uval 
    plot(Time, Reflux, type="l", bty="n")
# add yval 
    y.u <- diff(range(Tray47))/diff(range(Reflux))
    u0 <- min(Reflux)
    y0 <- min(Tray47)

    lines(Time, u0+(Tray47-y0)/y.u, lty=3, lwd=1.5, col="red")
    y.tick <- pretty(range(Tray47))
    axis(4, at=u0+(y.tick)/y.u, labels=y.tick, col="red", lty=3,
            lwd=1.5)
# restore previous plot margins
    par(op)
    detach(refinery)
\end{ExampleCode}
\end{Examples}

\HeaderA{register.fd}{Register Functional Data Objects Using a Continuous Criterion}{register.fd}
\keyword{smooth}{register.fd}
\begin{Description}\relax
criterion.  By aligned is meant that the shape of each curve is matched
as closely as possible to that of the target by means of a smooth
increasing transformation of the argument, or a warping function.
\end{Description}
\begin{Usage}
\begin{verbatim}
register.fd(y0fd=NULL, yfd=NULL, WfdParobj=c(Lfdobj=2, lambda=1),
            conv=1e-04, iterlim=20, dbglev=1, periodic=FALSE, crit=2)
\end{verbatim}
\end{Usage}
\begin{Arguments}
\begin{ldescription}
\item[\code{y0fd}] a functional data object defining the target for registration.

If \code{yfd} is NULL and y0fd is a multivariate data object, then
y0fd is assigned to yfd and y0fd is replaced by its mean.

Alternatively, if \code{yfd} is a multivariate functional data
object and y0fd is missing, y0fd is replaced by the mean of
\code{y0fd}.  

Otherwise, y0fd must be a univariate functional data object taken as
the target to which \code{yfd} is registered.  

\item[\code{yfd}] a multivariate functional data object defining the functions to be
registered to target \code{y0fd}.  If it is NULL and \code{y0fd} is
a multivariate functional data object, yfd takes the value of 
\code{y0fd}.  

\item[\code{WfdParobj}] a functional parameter object for a single function.  This is used
as the initial value in the estimation of a function $W(t)$ that
defines the warping function $h(t)$ that registers a particular
curve. The object also contains information on a roughness penalty 
and smoothing parameter to control the roughness of $h(t)$.

Alternatively, this can be a vector or a list with components named
\code{Lfdobj} and \code{lambda}, which are passed as arguments to
\code{fdPar} to create the functional parameter form of WfdParobj
required by the rest of the register.fd algorithm.

The default \code{Lfdobj} of 2 penalizes curvature, thereby
preferring no warping of time, with \code{lambda} indicating the
strength of that preference.  A common alternative is \code{Lfdobj}
= 3, penalizing the rate of change of curvature.   

\item[\code{conv}] a criterion for convergence of the iterations.

\item[\code{iterlim}] a limit on the number of iterations.

\item[\code{dbglev}] either 0, 1, or 2.  This controls the amount information printed out
on each iteration, with 0 implying no output, 1 intermediate output
level, and 2 full output.  (If this is run with output buffering
such as used with S-Plus, it may be necessary to turn off the output
buffering to actually get the progress reports before the completion
of computations.)  

\item[\code{periodic}] a logical variable:  if \code{TRUE}, the functions are considered to
be periodic, in which case a constant can be added to all argument
values after they are warped. 

\item[\code{crit}] an integer that is either 1 or 2 that indicates the nature of the
continuous registration criterion that is used.  If 1, the criterion is
least squares, and if 2, the criterion is the minimum eigenvalue of a
cross-product matrix.  In general, criterion 2 is to be preferred.

\end{ldescription}
\end{Arguments}
\begin{Details}\relax
The warping function that smoothly and monotonely transforms the
argument is defined by \code{Wfd} is the same as that defines the
monotone smoothing function in for function \code{smooth.monotone.}
See the help file for that function for further details.
\end{Details}
\begin{Value}
a named list of length 3 containing the following components:

\begin{ldescription}
\item[\code{regfd}] A functional data object containing the registered functions.

\item[\code{Wfd}] A functional data object containing the functions $h W(t)$
that define the warping functions $h(t)$.

\item[\code{shift}] If the functions are periodic, this is a vector of time shifts.

\end{ldescription}
\end{Value}
\begin{Source}\relax
Ramsay, James O., and Silverman, Bernard W. (2006), \emph{Functional
Data Analysis, 2nd ed.}, Springer, New York.

Ramsay, James O., and Silverman, Bernard W. (2002), \emph{Applied
Functional Data Analysis}, Springer, New York, ch. 6 \& 7.
\end{Source}
\begin{SeeAlso}\relax
\code{\LinkA{smooth.monotone}{smooth.monotone}}, 
\code{\LinkA{smooth.morph}{smooth.morph}}
\end{SeeAlso}
\begin{Examples}
\begin{ExampleCode}
#See the analyses of the growth data for examples.
##
## 1.  Simplest call
##
# Specify smoothing weight 
lambda.gr2.3 <- .03

# Specify what to smooth, namely the rate of change of curvature
Lfdobj.growth    <- 2 

# Establish a B-spline basis
nage <- length(growth$age)
norder.growth <- 6
nbasis.growth <- nage + norder.growth - 2
rng.growth <- range(growth$age)
# 1 18 
wbasis.growth <- create.bspline.basis(rangeval=rng.growth,
                   nbasis=nbasis.growth, norder=norder.growth,
                   breaks=growth$age)

# Smooth consistent with the analysis of these data
# in afda-ch06.R, and register to individual smooths:  
cvec0.growth <- matrix(0,nbasis.growth,1)
Wfd0.growth  <- fd(cvec0.growth, wbasis.growth)
growfdPar2.3 <- fdPar(Wfd0.growth, Lfdobj.growth, lambda.gr2.3)
# Create a functional data object for all the boys
hgtmfd.all <- with(growth, smooth.basis(age, hgtm, growfdPar2.3))

nBoys <- 2
# nBoys <- dim(growth[["hgtm"]])[2]
# register.fd takes time, so use only 2 curves as an illustration
# to minimize compute time in this example;  

#Alternative to subsetting later is to subset now:  
#hgtmfd.all<-with(growth,smooth.basis(age, hgtm[,1:nBoys],growfdPar2.3))

# Register the growth velocity rather than the
# growth curves directly 
smBv <- deriv(hgtmfd.all$fd, 1)

# This takes time, so limit the number of curves registered to nBoys

## Not run: 
smB.reg.0 <- register.fd(smBv[1:nBoys])

smB.reg.1 <- register.fd(smBv[1:nBoys],WfdParobj=c(Lfdobj=Lfdobj.growth, lambda=lambda.gr2.3))

##
## 2.  Call providing the target
##

smBv.mean <- deriv(mean(hgtmfd.all$fd[1:nBoys]), 1)
smB.reg.2a <- register.fd(smBv.mean, smBv[1:nBoys],
               WfdParobj=c(Lfdobj=Lfdobj.growth, lambda=lambda.gr2.3))

smBv.mean <- mean(smBv[1:nBoys]) 
smB.reg.2 <- register.fd(smBv.mean, smBv[1:nBoys],
               WfdParobj=c(Lfdobj=Lfdobj.growth, lambda=lambda.gr2.3))
all.equal(smB.reg.1, smB.reg.2) 

##
## 3.  Call using WfdParobj
##

# Create a dummy functional data object
# to hold the functional data objects for the
# time warping function
# ... start with a zero matrix (nbasis.growth, nBoys) 
smBc0 <- matrix(0, nbasis.growth, nBoys)
# ... convert to a functional data object 
smBwfd0 <- fd(smBc0, wbasis.growth)
# ... convert to a functional parameter object 
smB.wfdPar <- fdPar(smBwfd0, Lfdobj.growth, lambda.gr2.3)

smB.reg.3<- register.fd(smBv[1:nBoys], WfdParobj=smB.wfdPar)
all.equal(smB.reg.1, smB.reg.3)
## End(Not run)

\end{ExampleCode}
\end{Examples}

\HeaderA{sd.fd}{Standard Deviation of Functional Data}{sd.fd}
\aliasA{std.fd}{sd.fd}{std.fd}
\aliasA{stddev.fd}{sd.fd}{stddev.fd}
\aliasA{stdev.fd}{sd.fd}{stdev.fd}
\keyword{smooth}{sd.fd}
\begin{Description}\relax
Evaluate the standard deviation of a set of functions in a functional
data object.
\end{Description}
\begin{Usage}
\begin{verbatim}
sd.fd(fdobj)
std.fd(fdobj)
stdev.fd(fdobj)
stddev.fd(fdobj)
\end{verbatim}
\end{Usage}
\begin{Arguments}
\begin{ldescription}
\item[\code{fdobj}] a functional data object.

\end{ldescription}
\end{Arguments}
\begin{Details}\relax
The multiple aliases are provided for compatibility with previous
versions and with other languages.  The name for the standard
deviation function in R is 'sd'.  Matlab uses 'std'.  S-Plus and
Microsoft Excal use 'stdev'.  'stddev' was used in a previous version
of the 'fda' package and is retained for compatibility.
\end{Details}
\begin{Value}
a functional data object with a single replication
that contains the standard deviation of the one or several functions in
the object \code{fdobj}.
\end{Value}
\begin{SeeAlso}\relax
\code{\LinkA{mean.fd}{mean.fd}}, 
\code{\LinkA{sum.fd}{sum.fd}}, 
\code{\LinkA{center.fd}{center.fd}}
\end{SeeAlso}
\begin{Examples}
\begin{ExampleCode}
liptime  <- seq(0,1,.02)
liprange <- c(0,1)

#  -------------  create the fd object -----------------
#       use 31 order 6 splines so we can look at acceleration

nbasis <- 51
norder <- 6
lipbasis <- create.bspline.basis(liprange, nbasis, norder)
lipbasis <- create.bspline.basis(liprange, nbasis, norder)

#  ------------  apply some light smoothing to this object  -------

Lfdobj   <- int2Lfd(4)
lambda   <- 1e-12
lipfdPar <- fdPar(lipbasis, Lfdobj, lambda)

lipfd <- smooth.basis(liptime, lip, lipfdPar)$fd
names(lipfd$fdnames) = c("Normalized time", "Replications", "mm")

lipstdfd <- sd.fd(lipfd)
plot(lipstdfd)

all.equal(lipstdfd, std.fd(lipfd))
all.equal(lipstdfd, stdev.fd(lipfd))
all.equal(lipstdfd, stddev.fd(lipfd))

\end{ExampleCode}
\end{Examples}

\HeaderA{smooth.basis}{Smooth Data with an Indirectly Specified Roughness Penalty}{smooth.basis}
\keyword{smooth}{smooth.basis}
\begin{Description}\relax
This is the main function for smoothing data using a roughness
penalty.  Unlike function \code{data2fd}, which does not employ a
rougness penalty, this function controls the nature and degree of
smoothing by penalyzing a measure of rougness.  Roughness is definable
in a wide variety of ways using either derivatives or a linear
differential operator.
\end{Description}
\begin{Usage}
\begin{verbatim}
smooth.basis(argvals, y, fdParobj, wtvec=rep(1, length(argvals)),
             fdnames=NULL)
\end{verbatim}
\end{Usage}
\begin{Arguments}
\begin{ldescription}
\item[\code{argvals}] a vector of argument values correspond to the observations in array
\code{y}.

\item[\code{y}] an array containing values of curves at discrete sampling points or
argument values. If the array is a matrix, the rows must correspond
to argument values and columns to replications, and it will be
assumed that there is only one variable per observation.  If
\code{y} is a three-dimensional array, the first dimension
corresponds to argument values, the second to replications, and the
third to variables within replications.  If \code{y} is a vector,
only one replicate and variable are assumed.  

\item[\code{fdParobj}] a functional parameter object, a functional data object or a
functional basis object.  If the object is a functional parameter
object, then the linear differential operator object and the
smoothing parameter in this object define the roughness penalty.  If
the object is a functional data object, the basis within this object
is used without a roughness penalty, and this is also the case if
the object is a functional basis object.  In these latter two cases,
\code{smooth.basis} is essentially the same as \code{data2fd}.

\item[\code{wtvec}] a vector of the same length as \code{argvals} containing weights for
the values to be smoothed. 

\item[\code{fdnames}] a list of length 3 containing character vectors of names for the
following: 

\Itemize{
\item[args] name for each observation or point in time at which data are
collected for each 'rep', unit or subject.

\item[reps] name for each 'rep', unit or subject.

\item[fun] name for each 'fun' or (response) variable measured repeatedly
(per 'args') for each 'rep'.

}

\end{ldescription}
\end{Arguments}
\begin{Details}\relax
If the smoothing parameter \code{lambda} is zero, there is no penalty
on roughness.  As lambda increases, usually in logarithmic terms, the
penalty on roughness increases and the fitted curves become more and
more smooth.  Ultimately, the curves are forced to have zero roughness
in the sense of being in the null space of the linear differential
operator object \code{Lfdobj}that is a member of the \code{fdParobj}.

For example, a common choice of roughness penalty is the integrated
square of the second derivative.  This penalizes curvature.  Since the
second derivative of a straight line is zero, very large values of
\code{lambda} will force the fit to become linear.  It is also
possible to control the amount of roughness by using a degrees of
freedom measure.  The value equivalent to \code{lambda} is found in
the list returned by the function.  On the other hand, it is possible
to specify a degrees of freedom value, and then use function
\code{df2lambda} to determine the equivalent value of \code{lambda}.
One should not put complete faith in any automatic method for
selecting \code{lambda}, including the GCV method. There are many
reasons for this.  For example, if derivatives are required, then the
smoothing level that is automatically selected may give unacceptably
rough derivatives.  These methods are also highly sensitive to the
assumption of independent errors, which is usually dubious with
functional data.  The best advice is to start with the value
minimizing the \code{gcv} measure, and then explore \code{lambda}
values a few log units up and down from this value to see what the
smoothing function and its derivatives look like.  The function
\code{plotfit.fd} was designed for this purpose.

An alternative to using \code{smooth.basis} is to first represent
the data in a basis system with reasonably high resolution using
\code{data2fd}, and then smooth the resulting functional data object
using function \code{smooth.fd}.
\end{Details}
\begin{Value}
an object of class \code{fdSmooth}, which is a names list of length 6
with the following components: 

\begin{ldescription}
\item[\code{fd}] a functional data object containing a smooth of the data. 

\item[\code{df}] a degrees of freedom measure of the smooth

\item[\code{gcv}] the value of the generalized cross-validation or GCV criterion.  If
there are multiple curves, this is a vector of values, one per
curve.  If the smooth is multivariate, the result is a matrix of gcv
values, with columns corresponding to variables.

\deqn{gcv = n*SSE/((n-df)^2)}{}

\item[\code{SSE}] the error sums of squares.  SSE is a vector or a matrix of the same
size as GCV. 

\item[\code{penmat}] the penalty matrix.

\item[\code{y2cMap}] the matrix mapping the data to the coefficients.

\end{ldescription}
\end{Value}
\begin{SeeAlso}\relax
\code{\LinkA{data2fd}{data2fd}}, \code{\LinkA{df2lambda}{df2lambda}}, 
\code{\LinkA{lambda2df}{lambda2df}}, \code{\LinkA{lambda2gcv}{lambda2gcv}}, 
\code{\LinkA{plot.fd}{plot.fd}}, \code{\LinkA{project.basis}{project.basis}}, 
\code{\LinkA{smooth.fd}{smooth.fd}}, \code{\LinkA{smooth.monotone}{smooth.monotone}}, 
\code{\LinkA{smooth.pos}{smooth.pos}}, \code{\LinkA{smooth.basisPar}{smooth.basisPar}}
\end{SeeAlso}
\begin{Examples}
\begin{ExampleCode}
##
## Example 1:  Inappropriate smoothing  
##
# A toy example that creates problems with
# data2fd:  (0,0) -> (0.5, -0.25) -> (1,1)
b2.3 <- create.bspline.basis(norder=2, breaks=c(0, .5, 1))
# 3 bases, order 2 = degree 1 =
# continuous, bounded, locally linear
fdPar2 <- fdPar(b2.3, Lfdobj=2, lambda=1)

## Not run: 
# Penalize excessive slope Lfdobj=1;  
# second derivative Lfdobj=2 is discontinuous,
# so the following generates an error:
  fd2.3s0 <- smooth.basis(0:1, 0:1, fdPar2)
Derivative of order 2 cannot be taken for B-spline of order 2 
Probable cause is a value of the nbasis argument
 in function create.basis.fd that is too small.
Error in bsplinepen(basisobj, Lfdobj, rng) :
## End(Not run)

##
## Example 2.  Better 
##
b3.4 <- create.bspline.basis(norder=3, breaks=c(0, .5, 1))
# 4 bases, order 3 = degree 2 =
# continuous, bounded, locally quadratic 
fdPar3 <- fdPar(b3.4, lambda=1)
# Penalize excessive slope Lfdobj=1;  
# second derivative Lfdobj=2 is discontinuous.
fd3.4s0 <- smooth.basis(0:1, 0:1, fdPar3)

plot(fd3.4s0$fd)
# same plot via plot.fdSmooth
plot(fd3.4s0) 

##
## Example 3.  lambda = 1, 0.0001, 0
##
#  Example 3.  lambda = 1 
#  Shows the effects of three levels of smoothing
#  where the size of the third derivative is penalized.
#  The null space contains quadratic functions.
x <- seq(-1,1,0.02)
y <- x + 3*exp(-6*x^2) + rnorm(rep(1,101))*0.2
#  set up a saturated B-spline basis
basisobj <- create.bspline.basis(c(-1,1), 101)

fdPar1 <- fdPar(basisobj, 2, lambda=1)
result1  <- smooth.basis(x, y, fdPar1)
with(result1, c(df, gcv, SSE))

#  Example 2.  lambda = 0.0001
fdPar.0001 <- fdPar(basisobj, 2, lambda=0.0001)
result2  <- smooth.basis(x, y, fdPar.0001)
with(result2, c(df, gcv, SSE))
# less smoothing, more degrees of freedom,
# smaller gcv, smaller SSE 

#  Example 3.  lambda = 0
fdPar0 <- fdPar(basisobj, 2, lambda=0)
result3  <- smooth.basis(x, y, fdPar0)
with(result3, c(df, gcv, SSE))
# Saturate fit:  number of observations = nbasis 
# with no smoothing, so degrees of freedom = nbasis,
# gcv = Inf indicating overfitting;
# SSE = 0 (to within roundoff error)

plot(x,y)           # plot the data
lines(result1[['fd']], lty=2)  #  add heavily penalized smooth
lines(result2[['fd']], lty=1)  #  add reasonably penalized smooth
lines(result3[['fd']], lty=3)  #  add smooth without any penalty
legend(-1,3,c("1","0.0001","0"),lty=c(2,1,3))

plotfit.fd(y, x, result2[['fd']])  # plot data and smooth

##
## Example 4.  Supersaturated
##
basis104 <- create.bspline.basis(c(-1,1), 104)

fdPar104.0 <- fdPar(basis104, 2, lambda=0) 
result104.0  <- smooth.basis(x, y, fdPar104.0)
with(result104.0, c(df, gcv, SSE))

plotfit.fd(y, x, result104.0[['fd']], nfine=501)
# perfect (over)fit
# Need lambda > 0.

##
## Example 5.  gait
##
gaittime  <- (1:20)/21
gaitrange <- c(0,1)
gaitbasis <- create.fourier.basis(gaitrange,21)
lambda    <- 10^(-11.5)
harmaccelLfd <- vec2Lfd(c(0, 0, (2*pi)^2, 0))

gaitfdPar <- fdPar(gaitbasis, harmaccelLfd, lambda)
gaitfd <- smooth.basis(gaittime, gait, gaitfdPar)$fd
## Not run: 
# by default creats multiple plots, asking for a click between plots 
plotfit.fd(gait, gaittime, gaitfd)
## End(Not run)
\end{ExampleCode}
\end{Examples}

\HeaderA{smooth.basisPar}{Smooth Data Using a Directly Specified Roughness Penalty}{smooth.basisPar}
\keyword{smooth}{smooth.basisPar}
\begin{Description}\relax
Smooth (argvals, y) data with roughness penalty defined by the
remaining arguments.
\end{Description}
\begin{Usage}
\begin{verbatim}
smooth.basisPar(argvals, y, fdobj=NULL, Lfdobj=NULL,
      lambda=0, estimate=TRUE, penmat=NULL,
      wtvec=rep(1, length(argvals)), fdnames=NULL)
\end{verbatim}
\end{Usage}
\begin{Arguments}
\begin{ldescription}
\item[\code{argvals}] a vector of argument values correspond to the observations in array
\code{y}.

\item[\code{y}] an array containing values of curves at discrete sampling points or
argument values. If the array is a matrix, the rows must correspond
to argument values and columns to replications, and it will be
assumed that there is only one variable per observation.  If
\code{y} is a three-dimensional array, the first dimension
corresponds to argument values, the second to replications, and the
third to variables within replications.  If \code{y} is a vector,
only one replicate and variable are assumed.  

\item[\code{fdobj}] One of the following:

\Itemize{
\item[fd] a functional data object (class \code{fd})
\item[basisfd] a functional basis object (class \code{basisfd}, which is
converted to a functional data object with the identity matrix
as the coefficient matrix. 

\item[fdPar] a functional parameter object (class \code{fdPar})

\item[integer] an integer giving the order of a B-spline basis, which is
further converted to a functional data object with the identity
matrix as the coefficient matrix.  

\item[matrix or array] replaced by fd(fdobj)     
\item[NULL] Defaults to fdobj = create.bspline.basis(argvals).

}

\item[\code{Lfdobj}] either a nonnegative integer or a linear differential operator
object.  If NULL and fdobj[['type']] == 'bspline', Lfdobj =
int2Lfd(max(0, norder-2)), where norder = order of fdobj.   

\item[\code{lambda}] a nonnegative real number specifying the amount of smoothing
to be applied to the estimated functional parameter.

\item[\code{estimate}] a logical value:  if \code{TRUE}, the functional parameter is
estimated, otherwise, it is held fixed.

\item[\code{penmat}] a roughness penalty matrix.  Including this can eliminate the need
to compute this matrix over and over again in some types of
calculations.

\item[\code{wtvec}] a vector of the same length as \code{argvals} containing weights for
the values to be smoothed. 

\item[\code{fdnames}] a list of length 3 containing character vectors of names for the
following: 

\Itemize{
\item[args] name for each observation or point in time at which data are
collected for each 'rep', unit or subject.

\item[reps] name for each 'rep', unit or subject.

\item[fun] name for each 'fun' or (response) variable measured repeatedly
(per 'args') for each 'rep'.

}

\end{ldescription}
\end{Arguments}
\begin{Details}\relax
1.  if(is.null(fdobj))fdobj <- create.bspline.basis(argvals).  Else
if(is.integer(fdobj)) fdobj <- create.bspline.basis(argvals, norder =
fdobj) 

2.  fdPar

3.  smooth.basis
\end{Details}
\begin{Value}
The output of a call to 'smooth.basis', which is a named list of
length 6 and class \code{fdSmooth} with the following components:  

\begin{ldescription}
\item[\code{fd}] a functional data object that smooths the data.

\item[\code{df}] a degrees of freedom measure of the smooth

\item[\code{gcv}] the value of the generalized cross-validation or GCV criterion.  If
there are multiple curves, this is a vector of values, one per
curve.  If the smooth is multivariate, the result is a matrix of gcv
values, with columns corresponding to variables.  

\item[\code{SSE}] the error sums of squares.  SSE is a vector or a matrix of the same
size as 'gcv'. 

\item[\code{penmat}] the penalty matrix.

\item[\code{y2cMap}] the matrix mapping the data to the coefficients.

\end{ldescription}
\end{Value}
\begin{References}\relax
Ramsay, James O., and Silverman, Bernard W. (2005), \emph{Functional 
Data Analysis, 2nd ed.}, Springer, New York. 

Ramsay, James O., and Silverman, Bernard W. (2002), \emph{Applied
Functional Data Analysis}, Springer, New York.
\end{References}
\begin{SeeAlso}\relax
\code{\LinkA{Data2fd}{Data2fd}}, 
\code{\LinkA{df2lambda}{df2lambda}}, 
\code{\LinkA{fdPar}{fdPar}}, 
\code{\LinkA{lambda2df}{lambda2df}}, 
\code{\LinkA{lambda2gcv}{lambda2gcv}}, 
\code{\LinkA{plot.fd}{plot.fd}}, 
\code{\LinkA{project.basis}{project.basis}}, 
\code{\LinkA{smooth.basis}{smooth.basis}}, 
\code{\LinkA{smooth.fd}{smooth.fd}}, 
\code{\LinkA{smooth.monotone}{smooth.monotone}}, 
\code{\LinkA{smooth.pos}{smooth.pos}}
\end{SeeAlso}
\begin{Examples}
\begin{ExampleCode}
##
## simplest call
##
girlGrowthSm <- with(growth, smooth.basisPar(argvals=age, y=hgtf))
plot(girlGrowthSm$fd, xlab="age", ylab="height (cm)",
         main="Girls in Berkeley Growth Study" )
plot(deriv(girlGrowthSm$fd), xlab="age", ylab="growth rate (cm / year)",
         main="Girls in Berkeley Growth Study" )
plot(deriv(girlGrowthSm$fd, 2), xlab="age",
        ylab="growth acceleration (cm / year^2)",
        main="Girls in Berkeley Growth Study" )
#  Shows the effects of smoothing
#  where the size of the third derivative is penalized.
#  The null space contains quadratic functions.

##
## Another simple call
##
lipSm <- smooth.basisPar(liptime, lip)
plot(lipSm)
# oversmoothing
plot(smooth.basisPar(liptime, lip, lambda=1e-9))
# more sensible 

##
## A third example 
##

x <- seq(-1,1,0.02)
y <- x + 3*exp(-6*x^2) + sin(1:101)/2
# sin not rnorm to make it easier to compare
# results across platforms 

#  set up a saturated B-spline basis
basisobj101 <- create.bspline.basis(x)
fdParobj101 <- fdPar(basisobj101, 2, lambda=1)
result101  <- smooth.basis(x, y, fdParobj101)

resultP <- smooth.basisPar(argvals=x, y=y, fdobj=basisobj101, lambda=1)

all.equal(result101, resultP)

# TRUE 

result4 <- smooth.basisPar(argvals=x, y=y, fdobj=4, lambda=1)

all.equal(resultP, result4)

# TRUE 

result4. <- smooth.basisPar(argvals=x, y=y, lambda=1)

all.equal(resultP, result4.)

# TRUE

with(result4, c(df, gcv)) #  display df and gcv measures

result4.4 <- smooth.basisPar(argvals=x, y=y, lambda=1e-4)
with(result4.4, c(df, gcv)) #  display df and gcv measures
# less smoothing, more degrees of freedom, better fit 

plot(result4.4)
lines(result4, col='green')
lines(result4$fd, col='green') # same as lines(result4, ...)

result4.0 <- smooth.basisPar(x, y, basisobj101, lambda=0)

result4.0a <- smooth.basisPar(x, y, lambda=0)

all.equal(result4.0, result4.0a)


with(result4.0, c(df, gcv)) #  display df and gcv measures
# no smoothing, degrees of freedom = number of points 
# but generalized cross validation = Inf
# suggesting overfitting.  

##
## fdnames?
##
girlGrow12 <- with(growth, smooth.basisPar(argvals=age, y=hgtf[, 1:2],
              fdnames=c('age', 'girl', 'height')) ) 
girlGrow12. <- with(growth, smooth.basisPar(argvals=age, y=hgtf[, 1:2],
    fdnames=list(age=age, girl=c('Carol', 'Sally'), value='height')) )

\end{ExampleCode}
\end{Examples}

\HeaderA{smooth.fd}{Smooth a Functional Data Object Using an Indirectly Specified
Roughness Penalty}{smooth.fd}
\keyword{smooth}{smooth.fd}
\begin{Description}\relax
Smooth data already converted to a functional data object, fdobj,
using criteria consolidated in a functional data parameter object,
fdParobj.  For example, data may have been converted to a functional
data object using function \code{data2fd} using a fairly large set of
basis functions.  This 'fdobj' is then smoothed as specified in
'fdParobj'.
\end{Description}
\begin{Usage}
\begin{verbatim}
smooth.fd(fdobj, fdParobj)
\end{verbatim}
\end{Usage}
\begin{Arguments}
\begin{ldescription}
\item[\code{fdobj}] a functional data object to be smoothed.

\item[\code{fdParobj}] a functional parameter object. This object is defined by a roughness
penalty in slot \code{Lfd} and a smoothing parameter lambda in slot
\code{lambda}, and this information is used to further smooth argument \code{fdobj}.

\end{ldescription}
\end{Arguments}
\begin{Value}
a functional data object.
\end{Value}
\begin{SeeAlso}\relax
\code{\LinkA{smooth.basis}{smooth.basis}}, 
\code{\LinkA{data2fd}{data2fd}}
\end{SeeAlso}
\begin{Examples}
\begin{ExampleCode}

#  Shows the effects of two levels of smoothing
#  where the size of the third derivative is penalized.
#  The null space contains quadratic functions.
x <- seq(-1,1,0.02)
y <- x + 3*exp(-6*x^2) + rnorm(rep(1,101))*0.2
#  set up a saturated B-spline basis
basisobj <- create.bspline.basis(c(-1,1),81)
#  convert to a functional data object that interpolates the data.
result <- smooth.basis(x, y, basisobj)
yfd  <- result$fd

#  set up a functional parameter object with smoothing
#  parameter 1e-6 and a penalty on the 3rd derivative.
yfdPar <- fdPar(yfd, 2, 1e-6)
yfd1 <- smooth.fd(yfd, yfdPar)

## Not run: 
# FIXME: using 3rd derivative here gave error?????
yfdPar3 <- fdPar(yfd, 3, 1e-6)
yfd1.3 <- smooth.fd(yfd, yfdPar3)
#Error in bsplinepen(basisobj, Lfdobj, rng) : 
#       Penalty matrix cannot be evaluated
#  for derivative of order 3 for B-splines of order 4
## End(Not run)

#  set up a functional parameter object with smoothing
#  parameter 1 and a penalty on the 3rd derivative.
yfdPar <- fdPar(yfd, 2, 1)
yfd2 <- smooth.fd(yfd, yfdPar)
#  plot the data and smooth
plot(x,y)           # plot the data
lines(yfd1, lty=1)  #  add moderately penalized smooth
lines(yfd2, lty=3)  #  add heavily  penalized smooth
legend(-1,3,c("0.000001","1"),lty=c(1,3))
#  plot the data and smoothing using function plotfit.fd
plotfit.fd(y, x, yfd1)  # plot data and smooth

\end{ExampleCode}
\end{Examples}

\HeaderA{smooth.fdPar}{Smooth a functional data object using a directly specified roughness
penalty}{smooth.fdPar}
\keyword{smooth}{smooth.fdPar}
\begin{Description}\relax
Smooth data already converted to a functional data object, fdobj,
using directly specified criteria.
\end{Description}
\begin{Usage}
\begin{verbatim}
smooth.fdPar(fdobj, Lfdobj=int2Lfd(0), lambda=0,
             estimate=TRUE, penmat=NULL) 
\end{verbatim}
\end{Usage}
\begin{Arguments}
\begin{ldescription}
\item[\code{fdobj}] a functional data object to be smoothed.    

\item[\code{Lfdobj}] either a nonnegative integer or a linear differential operator
object 

\item[\code{lambda}] a nonnegative real number specifying the amount of smoothing
to be applied to the estimated functional parameter.

\item[\code{estimate}] a logical value:  if \code{TRUE}, the functional parameter is
estimated, otherwise, it is held fixed.

\item[\code{penmat}] a roughness penalty matrix.  Including this can eliminate the need
to compute this matrix over and over again in some types of
calculations.

\end{ldescription}
\end{Arguments}
\begin{Details}\relax
1.  fdPar

2.  smooth.fd
\end{Details}
\begin{Value}
a functional data object.
\end{Value}
\begin{References}\relax
Ramsay, James O., and Silverman, Bernard W. (2005), \emph{Functional 
Data Analysis, 2nd ed.}, Springer, New York. 

Ramsay, James O., and Silverman, Bernard W. (2002), \emph{Applied
Functional Data Analysis}, Springer, New York.
\end{References}
\begin{SeeAlso}\relax
\code{\LinkA{smooth.fd}{smooth.fd}}, 
\code{\LinkA{fdPar}{fdPar}}, 
\code{\LinkA{smooth.basis}{smooth.basis}}, 
\code{\LinkA{smooth.pos}{smooth.pos}}, 
\code{\LinkA{smooth.morph}{smooth.morph}}
\end{SeeAlso}
\begin{Examples}
\begin{ExampleCode}
#  Shows the effects of two levels of smoothing
#  where the size of the third derivative is penalized.
#  The null space contains quadratic functions.
x <- seq(-1,1,0.02)
y <- x + 3*exp(-6*x^2) + rnorm(rep(1,101))*0.2
#  set up a saturated B-spline basis
basisobj <- create.bspline.basis(c(-1,1),81)
#  convert to a functional data object that interpolates the data.
result <- smooth.basis(x, y, basisobj)
yfd  <- result$fd
#  set up a functional parameter object with smoothing
#  parameter 1e-6 and a penalty on the 2nd derivative.
yfdPar <- fdPar(yfd, 2, 1e-6)
yfd1 <- smooth.fd(yfd, yfdPar)

yfd1. <- smooth.fdPar(yfd, 2, 1e-6)
all.equal(yfd1, yfd1.)
# TRUE

#  set up a functional parameter object with smoothing
#  parameter 1 and a penalty on the 2nd derivative.
yfd2 <- smooth.fdPar(yfd, 2, 1)

#  plot the data and smooth
plot(x,y)           # plot the data
lines(yfd1, lty=1)  #  add moderately penalized smooth
lines(yfd2, lty=3)  #  add heavily  penalized smooth
legend(-1,3,c("0.000001","1"),lty=c(1,3))
#  plot the data and smoothing using function plotfit.fd
plotfit.fd(y, x, yfd1)  # plot data and smooth

\end{ExampleCode}
\end{Examples}

\HeaderA{smooth.monotone}{Monotone Smoothing of Data}{smooth.monotone}
\keyword{smooth}{smooth.monotone}
\begin{Description}\relax
When the discrete data that are observed reflect a smooth strictly
increasing or strictly decreasing function, it is often desirable to
smooth the data with a strictly monotone function, even though the
data themselves may not be monotone due to observational error.  An
example is when data are collected on the size of a growing organism
over time.  This function computes such a smoothing function, but,
unlike other smoothing functions, for only for one curve at a time.
The smoothing function minimizes a weighted error sum of squares
criterion.  This minimization requires iteration, and therefore is
more computationally intensive than normal smoothing.

The monotone smooth is beta[1]+beta[2]*integral(exp(Wfdobj)), where
Wfdobj is a functional data object.  Since exp(Wfdobj)>0, its integral
is monotonically increasing.
\end{Description}
\begin{Usage}
\begin{verbatim}
smooth.monotone(x, y, WfdParobj, wt=rep(1,nobs),
                zmat=matrix(1,nobs,1), conv=.0001, iterlim=20,
                active=c(FALSE,rep(TRUE,ncvec-1)), dbglev=1)
\end{verbatim}
\end{Usage}
\begin{Arguments}
\begin{ldescription}
\item[\code{x}] a vector of argument values.

\item[\code{y}] a vector of data values.  This function can only smooth
one set of data at a time.

\item[\code{WfdParobj}] a functional parameter object that provides an initial
value for the coefficients defining function $W(t)$,
and a roughness penalty on this function.

\item[\code{wt}] a vector of weights to be used in the smoothing.

\item[\code{zmat}] a design matrix or a matrix of covariate values that also
define the smooth of the data.

\item[\code{conv}] a convergence criterion.

\item[\code{iterlim}] the maximum number of iterations allowed in the minimization
of error sum of squares.

\item[\code{active}] a logical vector specifying which coefficients defining
$W(t)$ are estimated.  Normally, the first coefficient
is fixed.

\item[\code{dbglev}] either 0, 1, or 2.  This controls the amount information printed out on
each iteration, with 0 implying no output, 1 intermediate output level,
and 2 full output.  If either level 1 or 2 is specified, it can be
helpful to turn off the output buffering feature of S-PLUS.

\end{ldescription}
\end{Arguments}
\begin{Details}\relax
The smoothing function  $f(x)$ is determined by
three objects that need to be estimated from the data:

\Itemize{
\item $W(x)$, a functional data object that is first
exponentiated and then the result integrated.  This is the heart
of the monotone smooth.  The closer $W(x)$ is to zero, the
closer the monotone smooth becomes a straight line.  The closer
$W(x)$ becomes a constant, the more the monotone smoother
becomes an exponential function.  It is assumed that $W(0) = 0.$

\item $b0$, an intercept term that determines the value of the
smoothing function at $x = 0$. 

\item $b1$, a regression coefficient that determines the slope
of the smoothing function at $x = 0$. 
}
In addition, it is possible to have the intercept $b0$
depend in turn on the values of one or more covariates through the
design matrix \code{Zmat} as follows:
$b0 = Z c$. In this case, the single
intercept coefficient is replaced by the regression coefficients
in vector $c$ multipying the design matrix.
\end{Details}
\begin{Value}
a named list of length 5 containing:

\begin{ldescription}
\item[\code{Wfdobj}] a functional data object defining function $W(x)$ that optimizes the fit
to the data of the monotone function that it defines. 

\item[\code{beta}] the optimal regression coefficient values.

\item[\code{Flist}] a named list containing three results for the final converged solution:
(1)
\bold{f}: the optimal function value being minimized,
(2)
\bold{grad}: the gradient vector at the optimal solution,   and
(3)
\bold{norm}: the norm of the gradient vector at the optimal solution.

\item[\code{iternum}] the number of iterations.

\item[\code{iterhist}] \code{iternum+1} by 5 matrix containing the iteration
history.

\end{ldescription}
\end{Value}
\begin{References}\relax
Ramsay, James O., and Silverman, Bernard W. (2005), \emph{Functional 
Data Analysis, 2nd ed.}, Springer, New York. 

Ramsay, James O., and Silverman, Bernard W. (2002), \emph{Applied
Functional Data Analysis}, Springer, New York.
\end{References}
\begin{SeeAlso}\relax
\code{\LinkA{smooth.basis}{smooth.basis}}, 
\code{\LinkA{smooth.pos}{smooth.pos}}, 
\code{\LinkA{smooth.morph}{smooth.morph}}
\end{SeeAlso}
\begin{Examples}
\begin{ExampleCode}

#  Estimate the acceleration functions for growth curves
#  See the analyses of the growth data.
#  Set up the ages of height measurements for Berkeley data
age <- c( seq(1, 2, 0.25), seq(3, 8, 1), seq(8.5, 18, 0.5))
#  Range of observations
rng <- c(1,18)
#  First set up a basis for monotone smooth
#  We use b-spline basis functions of order 6
#  Knots are positioned at the ages of observation.
norder <- 6
nage   <- 31
nbasis <- nage + norder - 2
wbasis <- create.bspline.basis(rng, nbasis, norder, age)
#  starting values for coefficient
cvec0 <- matrix(0,nbasis,1)
Wfd0  <- fd(cvec0, wbasis)
#  set up functional parameter object
Lfdobj    <- 3          #  penalize curvature of acceleration
lambda    <- 10^(-0.5)  #  smoothing parameter
growfdPar <- fdPar(Wfd0, Lfdobj, lambda)
#  Set up wgt vector
wgt   <- rep(1,nage)
#  Smooth the data for the first girl
hgt1 = growth$hgtf[,1]
result <- smooth.monotone(age, hgt1, growfdPar, wgt)
#  Extract the functional data object and regression
#  coefficients
Wfd  <- result$Wfdobj
beta <- result$beta
#  Evaluate the fitted height curve over a fine mesh
agefine <- seq(1,18,len=101)
hgtfine <- beta[1] + beta[2]*eval.monfd(agefine, Wfd)
#  Plot the data and the curve
plot(age, hgt1, type="p")
lines(agefine, hgtfine)
#  Evaluate the acceleration curve
accfine <- beta[2]*eval.monfd(agefine, Wfd, 2)
#  Plot the acceleration curve
plot(agefine, accfine, type="l")
lines(c(1,18),c(0,0),lty=4)

\end{ExampleCode}
\end{Examples}

\HeaderA{smooth.morph}{Estimates a Smooth Warping Function}{smooth.morph}
\keyword{smooth}{smooth.morph}
\begin{Description}\relax
This function is nearly identical to \code{smooth.monotone}
but is intended to compute a smooth monotone transformation
$h(t)$ of argument $t$ such that
$h(0) = 0$ and $h(TRUE) = TRUE$, where $t$ is
the upper limit of $t$.  This function is used primarily
to register curves.
\end{Description}
\begin{Usage}
\begin{verbatim}
smooth.morph(x, y, WfdParobj, wt=rep(1,nobs),
             conv=.0001, iterlim=20,
             active=c(FALSE,rep(TRUE,ncvec-1)),
             dbglev=1)
\end{verbatim}
\end{Usage}
\begin{Arguments}
\begin{ldescription}
\item[\code{x}] a vector of argument values.

\item[\code{y}] a vector of data values.  This function can only smooth
one set of data at a time.

\item[\code{WfdParobj}] a functional parameter object that provides an initial
value for the coefficients defining function $W(t)$,
and a roughness penalty on this function.

\item[\code{wt}] a vector of weights to be used in the smoothing.

\item[\code{conv}] a convergence criterion.

\item[\code{iterlim}] the maximum number of iterations allowed in the minimization
of error sum of squares.

\item[\code{active}] a logical vector specifying which coefficients defining
$W(t)$ are estimated.  Normally, the first coefficient
is fixed.

\item[\code{dbglev}] either 0, 1, or 2.  This controls the amount information printed out on
each iteration, with 0 implying no output, 1 intermediate output level,
and 2 full output.  If either level 1 or 2 is specified, it can be
helpful to turn off the output buffering feature of S-PLUS.

\end{ldescription}
\end{Arguments}
\begin{Value}
A named list of length 4 containing:

\begin{ldescription}
\item[\code{Wfdobj}] a functional data object defining function $W(x)$ that that
optimizes the fit to the data of the monotone function that it defines.

\item[\code{Flist}] a named list containing three results for the final converged solution:
(1)
\bold{f}: the optimal function value being minimized,
(2)
\bold{grad}: the gradient vector at the optimal solution,   and
(3)
\bold{norm}: the norm of the gradient vector at the optimal solution.

\item[\code{iternum}] the number of iterations.

\item[\code{iternum}] the number of iterations.

\item[\code{iterhist}] a \code{} by 5 matrix containing the iteration
history.

\end{ldescription}
\end{Value}
\begin{SeeAlso}\relax
\code{\LinkA{smooth.monotone}{smooth.monotone}}, 
\code{\LinkA{landmarkreg}{landmarkreg}}, 
\code{\LinkA{register.fd}{register.fd}}
\end{SeeAlso}

\HeaderA{smooth.pos}{Smooth Data with a Positive Function}{smooth.pos}
\keyword{smooth}{smooth.pos}
\begin{Description}\relax
A set of data is smoothed with a functional data object that only
takes positive values.  For example, this function can be used
to estimate a smooth variance function from a set of squared residuals.
A function $W(t)$ is estimated such that that the smoothing
function is $exp[W(t)]$.
\end{Description}
\begin{Usage}
\begin{verbatim}
smooth.pos(argvals, y, WfdParobj, wt=rep(1,nobs),
           conv=.0001, iterlim=20, dbglev=1)
\end{verbatim}
\end{Usage}
\begin{Arguments}
\begin{ldescription}
\item[\code{argvals}] a vector of argument values.

\item[\code{y}] a vector of data values.  This function can only smooth
one set of data at a time.

\item[\code{WfdParobj}] a functional parameter object that provides an initial
value for the coefficients defining function $W(t)$,
and a roughness penalty on this function.

\item[\code{wt}] a vector of weights to be used in the smoothing.

\item[\code{conv}] a convergence criterion.

\item[\code{iterlim}] the maximum number of iterations allowed in the minimization
of error sum of squares.

\item[\code{dbglev}] either 0, 1, or 2.  This controls the amount information printed out on
each iteration, with 0 implying no output, 1 intermediate output level,
and 2 full output.  If either level 1 or 2 is specified, it can be
helpful to turn off the output buffering feature of S-PLUS.

\end{ldescription}
\end{Arguments}
\begin{Value}
a named list of length 4 containing:

\begin{ldescription}
\item[\code{Wfdobj}] a functional data object defining function $W(x)$ that that
optimizes the fit to the data of the monotone function that it defines.

\item[\code{Flist}] a named list containing three results for the final converged solution:
(1)
\bold{f}: the optimal function value being minimized,
(2)
\bold{grad}: the gradient vector at the optimal solution,   and
(3)
\bold{norm}: the norm of the gradient vector at the optimal solution.

\item[\code{iternum}] the number of iterations.

\item[\code{iternum}] the number of iterations.

\item[\code{iterhist}] a \code{iternum+1} by 5 matrix containing the iteration
history.

\end{ldescription}
\end{Value}
\begin{SeeAlso}\relax
\code{\LinkA{smooth.monotone}{smooth.monotone}}, 
\code{\LinkA{smooth.morph}{smooth.morph}}
\end{SeeAlso}
\begin{Examples}
\begin{ExampleCode}
#See the analyses of the daily weather data for examples.
\end{ExampleCode}
\end{Examples}

\HeaderA{sum.fd}{Sum of Functional Data}{sum.fd}
\keyword{smooth}{sum.fd}
\begin{Description}\relax
Evaluate the sum of a set of functions in a functional data object.
\end{Description}
\begin{Usage}
\begin{verbatim}
sum.fd(..., na.rm)
\end{verbatim}
\end{Usage}
\begin{Arguments}
\begin{ldescription}
\item[\code{...}] a functional data object to sum.

\item[\code{na.rm}] Not used.
\end{ldescription}
\end{Arguments}
\begin{Value}
a functional data object with a single replication
that contains the sum of the functions in the object \code{fd}.
\end{Value}
\begin{SeeAlso}\relax
\code{\LinkA{mean.fd}{mean.fd}}, 
\code{\LinkA{std.fd}{std.fd}}, 
\code{\LinkA{stddev.fd}{stddev.fd}}, 
\code{\LinkA{center.fd}{center.fd}}
\end{SeeAlso}

\HeaderA{summary.basisfd}{Summarize a Functional Data Object}{summary.basisfd}
\keyword{smooth}{summary.basisfd}
\begin{Description}\relax
Provide a compact summary of the characteristics of a
functional data object.
\end{Description}
\begin{Usage}
\begin{verbatim}
summary.basisfd(object, ...)
\end{verbatim}
\end{Usage}
\begin{Arguments}
\begin{ldescription}
\item[\code{object}] a functional data object.

\item[\code{...}] Other arguments to match generic
\end{ldescription}
\end{Arguments}
\begin{Value}
a displayed summary of the bivariate functional data object.
\end{Value}
\begin{References}\relax
Ramsay, James O., and Silverman, Bernard W. (2005), \emph{Functional 
Data Analysis, 2nd ed.}, Springer, New York. 

Ramsay, James O., and Silverman, Bernard W. (2002), \emph{Applied
Functional Data Analysis}, Springer, New York.
\end{References}

\HeaderA{summary.bifd}{Summarize a Bivariate Functional Data Object}{summary.bifd}
\keyword{smooth}{summary.bifd}
\begin{Description}\relax
Provide a compact summary of the characteristics of a
bivariate functional data object.
\end{Description}
\begin{Usage}
\begin{verbatim}
## S3 method for class 'bifd':
summary(object, ...)
summary.bifd(object, ...)
\end{verbatim}
\end{Usage}
\begin{Arguments}
\begin{ldescription}
\item[\code{object}] a bivariate functional data object.

\item[\code{...}] Other arguments to match the generic function for 'summary'
\end{ldescription}
\end{Arguments}
\begin{Value}
a displayed summary of the bivariate functional data object.
\end{Value}
\begin{SeeAlso}\relax
\code{\LinkA{summary}{summary}},
\end{SeeAlso}

\HeaderA{summary.fd}{Summarize a Functional Data Object}{summary.fd}
\keyword{smooth}{summary.fd}
\begin{Description}\relax
Provide a compact summary of the characteristics of a functional
data object.
\end{Description}
\begin{Usage}
\begin{verbatim}
## S3 method for class 'fd':
summary(object,...)
summary.fd(object,...)
\end{verbatim}
\end{Usage}
\begin{Arguments}
\begin{ldescription}
\item[\code{object}] a functional data object.

\item[\code{...}] Other arguments to match the generic for 'summary'
\end{ldescription}
\end{Arguments}
\begin{Value}
a displayed summary of the functional data object.
\end{Value}
\begin{SeeAlso}\relax
\code{\LinkA{summary}{summary}},
\end{SeeAlso}

\HeaderA{summary.fdPar}{Summarize a Functional Parameter Object}{summary.fdPar}
\keyword{smooth}{summary.fdPar}
\begin{Description}\relax
Provide a compact summary of the characteristics of a functional
parameter object.
\end{Description}
\begin{Usage}
\begin{verbatim}
## S3 method for class 'fdPar':
summary(object, ...)
summary.fdPar(object, ...)
\end{verbatim}
\end{Usage}
\begin{Arguments}
\begin{ldescription}
\item[\code{object}] a functional parameter object.

\item[\code{...}] Other arguments to match the generic 'summary' function
\end{ldescription}
\end{Arguments}
\begin{Value}
a displayed summary of the functional parameter object.
\end{Value}
\begin{SeeAlso}\relax
\code{\LinkA{summary}{summary}},
\end{SeeAlso}

\HeaderA{summary.Lfd}{Summarize a Linear Differential Operator Object}{summary.Lfd}
\keyword{smooth}{summary.Lfd}
\begin{Description}\relax
Provide a compact summary of the characteristics of a
linear differential operator object.
\end{Description}
\begin{Usage}
\begin{verbatim}
## S3 method for class 'Lfd':
summary(object, ...)
summary.Lfd(object, ...)
\end{verbatim}
\end{Usage}
\begin{Arguments}
\begin{ldescription}
\item[\code{object}] a linear differential operator object.

\item[\code{...}] Other arguments to match the generic 'summary' function
\end{ldescription}
\end{Arguments}
\begin{Value}
a displayed summary of the linear differential operator object.
\end{Value}
\begin{SeeAlso}\relax
\code{\LinkA{summary}{summary}},
\end{SeeAlso}

\HeaderA{svd2}{singular value decomposition with automatic error handling}{svd2}
\keyword{array}{svd2}
\begin{Description}\relax
The 'svd' function in R 2.5.1 occasionally throws an error
with a cryptic message.  In some such cases, changing the
LINPACK argument works.
\end{Description}
\begin{Usage}
\begin{verbatim}
  svd2(x, nu = min(n, p), nv = min(n, p), LINPACK = FALSE)
\end{verbatim}
\end{Usage}
\begin{Arguments}
\begin{ldescription}
\item[\code{x, nu, nv, LINPACK}] as for the 'svd' function in the 'base' package.

\end{ldescription}
\end{Arguments}
\begin{Details}\relax
In R 2.5.1, the 'svd' function sometimes stops with a cryptic error
message for a matrix x for which a second call to 'svd' with !LINPACK
will produce an answer.  When such conditions occur, assign 'x' with
attributes 'nu', 'nv', and 'LINPACK' to '.svd.LINPACK.error.matrix'
in 'env = .GlobalEnv'.

Except for these rare pathologies, 'svd2' should work the same as
'svd'.
\end{Details}
\begin{Value}
a list with components d, u, and v, as described in the help file for
'svd' in the 'base' package.
\end{Value}
\begin{SeeAlso}\relax
\code{\LinkA{svd}{svd}},
\end{SeeAlso}

\HeaderA{TaylorSpline}{Taylor representation of a B-Spline}{TaylorSpline}
\methaliasA{TaylorSpline.dierckx}{TaylorSpline}{TaylorSpline.dierckx}
\methaliasA{TaylorSpline.fd}{TaylorSpline}{TaylorSpline.fd}
\methaliasA{TaylorSpline.fdPar}{TaylorSpline}{TaylorSpline.fdPar}
\methaliasA{TaylorSpline.fdSmooth}{TaylorSpline}{TaylorSpline.fdSmooth}
\keyword{smooth}{TaylorSpline}
\keyword{manip}{TaylorSpline}
\begin{Description}\relax
Convert B-Spline coefficients into a local Taylor series
representation expanded about the midpoint between each pair of
distinct knots.
\end{Description}
\begin{Usage}
\begin{verbatim}
TaylorSpline(object, ...)
## S3 method for class 'fd':
TaylorSpline(object, ...)
## S3 method for class 'fdPar':
TaylorSpline(object, ...)
## S3 method for class 'fdSmooth':
TaylorSpline(object, ...)
## S3 method for class 'dierckx':
TaylorSpline(object, ...)
\end{verbatim}
\end{Usage}
\begin{Arguments}
\begin{ldescription}
\item[\code{ object }] a spline object, e.g., of class 'dierckx'.

\item[\code{...}] optional arguments 
\end{ldescription}
\end{Arguments}
\begin{Details}\relax
1.  Is \code{object} a spline object with a B-spline basis?  If no,
throw an error.

2.  Find \code{knots} and \code{midpoints}.

3.  Obtain coef(object).

4.  Determine the number of dimensions of coef(object) and create
empty \code{coef} and \code{deriv} arrays to match.  Then fill the
arrays.
\end{Details}
\begin{Value}
a list with the following components:

\begin{ldescription}
\item[\code{knots}] a numeric vector of knots(object, interior=FALSE)

\item[\code{midpoints}] midpoints of intervals defined by unique(knots)

\item[\code{coef}] A matrix of dim = c(nKnots-1, norder) containing the coeffients of
a polynomial in (x-midpoints[i]) for interval i, where nKnots =
length(unique(knots)).

\item[\code{deriv}] A matrix of dim = c(nKnots-1, norder) containing the derivatives
of the spline evaluated at \code{midpoints}.

\end{ldescription}

normal-bracket38bracket-normal
\end{Value}
\begin{Author}\relax
Spencer Graves
\end{Author}
\begin{SeeAlso}\relax
\code{\LinkA{fd}{fd}}
\code{\LinkA{create.bspline.basis}{create.bspline.basis}}
\end{SeeAlso}
\begin{Examples}
\begin{ExampleCode}
##
## The simplest b-spline basis:  order 1, degree 0, zero interior knots:
##       a single step function
##
library(DierckxSpline)
bspl1.1 <- create.bspline.basis(norder=1, breaks=0:1)
# ... jump to pi to check the code
fd.bspl1.1pi <- fd(pi, basisobj=bspl1.1)
bspl1.1pi <- TaylorSpline(fd.bspl1.1pi)


##
## Cubic spline:  4  basis functions
##
bspl4 <- create.bspline.basis(nbasis=4)
plot(bspl4)
parab4.5 <- fd(c(3, -1, -1, 3)/3, bspl4)
# = 4*(x-.5)
TaylorSpline(parab4.5)

##
## A more realistic example
##
data(titanium)
#  Cubic spline with 5 interior knots (6 segments)
titan10 <- with(titanium, curfit.free.knot(x, y))
(titan10T <- TaylorSpline(titan10) )

\end{ExampleCode}
\end{Examples}

\HeaderA{tperm.fd}{Permutation t-test for two groups of functional data objects.}{tperm.fd}
\keyword{smooth}{tperm.fd}
\begin{Description}\relax
tperm.fd creates a null distribution for a test of no difference between two
groups of functional data objects.
\end{Description}
\begin{Usage}
\begin{verbatim}
 tperm.fd(x1fd, x2fd, nperm=200, q=0.95, argvals=NULL, plotres=TRUE)
\end{verbatim}
\end{Usage}
\begin{Arguments}
\begin{ldescription}
\item[\code{x1fd}] a functional data object giving the first group of functional observations.

\item[\code{x2fd}] a functional data object giving the second group of functional
observations.

\item[\code{nperm}] number of permutations to use in creating the null distribution.

\item[\code{argvals}] If \code{yfdPar} is a \code{fd} object, the points at which to evaluate
the point-wise F-statistic.

\item[\code{q}] Critical quantile of the null distribution to compare to the observed
F-statistic.

\item[\code{plotres}] Argument to plot a visual display of the null distribution displaying the
\code{q}th quantile and observed F-statistic.

\end{ldescription}
\end{Arguments}
\begin{Details}\relax
The usual t-statistic is calculated pointwise and the test based on the
maximal value. If \code{argvals} is not specified,
it defaults to 101 equally-spaced points on the range of \code{yfdPar}.
\end{Details}
\begin{Value}
A list with components
\begin{ldescription}
\item[\code{pval}] the observed p-value of the permutation test.
\item[\code{qval}] the \code{q}th quantile of the null distribution.
\item[\code{Tobs}] the observed maximal t-statistic.
\item[\code{Tnull}] a vector of length \code{nperm} giving the observed values of the
permutation distribution.

\item[\code{Tvals}] the pointwise values of the observed t-statistic.
\item[\code{Tnullvals}] the pointwise values of of the permutation observations.
\item[\code{pvals.pts}] pointwise p-values of the t-statistic.
\item[\code{qvals.pts}] pointwise \code{q}th quantiles of the null distribution
\item[\code{argvals}] argument values for evaluating the F-statistic if \code{yfdPar}is
a functional data object.

\end{ldescription}

normal-bracket47bracket-normal
\end{Value}
\begin{Source}\relax
Ramsay, James O., and Silverman, Bernard W. (2006), \emph{Functional
Data Analysis, 2nd ed.}, Springer, New York.
\end{Source}
\begin{SeeAlso}\relax
\code{\LinkA{fRegress}{fRegress}}
\code{\LinkA{Fstat.fd}{Fstat.fd}}
\end{SeeAlso}

\HeaderA{var.fd}{Variance, Covariance, and Correlation Surfaces for
Functional Data Object(s)}{var.fd}
\keyword{smooth}{var.fd}
\begin{Description}\relax
Compute variance, covariance, and / or correlation functions for
functional data.  

These are two-argument functions and therefore define surfaces. If
only one functional data object is supplied, its variance or
correlation function is computed.  If two are supplied, the covariance
or correlation function between them is 
computed.
\end{Description}
\begin{Usage}
\begin{verbatim}
var.fd(fdobj1, fdobj2=fdobj1)
\end{verbatim}
\end{Usage}
\begin{Arguments}
\begin{ldescription}
\item[\code{fdobj1, fdobj2}] a functional data object.

\end{ldescription}
\end{Arguments}
\begin{Details}\relax
a two-argument or bivariate functional data object representing the
variance, covariance or correlation surface for a single functional
data object or the covariance between two functional data objects or
between different variables in a multivariate functional data object.
\end{Details}
\begin{Value}
An list object of class \code{bifd} with the following components:

\begin{ldescription}
\item[\code{coefs}] the coefficient array with dimensions fdobj1[["basis"]][["nbasis"]]
by fdobj2[["basis"]][["nbasis"]] giving the coefficients of the 
covariance matrix in terms of the bases used by fdobj1 and
fdobj2.  

\item[\code{sbasis}] fdobj1[["basis"]]

\item[\code{tbasis}] fdobj2[["basis"]]

\item[\code{bifdnames}] dimnames list for a 4-dimensional 'coefs' array.  If
length(dim(coefs)) is only 2 or 3, the last 2 or 1 component of
bifdnames is not used with dimnames(coefs).  

\end{ldescription}


Examples below illustrate this structure in simple
cases.
\end{Value}
\begin{SeeAlso}\relax
\code{\LinkA{mean.fd}{mean.fd}}, 
\code{\LinkA{sd.fd}{sd.fd}}, 
\code{\LinkA{std.fd}{std.fd}}
\code{\LinkA{stdev.fd}{stdev.fd}}
\end{SeeAlso}
\begin{Examples}
\begin{ExampleCode}
##
## Example with 2 different bases 
##
daybasis3 <- create.fourier.basis(c(0, 365))
daybasis5 <- create.fourier.basis(c(0, 365), 5)
tempfd3 <- with(CanadianWeather, data2fd(dailyAv[,,"Temperature.C"], 
       day.5, daybasis3, argnames=list("Day", "Station", "Deg C")) )
precfd5 <- with(CanadianWeather, data2fd(dailyAv[,,"log10precip"],
       day.5, daybasis5, argnames=list("Day", "Station", "Deg C")) )

# Compare with structure described above under 'value':
str(tempPrecVar3.5 <- var.fd(tempfd3, precfd5))

##
## Example with 2 variables, same bases
##
gaitbasis3 <- create.fourier.basis(nbasis=3)
str(gaitfd3 <- data2fd(gait, basisobj=gaitbasis3))
str(gaitVar.fd3 <- var.fd(gaitfd3))

# Check the answers with manual computations 
all.equal(var(t(gaitfd3$coefs[,,1])), gaitVar.fd3$coefs[,,,1])
# TRUE
all.equal(var(t(gaitfd3$coefs[,,2])), gaitVar.fd3$coefs[,,,3])
# TRUE
all.equal(var(t(gaitfd3$coefs[,,2]), t(gaitfd3$coefs[,,1])),
          gaitVar.fd3$coefs[,,,2])
# TRUE

# NOTE:
dimnames(gaitVar.fd3$coefs)[[4]]
# [1] Hip-Hip
# [2] Knee-Hip 
# [3] Knee-Knee
# If [2] were "Hip-Knee", then
# gaitVar.fd3$coefs[,,,2] would match 
#var(t(gaitfd3$coefs[,,1]), t(gaitfd3$coefs[,,2]))
# *** It does NOT.  Instead, it matches:  
#var(t(gaitfd3$coefs[,,2]), t(gaitfd3$coefs[,,1])),

##
## The following produces contour and perspective plots
##
# Evaluate at a 53 by 53 grid for plotting

daybasis65 <- create.fourier.basis(rangeval=c(0, 365), nbasis=65)

daytempfd <- with(CanadianWeather, data2fd(dailyAv[,,"Temperature.C"],
       day.5, daybasis65, argnames=list("Day", "Station", "Deg C")) )
str(tempvarbifd <- var.fd(daytempfd))

str(tempvarmat  <- eval.bifd(weeks,weeks,tempvarbifd))
# dim(tempvarmat)= c(53, 53)

op <- par(mfrow=c(1,2), pty="s")
#contour(tempvarmat, xlab="Days", ylab="Days")
contour(weeks, weeks, tempvarmat, 
        xlab="Daily Average Temperature",
        ylab="Daily Average Temperature",
        main=paste("Variance function across locations\n",
          "for Canadian Anual Temperature Cycle"),
        cex.main=0.8, axes=FALSE)
axisIntervals(1, atTick1=seq(0, 365, length=5), atTick2=NA, 
            atLabels=seq(1/8, 1, 1/4)*365,
            labels=paste("Q", 1:4) )
axisIntervals(2, atTick1=seq(0, 365, length=5), atTick2=NA, 
            atLabels=seq(1/8, 1, 1/4)*365,
            labels=paste("Q", 1:4) )
persp(weeks, weeks, tempvarmat,
      xlab="Days", ylab="Days", zlab="Covariance")
mtext("Temperature Covariance", line=-4, outer=TRUE)
par(op)

\end{ExampleCode}
\end{Examples}

\HeaderA{varmx.cca.fd}{Rotation of Functional Canonical Components with VARIMAX}{varmx.cca.fd}
\keyword{smooth}{varmx.cca.fd}
\begin{Description}\relax
Results of canonical correlation analysis are often easier to interpret if
they are rotated.  Among the many possible ways in which this rotation can be
defined, the VARIMAX criterion seems to give satisfactory results most
of the time.
\end{Description}
\begin{Usage}
\begin{verbatim}
varmx.cca.fd(ccafd, nx=201)
\end{verbatim}
\end{Usage}
\begin{Arguments}
\begin{ldescription}
\item[\code{ccafd}] an object of class "cca.fd" that is produced by function
\code{cca.fd}.

\item[\code{nx}] the number of points in a fine mesh of points that is
required to approximate canonical variable functional
data objects.

\end{ldescription}
\end{Arguments}
\begin{Value}
a rotated version of argument \code{cca.fd}.
\end{Value}
\begin{SeeAlso}\relax
\code{\LinkA{varmx}{varmx}}, 
\code{\LinkA{varmx.pca.fd}{varmx.pca.fd}}
\end{SeeAlso}

\HeaderA{varmx.pca.fd}{Rotation of Functional Principal Components with VARIMAX
Criterion}{varmx.pca.fd}
\keyword{smooth}{varmx.pca.fd}
\begin{Description}\relax
Principal components are often easier to interpret if they are
rotated.  Among the many possible ways in which this rotation can be
defined, the VARIMAX criterion seems to give satisfactory results most
of the time.
\end{Description}
\begin{Usage}
\begin{verbatim}
varmx.pca.fd(pcafd, nharm=scoresd[2], nx=501)
\end{verbatim}
\end{Usage}
\begin{Arguments}
\begin{ldescription}
\item[\code{pcafd}] an object of class \code{pca.fd} that is produced by function
\code{pca.fd}.

\item[\code{nharm}] the number of harmonics or principal components to be
rotated.

\item[\code{nx}] the number of argument values in a fine mesh
used to define the harmonics to be
rotated.

\end{ldescription}
\end{Arguments}
\begin{Value}
a rotated principal components analysis object of class
\code{pca.fd}.
\end{Value}
\begin{SeeAlso}\relax
\code{\LinkA{varmx}{varmx}}, 
\code{\LinkA{varmx.cca.fd}{varmx.cca.fd}}
\end{SeeAlso}

\HeaderA{varmx}{Rotate a Matrix of Component Loadings using the VARIMAX Criterion}{varmx}
\keyword{smooth}{varmx}
\begin{Description}\relax
The matrix being rotated contains the values of the component
functional data objects computed in either a principal
components analysis or a canonical correlation analysis.
The values are computed over a fine mesh of argument values.
\end{Description}
\begin{Usage}
\begin{verbatim}
varmx(amat)
\end{verbatim}
\end{Usage}
\begin{Arguments}
\begin{ldescription}
\item[\code{amat}] the matrix to be rotated.  The number of rows is
equal to the number of argument values \code{nx} used
in a fine mesh.  The number of columns is the number of
components to be rotated.

\end{ldescription}
\end{Arguments}
\begin{Details}\relax
The VARIMAX criterion is the variance of the squared component values.
As this criterion is maximized with respect to a rotation of the
space spanned by the columns of the matrix, the squared loadings
tend more and more to be either near 0 or near 1, and this tends to
help with the process of labelling or interpreting the rotated matrix.
\end{Details}
\begin{Value}
a square rotation matrix of order equal to the number
of components that are rotated.  A rotation matrix
$T$ has that property that $T'T = TT' = I$.
\end{Value}
\begin{SeeAlso}\relax
\code{\LinkA{varmx.pca.fd}{varmx.pca.fd}}, 
\code{\LinkA{varmx.cca.fd}{varmx.cca.fd}}
\end{SeeAlso}

\HeaderA{vec2Lfd}{Make a Linear Differential Operator Object from a Vector}{vec2Lfd}
\keyword{smooth}{vec2Lfd}
\begin{Description}\relax
A linear differential operator object of order $m$ is
constructed from the number in a vector of length $m$.
\end{Description}
\begin{Usage}
\begin{verbatim}
vec2Lfd(bwtvec, rangeval=c(0,1))
\end{verbatim}
\end{Usage}
\begin{Arguments}
\begin{ldescription}
\item[\code{bwtvec}] a vector of coefficients to define the linear differential
operator object

\item[\code{rangeval}] a vector of length 2 specifying the range over which the
operator is defined

\end{ldescription}
\end{Arguments}
\begin{Value}
a linear differential operator object
\end{Value}
\begin{SeeAlso}\relax
\code{\LinkA{int2Lfd}{int2Lfd}}, 
\code{\LinkA{Lfd}{Lfd}}
\end{SeeAlso}
\begin{Examples}
\begin{ExampleCode}
#  define the harmonic acceleration operator used in the
#  analysis of the daily temperature data
harmaccelLfd <- vec2Lfd(c(0,(2*pi/365)^2,0), c(0,365))
\end{ExampleCode}
\end{Examples}

\HeaderA{zerofind}{Does the range of the input contain 0?}{zerofind}
\keyword{logic}{zerofind}
\begin{Description}\relax
Returns TRUE if range of the argument includes 0 and FALSES if not.
\end{Description}
\begin{Usage}
\begin{verbatim}
  zerofind(fmat)
\end{verbatim}
\end{Usage}
\begin{Arguments}
\begin{ldescription}
\item[\code{fmat}] An object from which 'range' returns two numbers.  

\end{ldescription}
\end{Arguments}
\begin{Value}
A logical value TRUE or FALSE.
\end{Value}
\begin{SeeAlso}\relax
\code{\LinkA{range}{range}}
\end{SeeAlso}
\begin{Examples}
\begin{ExampleCode}
zerofind(1:5)
# FALSE
zerofind(0:3)
# TRUE 
\end{ExampleCode}
\end{Examples}

\end{document}
