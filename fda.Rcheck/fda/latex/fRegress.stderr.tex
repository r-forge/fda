\HeaderA{fRegress.stderr}{Compute Standard errors of Coefficient Functions
Estimated by Functional Regression
Analysis}{fRegress.stderr}
\keyword{smooth}{fRegress.stderr}
\begin{Description}\relax
Function \code{fRegress} carries out a functional regression analysis
of the concurrent kind, and estimates a regression coefficient function
corresponding to each independent variable, whether it is scalar or
functional.  This function uses the list that is output by \code{fRegress}
to provide standard error functions for each regression function.  These
standard error functions are pointwise, meaning that sampling standard
deviation functions only are computed, and not sampling covariances.
\end{Description}
\begin{Usage}
\begin{verbatim}
fRegress.stderr(fRegressList, y2cMap, SigmaE)
\end{verbatim}
\end{Usage}
\begin{Arguments}
\begin{ldescription}
\item[\code{fRegressList}] the named list of length six that is returned from a call to function
\code{fRegress}.

\item[\code{y2cMap}] a matrix that contains the linear transformation that
takes the raw data values into the coefficients defining a smooth
functional data object. Typically, this matrix is returned from a
call to function \code{smooth.basis} that generates the
dependent variable objects.  If the dependent variable is scalar,
this matrix is an identity matrix of order equal to the length
of the vector.

\item[\code{SigmaE}] either a matrix or a bivariate functional data object
according to whether the dependent variable is scalar or functional,
respectively.
This object has a number of replications equal to
the length of the dependent variable object.  It contains an estimate
of the variance-covariance matrix or function for the residuals.

\end{ldescription}
\end{Arguments}
\begin{Value}
a named list of length 3 containing:

\begin{ldescription}
\item[\code{betastderrlist}] a list object of length
the number of independent variables. Each member contains a
functional parameter object
for the standard error of a regression function.

\item[\code{bvar}] a symmetric matrix containing sampling variances and
covariances for the matrix of regression coefficients
for the regression functions.  These are stored
column-wise in defining BVARIANCE.

\item[\code{c2bMap}] a matrix containing the mapping from response variable
coefficients to coefficients for regression coefficients.

\end{ldescription}
\end{Value}
\begin{SeeAlso}\relax
\code{\LinkA{fRegress}{fRegress}}, 
\code{\LinkA{fRegress.CV}{fRegress.CV}}
\end{SeeAlso}
\begin{Examples}
\begin{ExampleCode}
#See the weather data analyses in the file this-is-escaped-codenormal-bracket29bracket-normal for
#examples of the use of function this-is-escaped-codenormal-bracket30bracket-normal.
\end{ExampleCode}
\end{Examples}

