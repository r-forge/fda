\HeaderA{plot.fd}{Plot a Functional Data Object}{plot.fd}
\aliasA{plot.fdSmooth}{plot.fd}{plot.fdSmooth}
\keyword{smooth}{plot.fd}
\keyword{hplot}{plot.fd}
\begin{Description}\relax
Functional data observations, or a derivative of them, are plotted.
These may be either plotted simultaneously, as \code{matplot} does for
multivariate data, or one by one with a mouse click to move from one
plot to another.  The function also accepts the other plot
specification arguments that the regular \code{plot} does.  Calling
\code{plot} with an \code{fdSmooth} object is simply plots its
code{fd} component.
\end{Description}
\begin{Usage}
\begin{verbatim}
## S3 method for class 'fd':
plot(x, y, Lfdobj=0, href=TRUE, titles=NULL,
                    xlim=NULL, ylim=NULL, xlab=NULL,
                    ylab=NULL, ask=FALSE, nx=201, ...)
## S3 method for class 'fdSmooth':
plot(x, y, Lfdobj=0, href=TRUE, titles=NULL,
                    xlim=NULL, ylim=NULL, xlab=NULL,
                    ylab=NULL, ask=FALSE, nx=201, ...)
\end{verbatim}
\end{Usage}
\begin{Arguments}
\begin{ldescription}
\item[\code{x}] functional data object(s) to be plotted.

\item[\code{y}] sequence of points at which to evaluate the functions 'x' and plot
on the horizontal axis.  Defaults to seq(rangex[1], rangex[2],
length = nx).

NOTE:  This will be the values on the horizontal axis, NOT the
vertical axis.  

\item[\code{Lfdobj}] either a nonnegative integer or a linear differential operator
object. If present, the derivative or the value of applying the
operator is plotted rather than the functions themselves.

\item[\code{href}] a logical variable:  If \code{TRUE}, add a horizontal reference line
at 0.  

\item[\code{titles}] a vector of strings for identifying curves

\item[\code{xlab}] a label for the horizontal axis.

\item[\code{ylab}] a label for the vertical axis.

\item[\code{xlim}] a vector of length 2 containing axis limits for the horizontal axis.

\item[\code{ylim}] a vector of length 2 containing axis limits for the vertical axis.

\item[\code{ask}] a logical value:  If \code{TRUE}, each curve is shown separately, and
the plot advances with a mouse click

\item[\code{nx}] the number of points to use to define the plot.  The default is
usually enough, but for a highly variable function more may be
required.

\item[\code{... }] additional plotting arguments that can be used with function
\code{plot}

\end{ldescription}
\end{Arguments}
\begin{Details}\relax
Note that for multivariate data, a
suitable array must first be defined using the \code{par} function.
\end{Details}
\begin{Value}
'done'
\end{Value}
\begin{Section}{Side Effects}
a plot of the functional observations
\end{Section}
\begin{SeeAlso}\relax
\code{\LinkA{lines.fd}{lines.fd}}, \code{\LinkA{plotfit.fd}{plotfit.fd}}
\end{SeeAlso}
\begin{Examples}
\begin{ExampleCode}
##
## plot.df
##
#daytime   <- (1:365)-0.5
#dayrange  <- c(0,365)
#dayperiod <- 365
#nbasis     <- 65
#dayrange  <- c(0,365)

daybasis65 <- create.fourier.basis(c(0, 365), 65)
harmaccelLfd <- vec2Lfd(c(0,(2*pi/365)^2,0), c(0, 365))

harmfdPar     <- fdPar(daybasis65, harmaccelLfd, lambda=1e5)

daytempfd <- with(CanadianWeather, data2fd(
        dailyAv[,,"Temperature.C"], day.5, 
        daybasis65,argnames=list("Day", "Station", "Deg C")) )

#  plot all the temperature functions for the monthly weather data
plot(daytempfd, main="Temperature Functions")

## Not run: 
# To plot one at a time:  
# The following pauses to request page changes.

plot(daytempfd, main="Temperature Functions", ask=TRUE)
## End(Not run)

##
## plot.fdSmooth
##
b3.4 <- create.bspline.basis(norder=3, breaks=c(0, .5, 1))
# 4 bases, order 3 = degree 2 =
# continuous, bounded, locally quadratic 
fdPar3 <- fdPar(b3.4, lambda=1)
# Penalize excessive slope Lfdobj=1;  
# second derivative Lfdobj=2 is discontinuous.
fd3.4s0 <- smooth.basis(0:1, 0:1, fdPar3)

# using plot.fd directly 
plot(fd3.4s0$fd)

# same plot via plot.fdSmooth 
plot(fd3.4s0)

\end{ExampleCode}
\end{Examples}

