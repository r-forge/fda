\HeaderA{polyg}{Polygonal Basis Function Values}{polyg}
\keyword{smooth}{polyg}
\begin{Description}\relax
Evaluates a set of polygonal basis functions, or a derivative of these
functions, at a set of arguments.
\end{Description}
\begin{Usage}
\begin{verbatim}
polyg(x, argvals, nderiv=0)
\end{verbatim}
\end{Usage}
\begin{Arguments}
\begin{ldescription}
\item[\code{x}] a vector of argument values at which the polygonal basis functions are to
evaluated.

\item[\code{argvals}] a strictly increasing set of argument values containing the range of x
within it that defines the polygonal basis.  The default is x itself.

\item[\code{nderiv}] the order of derivative to be evaluated.  The derivative must not exceed
one.  The default derivative is 0, meaning that the basis functions
themselves are evaluated.

\end{ldescription}
\end{Arguments}
\begin{Value}
a matrix of function values.  The number of rows equals the number of
arguments, and the number of columns equals the number of basis
\end{Value}
\begin{SeeAlso}\relax
\code{\LinkA{create.polygonal.basis}{create.polygonal.basis}}, 
\code{\LinkA{polygpen}{polygpen}}
\end{SeeAlso}
\begin{Examples}
\begin{ExampleCode}

#  set up a set of 21 argument values
x <- seq(0,1,0.05)
#  set up a set of 11 argument values
argvals <- seq(0,1,0.1)
#  with the default period (1) and derivative (0)
basismat <- polyg(x, argvals)
#  plot the basis functions
matplot(x, basismat, type="l")

\end{ExampleCode}
\end{Examples}

