\HeaderA{create.polygonal.basis}{Create a Polygonal Basis}{create.polygonal.basis}
\keyword{smooth}{create.polygonal.basis}
\begin{Description}\relax
A basis is set up for constructing polygonal lines, consisting of 
straight line segments that join together.
\end{Description}
\begin{Usage}
\begin{verbatim}
create.polygonal.basis(argvals=NULL, dropind=NULL,
                       quadvals=NULL, values=NULL)
\end{verbatim}
\end{Usage}
\begin{Arguments}
\begin{ldescription}
\item[\code{argvals}] a strictly increasing vector of argument values at which line
segments join to form a polygonal line.

\item[\code{dropind}] a vector of integers specifiying the basis functions to
be dropped, if any.  For example, if it is required that
a function be zero at the left boundary, this is achieved
by dropping the first basis function, the only one that
is nonzero at that point. Default value NULL.

\item[\code{quadvals}] a matrix with two columns and a number of rows equal to the number
of argument values used to approximate an integral using Simpson's
rule.  The first column contains these argument values.  
A minimum of 5 values are required for
each inter-knot interval, and that is often enough. These
are equally spaced between two adjacent knots.
The second column contains the weights used for Simpson's
rule.  These are proportional to 1, 4, 2, 4, ..., 2, 4, 1.

\item[\code{values}] a list containing the basis functions and their derivatives
evaluated at the quadrature points contained in the first
column of \code{ quadvals }.

\end{ldescription}
\end{Arguments}
\begin{Details}\relax
The actual basis functions consist of triangles, each with its apex
over an argument value. Note that in effect the polygonal basis is
identical to a B-spline basis of order 2 and a knot or break value at
each argument value.  The range of the polygonal basis is set to the
interval defined by the smallest and largest argument values.
\end{Details}
\begin{Value}
a basis object with the type \code{polyg}.
\end{Value}
\begin{SeeAlso}\relax
\code{\LinkA{basisfd}{basisfd}}, 
\code{\LinkA{create.bspline.basis}{create.bspline.basis}}, 
\code{\LinkA{create.constant.basis}{create.constant.basis}}, 
\code{\LinkA{create.exponential.basis}{create.exponential.basis}}, 
\code{\LinkA{create.fourier.basis}{create.fourier.basis}}, 
\code{\LinkA{create.monomial.basis}{create.monomial.basis}}, 
\code{\LinkA{create.polynomial.basis}{create.polynomial.basis}}, 
\code{\LinkA{create.power.basis}{create.power.basis}}
\end{SeeAlso}
\begin{Examples}
\begin{ExampleCode}
#  Create a polygonal basis over the interval [0,1]
#  with break points at 0, 0.1, ..., 0.95, 1
(basisobj <- create.polygonal.basis(seq(0,1,0.1)))
#  plot the basis
plot(basisobj)
\end{ExampleCode}
\end{Examples}

