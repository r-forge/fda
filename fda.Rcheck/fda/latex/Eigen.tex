\HeaderA{Eigen}{Eigenanalysis preserving dimnames}{Eigen}
\keyword{array}{Eigen}
\begin{Description}\relax
Compute eigenvalues and vectors, assigning names to the eigenvalues
and dimnames to the eigenvectors.
\end{Description}
\begin{Usage}
\begin{verbatim}
Eigen(x, symmetric, only.values = FALSE, EISPACK = FALSE,
      valuenames )
\end{verbatim}
\end{Usage}
\begin{Arguments}
\begin{ldescription}
\item[\code{x}] a square matrix whose spectral decomposition is to be computed.  

\item[\code{symmetric}] logical:  If TRUE, the matrix is assumed to be symmetric (or 
Hermitian if complex) and only its lower triangle (diagonal
included) is used.  If 'symmetric' is not specified, the
matrix is inspected for symmetry.

\item[\code{only.values}] if 'TRUE', only the eigenvalues are computed and returned, otherwise
both eigenvalues and eigenvectors are returned. 

\item[\code{EISPACK}] logical. Should EISPACK be used (for compatibility with R < 1.7.0)?

\item[\code{valuenames}] character vector of length nrow(x) or a character string that can be
extended to that length by appening 1:nrow(x).

The default depends on symmetric and whether
\code{\LinkA{rownames}{rownames}} == \code{\LinkA{colnames}{colnames}}:  If
\code{\LinkA{rownames}{rownames}} == \code{\LinkA{colnames}{colnames}} and
symmetric = TRUE (either specified or determined by
inspection), the default is "paste('ev', 1:nrow(x), sep='')".
Otherwise, the default is colnames(x) unless this is NULL. 

\end{ldescription}
\end{Arguments}
\begin{Details}\relax
1.  Check 'symmetric'  

2.  ev <- eigen(x, symmetric, only.values = FALSE, EISPACK = FALSE);
see \code{\LinkA{eigen}{eigen}} for more details.  

3.  rNames = rownames(x);  if this is NULL, rNames = if(symmetric)
paste('x', 1:nrow(x), sep='') else paste('xcol', 1:nrow(x)).  

4.  Parse 'valuenames', assign to names(ev[['values']]).  

5.  dimnames(ev[['vectors']]) <- list(rNames, valuenames) 

NOTE:  This naming convention is fairly obvious if 'x' is symmetric.
Otherwise, dimensional analysis suggests problems with almost any
naming convention.  To see this, consider the following simple
example:

\deqn{
X <- matrix(1:4, 2, dimnames=list(LETTERS[1:2], letters[3:4]))
}{}
\Tabular{rrr}{
& c & d \\
A & 1 & 3 \\
B & 2 & 4 \\
}
\deqn{
X.inv <- solve(X)
}{}
\Tabular{rrr}{
& A & B \\
c & -2 & 1.5 \\
d & 1 & -0.5 \\
}

One way of interpreting this is to assume that colnames are really
reciprocals of the units.  Thus, in this example, X[1,1] is in units
of 'A/c' and X.inv[1,1] is in units of 'c/A'.  This would make any
matrix with the same row and column names potentially dimensionless.
Since eigenvalues are essentially the diagonal of a diagonal matrix,
this would mean that eigenvalues are dimensionless, and their names
are merely placeholders.
\end{Details}
\begin{Value}
a list with components values and (if only.values = FALSE)
vectors, as described in \code{\LinkA{eigen}{eigen}}.
\end{Value}
\begin{Author}\relax
Spencer Graves
\end{Author}
\begin{SeeAlso}\relax
\code{\LinkA{eigen}{eigen}},
\code{\LinkA{svd}{svd}}
\code{\LinkA{qr}{qr}}
\code{\LinkA{chol}{chol}}
\end{SeeAlso}
\begin{Examples}
\begin{ExampleCode}
X <- matrix(1:4, 2, dimnames=list(LETTERS[1:2], letters[3:4]))
eigen(X)
Eigen(X)
Eigen(X, valuenames='eigval')

Y <- matrix(1:4, 2, dimnames=list(letters[5:6], letters[5:6]))
Eigen(Y)

Eigen(Y, symmetric=TRUE)
# only the lower triangle is used;
# the upper triangle is ignored.  
\end{ExampleCode}
\end{Examples}

