\HeaderA{density.fd}{Compute a Probability Density Function}{density.fd}
\keyword{smooth}{density.fd}
\begin{Description}\relax
Like the regular S-PLUS function \code{density}, this function
computes a probability density function for a sample of values of a
random variable.  However, in this case the density function is defined
by a functional parameter object \code{WfdParobj} along with a normalizing
constant \code{C}.

The density function $p(x)$ has the form
\code{p(x) = C exp[W(x)]}
where function $W(x)$ is defined by the functional data object
\code{WfdParobj}.
\end{Description}
\begin{Usage}
\begin{verbatim}
density.fd(x, WfdParobj, conv=0.0001, iterlim=20,
           active=2:nbasis, dbglev=1, ...)
\end{verbatim}
\end{Usage}
\begin{Arguments}
\begin{ldescription}
\item[\code{x}] a strictly increasing set variable values.
These observations may be one of two forms:
\Enumerate{
\item a vector of observatons $x_i$
\item a two-column matrix, with the observations $x_i$ in the
first column, and frequencies $f_i$ in the second.
}
The first option corresponds to all $f_i = 1$.

\item[\code{WfdParobj}] a functional parameter object specifying the initial
value, basis object, roughness penalty and smoothing
parameter defining function $W(t).$

\item[\code{conv}] a positive constant defining the convergence criterion.

\item[\code{iterlim}] the maximum number of iterations allowed.

\item[\code{active}] a logical vector of length equal to the number of coefficients
defining \code{Wfdobj}. If an entry is TRUE, the corresponding
coefficient is estimated, and if FALSE, it is held at the value defining the
argument \code{Wfdobj}.  Normally the first coefficient is set to 0
and not estimated, since it is assumed that $W(0) = 0$.

\item[\code{dbglev}] either 0, 1, or 2.  This controls the amount information printed out on
each iteration, with 0 implying no output, 1 intermediate output level,
and 2 full output.  If levels 1 and 2 are used, it is helpful to
turn off the output buffering option in S-PLUS.

\item[\code{...}] Other arguments to match the generic function 'density'
\end{ldescription}
\end{Arguments}
\begin{Details}\relax
The goal of the function is provide a smooth density function
estimate that approaches some target density by an amount that is
controlled by the linear differential operator \code{Lfdobj} and
the penalty parameter. For example, if the second derivative of
$W(t)$ is penalized heavily, this will force the function to
approach a straight line, which in turn will force the density function
itself to be nearly normal or Gaussian.  Similarly, to each textbook
density function there corresponds a $W(t)$, and to each of these
in turn their corresponds a linear differential operator that will, when
apply to $W(t)$, produce zero as a result.
To plot the density function or to evaluate it, evaluate \code{Wfdobj},
exponentiate the resulting vector, and then divide by the normalizing
constant \code{C}.
\end{Details}
\begin{Value}
a named list of length 4 containing:

\begin{ldescription}
\item[\code{Wfdobj}] a functional data object defining function $W(x)$ that that
optimizes the fit to the data of the monotone function that it defines.

\item[\code{C}] the normalizing constant.

\item[\code{Flist}] a named list containing three results for the final converged solution:
(1)
\bold{f}: the optimal function value being minimized,
(2)
\bold{grad}: the gradient vector at the optimal solution,   and
(3)
\bold{norm}: the norm of the gradient vector at the optimal solution.

\item[\code{iternum}] the number of iterations.

\item[\code{iterhist}] a \code{iternum+1} by 5 matrix containing the iteration
history.

\end{ldescription}
\end{Value}
\begin{SeeAlso}\relax
\code{\LinkA{intensity.fd}{intensity.fd}}
\code{\LinkA{density}{density}}
\end{SeeAlso}
\begin{Examples}
\begin{ExampleCode}

#  set up range for density
rangeval <- c(-3,3)
#  set up some standard normal data
x <- rnorm(50)
#  make sure values within the range
x[x < -3] <- -2.99
x[x >  3] <-  2.99
#  set up basis for W(x)
basisobj <- create.bspline.basis(rangeval, 11)
#  set up initial value for Wfdobj
Wfd0 <- fd(matrix(0,11,1), basisobj)
WfdParobj <- fdPar(Wfd0)
#  estimate density
denslist <- density.fd(x, WfdParobj)
#  plot density
xval <- seq(-3,3,.2)
wval <- eval.fd(xval, denslist$Wfdobj)
pval <- exp(wval)/denslist$C
plot(xval, pval, type="l", ylim=c(0,0.4))
points(x,rep(0,50))

\end{ExampleCode}
\end{Examples}

