\HeaderA{handwrit}{Cursive handwriting samples}{handwrit}
\keyword{datasets}{handwrit}
\begin{Description}\relax
20 cursive samples of 1401 (x, y,) coordinates for writing "fda"
\end{Description}
\begin{Format}\relax
An array of dimensions (1401, 20, 2) giving 1401 pairs of (x, y)
coordinates for each of 20 replicates of cursively writing "fda"
\end{Format}
\begin{Details}\relax
These data are the X-Y coordinates of 20 replications of writing
the script "fda".  The subject was Jim Ramsay.  Each replication
is represented by 1401 coordinate values.  The scripts have been 
extensively pre-processed.  They have been adjusted to a common
length that corresponds to 2.3 seconds or 2300 milliseconds, and
they have already been registered so that important features in
each script are aligned.

This analysis is designed to illustrate techniques for working
with functional data having rather high frequency variation and
represented by thousands of data points per record.  Comments
along the way explain the choices of analysis that were made.

The final result of the analysis is a third order linear 
differential equation for each coordinate forced by a 
constant and by time.  The equations are able to reconstruct
the scripts to a fairly high level of accuracy, and are also
able to accommodate a substantial amount of the variation in
the observed scripts across replications.  by contrast, a 
second order equation was found to be completely inadequate.

An interesting suprise in the results is the role placed by
a 120 millisecond cycle such that sharp features such as cusps
correspond closely to this period.  This 110-120 msec cycle
seems is usually seen in human movement data involving rapid
movements, such as speech, juggling and so on.

These 20 records have already been normalized to a common time
interval of 2300 milliseconds and have beeen also registered so that
prominent features occur at the same times across replications.  Time
will be measured in (approximate) milliseconds and space in meters.
The data will require a small amount of smoothing, since an error of
0.5 mm is characteristic of the OPTOTRAK 3D measurement system used to
collect the data.

Milliseconds were chosen as a time scale in order to make the ratio of
the time unit to the inter-knot interval not too far from one.
Otherwise, smoothing parameter values may be extremely small or
extremely large.

The basis functions will be B-splines, with a spline placed at each
knot.  One may question whether so many basis functions are required,
but this decision is found to be essential for stable derivative
estimation up to the third order at and near the boundaries.  

Order 7 was used to get a smooth third derivative, which requires
penalizing the size of the 5th derivative, which in turn requires an
order of at least 7.  This implies norder + no. of interior knots =
1399 + 7 = 1406 basis functions.  

The smoothing parameter value 1e8 was chosen to obtain a fitting error
of about 0.5 mm, the known error level in the OPTOTRACK equipment.
\end{Details}
\begin{Source}\relax
Ramsay, James O., and Silverman, Bernard W. (2006), \emph{Functional
Data Analysis, 2nd ed.}, Springer, New York.
\end{Source}
\begin{Examples}
\begin{ExampleCode}
plot(handwrit[, 1, 1], handwrit[, 1, 2], type="l")
\end{ExampleCode}
\end{Examples}

