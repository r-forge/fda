\HeaderA{lambda2gcv}{Compute GCV Criterion}{lambda2gcv}
\keyword{smooth}{lambda2gcv}
\begin{Description}\relax
The generalized cross-validation or GCV criterion is often used to
select an appropriate smoothing parameter value, by finding the
smoothing parameter that minimizes GCV.  This function locates that
value.
\end{Description}
\begin{Usage}
\begin{verbatim}
  lambda2gcv(log10lambda, argvals, y, fdParobj, wtvec=rep(1,length(argvals)))
\end{verbatim}
\end{Usage}
\begin{Arguments}
\begin{ldescription}
\item[\code{log10lambda}] the logarithm (base 10) of the smoothing parameter

\item[\code{argvals}] a vector of argument values.

\item[\code{y}] the data to be smoothed.

\item[\code{fdParobj}] a functional parameter object defining the smooth.

\item[\code{wtvec}] a weight vector used in the smoothing.

\end{ldescription}
\end{Arguments}
\begin{Details}\relax
Currently, \code{lambda2gcv}
\end{Details}
\begin{Value}
1.  \eqn{fdParobj[['lambda']] <- 10^log10lambda}{}

2.  smoothlist <- smooth.basks(argvals, y, fdParobj, wtvec)

3.  return(smoothlist[['gcv']])
\end{Value}
\begin{SeeAlso}\relax
\code{\LinkA{smooth.basis}{smooth.basis}}
\code{\LinkA{fdPar}{fdPar}}
\end{SeeAlso}

