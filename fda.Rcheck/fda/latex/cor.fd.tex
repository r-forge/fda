\HeaderA{cor.fd}{Correlation matrix from functional data object(s)}{cor.fd}
\keyword{smooth}{cor.fd}
\begin{Description}\relax
Compute a correlation matrix for one or two functional data objects.
\end{Description}
\begin{Usage}
\begin{verbatim}
cor.fd(evalarg1, fdobj1, evalarg2=evalarg1, fdobj2=fdobj1)
\end{verbatim}
\end{Usage}
\begin{Arguments}
\begin{ldescription}
\item[\code{evalarg1}] a vector of argument values for fdobj1.   

\item[\code{evalarg2}] a vector of argument values for fdobj2.  

\item[\code{fdobj1, fdobj2}] functional data objects 

\end{ldescription}
\end{Arguments}
\begin{Details}\relax
1.  var1 <- var.fd(fdobj1) 

2.  evalVar1 <- eval.bifd(evalarg1, evalarg1, var1)


3.  if(missing(fdobj2)) Convert evalVar1 to correlations


4.  else:  

4.1.  var2 <- var.fd(fdobj2)

4.2.  evalVar2 <- eval.bifd(evalarg2, evalarg2, var2)

4.3.  var12 <- var.df(fdobj1, fdobj2)

4.4.  evalVar12 <- eval.bifd(evalarg1, evalarg2, var12)

4.5.  Convert evalVar12 to correlations
\end{Details}
\begin{Value}
A matrix or array:

With one or two functional data objects, fdobj1 and possibly fdobj2,
the value is a matrix of dimensions length(evalarg1) by length(evalarg2) giving the
correlations at those points of fdobj1 if missing(fdojb2) or of
correlations between eval.fd(evalarg1, fdobj1) and eval.fd(evalarg2,
fdobj2).

With a single multivariate data object with k variables, the value is
a 4-dimensional array of dim = c(nPts, nPts, 1, choose(k+1, 2)), where
nPts = length(evalarg1).
\end{Value}
\begin{SeeAlso}\relax
\code{\LinkA{mean.fd}{mean.fd}}, 
\code{\LinkA{sd.fd}{sd.fd}}, 
\code{\LinkA{std.fd}{std.fd}}
\code{\LinkA{stdev.fd}{stdev.fd}}
\code{\LinkA{var.fd}{var.fd}}
\end{SeeAlso}
\begin{Examples}
\begin{ExampleCode}
daybasis3 <- create.fourier.basis(c(0, 365))
daybasis5 <- create.fourier.basis(c(0, 365), 5)
tempfd3 <- with(CanadianWeather, data2fd(
       dailyAv[,,"Temperature.C"], day.5,
       daybasis3, argnames=list("Day", "Station", "Deg C")) )
precfd5 <- with(CanadianWeather, data2fd(
       dailyAv[,,"log10precip"], day.5,
       daybasis5, argnames=list("Day", "Station", "Deg C")) )

# Correlation matrix for a single functional data object
(tempCor3 <- cor.fd(seq(0, 356, length=4), tempfd3))

# Cross correlation matrix between two functional data objects 
# Compare with structure described above under 'value':
(tempPrecCor3.5 <- cor.fd(seq(0, 365, length=4), tempfd3,
                          seq(0, 356, length=6), precfd5))

# The following produces contour and perspective plots

daybasis65 <- create.fourier.basis(rangeval=c(0, 365), nbasis=65)
daytempfd <- with(CanadianWeather, data2fd(
       dailyAv[,,"Temperature.C"], day.5,
       daybasis65, argnames=list("Day", "Station", "Deg C")) )
dayprecfd <- with(CanadianWeather, data2fd(
       dailyAv[,,"log10precip"], day.5,
       daybasis65, argnames=list("Day", "Station", "log10(mm)")) )

str(tempPrecCor <- cor.fd(weeks, daytempfd, weeks, dayprecfd))
# dim(tempPrecCor)= c(53, 53)

op <- par(mfrow=c(1,2), pty="s")
contour(weeks, weeks, tempPrecCor, 
        xlab="Average Daily Temperature",
        ylab="Average Daily log10(precipitation)",
        main=paste("Correlation function across locations\n",
          "for Canadian Anual Temperature Cycle"),
        cex.main=0.8, axes=FALSE)
axisIntervals(1, atTick1=seq(0, 365, length=5), atTick2=NA, 
            atLabels=seq(1/8, 1, 1/4)*365,
            labels=paste("Q", 1:4) )
axisIntervals(2, atTick1=seq(0, 365, length=5), atTick2=NA, 
            atLabels=seq(1/8, 1, 1/4)*365,
            labels=paste("Q", 1:4) )
persp(weeks, weeks, tempPrecCor,
      xlab="Days", ylab="Days", zlab="Correlation")
mtext("Temperature-Precipitation Correlations", line=-4, outer=TRUE)
par(op)

# Correlations and cross correlations
# in a bivariate functional data object
gaitbasis5 <- create.fourier.basis(nbasis=5)
gaitfd5 <- data2fd(gait, basisobj=gaitbasis5)

gait.t3 <- (0:2)/2
(gaitCor3.5 <- cor.fd(gait.t3, gaitfd5))
# Check the answers with manual computations
gait3.5 <- eval.fd(gait.t3, gaitfd5)
all.equal(cor(t(gait3.5[,,1])), gaitCor3.5[,,,1])
# TRUE
all.equal(cor(t(gait3.5[,,2])), gaitCor3.5[,,,3])
# TRUE
all.equal(cor(t(gait3.5[,,2]), t(gait3.5[,,1])),
               gaitCor3.5[,,,2])
# TRUE

# NOTE:
dimnames(gaitCor3.5)[[4]]
# [1] Hip-Hip
# [2] Knee-Hip 
# [3] Knee-Knee
# If [2] were "Hip-Knee", then
# gaitCor3.5[,,,2] would match 
# cor(t(gait3.5[,,1]), t(gait3.5[,,2]))
# *** It does NOT.  Instead, it matches:  
# cor(t(gait3.5[,,2]), t(gait3.5[,,1]))

\end{ExampleCode}
\end{Examples}

