\HeaderA{fourier}{Fourier Basis Function Values}{fourier}
\keyword{smooth}{fourier}
\begin{Description}\relax
Evaluates a set of Fourier basis functions, or a derivative of these
functions, at a set of arguments.
\end{Description}
\begin{Usage}
\begin{verbatim}
fourier(x, nbasis=n, period=span, nderiv=0)
\end{verbatim}
\end{Usage}
\begin{Arguments}
\begin{ldescription}
\item[\code{x}] a vector of argument values at which the Fourier basis functions are
to be evaluated.

\item[\code{nbasis}] the number of basis functions in the Fourier basis.  The first basis
function is the constant function, followed by sets of  sine/cosine
pairs.  Normally the number of basis functions will be an odd.  The
default number is the number of argument values.

\item[\code{period}] the width of an interval over which all sine/cosine basis functions
repeat themselves. The default is the difference between the largest
and smallest argument values.

\item[\code{nderiv}] the derivative to be evaluated.  The derivative must not exceed the
order.  The default derivative is 0, meaning that the basis functions
themselves are evaluated.

\end{ldescription}
\end{Arguments}
\begin{Value}
a matrix of function values.  The number of rows equals the number of
arguments, and the number of columns equals the number of basis functions.
\end{Value}
\begin{SeeAlso}\relax
\code{\LinkA{fourierpen}{fourierpen}}
\end{SeeAlso}
\begin{Examples}
\begin{ExampleCode}

#  set up a set of 11 argument values
x <- seq(0,1,0.1)
names(x) <- paste("x", 0:10, sep="")
#  compute values for five Fourier basis functions
#  with the default period (1) and derivative (0)
(basismat <- fourier(x, 5))

# Create a false Fourier basis, i.e., nbasis = 1
# = a constant function
fourier(x, 1)

\end{ExampleCode}
\end{Examples}

