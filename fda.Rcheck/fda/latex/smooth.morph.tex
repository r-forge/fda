\HeaderA{smooth.morph}{Estimates a Smooth Warping Function}{smooth.morph}
\keyword{smooth}{smooth.morph}
\begin{Description}\relax
This function is nearly identical to \code{smooth.monotone}
but is intended to compute a smooth monotone transformation
$h(t)$ of argument $t$ such that
$h(0) = 0$ and $h(TRUE) = TRUE$, where $t$ is
the upper limit of $t$.  This function is used primarily
to register curves.
\end{Description}
\begin{Usage}
\begin{verbatim}
smooth.morph(x, y, WfdParobj, wt=rep(1,nobs),
             conv=.0001, iterlim=20,
             active=c(FALSE,rep(TRUE,ncvec-1)),
             dbglev=1)
\end{verbatim}
\end{Usage}
\begin{Arguments}
\begin{ldescription}
\item[\code{x}] a vector of argument values.

\item[\code{y}] a vector of data values.  This function can only smooth
one set of data at a time.

\item[\code{WfdParobj}] a functional parameter object that provides an initial
value for the coefficients defining function $W(t)$,
and a roughness penalty on this function.

\item[\code{wt}] a vector of weights to be used in the smoothing.

\item[\code{conv}] a convergence criterion.

\item[\code{iterlim}] the maximum number of iterations allowed in the minimization
of error sum of squares.

\item[\code{active}] a logical vector specifying which coefficients defining
$W(t)$ are estimated.  Normally, the first coefficient
is fixed.

\item[\code{dbglev}] either 0, 1, or 2.  This controls the amount information printed out on
each iteration, with 0 implying no output, 1 intermediate output level,
and 2 full output.  If either level 1 or 2 is specified, it can be
helpful to turn off the output buffering feature of S-PLUS.

\end{ldescription}
\end{Arguments}
\begin{Value}
A named list of length 4 containing:

\begin{ldescription}
\item[\code{Wfdobj}] a functional data object defining function $W(x)$ that that
optimizes the fit to the data of the monotone function that it defines.

\item[\code{Flist}] a named list containing three results for the final converged solution:
(1)
\bold{f}: the optimal function value being minimized,
(2)
\bold{grad}: the gradient vector at the optimal solution,   and
(3)
\bold{norm}: the norm of the gradient vector at the optimal solution.

\item[\code{iternum}] the number of iterations.

\item[\code{iternum}] the number of iterations.

\item[\code{iterhist}] a \code{} by 5 matrix containing the iteration
history.

\end{ldescription}
\end{Value}
\begin{SeeAlso}\relax
\code{\LinkA{smooth.monotone}{smooth.monotone}}, 
\code{\LinkA{landmarkreg}{landmarkreg}}, 
\code{\LinkA{register.fd}{register.fd}}
\end{SeeAlso}

