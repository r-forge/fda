\HeaderA{varmx}{Rotate a Matrix of Component Loadings using the VARIMAX Criterion}{varmx}
\keyword{smooth}{varmx}
\begin{Description}\relax
The matrix being rotated contains the values of the component
functional data objects computed in either a principal
components analysis or a canonical correlation analysis.
The values are computed over a fine mesh of argument values.
\end{Description}
\begin{Usage}
\begin{verbatim}
varmx(amat)
\end{verbatim}
\end{Usage}
\begin{Arguments}
\begin{ldescription}
\item[\code{amat}] the matrix to be rotated.  The number of rows is
equal to the number of argument values \code{nx} used
in a fine mesh.  The number of columns is the number of
components to be rotated.

\end{ldescription}
\end{Arguments}
\begin{Details}\relax
The VARIMAX criterion is the variance of the squared component values.
As this criterion is maximized with respect to a rotation of the
space spanned by the columns of the matrix, the squared loadings
tend more and more to be either near 0 or near 1, and this tends to
help with the process of labelling or interpreting the rotated matrix.
\end{Details}
\begin{Value}
a square rotation matrix of order equal to the number
of components that are rotated.  A rotation matrix
$T$ has that property that $T'T = TT' = I$.
\end{Value}
\begin{SeeAlso}\relax
\code{\LinkA{varmx.pca.fd}{varmx.pca.fd}}, 
\code{\LinkA{varmx.cca.fd}{varmx.cca.fd}}
\end{SeeAlso}

