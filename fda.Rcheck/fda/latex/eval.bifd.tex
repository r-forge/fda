\HeaderA{eval.bifd}{Values a Two-argument Functional Data Object}{eval.bifd}
\keyword{smooth}{eval.bifd}
\begin{Description}\relax
A vector of argument values for the first argument \code{s} of the
functional data object to be evaluated.
\end{Description}
\begin{Usage}
\begin{verbatim}
eval.bifd(sevalarg, tevalarg, bifd, sLfdobj=0, tLfdobj=0)
\end{verbatim}
\end{Usage}
\begin{Arguments}
\begin{ldescription}
\item[\code{sevalarg}] a vector of argument values for the first argument \code{s} of the 
functional data object to be evaluated.

\item[\code{tevalarg}] a vector of argument values for the second argument \code{t} of the 
functional data object to be evaluated.

\item[\code{bifd}] a two-argument functional data object.

\item[\code{sLfdobj}] either a nonnegative integer or a linear differential operator
object.  If present, the derivative or the value of applying the
operator to the object as a function of the first argument \code{s}
is evaluated rather than the functions themselves.

\item[\code{tLfdobj}] either a nonnegative integer or a linear differential operator
object.  If present, the derivative or the value of applying the
operator to the object as a function of the second argument \code{t}
is evaluated rather than the functions themselves.

\end{ldescription}
\end{Arguments}
\begin{Value}
an array of 2, 3, or 4 dimensions containing the function values.  The
first dimension corresponds to the argument values in sevalarg, the
second to argument values in tevalarg, the third if present to
replications, and the fourth if present to functions.
\end{Value}
\begin{Examples}
\begin{ExampleCode}
daybasis   <- create.fourier.basis(c(0,365), 365)
harmLcoef  <- c(0,(2*pi/365)^2,0)
harmLfd    <- vec2Lfd(harmLcoef, c(0,365))
templambda <- 1.0
tempfdPar  <- fdPar(daybasis, harmLfd, lambda=1)
tempfd     <- smooth.basis(day.5,
          CanadianWeather$dailyAv[,,"Temperature.C"], tempfdPar)$fd
#    define the variance-covariance bivariate fd object
tempvarbifd <- var.fd(tempfd)
#    evaluate the variance-covariance surface and plot
weektime    <- seq(0,365,len=53)
tempvarmat  <- eval.bifd(weektime,weektime,tempvarbifd)
#    make a perspective plot of the variance function
persp(tempvarmat)
\end{ExampleCode}
\end{Examples}

