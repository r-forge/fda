\HeaderA{center.fd}{Center Functional Data}{center.fd}
\keyword{smooth}{center.fd}
\begin{Description}\relax
Subtract the pointwise mean from each of the functions
in a functional data object; that is, to center them on the mean function.
\end{Description}
\begin{Usage}
\begin{verbatim}
center.fd(fdobj)
\end{verbatim}
\end{Usage}
\begin{Arguments}
\begin{ldescription}
\item[\code{fdobj}] a functional data object to be centered.

\end{ldescription}
\end{Arguments}
\begin{Value}
a functional data object whose mean is zero.
\end{Value}
\begin{SeeAlso}\relax
\code{\LinkA{mean.fd}{mean.fd}}, 
\code{\LinkA{sum.fd}{sum.fd}}, 
\code{\LinkA{stddev.fd}{stddev.fd}}, 
\code{\LinkA{std.fd}{std.fd}}
\end{SeeAlso}
\begin{Examples}
\begin{ExampleCode}
daytime    <- (1:365)-0.5
daybasis   <- create.fourier.basis(c(0,365), 365)
harmLcoef  <- c(0,(2*pi/365)^2,0)
harmLfd    <- vec2Lfd(harmLcoef, c(0,365))
templambda <- 0.01
tempfdPar  <- fdPar(daybasis, harmLfd, templambda)
tempfd     <- smooth.basis(daytime,
       CanadianWeather$dailyAv[,,"Temperature.C"], tempfdPar)$fd
tempctrfd  <- center.fd(tempfd)

plot(tempctrfd, xlab="Day", ylab="deg. C",
     main = "Centered temperature curves")
\end{ExampleCode}
\end{Examples}

