\HeaderA{growth}{Berkeley Growth Study data}{growth}
\keyword{datasets}{growth}
\begin{Description}\relax
A list containing the heights of 39 boys and 54 girls from age 1 to 18
and the ages at which they were collected.
\end{Description}
\begin{Format}\relax
This list contains the following components:
\describe{
\item[hgtm] a 31 by 39 numeric matrix giving the heights in centimeters of
39 boys at 31 ages.    

\item[hgtf] a 31 by 54 numeric matrix giving the heights in centimeters of
54 girls at 31 ages.  

\item[age] a numeric vector of length 31 giving the ages at which the
heights were measured.  

}
\end{Format}
\begin{Details}\relax
The ages are not equally spaced.
\end{Details}
\begin{Source}\relax
Ramsay, James O., and Silverman, Bernard W. (2006), \emph{Functional
Data Analysis, 2nd ed.}, Springer, New York. 

Ramsay, James O., and Silverman, Bernard W. (2002), \emph{Applied
Functional Data Analysis}, Springer, New York, ch. 6. 

Tuddenham, R. D., and Snyder, M. M. (1954) "Physical growth of
California boys and girls from birth to age 18", \emph{University of
California Publications in Child Development}, 1, 183-364.
\end{Source}
\begin{Examples}
\begin{ExampleCode}
with(growth, matplot(age, hgtf[, 1:10], type="b"))
\end{ExampleCode}
\end{Examples}

