\HeaderA{eval.posfd}{Evaluate a Positive Functional Data Object}{eval.posfd}
\keyword{smooth}{eval.posfd}
\begin{Description}\relax
Evaluate a positive functional data object at specified argument values,
or evaluate a derivative of the functional object.
\end{Description}
\begin{Usage}
\begin{verbatim}
eval.posfd(evalarg, Wfdobj, Lfdobj=int2Lfd(0))
\end{verbatim}
\end{Usage}
\begin{Arguments}
\begin{ldescription}
\item[\code{evalarg}] a vector of argument values at which the functional data object is to be
evaluated.

\item[\code{Wfdobj}] a functional data object that defines the positive function to be
evaluated.  Only univariate functions are permitted.

\item[\code{Lfdobj}] a nonnegative integer specifying a derivative to be evaluated.  AT
this time of writing, permissible derivative values are 0, 1 or 2.
A linear differential operator is not allowed.

\end{ldescription}
\end{Arguments}
\begin{Details}\relax
A positive function data object $h(t)$ is defined by $h(t) =[exp Wfd](t)$.
The function \code{Wfdobj} that defines the positive function is
usually estimated by positive smoothing function
\code{smooth.positive}
\end{Details}
\begin{Value}
a matrix containing the positive function
values.  The first dimension corresponds to the argument values in
\code{evalarg} and
the second to replications.
\end{Value}
\begin{SeeAlso}\relax
\code{\LinkA{eval.fd}{eval.fd}}, 
\code{\LinkA{eval.monfd}{eval.monfd}}
\end{SeeAlso}

