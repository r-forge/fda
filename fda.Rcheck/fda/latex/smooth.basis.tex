\HeaderA{smooth.basis}{Smooth Data with an Indirectly Specified Roughness Penalty}{smooth.basis}
\keyword{smooth}{smooth.basis}
\begin{Description}\relax
This is the main function for smoothing data using a roughness
penalty.  Unlike function \code{data2fd}, which does not employ a
rougness penalty, this function controls the nature and degree of
smoothing by penalyzing a measure of rougness.  Roughness is definable
in a wide variety of ways using either derivatives or a linear
differential operator.
\end{Description}
\begin{Usage}
\begin{verbatim}
smooth.basis(argvals, y, fdParobj, wtvec=rep(1, length(argvals)),
             fdnames=NULL)
\end{verbatim}
\end{Usage}
\begin{Arguments}
\begin{ldescription}
\item[\code{argvals}] a vector of argument values correspond to the observations in array
\code{y}.

\item[\code{y}] an array containing values of curves at discrete sampling points or
argument values. If the array is a matrix, the rows must correspond
to argument values and columns to replications, and it will be
assumed that there is only one variable per observation.  If
\code{y} is a three-dimensional array, the first dimension
corresponds to argument values, the second to replications, and the
third to variables within replications.  If \code{y} is a vector,
only one replicate and variable are assumed.

\item[\code{fdParobj}] a functional parameter object, a functional data object or a
functional basis object.  If the object is a functional parameter
object, then the linear differential operator object and the
smoothing parameter in this object define the roughness penalty.  If
the object is a functional data object, the basis within this object
is used without a roughness penalty, and this is also the case if
the object is a functional basis object.  In these latter two cases,
\code{smooth.basis} is essentially the same as \code{data2fd}.

\item[\code{wtvec}] a vector of the same length as \code{argvals} containing weights for
the values to be smoothed.

\item[\code{fdnames}] a list of length 3 containing character vectors of names for the
following:

\Itemize{
\item[args] name for each observation or point in time at which data are
collected for each 'rep', unit or subject.

\item[reps] name for each 'rep', unit or subject.

\item[fun] name for each 'fun' or (response) variable measured repeatedly
(per 'args') for each 'rep'.

}

\end{ldescription}
\end{Arguments}
\begin{Details}\relax
If the smoothing parameter \code{lambda} is zero, there is no penalty
on roughness.  As lambda increases, usually in logarithmic terms, the
penalty on roughness increases and the fitted curves become more and
more smooth.  Ultimately, the curves are forced to have zero roughness
in the sense of being in the null space of the linear differential
operator object \code{Lfdobj}that is a member of the \code{fdParobj}.

For example, a common choice of roughness penalty is the integrated
square of the second derivative.  This penalizes curvature.  Since the
second derivative of a straight line is zero, very large values of
\code{lambda} will force the fit to become linear.  It is also
possible to control the amount of roughness by using a degrees of
freedom measure.  The value equivalent to \code{lambda} is found in
the list returned by the function.  On the other hand, it is possible
to specify a degrees of freedom value, and then use function
\code{df2lambda} to determine the equivalent value of \code{lambda}.
One should not put complete faith in any automatic method for
selecting \code{lambda}, including the GCV method. There are many
reasons for this.  For example, if derivatives are required, then the
smoothing level that is automatically selected may give unacceptably
rough derivatives.  These methods are also highly sensitive to the
assumption of independent errors, which is usually dubious with
functional data.  The best advice is to start with the value
minimizing the \code{gcv} measure, and then explore \code{lambda}
values a few log units up and down from this value to see what the
smoothing function and its derivatives look like.  The function
\code{plotfit.fd} was designed for this purpose.

An alternative to using \code{smooth.basis} is to first represent
the data in a basis system with reasonably high resolution using
\code{data2fd}, and then smooth the resulting functional data object
using function \code{smooth.fd}.
\end{Details}
\begin{Value}
an object of class \code{fdSmooth}, which is a named list of length 6
with the following components:

\begin{ldescription}
\item[\code{fd}] a functional data object containing a smooth of the data.

\item[\code{df}] a degrees of freedom measure of the smooth

\item[\code{gcv}] the value of the generalized cross-validation or GCV criterion.  If
there are multiple curves, this is a vector of values, one per
curve.  If the smooth is multivariate, the result is a matrix of gcv
values, with columns corresponding to variables.

\deqn{gcv = n*SSE/((n-df)^2)}{}

\item[\code{SSE}] the error sums of squares.  SSE is a vector or a matrix of the same
size as GCV.

\item[\code{penmat}] the penalty matrix.

\item[\code{y2cMap}] the matrix mapping the data to the coefficients.

\item[\code{argvals, y}] input arguments
\end{ldescription}
\end{Value}
\begin{SeeAlso}\relax
\code{\LinkA{data2fd}{data2fd}}, \code{\LinkA{df2lambda}{df2lambda}},
\code{\LinkA{lambda2df}{lambda2df}}, \code{\LinkA{lambda2gcv}{lambda2gcv}},
\code{\LinkA{plot.fd}{plot.fd}}, \code{\LinkA{project.basis}{project.basis}},
\code{\LinkA{smooth.fd}{smooth.fd}}, \code{\LinkA{smooth.monotone}{smooth.monotone}},
\code{\LinkA{smooth.pos}{smooth.pos}}, \code{\LinkA{smooth.basisPar}{smooth.basisPar}}
\end{SeeAlso}
\begin{Examples}
\begin{ExampleCode}
##
## Example 1:  Inappropriate smoothing
##
# A toy example that creates problems with
# data2fd:  (0,0) -> (0.5, -0.25) -> (1,1)
b2.3 <- create.bspline.basis(norder=2, breaks=c(0, .5, 1))
# 3 bases, order 2 = degree 1 =
# continuous, bounded, locally linear
fdPar2 <- fdPar(b2.3, Lfdobj=2, lambda=1)

## Not run: 
# Penalize excessive slope Lfdobj=1;
# second derivative Lfdobj=2 is discontinuous,
# so the following generates an error:
  fd2.3s0 <- smooth.basis(0:1, 0:1, fdPar2)
Derivative of order 2 cannot be taken for B-spline of order 2
Probable cause is a value of the nbasis argument
 in function create.basis.fd that is too small.
Error in bsplinepen(basisobj, Lfdobj, rng) :
## End(Not run)

##
## Example 2.  Better
##
b3.4 <- create.bspline.basis(norder=3, breaks=c(0, .5, 1))
# 4 bases, order 3 = degree 2 =
# continuous, bounded, locally quadratic
fdPar3 <- fdPar(b3.4, lambda=1)
# Penalize excessive slope Lfdobj=1;
# second derivative Lfdobj=2 is discontinuous.
fd3.4s0 <- smooth.basis(0:1, 0:1, fdPar3)

plot(fd3.4s0$fd)
# same plot via plot.fdSmooth
plot(fd3.4s0)

##
## Example 3.  lambda = 1, 0.0001, 0
##
#  Shows the effects of three levels of smoothing
#  where the size of the third derivative is penalized.
#  The null space contains quadratic functions.
x <- seq(-1,1,0.02)
y <- x + 3*exp(-6*x^2) + rnorm(rep(1,101))*0.2
#  set up a saturated B-spline basis
basisobj <- create.bspline.basis(c(-1,1), 101)

fdPar1 <- fdPar(basisobj, 2, lambda=1)
result1  <- smooth.basis(x, y, fdPar1)
with(result1, c(df, gcv, SSE))

##
## Example 4.  lambda = 0.0001
##
fdPar.0001 <- fdPar(basisobj, 2, lambda=0.0001)
result2  <- smooth.basis(x, y, fdPar.0001)
with(result2, c(df, gcv, SSE))
# less smoothing, more degrees of freedom,
# smaller gcv, smaller SSE

##
##  Example 5.  lambda = 0
##
fdPar0 <- fdPar(basisobj, 2, lambda=0)
result3  <- smooth.basis(x, y, fdPar0)
with(result3, c(df, gcv, SSE))
# Saturate fit:  number of observations = nbasis
# with no smoothing, so degrees of freedom = nbasis,
# gcv = Inf indicating overfitting;
# SSE = 0 (to within roundoff error)

plot(x,y)           # plot the data
lines(result1[['fd']], lty=2)  #  add heavily penalized smooth
lines(result2[['fd']], lty=1)  #  add reasonably penalized smooth
lines(result3[['fd']], lty=3)  #  add smooth without any penalty
legend(-1,3,c("1","0.0001","0"),lty=c(2,1,3))

plotfit.fd(y, x, result2[['fd']])  # plot data and smooth

##
## Example 6.  Supersaturated
##
basis104 <- create.bspline.basis(c(-1,1), 104)

fdPar104.0 <- fdPar(basis104, 2, lambda=0)
result104.0  <- smooth.basis(x, y, fdPar104.0)
with(result104.0, c(df, gcv, SSE))

plotfit.fd(y, x, result104.0[['fd']], nfine=501)
# perfect (over)fit
# Need lambda > 0.

##
## Example 7.  gait
##
gaittime  <- (1:20)/21
gaitrange <- c(0,1)
gaitbasis <- create.fourier.basis(gaitrange,21)
lambda    <- 10^(-11.5)
harmaccelLfd <- vec2Lfd(c(0, 0, (2*pi)^2, 0))

gaitfdPar <- fdPar(gaitbasis, harmaccelLfd, lambda)
gaitfd <- smooth.basis(gaittime, gait, gaitfdPar)$fd
## Not run: 
# by default creates multiple plots, asking for a click between plots
plotfit.fd(gait, gaittime, gaitfd)
## End(Not run)
\end{ExampleCode}
\end{Examples}

