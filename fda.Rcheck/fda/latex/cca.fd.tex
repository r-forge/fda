\HeaderA{cca.fd}{Functional Canonical Correlation Analysis}{cca.fd}
\keyword{smooth}{cca.fd}
\begin{Description}\relax
Carry out a functional canonical correlation analysis with
regularization or roughness penalties on the estimated
canonical variables.
\end{Description}
\begin{Usage}
\begin{verbatim}
cca.fd(fdobj1, fdobj2=fdobj1, ncan = 2,
       ccafdParobj1=fdPar(basisobj1, 2, 1e-10),
       ccafdParobj2=ccafdParobj1, centerfns=TRUE)
\end{verbatim}
\end{Usage}
\begin{Arguments}
\begin{ldescription}
\item[\code{fdobj1}] a functional data object.

\item[\code{fdobj2}] a functional data object.  By default this is \code{ fdobj1 }, in
which case the first argument must be a bivariate funnctional data
object.

\item[\code{ncan}] the number of canonical variables and weight functions to be
computed.  The default is 2.

\item[\code{ccafdParobj1}] a functional parameter object defining the first set of canonical
weight functions.  The object may contain specifications for a
roughness penalty. The default is defined using the same basis
as that used for \code{ fdobj1 } with a slight penalty on its
second derivative.

\item[\code{ccafdParobj2}] a functional parameter object defining the second set of canonical
weight functions.  The object may contain specifications for a
roughness penalty. The default is \code{ ccafdParobj1 }.

\item[\code{centerfns}] if TRUE, the functions are centered prior to analysis. This is the
default.

\end{ldescription}
\end{Arguments}
\begin{Value}
an object of class \code{cca.fd} with the 5 slots:

\begin{ldescription}
\item[\code{ccwtfd1}] a functional data object for the first
canonical variate weight function

\item[\code{ccwtfd2}] a functional data object for the second
canonical variate weight function

\item[\code{cancorr}] a vector of canonical correlations

\item[\code{ccavar1}] a matrix of scores on the first canonical variable.

\item[\code{ccavar2}] a matrix of scores on the second canonical variable.

\end{ldescription}
\end{Value}
\begin{SeeAlso}\relax
\code{\LinkA{plot.cca.fd}{plot.cca.fd}}, 
\code{\LinkA{varmx.cca.fd}{varmx.cca.fd}}, 
\code{\LinkA{pca.fd}{pca.fd}}
\end{SeeAlso}
\begin{Examples}
\begin{ExampleCode}
#  Canonical correlation analysis of knee-hip curves

gaittime  <- (1:20)/21
gaitrange <- c(0,1)
gaitbasis <- create.fourier.basis(gaitrange,21)
lambda    <- 10^(-11.5)
harmaccelLfd <- vec2Lfd(c(0, 0, (2*pi)^2, 0))

gaitfdPar <- fdPar(gaitbasis, harmaccelLfd, lambda)
gaitfd <- smooth.basis(gaittime, gait, gaitfdPar)$fd

ccafdPar <- fdPar(gaitfd, harmaccelLfd, 1e-8)
ccafd0    <- cca.fd(gaitfd[,1], gaitfd[,2], ncan=3, ccafdPar, ccafdPar)
#  compute a VARIMAX rotation of the canonical variables
ccafd <- varmx.cca.fd(ccafd0)
#  plot the canonical weight functions
op <- par(mfrow=c(2,1))
#plot.cca.fd(ccafd, cex=1.2, ask=TRUE)
#plot.cca.fd(ccafd, cex=1.2)
#  display the canonical correlations
#round(ccafd$ccacorr[1:6],3)
par(op)
\end{ExampleCode}
\end{Examples}

