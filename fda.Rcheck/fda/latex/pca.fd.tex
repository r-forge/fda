\HeaderA{pca.fd}{Functional Principal Components Analysis}{pca.fd}
\keyword{smooth}{pca.fd}
\begin{Description}\relax
Functional Principal components analysis aims to display types of
variation across a sample of functions.  Principal components analysis
is an exploratory data analysis that tends to be an early part of many
projects.  These modes of variation are called $principal components$
or $harmonics.$  This function computes these harmonics, the
eigenvalues that indicate how important each mode of variation, and
harmonic scores for individual functions. If the functions are
multivariate, these harmonics are combined into a composite function
that summarizes joint variation among the several functions that make
up a multivariate functional observation.
\end{Description}
\begin{Usage}
\begin{verbatim}
pca.fd(fdobj, nharm = 2, harmfdPar=fdPar(fdobj),
       centerfns = TRUE)
\end{verbatim}
\end{Usage}
\begin{Arguments}
\begin{ldescription}
\item[\code{fdobj}] a functional data object.

\item[\code{nharm}] the number of harmonics or principal components to compute.

\item[\code{harmfdPar}] a functional parameter object that defines the
harmonic or principal component functions to be estimated.

\item[\code{centerfns}] a logical value:
if TRUE, subtract the mean function from each function before
computing principal components.

\end{ldescription}
\end{Arguments}
\begin{Value}
an object of class "pca.fd" with these named entries:

\begin{ldescription}
\item[\code{harmonics}] a functional data object for the harmonics or eigenfunctions

\item[\code{values}] the complete set of eigenvalues

\item[\code{scores}] s matrix of scores on the principal components or harmonics

\item[\code{varprop}] a vector giving the proportion of variance explained
by each eigenfunction

\item[\code{meanfd}] a functional data object giving the mean function

\end{ldescription}
\end{Value}
\begin{SeeAlso}\relax
\code{\LinkA{cca.fd}{cca.fd}},
\code{\LinkA{pda.fd}{pda.fd}}
\end{SeeAlso}
\begin{Examples}
\begin{ExampleCode}

#  carry out a PCA of temperature
#  penalize harmonic acceleration, use varimax rotation

daybasis65 <- create.fourier.basis(c(0, 365), nbasis=65, period=365)

harmaccelLfd <- vec2Lfd(c(0,(2*pi/365)^2,0), c(0, 365))
harmfdPar     <- fdPar(daybasis65, harmaccelLfd, lambda=1e5)
daytempfd <- data2fd(CanadianWeather$dailyAv[,,"Temperature.C"],
      day.5, daybasis65, argnames=list("Day", "Station", "Deg C"))

daytemppcaobj <- pca.fd(daytempfd, nharm=4, harmfdPar)
daytemppcaVarmx <- varmx.pca.fd(daytemppcaobj)
#  plot harmonics
op <- par(mfrow=c(2,2))
plot.pca.fd(daytemppcaobj, cex.main=0.9)

plot.pca.fd(daytemppcaVarmx, cex.main=0.9)
par(op)

plot(daytemppcaobj$harmonics)
plot(daytemppcaVarmx$harmonics)
\end{ExampleCode}
\end{Examples}

