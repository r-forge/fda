\HeaderA{create.polynomial.basis}{Create a Polynomial Basis}{create.polynomial.basis}
\keyword{smooth}{create.polynomial.basis}
\begin{Description}\relax
Creates a set of basis functions consisting of powers
of the argument shifted by a constant.
\end{Description}
\begin{Usage}
\begin{verbatim}
create.polynomial.basis(rangeval=c(0, 1), nbasis=2, ctr=0,
                   dropind=NULL, quadvals=NULL, values=NULL,
                   basisvalues=NULL, names='polynom')
\end{verbatim}
\end{Usage}
\begin{Arguments}
\begin{ldescription}
\item[\code{rangeval}] a vector of length 2 defining the range.

\item[\code{nbasis}] the number of basis functions. The default is 2,
which defines a basis for straight lines.

\item[\code{ctr}] this value is used to shift the argument prior to taking
its power.

\item[\code{dropind}] a vector of integers specifiying the basis functions to
be dropped, if any.  For example, if it is required that
a function be zero at the left boundary, this is achieved
by dropping the first basis function, the only one that
is nonzero at that point.

\item[\code{quadvals}] a matrix with two columns and a number of rows equal to the number
of quadrature points for numerical evaluation of the penalty
integral.  The first column of \code{quadvals} contains the
quadrature points, and the second column the quadrature weights.  A
minimum of 5 values are required for each inter-knot interval, and
that is often enough.  For Simpson's rule, these points are equally
spaced, and the weights are proportional to 1, 4, 2, 4, ..., 2, 4,
1.

\item[\code{values}] a list of matrices with one row for each row of \code{quadvals} and
one column for each basis function.  The elements of the list
correspond to the basis functions and their derivatives evaluated at
the quadrature points contained in the first column of
\code{quadvals}.

\item[\code{basisvalues}] A list of lists, allocated by code such as vector("list",1).  This
field is designed to avoid evaluation of a basis system repeatedly
at a set of argument values.  Each list within the vector
corresponds to a specific set of argument values, and must have at
least two components, which may be tagged as you wish.  `The first
component in an element of the list vector contains the argument
values.  The second component in an element of the list vector
contains a matrix of values of the basis functions evaluated at the
arguments in the first component.  The third and subsequent
components, if present, contain matrices of values their derivatives
up to a maximum derivative order.  Whenever function getbasismatrix
is called, it checks the first list in each row to see, first, if
the number of argument values corresponds to the size of the first
dimension, and if this test succeeds, checks that all of the
argument values match.  This takes time, of course, but is much
faster than re-evaluation of the basis system.  Even this time can
be avoided by direct retrieval of the desired array.  For example,
you might set up a vector of argument values called "evalargs" along
with a matrix of basis function values for these argument values
called "basismat".  You might want too use names like "args" and
"values", respectively for these.  You would then assign them to
\code{basisvalues} with code such as the following:

basisobj\$basisvalues <- vector("list",1)

basisobj\$basisvalues[[1]] <- list(args=evalargs,
values=basismat)

\item[\code{names}] either a character vector of the same length as the number of basis
functions or a simple stem used to construct such a vector.

For \code{polynom} bases, this defaults to paste('polynom',
1:nbreaks, sep='').

\end{ldescription}
\end{Arguments}
\begin{Details}\relax
The only difference between a monomial and a polynomial basis
is the use of a shift value.  This helps to avoid rounding error
when the argument values are a long way from zero.
\end{Details}
\begin{Value}
a basis object with the type \code{polynom}.
\end{Value}
\begin{SeeAlso}\relax
\code{\LinkA{basisfd}{basisfd}},
\code{\LinkA{create.basis}{create.basis}},
\code{\LinkA{create.bspline.basis}{create.bspline.basis}},
\code{\LinkA{create.constant.basis}{create.constant.basis}},
\code{\LinkA{create.fourier.basis}{create.fourier.basis}},
\code{\LinkA{create.exponential.basis}{create.exponential.basis}},
\code{\LinkA{create.monomial.basis}{create.monomial.basis}},
\code{\LinkA{create.polygonal.basis}{create.polygonal.basis}},
\code{\LinkA{create.power.basis}{create.power.basis}}
\end{SeeAlso}
\begin{Examples}
\begin{ExampleCode}
#  Create a polynomial basis over the years in the 20th century
#  and center the basis functions on 1950.
basisobj <- create.polynomial.basis(c(1900, 2000), nbasis=3, ctr=1950)
#  plot the basis
# The following should work but doesn't;  2007.05.01
#plot(basisobj)
\end{ExampleCode}
\end{Examples}

