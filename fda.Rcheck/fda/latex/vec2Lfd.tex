\HeaderA{vec2Lfd}{Make a Linear Differential Operator Object from a Vector}{vec2Lfd}
\keyword{smooth}{vec2Lfd}
\begin{Description}\relax
A linear differential operator object of order $m$ is
constructed from the number in a vector of length $m$.
\end{Description}
\begin{Usage}
\begin{verbatim}
vec2Lfd(bwtvec, rangeval=c(0,1))
\end{verbatim}
\end{Usage}
\begin{Arguments}
\begin{ldescription}
\item[\code{bwtvec}] a vector of coefficients to define the linear differential
operator object

\item[\code{rangeval}] a vector of length 2 specifying the range over which the
operator is defined

\end{ldescription}
\end{Arguments}
\begin{Value}
a linear differential operator object
\end{Value}
\begin{SeeAlso}\relax
\code{\LinkA{int2Lfd}{int2Lfd}}, 
\code{\LinkA{Lfd}{Lfd}}
\end{SeeAlso}
\begin{Examples}
\begin{ExampleCode}
#  define the harmonic acceleration operator used in the
#  analysis of the daily temperature data
harmaccelLfd <- vec2Lfd(c(0,(2*pi/365)^2,0), c(0,365))
\end{ExampleCode}
\end{Examples}

