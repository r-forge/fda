\HeaderA{odesolv}{Numerical Solution mth Order Differential Equation System}{odesolv}
\keyword{smooth}{odesolv}
\begin{Description}\relax
The system of differential equations is linear, with
possibly time-varying coefficient functions.
The numerical solution is computed with the Runge-Kutta method.
\end{Description}
\begin{Usage}
\begin{verbatim}
odesolv(bwtlist, ystart=diag(rep(1,norder)),
        h0=width/100, hmin=width*1e-10, hmax=width*0.5,
        EPS=1e-4, MAXSTP=1000)
\end{verbatim}
\end{Usage}
\begin{Arguments}
\begin{ldescription}
\item[\code{bwtlist}] a list whose members are functional parameter objects
defining the weight functions for the linear differential
equation.

\item[\code{ystart}] a vector of initial values for the equations.  These
are the values at time 0 of the solution and its first
m - 1 derivatives.

\item[\code{h0}] a positive initial step size.

\item[\code{hmin}] the minimum allowable step size.

\item[\code{hmax}] the maximum allowable step size.

\item[\code{EPS}] a convergence criterion.

\item[\code{MAXSTP}] the maximum number of steps allowed.

\end{ldescription}
\end{Arguments}
\begin{Details}\relax
This function is required to compute a set of solutions of an
estimated linear differential equation in order compute a fit
to the data that solves the equation.  Such a fit will be a
linear combinations of m independent solutions.
\end{Details}
\begin{Value}
a named list of length 2 containing

\begin{ldescription}
\item[\code{tp}] a vector of time values at which the system is evaluated

\item[\code{yp}] a matrix of variable values corresponding to \code{tp}.

\end{ldescription}
\end{Value}
\begin{SeeAlso}\relax
\code{\LinkA{pda.fd}{pda.fd}},
\end{SeeAlso}
\begin{Examples}
\begin{ExampleCode}
#See the analyses of the lip data.
\end{ExampleCode}
\end{Examples}

