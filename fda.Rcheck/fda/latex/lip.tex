\HeaderA{lip}{Lip motion}{lip}
\aliasA{lipmarks}{lip}{lipmarks}
\aliasA{liptime}{lip}{liptime}
\keyword{datasets}{lip}
\begin{Description}\relax
51 measurements of the position of the lower lip every 7 milliseconds
for 20 repitions of the syllable 'bob'.
\end{Description}
\begin{Usage}
\begin{verbatim}
lip
lipmarks 
liptime
\end{verbatim}
\end{Usage}
\begin{Format}\relax
\item[lip] a matrix of dimension c(51, 20) giving the position of the lower
lip every 7 milliseconds for 350 miliseconds.

\item[lipmarks] a matrix of dimension c(20, 2) giving the positions of the
'leftElbow' and 'rightElbow' in each of the 20 repitions of the
syllable 'bob'.  

\item[liptime] time in seconds from the start = seq(0, 0.35, 51) = every 7
milliseconds.  
\end{Format}
\begin{Details}\relax
These are rather simple data, involving the movement of the lower lip
while saying "bob".  There are 20 replications and 51 sampling points.
The data are used to illustrate two techniques:  landmark registration
and principal differental analysis.  
Principal differential analysis estimates a linear differential equation
that can be used to describe not only the observed curves, but also a 
certain number of their derivatives.  
For a rather more elaborate example of principal differential analysis, 
see the handwriting data.

See the \code{lip} \code{demo}.
\end{Details}
\begin{Source}\relax
Ramsay, James O., and Silverman, Bernard W. (2006), \emph{Functional
Data Analysis, 2nd ed.}, Springer, New York, sections 19.2 and
19.3.
\end{Source}
\begin{Examples}
\begin{ExampleCode}
#  See the this-is-escaped-codenormal-bracket21bracket-normal this-is-escaped-codenormal-bracket22bracket-normal.  
\end{ExampleCode}
\end{Examples}

