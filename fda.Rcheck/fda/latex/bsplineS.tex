\HeaderA{bsplineS}{B-spline Basis Function Values}{bsplineS}
\keyword{smooth}{bsplineS}
\begin{Description}\relax
Evaluates a set of B-spline basis functions, or a derivative of these
functions, at a set of arguments.
\end{Description}
\begin{Usage}
\begin{verbatim}
bsplineS(x, breaks, norder=4, nderiv=0)
\end{verbatim}
\end{Usage}
\begin{Arguments}
\begin{ldescription}
\item[\code{x}] A vector of argument values at which the B-spline basis functions
are to be evaluated.

\item[\code{breaks}] A strictly increasing set of break values defining the B-spline
basis.  The argument values \code{x} should be within the interval
spanned by the break values.

\item[\code{norder}] The order of the B-spline basis functions.  The order less one is
the degree of the piece-wise polynomials that make up any B-spline
function. The default is order 4, meaning piece-wise cubic.

\item[\code{nderiv}] A nonnegative integer specifying the order of derivative to be
evaluated.  The derivative must not exceed the order.  The default 
derivative is 0, meaning that the basis functions themselves are
evaluated. 

\end{ldescription}
\end{Arguments}
\begin{Value}
a matrix of function values.  The number of rows equals the number of 
arguments, and the number of columns equals the number of basis
functions.
\end{Value}
\begin{Examples}
\begin{ExampleCode}
# Minimal example:  A B-spline of order 1 (i.e., a step function)
# with 0 interior knots:
bsplineS(seq(0, 1, .2), 0:1, 1, 0)

#  set up break values at 0.0, 0.2,..., 0.8, 1.0.
breaks <- seq(0,1,0.2)
#  set up a set of 11 argument values
x <- seq(0,1,0.1)
#  the order willl be 4, and the number of basis functions
#  is equal to the number of interior break values (4 here)
#  plus the order, for a total here of 8.
norder <- 4
#  compute the 11 by 8 matrix of basis function values
basismat <- bsplineS(x, breaks, norder)
\end{ExampleCode}
\end{Examples}

