\HeaderA{sd.fd}{Standard Deviation of Functional Data}{sd.fd}
\aliasA{std.fd}{sd.fd}{std.fd}
\aliasA{stddev.fd}{sd.fd}{stddev.fd}
\aliasA{stdev.fd}{sd.fd}{stdev.fd}
\keyword{smooth}{sd.fd}
\begin{Description}\relax
Evaluate the standard deviation of a set of functions in a functional
data object.
\end{Description}
\begin{Usage}
\begin{verbatim}
sd.fd(fdobj)
std.fd(fdobj)
stdev.fd(fdobj)
stddev.fd(fdobj)
\end{verbatim}
\end{Usage}
\begin{Arguments}
\begin{ldescription}
\item[\code{fdobj}] a functional data object.

\end{ldescription}
\end{Arguments}
\begin{Details}\relax
The multiple aliases are provided for compatibility with previous
versions and with other languages.  The name for the standard
deviation function in R is 'sd'.  Matlab uses 'std'.  S-Plus and
Microsoft Excal use 'stdev'.  'stddev' was used in a previous version
of the 'fda' package and is retained for compatibility.
\end{Details}
\begin{Value}
a functional data object with a single replication
that contains the standard deviation of the one or several functions in
the object \code{fdobj}.
\end{Value}
\begin{SeeAlso}\relax
\code{\LinkA{mean.fd}{mean.fd}}, 
\code{\LinkA{sum.fd}{sum.fd}}, 
\code{\LinkA{center.fd}{center.fd}}
\end{SeeAlso}
\begin{Examples}
\begin{ExampleCode}
liptime  <- seq(0,1,.02)
liprange <- c(0,1)

#  -------------  create the fd object -----------------
#       use 31 order 6 splines so we can look at acceleration

nbasis <- 51
norder <- 6
lipbasis <- create.bspline.basis(liprange, nbasis, norder)
lipbasis <- create.bspline.basis(liprange, nbasis, norder)

#  ------------  apply some light smoothing to this object  -------

Lfdobj   <- int2Lfd(4)
lambda   <- 1e-12
lipfdPar <- fdPar(lipbasis, Lfdobj, lambda)

lipfd <- smooth.basis(liptime, lip, lipfdPar)$fd
names(lipfd$fdnames) = c("Normalized time", "Replications", "mm")

lipstdfd <- sd.fd(lipfd)
plot(lipstdfd)

all.equal(lipstdfd, std.fd(lipfd))
all.equal(lipstdfd, stdev.fd(lipfd))
all.equal(lipstdfd, stddev.fd(lipfd))

\end{ExampleCode}
\end{Examples}

