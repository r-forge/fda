\HeaderA{Lfd}{Define a Linear Differential Operator Object}{Lfd}
\keyword{smooth}{Lfd}
\begin{Description}\relax
A linear differential operator of order $m$ is defined,
usually to specify a roughness penalty.
\end{Description}
\begin{Usage}
\begin{verbatim}
Lfd(nderiv=0, bwtlist=vector("list", 0))
\end{verbatim}
\end{Usage}
\begin{Arguments}
\begin{ldescription}
\item[\code{nderiv}] a nonnegative integer specifying the order $m$ of the
highest order derivative in the operator

\item[\code{bwtlist}] a list of length $m$.  Each member contains a
functional data object that acts as a weight function for a
derivative.  The first member weights the function, the
second the first derivative, and so on up to order $m-1$.

\end{ldescription}
\end{Arguments}
\begin{Details}\relax
To check that an object is of this class, use functions
\code{is.Lfd} or \code{int2Lfd}.

Linear differential operator objects are often used to
define roughness penalties for smoothing towards a
"hypersmooth" function that is annihilated by the operator.
For example, the harmonic acceleration operator used in the
analysis of the Canadian daily weather data annihilates linear
combinations of $1, sin(2 pi t/365)$ and $cos(2 pi t/365)$,
and the larger the smoothing parameter, the closer the smooth
function will be to a function of this shape.

Function \code{pda.fd} estimates a linear differential
operator object that comes as close as possible to annihilating
a functional data object.

A linear differential operator of order $m$ is a
linear combination of the derivatives of a functional
data object up to order $m$.  The derivatives of
orders 0, 1, ..., $m-1$ can each be multiplied
by a weight function $b(t)$ that may or may not vary with
argument $t$.

If the notation $D^j$ is taken to
mean "take the derivative of order $j$", then a linear
differental operator $L$ applied to function $x$
has the expression

$Lx(t) = b_0(t) x(t) + b_1(t)Dx(t) + ... + b_\{m-1\}(t) D^\{m-1\} x(t)
+ D^mx(t)$
\end{Details}
\begin{Value}
a linear differential operator object
\end{Value}
\begin{SeeAlso}\relax
\code{\LinkA{int2Lfd}{int2Lfd}}, 
\code{\LinkA{vec2Lfd}{vec2Lfd}}, 
\code{\LinkA{fdPar}{fdPar}}, 
\code{\LinkA{pda.fd}{pda.fd}}
\end{SeeAlso}
\begin{Examples}
\begin{ExampleCode}
#  Set up the harmonic acceleration operator
dayrange  <- c(0,365)
Lbasis  <- create.constant.basis(dayrange)
Lcoef   <- matrix(c(0,(2*pi/365)^2,0),1,3)
bfdobj  <- fd(Lcoef,Lbasis)
bwtlist <- fd2list(bfdobj)
harmaccelLfd <- Lfd(3, bwtlist)
\end{ExampleCode}
\end{Examples}

