\HeaderA{create.monomial.basis}{Create a Monomial Basis}{create.monomial.basis}
\keyword{smooth}{create.monomial.basis}
\begin{Description}\relax
Creates a set of basis functions consisting of powers
of the argument.
\end{Description}
\begin{Usage}
\begin{verbatim}
create.monomial.basis(rangeval=c(0, 1), nbasis=NULL,
         exponents=NULL, dropind=NULL, quadvals=NULL,
         values=NULL, basisvalues=NULL, names='monomial')
\end{verbatim}
\end{Usage}
\begin{Arguments}
\begin{ldescription}
\item[\code{rangeval}] a vector of length 2 containing the initial and final
values of the interval over which the functional
data object can be evaluated.

\item[\code{nbasis}] the number of basis functions = \code{length(exponents)}.  Default =
if(is.null(exponents)) 2 else length(exponents).

\item[\code{exponents}] the nonnegative integer powers to be used.  By default,
these are 0, 1, 2, ..., (nbasis-1).

\item[\code{dropind}] a vector of integers specifiying the basis functions to be dropped,
if any.  For example, if it is required that a function be zero at
the left boundary when rangeval[1] = 0, this is achieved by dropping
the first basis function, the only one that is nonzero at that
point.

\item[\code{quadvals}] a matrix with two columns and a number of rows equal to the number
of quadrature points for numerical evaluation of the penalty
integral.  The first column of \code{quadvals} contains the
quadrature points, and the second column the quadrature weights.  A
minimum of 5 values are required for each inter-knot interval, and
that is often enough.  For Simpson's rule, these points are equally
spaced, and the weights are proportional to 1, 4, 2, 4, ..., 2, 4,
1.

\item[\code{values}] a list of matrices with one row for each row of \code{quadvals} and
one column for each basis function.  The elements of the list
correspond to the basis functions and their derivatives evaluated at
the quadrature points contained in the first column of
\code{quadvals}.

\item[\code{basisvalues}] A list of lists, allocated by code such as vector("list",1).  This
field is designed to avoid evaluation of a basis system repeatedly
at a set of argument values.  Each list within the vector
corresponds to a specific set of argument values, and must have at
least two components, which may be tagged as you wish.  `The first
component in an element of the list vector contains the argument
values.  The second component in an element of the list vector
contains a matrix of values of the basis functions evaluated at the
arguments in the first component.  The third and subsequent
components, if present, contain matrices of values their derivatives
up to a maximum derivative order.  Whenever function getbasismatrix
is called, it checks the first list in each row to see, first, if
the number of argument values corresponds to the size of the first
dimension, and if this test succeeds, checks that all of the
argument values match.  This takes time, of course, but is much
faster than re-evaluation of the basis system.  Even this time can
be avoided by direct retrieval of the desired array.  For example,
you might set up a vector of argument values called "evalargs" along
with a matrix of basis function values for these argument values
called "basismat".  You might want too use names like "args" and
"values", respectively for these.  You would then assign them to
\code{basisvalues} with code such as the following:

basisobj\$basisvalues <- vector("list",1)

basisobj\$basisvalues[[1]] <- list(args=evalargs,
values=basismat)

\item[\code{names}] either a character vector of the same length as the number of basis
functions or a simple stem used to construct such a vector.

For \code{monomial} bases, this defaults to paste('monomial',
1:nbreaks, sep='').

\end{ldescription}
\end{Arguments}
\begin{Value}
a basis object with the type \code{monom}.
\end{Value}
\begin{SeeAlso}\relax
\code{\LinkA{basisfd}{basisfd}},
\code{link\{create.basis\}}
\code{\LinkA{create.bspline.basis}{create.bspline.basis}},
\code{\LinkA{create.constant.basis}{create.constant.basis}},
\code{\LinkA{create.fourier.basis}{create.fourier.basis}},
\code{\LinkA{create.exponential.basis}{create.exponential.basis}},
\code{\LinkA{create.polygonal.basis}{create.polygonal.basis}},
\code{\LinkA{create.polynomial.basis}{create.polynomial.basis}},
\code{\LinkA{create.power.basis}{create.power.basis}}
\end{SeeAlso}
\begin{Examples}
\begin{ExampleCode}
##
## simplest example: one constant 'basis function'
##
m0 <- create.monomial.basis(nbasis=1)
plot(m0)

##
## Create a monomial basis over the interval [-1,1]
##  consisting of the first three powers of t
##
basisobj <- create.monomial.basis(c(-1,1), 5)
#  plot the basis
plot(basisobj)
\end{ExampleCode}
\end{Examples}

