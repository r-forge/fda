\HeaderA{fda-package}{Functional Data Analysis in R}{fda.Rdash.package}
\aliasA{fda}{fda-package}{fda}
\keyword{smooth}{fda-package}
\begin{Description}\relax
Functions and data sets companion to Ramsay, J. O., and Silverman,
B. W. (2005) Functional Data Analysis, 2nd ed. and (2002) Applied
Functional Data Analysis (Springer).  This includes finite bases
approximations (such as splines and Fourier series) to functions fit
to data smoothing on the integral of the squared deviations from an
arbitrary differential operator.
\end{Description}
\begin{Details}\relax
\Tabular{ll}{
Package: & fda\\
Type: & Package\\
Version: & 2.0.5\\
Date: & 2008-05-05\\
License: & GPL-2\\
LazyLoad: & yes\\
}
\end{Details}
\begin{Author}\relax
J. O. Ramsay,

Maintainer:  J. O. Ramsay <ramsay@psych.mcgill.ca>
\end{Author}
\begin{References}\relax
Ramsay, James O., and Silverman, Bernard W. (2005), \emph{Functional
Data Analysis, 2nd ed.}, Springer, New York.

Ramsay, James O., and Silverman, Bernard W. (2002), \emph{Applied
Functional Data Analysis}, Springer, New York.
\end{References}
\begin{Examples}
\begin{ExampleCode}
##
## Simple smoothing
##
girlGrowthSm <- with(growth, smooth.basisPar(argvals=age, y=hgtf))
plot(girlGrowthSm$fd, xlab="age", ylab="height (cm)",
         main="Girls in Berkeley Growth Study" )
plot(deriv(girlGrowthSm$fd), xlab="age", ylab="growth rate (cm / year)",
         main="Girls in Berkeley Growth Study" )
plot(deriv(girlGrowthSm$fd, 2), xlab="age",
        ylab="growth acceleration (cm / year^2)",
        main="Girls in Berkeley Growth Study" )
##
## Simple basis
##
bspl1.2 <- create.bspline.basis(norder=1, breaks=c(0,.5, 1))
plot(bspl1.2)
# 2 bases, order 1 = degree 0 = step functions:
# (1) constant 1 between 0 and 0.5 and 0 otherwise
# (2) constant 1 between 0.5 and 1 and 0 otherwise.

fd1.2 <- Data2fd(0:1, basisobj=bspl1.2)
op <- par(mfrow=c(2,1))
plot(bspl1.2, main='bases')
plot(fd1.2, main='fit')
par(op)
# A step function:  0 to time=0.5, then 1 after

\end{ExampleCode}
\end{Examples}

