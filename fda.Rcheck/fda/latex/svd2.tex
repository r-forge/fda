\HeaderA{svd2}{singular value decomposition with automatic error handling}{svd2}
\keyword{array}{svd2}
\begin{Description}\relax
The 'svd' function in R 2.5.1 occasionally throws an error
with a cryptic message.  In some such cases, changing the
LINPACK argument has worked.
\end{Description}
\begin{Usage}
\begin{verbatim}
  svd2(x, nu = min(n, p), nv = min(n, p), LINPACK = FALSE)
\end{verbatim}
\end{Usage}
\begin{Arguments}
\begin{ldescription}
\item[\code{x, nu, nv, LINPACK}] as for the 'svd' function in the 'base' package.

\end{ldescription}
\end{Arguments}
\begin{Details}\relax
In R 2.5.1, the 'svd' function sometimes stops with a cryptic error
message for a matrix x for which a second call to 'svd' with !LINPACK
will produce an answer.  When such conditions occur, assign 'x' with
attributes 'nu', 'nv', and 'LINPACK' to '.svd.LINPACK.error.matrix'
in 'env = .GlobalEnv'.

Except for these rare pathologies, 'svd2' should work the same as
'svd'.
\end{Details}
\begin{Value}
a list with components d, u, and v, as described in the help file for
'svd' in the 'base' package.
\end{Value}
\begin{SeeAlso}\relax
\code{\LinkA{svd}{svd}},
\end{SeeAlso}

