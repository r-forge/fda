\HeaderA{tperm.fd}{Permutation t-test for two groups of functional data objects.}{tperm.fd}
\keyword{smooth}{tperm.fd}
\begin{Description}\relax
tperm.fd creates a null distribution for a test of no difference between two
groups of functional data objects.
\end{Description}
\begin{Usage}
\begin{verbatim}
 tperm.fd(x1fd, x2fd, nperm=200, q=0.95, argvals=NULL, plotres=TRUE)
\end{verbatim}
\end{Usage}
\begin{Arguments}
\begin{ldescription}
\item[\code{x1fd}] a functional data object giving the first group of functional observations.

\item[\code{x2fd}] a functional data object giving the second group of functional
observations.

\item[\code{nperm}] number of permutations to use in creating the null distribution.

\item[\code{argvals}] If \code{yfdPar} is a \code{fd} object, the points at which to evaluate
the point-wise F-statistic.

\item[\code{q}] Critical quantile of the null distribution to compare to the observed
F-statistic.

\item[\code{plotres}] Argument to plot a visual display of the null distribution displaying the
\code{q}th quantile and observed F-statistic.

\end{ldescription}
\end{Arguments}
\begin{Details}\relax
The usual t-statistic is calculated pointwise and the test based on the
maximal value. If \code{argvals} is not specified,
it defaults to 101 equally-spaced points on the range of \code{yfdPar}.
\end{Details}
\begin{Value}
A list with components
\begin{ldescription}
\item[\code{pval}] the observed p-value of the permutation test.
\item[\code{qval}] the \code{q}th quantile of the null distribution.
\item[\code{Tobs}] the observed maximal t-statistic.
\item[\code{Tnull}] a vector of length \code{nperm} giving the observed values of the
permutation distribution.

\item[\code{Tvals}] the pointwise values of the observed t-statistic.
\item[\code{Tnullvals}] the pointwise values of of the permutation observations.
\item[\code{pvals.pts}] pointwise p-values of the t-statistic.
\item[\code{qvals.pts}] pointwise \code{q}th quantiles of the null distribution
\item[\code{argvals}] argument values for evaluating the F-statistic if \code{yfdPar}is
a functional data object.

\end{ldescription}

normal-bracket47bracket-normal
\end{Value}
\begin{Source}\relax
Ramsay, James O., and Silverman, Bernard W. (2006), \emph{Functional
Data Analysis, 2nd ed.}, Springer, New York.
\end{Source}
\begin{SeeAlso}\relax
\code{\LinkA{fRegress}{fRegress}}
\code{\LinkA{Fstat.fd}{Fstat.fd}}
\end{SeeAlso}

