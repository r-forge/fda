\HeaderA{TaylorSpline}{Taylor representation of a B-Spline}{TaylorSpline}
\methaliasA{TaylorSpline.dierckx}{TaylorSpline}{TaylorSpline.dierckx}
\methaliasA{TaylorSpline.fd}{TaylorSpline}{TaylorSpline.fd}
\methaliasA{TaylorSpline.fdPar}{TaylorSpline}{TaylorSpline.fdPar}
\methaliasA{TaylorSpline.fdSmooth}{TaylorSpline}{TaylorSpline.fdSmooth}
\keyword{smooth}{TaylorSpline}
\keyword{manip}{TaylorSpline}
\begin{Description}\relax
Convert B-Spline coefficients into a local Taylor series
representation expanded about the midpoint between each pair of
distinct knots.
\end{Description}
\begin{Usage}
\begin{verbatim}
TaylorSpline(object, ...)
## S3 method for class 'fd':
TaylorSpline(object, ...)
## S3 method for class 'fdPar':
TaylorSpline(object, ...)
## S3 method for class 'fdSmooth':
TaylorSpline(object, ...)
## S3 method for class 'dierckx':
TaylorSpline(object, ...)
\end{verbatim}
\end{Usage}
\begin{Arguments}
\begin{ldescription}
\item[\code{ object }] a spline object, e.g., of class 'dierckx'.

\item[\code{...}] optional arguments 
\end{ldescription}
\end{Arguments}
\begin{Details}\relax
1.  Is \code{object} a spline object with a B-spline basis?  If no,
throw an error.

2.  Find \code{knots} and \code{midpoints}.

3.  Obtain coef(object).

4.  Determine the number of dimensions of coef(object) and create
empty \code{coef} and \code{deriv} arrays to match.  Then fill the
arrays.
\end{Details}
\begin{Value}
a list with the following components:

\begin{ldescription}
\item[\code{knots}] a numeric vector of knots(object, interior=FALSE)

\item[\code{midpoints}] midpoints of intervals defined by unique(knots)

\item[\code{coef}] A matrix of dim = c(nKnots-1, norder) containing the coeffients of
a polynomial in (x-midpoints[i]) for interval i, where nKnots =
length(unique(knots)).

\item[\code{deriv}] A matrix of dim = c(nKnots-1, norder) containing the derivatives
of the spline evaluated at \code{midpoints}.

\end{ldescription}

normal-bracket38bracket-normal
\end{Value}
\begin{Author}\relax
Spencer Graves
\end{Author}
\begin{SeeAlso}\relax
\code{\LinkA{fd}{fd}}
\code{\LinkA{create.bspline.basis}{create.bspline.basis}}
\end{SeeAlso}
\begin{Examples}
\begin{ExampleCode}
##
## The simplest b-spline basis:  order 1, degree 0, zero interior knots:
##       a single step function
##
library(DierckxSpline)
bspl1.1 <- create.bspline.basis(norder=1, breaks=0:1)
# ... jump to pi to check the code
fd.bspl1.1pi <- fd(pi, basisobj=bspl1.1)
bspl1.1pi <- TaylorSpline(fd.bspl1.1pi)


##
## Cubic spline:  4  basis functions
##
bspl4 <- create.bspline.basis(nbasis=4)
plot(bspl4)
parab4.5 <- fd(c(3, -1, -1, 3)/3, bspl4)
# = 4*(x-.5)
TaylorSpline(parab4.5)

##
## A more realistic example
##
data(titanium)
#  Cubic spline with 5 interior knots (6 segments)
titan10 <- with(titanium, curfit.free.knot(x, y))
(titan10T <- TaylorSpline(titan10) )

\end{ExampleCode}
\end{Examples}

