\HeaderA{create.fourier.basis}{Create a Fourier Basis}{create.fourier.basis}
\keyword{smooth}{create.fourier.basis}
\begin{Description}\relax
Create an Fourier basis object defining a set of Fourier
functions with specified period.
\end{Description}
\begin{Usage}
\begin{verbatim}
create.fourier.basis(rangeval=c(0, 1), nbasis=3,
              period=diff(rangeval), dropind=NULL, quadvals=NULL,
              values=NULL, basisvalues=NULL, names=NULL)
\end{verbatim}
\end{Usage}
\begin{Arguments}
\begin{ldescription}
\item[\code{rangeval}] a vector of length 2 containing the initial and final values of the
interval over which the functional data object can be evaluated.

\item[\code{nbasis}] positive odd integer:  If an even number is specified, it is rounded
up to the nearest odd integer to preserve the pairing of sine and
cosine functions.  An even number of basis functions  only makes
sense when there are always only an even number of observations at
equally spaced points;  that case can be accomodated using dropind =
nbasis-1 (because the bases are \code{const}, \code{sin},
\code{cos}, ...).

\item[\code{period}] the width of any interval over which the Fourier functions repeat
themselves or are periodic.

\item[\code{dropind}] an optional vector of integers specifiying basis functions to be
dropped.

\item[\code{quadvals}] an optional matrix with two columns and a number of rows equal to
the number of quadrature points for numerical evaluation of the
penalty integral.  The first column of \code{quadvals} contains the
quadrature points, and the second column the quadrature weights.  A
minimum of 5 values are required for each inter-knot interval, and
that is often enough.  For Simpson's rule, these points are equally
spaced, and the weights are proportional to 1, 4, 2, 4, ..., 2, 4,
1.

\item[\code{values}] an optional list of matrices with one row for each row of
\code{quadvals} and one column for each basis function.  The
elements of the list correspond to the basis functions and their
derivatives evaluated at the quadrature points contained in the
first column of \code{quadvals}.

\item[\code{basisvalues}] an optional list of lists, allocated by code such as
vector("list",1).  This field is designed to avoid evaluation of a
basis system repeatedly at a set of argument values.  Each sublist
corresponds to a specific set of argument values, and must have at
least two components:  a vector of argument values and a matrix of
the values the basis functions evaluated at the arguments in the
first component.  Third and subsequent components, if present,
contain matrices of values their derivatives.  Whenever function
getbasismatrix is called, it checks the first list in each row to
see, first, if the number of argument values corresponds to the size
of the first dimension, and if this test succeeds, checks that all
of the argument values match.  This takes time, of course, but is
much faster than re-evaluation of the basis system.  Even this time
can be avoided by direct retrieval of the desired array.  For
example, you might set up a vector of argument values called
"evalargs" along with a matrix of basis function values for these
argument values called "basismat".  You might want too use tags like
"args" and "values", respectively for these.  You would then assign
them to \code{basisvalues} with code such as the following:

basisobj\$basisvalues <- vector("list",1)

basisobj\$basisvalues[[1]] <- list(args=evalargs,
values=basismat)

\item[\code{names}] either a character vector of the same length as the number of basis
functions or a simple stem used to construct such a vector.

If \code{nbasis} = 3, \code{names} defaults to c('const', 'cos',
'sin').  If \code{nbasis} > 3, \code{names} defaults to c('const',
outer(c('cos', 'sin'), 1:((nbasis-1)/2), paste, sep='')).

If names = NA, no names are used.

\end{ldescription}
\end{Arguments}
\begin{Details}\relax
Functional data objects are constructed by specifying a set of basis
functions and a set of coefficients defining a linear combination of
these basis functions.  The Fourier basis is a system
that is usually used for periodic functions.  It has the advantages
of very fast computation and great flexibility.   If the data are
considered to be nonperiod, the Fourier basis is usually preferred.
The first Fourier basis function is the constant function.  The
remainder are sine and cosine pairs with integer multiples of the
base period. The number of basis functions generated is always odd.
\end{Details}
\begin{Value}
a basis object with the type \code{fourier}.
\end{Value}
\begin{SeeAlso}\relax
\code{\LinkA{basisfd}{basisfd}},
\code{\LinkA{create.bspline.basis}{create.bspline.basis}},
\code{\LinkA{create.constant.basis}{create.constant.basis}},
\code{\LinkA{create.exponential.basis}{create.exponential.basis}},
\code{\LinkA{create.monomial.basis}{create.monomial.basis}},
\code{\LinkA{create.polygonal.basis}{create.polygonal.basis}},
\code{\LinkA{create.polynomial.basis}{create.polynomial.basis}},
\code{\LinkA{create.power.basis}{create.power.basis}}
\end{SeeAlso}
\begin{Examples}
\begin{ExampleCode}
# Create a minimal Fourier basis for the monthly temperature data,
#  using 3 basis functions with period 12 months.
monthbasis3 <- create.fourier.basis(c(0,12) )
#  plot the basis
plot(monthbasis3)

# set up the Fourier basis for the monthly temperature data,
#  using 9 basis functions with period 12 months.
monthbasis <- create.fourier.basis(c(0,12), 9, 12.0)

#  plot the basis
plot(monthbasis)

# Create a false Fourier basis using 1 basis function.
falseFourierBasis <- create.fourier.basis(nbasis=1)
#  plot the basis:  constant
plot(falseFourierBasis)

\end{ExampleCode}
\end{Examples}

