\HeaderA{pda.fd}{Principal Differential Analysis}{pda.fd}
\keyword{smooth}{pda.fd}
\begin{Description}\relax
Principal differential analysis (PDA) estimates a
system of \eqn{n}{} linear differential equations that define functions
that fit the data and their derivatives.  There is an equation in the
system for each variable.  

Each equation has on its right side the highest order derivative that is used, 
and the order of this derivative, \eqn{m_j, j=1,...,n}{} can vary over equations.    

On the left side of equation equation is a linear combination of all the 
variables and all the derivatives of these variables up to order one less than 
the order \eqn{m_j}{} of the highest derivative.

In addition, the right side may contain linear combinations of forcing
functions as well, with the number of forcing functions varying over 
equations.

The linear combinations are defined by weighting functions multiplying each
variable, derivative, and forcing function in the equation.  These weighting
functions may be constant or vary over time.  They are each represented by a
functional parameter object, specifying a basis for an expansion of a
coefficient, a linear differential operator for smoothing purposes, a 
smoothing parameter value, and a logical variable indicating whether the
function is to be estimated, or kept fixed.
\end{Description}
\begin{Usage}
\begin{verbatim}
pda.fd(xfdlist, bwtlist=NULL,
       awtlist=NULL, ufdlist=NULL, nfine=501)
\end{verbatim}
\end{Usage}
\begin{Arguments}
\begin{ldescription}
\item[\code{xfdlist}] a list whose members are functional data objects representing each
variable in the system of differential equations.  Each of these objects
contain one or more curves to be represented by the corresponding differential 
equation.  The length of the list is equal to the number of 
differential equations. The number \eqn{N}{} of replications must be
the same for each member functional data object.

\item[\code{bwtlist}] this argument contains the weight coefficients that multiply, in the right
side of each equation, all the variables in the system, and all their 
derivatives, where the derivatives are used up to one less than the order
of the variable.   This argument has, in general, a three-level structure,
defined by a three-level hierarchy of list objects.  

At the top level, the argument is a single list of length equal to the number of 
variables. Each component of this list is itself a list 

At the second level, each component of the top level list is itself a list, also 
of length equal to the number of variables.  

At the third and bottom level, each component of a second level list is a list 
of length equal to the number of orders of derivatives appearing on the right       
side of the equation, including  the variable itself, a derivative of order 0. 
If m indicates the order of the equation, that is the order of the derivative
on the left side, then this list is length m.  

The components in the third level lists are functional parameter objects
defining estimates for weight functions.  For a first order equation,
for example, \eqn{m = 1}{} and the single component in each list contains a weight 
function for the variable.  Since each equation has a term involving each
variable in the system, a system of first order equations will have \eqn{n^2}{}
at the third level of this structure.

There MUST be a component for each weight function, even if the corresponding  
term does not appear in the equation.  In the case of a missing term, the 
corresponding component can be \code{NULL}, and it will be treated as a 
coefficient fixed at 0.  

However, in the case of a single differential equation, \code{bwtlist} can be
given a simpler structure, since in this case only \eqn{m}{} coefficients are
required.  Therefore, for a single equation, \code{bwtlist} can be a list
of length \eqn{m}{} with each component containing a functional parameter 
object for the corresponding derivative. 

\item[\code{awtlist}] a two-level list containing weight functions for forcing functions.

In addition to terms in each of the equations involving terms corresponding
to each derivative of each variable in the system, each equation can
also have a contribution from one or more exogenous variables, often
called \emph{forcing functions.}

This argument defines the weights multiplying these
forcing functions, and is a list of length \eqn{n}{}, the number of variables.
Each component of this is is in turn a list, each component of which contains
a functional parameter object defining a weight function for a forcing function.
If there are no forcing functions for an equation, this list can be \code{NULL}.
If none of the equations involve forcing functions, \code{awtlist} can be
\code{NULL}, which is its default value if it is not in the argument list.

\item[\code{ufdlist}] a two-level list containing forcing functions.  This list structure is 
identical to that for \code{awtlist}, the only difference being that the
components in the second level contain functional data objects for the 
forcing functions, rather than functional parameter objects.

\item[\code{nfine}] a number of values for a fine mesh.
The estimation of the differential equation involves discrete
numerical quadrature estimates of integrals, and these require
that functions be evaluated at a fine mesh of values of the
argument.  This argument defines the number to use.  The default value of 501
is reset to five times the largest number of basis functions used to represent 
any variable in the system, if this number is larger.

\end{ldescription}
\end{Arguments}
\begin{Value}
a named list of length 3 with components:

\begin{ldescription}
\item[\code{bwtlist}] a list array of the same dimensions as the
corresponding argument, containing the estimated or fixed weight
functions defining the system of linear differential equations.

\item[\code{resfdlist}] a list of length equal to the number of variables
or equations.  Each members is a functional data object giving the
residual functions or forcing functions defined as the left side
of the equation (the derivative of order m of a variable) minus
the linear fit on the right side.  The number of replicates for each
residual functional data object is the same as that for the variables.

\item[\code{awtlist}] a list of the same dimensions as the corresponding
argument.  Each member is an estimated or fixed weighting function for
a forcing function.

\end{ldescription}
\end{Value}
\begin{SeeAlso}\relax
\code{\LinkA{pca.fd}{pca.fd}}, 
\code{\LinkA{cca.fd}{cca.fd}}
\end{SeeAlso}
\begin{Examples}
\begin{ExampleCode}
#See analyses of daily weather data for examples.
##
##  set up objects for examples
##
#  constant basis for estimating weight functions
cbasis = create.constant.basis(c(0,1))
#  monomial basis: {1,t}  for estimating weight functions
mbasis = create.monomial.basis(c(0,1),2)
#  quartic spline basis with 54 basis functions for
#    defining functions to be analyzed
xbasis = create.bspline.basis(c(0,1),24,5)
#  set up functional parameter objects for weight bases
cfdPar = fdPar(cbasis)
mfdPar = fdPar(mbasis)
#  sampling points over [0,1]
tvec = seq(0,1,len=101)
##
##  Example 1:  a single first order constant coefficient unforced equation
##     Dx = -4*x  for  x(t) = exp(-4t)
beta    = 4
xvec    = exp(-beta*tvec)
xfd     = smooth.basis(tvec, xvec, xbasis)$fd
xfdlist = list(xfd)
bwtlist = list(cfdPar)
#  perform the principal differential analysis
result = pda.fd(xfdlist, bwtlist)
#  display weight coefficient for variable
bwtlistout = result$bwtlist
bwtfd      = bwtlistout[[1]]$fd
par(mfrow=c(1,1))
plot(bwtfd)
title("Weight coefficient for variable")
print(round(bwtfd$coefs,3))
#  display residual functions
reslist    = result$resfdlist
plot(reslist[[1]])
title("Residual function")
##
##  Example 2:  a single first order varying coefficient unforced equation
##     Dx(t) = -t*x(t) or x(t) = exp(-t^2/2)
bvec    = tvec
xvec    = exp(-tvec^2/2)
xfd     = smooth.basis(tvec, xvec, xbasis)$fd
xfdlist = list(xfd)
bwtlist = list(mfdPar)
#  perform the principal differential analysis
result = pda.fd(xfdlist, bwtlist)
#  display weight coefficient for variable
bwtlistout = result$bwtlist
bwtfd      = bwtlistout[[1]]$fd
par(mfrow=c(1,1))
plot(bwtfd)
title("Weight coefficient for variable")
print(round(bwtfd$coefs,3))
#  display residual function
reslist    = result$resfdlist
plot(reslist[[1]])
title("Residual function")
##
##  Example 3:  a single second order constant coefficient unforced equation
##     Dx(t) = -(2*pi)^2*x(t) or x(t) = sin(2*pi*t)
##
xvec    = sin(2*pi*tvec)
xfd     = smooth.basis(tvec, xvec, xbasis)$fd
xfdlist = list(xfd)
bwtlist = list(cfdPar,cfdPar)
#  perform the principal differential analysis
result = pda.fd(xfdlist, bwtlist)
#  display weight coefficients
bwtlistout = result$bwtlist
bwtfd1     = bwtlistout[[1]]$fd
bwtfd2     = bwtlistout[[2]]$fd
par(mfrow=c(2,1))
plot(bwtfd1)
title("Weight coefficient for variable")
plot(bwtfd2)
title("Weight coefficient for derivative of variable")
print(round(c(bwtfd1$coefs, bwtfd2$coefs),3))
print(bwtfd2$coefs)
#  display residual function
reslist    = result$resfdlist
par(mfrow=c(1,1))
plot(reslist[[1]])
title("Residual function")
##
##  Example 4:  two first order constant coefficient unforced equations
##     Dx1(t) = x2(t) and Dx2(t) = -x1(t)  
##   equivalent to  x1(t) = sin(2*pi*t)
##
xvec1     = sin(2*pi*tvec)
xvec2     = 2*pi*cos(2*pi*tvec)
xfd1      = smooth.basis(tvec, xvec1, xbasis)$fd
xfd2      = smooth.basis(tvec, xvec2, xbasis)$fd
xfdlist   = list(xfd1,xfd2)
bwtlist   = list(
                 list(
                      list(cfdPar),
                      list(cfdPar)
                     ),
                 list(
                      list(cfdPar),
                      list(cfdPar)
                     )
                )
#  perform the principal differential analysis
result = pda.fd(xfdlist, bwtlist)
#  display weight coefficients
bwtlistout = result$bwtlist
bwtfd11    = bwtlistout[[1]][[1]][[1]]$fd
bwtfd21    = bwtlistout[[2]][[1]][[1]]$fd
bwtfd12    = bwtlistout[[1]][[2]][[1]]$fd
bwtfd22    = bwtlistout[[2]][[2]][[1]]$fd
par(mfrow=c(2,2))
plot(bwtfd11)
title("Weight for variable 1 in equation 1")
plot(bwtfd21)
title("Weight for variable 2 in equation 1")
plot(bwtfd12)
title("Weight for variable 1 in equation 2")
plot(bwtfd22)
title("Weight for variable 2 in equation 2")
print(round(bwtfd11$coefs,3))
print(round(bwtfd21$coefs,3))
print(round(bwtfd12$coefs,3))
print(round(bwtfd22$coefs,3))
#  display residual functions
reslist = result$resfdlist
par(mfrow=c(2,1))
plot(reslist[[1]])
title("Residual function for variable 1")
plot(reslist[[2]])
title("Residual function for variable 2")
##
##  Example 5:  a single first order constant coefficient equation
##     Dx = -4*x  for  x(t) = exp(-4t) forced by u(t) = 2
##
beta    = 4
alpha   = 2
xvec0   = exp(-beta*tvec)
intv    = (exp(beta*tvec) - 1)/beta
xvec    = xvec0*(1 + alpha*intv)
xfd     = smooth.basis(tvec, xvec, xbasis)$fd
xfdlist = list(xfd)
bwtlist = list(cfdPar)
awtlist = list(cfdPar)
ufdlist = list(fd(1,cbasis))
#  perform the principal differential analysis
result = pda.fd(xfdlist, bwtlist, awtlist, ufdlist)
#  display weight coefficients
bwtlistout = result$bwtlist
bwtfd      = bwtlistout[[1]]$fd
awtlistout = result$awtlist
awtfd      = awtlistout[[1]]$fd
par(mfrow=c(2,1))
plot(bwtfd)
title("Weight for variable")
plot(awtfd)
title("Weight for forcing function")
#  display residual function
reslist = result$resfdlist
par(mfrow=c(1,1))
plot(reslist[[1]], ylab="residual")
title("Residual function")
##
##  Example 6:  two first order constant coefficient equations
##     Dx = -4*x    for  x(t) = exp(-4t)     forced by u(t) =  2
##     Dx = -4*t*x  for  x(t) = exp(-4t^2/2) forced by u(t) = -1
##
beta    = 4
xvec10  = exp(-beta*tvec)
alpha1  = 2
alpha2  = -1
xvec1   = xvec0 + alpha1*(1-xvec10)/beta
xvec20  = exp(-beta*tvec^2/2)
vvec    = exp(beta*tvec^2/2);
intv    = 0.01*(cumsum(vvec) - 0.5*vvec)
xvec2   = xvec20*(1 + alpha2*intv)
xfd1    = smooth.basis(tvec, xvec1, xbasis)$fd
xfd2    = smooth.basis(tvec, xvec2, xbasis)$fd
xfdlist = list(xfd1, xfd2)
bwtlist    = list(
                 list(
                      list(cfdPar),
                      list(cfdPar)
                     ),
                 list(
                      list(cfdPar),
                      list(mfdPar)
                     )
                )
awtlist = list(list(cfdPar), list(cfdPar))
ufdlist = list(list(fd(1,cbasis)), list(fd(1,cbasis)))
#  perform the principal differential analysis
result = pda.fd(xfdlist, bwtlist, awtlist, ufdlist)
# display weight functions for variables
bwtlistout = result$bwtlist
bwtfd11    = bwtlistout[[1]][[1]][[1]]$fd
bwtfd21    = bwtlistout[[2]][[1]][[1]]$fd
bwtfd12    = bwtlistout[[1]][[2]][[1]]$fd
bwtfd22    = bwtlistout[[2]][[2]][[1]]$fd
par(mfrow=c(2,2))
plot(bwtfd11)
title("weight on variable 1 in equation 1")
plot(bwtfd21)
title("weight on variable 2 in equation 1")
plot(bwtfd12)
title("weight on variable 1 in equation 2")
plot(bwtfd22)
title("weight on variable 2 in equation 2")
print(round(bwtfd11$coefs,3))
print(round(bwtfd21$coefs,3))
print(round(bwtfd12$coefs,3))
print(round(bwtfd22$coefs,3))
#  display weight functions for forcing functions
awtlistout = result$awtlist
awtfd1     = awtlistout[[1]][[1]]$fd
awtfd2     = awtlistout[[2]][[1]]$fd
par(mfrow=c(2,1))
plot(awtfd1)
title("weight on forcing function in equation 1")
plot(awtfd2)
title("weight on forcing function in equation 2")
#  display residual functions
reslist    = result$resfdlist
par(mfrow=c(2,1))
plot(reslist[[1]])
title("residual function for equation 1")
plot(reslist[[2]])
title("residual function for equation 2")
\end{ExampleCode}
\end{Examples}

