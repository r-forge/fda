\HeaderA{checkDims3}{Compare dimensions and dimnames of arrays}{checkDims3}
\aliasA{checkDim3}{checkDims3}{checkDim3}
\keyword{utilities}{checkDims3}
\begin{Description}\relax
Compare selected dimensions and dimnames of arrays, coercing objects
to 3-dimensional arrays and either give an error or force matching.
\end{Description}
\begin{Usage}
\begin{verbatim}
checkDim3(x, y=NULL, xdim=1, ydim=1, defaultNames='x',
         subset=c('xiny', 'yinx', 'neither'),
         xName=substring(deparse(substitute(x)), 1, 33),
         yName=substring(deparse(substitute(y)), 1, 33) )
checkDims3(x, y=NULL, xdim=2:3, ydim=2:3, defaultNames='x',
         subset=c('xiny', 'yinx', 'neither'),
         xName=substring(deparse(substitute(x)), 1, 33),
         yName=substring(deparse(substitute(y)), 1, 33) )
\end{verbatim}
\end{Usage}
\begin{Arguments}
\begin{ldescription}
\item[\code{x, y}] arrays to be compared.  If \code{y} is missing, \code{x} is used.

Currently, both \code{x} and \code{y} can have at most 3
dimensions.  If either has more, an error will be thrown.  If either
has fewer, it will be expanded to 3 dimensions using
\code{as.array3}. 

\item[\code{xdim, ydim}] For \code{checkDim3}, these are positive integers indicating which
dimension of \code{x} will be compared with which dimension of
\code{y}.  

For \code{checkDims3}, these are positive integer vectors of the same
length, passed one at a time to \code{checkDim3}.  The default here
is to force matching dimensions for \code{\LinkA{plotfit.fd}{plotfit.fd}}.  

\item[\code{defaultNames}] Either NULL, FALSE or a character string or vector or list.  If
NULL, no checking is done of dimnames.  If FALSE, an error is thrown
unless the corresponding dimensions of \code{x} and \code{y} match
exactly. 

If it is a character string, vector, or list, it is used as the
default names if neither \code{x} nor \code{y} have dimenames for
the compared dimensions.  If it is a character vector that is too
short, it is extended to the required length using
paste(defaultNames, 1:ni), where \code{ni} = the required length. 

If it is a list, it should have length (length(xdim)+1).  Each
component must be either a character vector or NULL.  If neither
\code{x} nor \code{y} have dimenames for the first compared
dimensions, defaultNames[[1]] will be used instead unless it is
NULL, in which case the last component of defaultNames will be
used.  If it is null, an error is thrown.  

\item[\code{subset}] If 'xiny', and any(dim(y)[ydim] < dim(x)[xdim]), an error is
thrown.  Else if any(dim(y)[ydim] > dim(x)[xdim]) the larger is
reduced to match the smaller.  If 'yinx', this procedure is
reversed.

If 'neither', any dimension mismatch generates an error.  

\item[\code{xName, yName}] names of the arguments \code{x} and \code{y}, used only to in error
messages.  

\end{ldescription}
\end{Arguments}
\begin{Details}\relax
For \code{checkDims3}, confirm that \code{xdim} and \code{ydim} have
the same length, and call \code{checkDim3} for each pair.  

For \code{checkDim3}, proceed as follows:

1.  if((xdim>3) | (ydim>3)) throw an error.

2.  ixperm <- list(1:3, c(2, 1, 3), c(3, 2, 1))[xdim];
iyperm <- list(1:3, c(2, 1, 3), c(3, 2, 1))[ydim];

3.  x3 <- aperm(as.array3(x), ixperm);
y3 <- aperm(as.array3(y), iyperm) 

4.  xNames <- dimnames(x3);  yNames <- dimnames(y3) 

5.  Check subset.  For example, for subset='xiny', use the following:
if(is.null(xNames)){
if(dim(x3)[1]>dim(y3)[1]) stop
else y. <- y3[1:dim(x3)[1],,]
dimnames(x) <- list(yNames[[1]], NULL, NULL) 
}
else {
if(is.null(xNames[[1]])){
if(dim(x3)[1]>dim(y3)[1]) stop
else y. <- y3[1:dim(x3)[1],,]
dimnames(x3)[[1]] <- yNames[[1]]
}
else {
if(any(!is.element(xNames[[1]], yNames[[1]])))stop
else y. <- y3[xNames[[1]],,]
}
}

6.  return(list(x=aperm(x3, ixperm), y=aperm(y., iyperm)))
\end{Details}
\begin{Value}
a list with components \code{x} and \code{y}.
\end{Value}
\begin{Author}\relax
Spencer Graves
\end{Author}
\begin{SeeAlso}\relax
\code{\LinkA{as.array3}{as.array3}}
\code{\LinkA{plotfit.fd}{plotfit.fd}}
\end{SeeAlso}
\begin{Examples}
\begin{ExampleCode}
# Select the first two rows of y 
stopifnot(all.equal( 
checkDim3(1:2, 3:5),
list(x=array(1:2, c(2,1,1), list(c('x1','x2'), NULL, NULL)), 
     y=array(3:4, c(2,1,1), list(c('x1','x2'), NULL, NULL)) )
)) 

# Select the first two rows of a matrix y 
stopifnot(all.equal(
checkDim3(1:2, matrix(3:8, 3)),
list(x=array(1:2,         c(2,1,1), list(c('x1','x2'), NULL, NULL)), 
     y=array(c(3:4, 6:7), c(2,2,1), list(c('x1','x2'), NULL, NULL)) )
))

# Select the first column of y
stopifnot(all.equal(
checkDim3(1:2, matrix(3:8, 3), 2, 2), 
list(x=array(1:2,         c(2,1,1), list(NULL, 'x', NULL)), 
     y=array(3:5, c(3,1,1), list(NULL, 'x', NULL)) )
))

# Select the first two rows and the first column of y
stopifnot(all.equal(
checkDims3(1:2, matrix(3:8, 3), 1:2, 1:2),
list(x=array(1:2, c(2,1,1), list(c('x1','x2'), 'x', NULL)), 
     y=array(3:4, c(2,1,1), list(c('x1','x2'), 'x', NULL)) ) 
))

# Select the first 2 rows of y 
x1 <- matrix(1:4, 2, dimnames=list(NULL, LETTERS[2:3]))
x1a <- x1. <- as.array3(x1)
dimnames(x1a)[[1]] <- c('x1', 'x2') 
y1 <- matrix(11:19, 3, dimnames=list(NULL, LETTERS[1:3]))
y1a <- y1. <- as.array3(y1) 
dimnames(y1a)[[1]] <- c('x1', 'x2', 'x3')

stopifnot(all.equal(
checkDim3(x1, y1),
list(x=x1a, y=y1a[1:2, , , drop=FALSE])
))

# Select columns 2 & 3 of y 
stopifnot(all.equal(
checkDim3(x1, y1, 2, 2),
list(x=x1., y=y1.[, 2:3, , drop=FALSE ])
))

# Select the first 2 rows and  columns 2 & 3 of y 
stopifnot(all.equal(
checkDims3(x1, y1, 1:2, 1:2),
list(x=x1a, y=y1a[1:2, 2:3, , drop=FALSE ])
)) 

# y = columns 2 and 3 of x 
x23 <- matrix(1:6, 2, dimnames=list(letters[2:3], letters[1:3]))
x23. <- as.array3(x23) 
stopifnot(all.equal(
checkDim3(x23, xdim=1, ydim=2),
list(x=x23., y=x23.[, 2:3,, drop=FALSE ])
))

# Transfer dimnames from y to x
x4a <- x4 <- matrix(1:4, 2)
y4 <- matrix(5:8, 2, dimnames=list(letters[1:2], letters[3:4]))
dimnames(x4a) <- dimnames(t(y4))
stopifnot(all.equal(
checkDims3(x4, y4, 1:2, 2:1),
list(x=as.array3(x4a), y=as.array3(y4))
))

# as used in plotfit.fd
daybasis65 <- create.fourier.basis(c(0, 365), 65)

daytempfd <- with(CanadianWeather, data2fd(
       dailyAv[,,"Temperature.C"], day.5, 
       daybasis65, argnames=list("Day", "Station", "Deg C")) )

defaultNms <- with(daytempfd, c(fdnames[2], fdnames[3], x='x'))
subset <- checkDims3(CanadianWeather$dailyAv[, , "Temperature.C"],
               daytempfd$coef, defaultNames=defaultNms)
# Problem:  dimnames(...)[[3]] = '1' 
# Fix:  
subset3 <- checkDims3(
        CanadianWeather$dailyAv[, , "Temperature.C", drop=FALSE],
               daytempfd$coef, defaultNames=defaultNms)
\end{ExampleCode}
\end{Examples}

