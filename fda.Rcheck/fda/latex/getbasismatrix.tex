\HeaderA{getbasismatrix}{Values of Basis Functions or their Derivatives}{getbasismatrix}
\keyword{smooth}{getbasismatrix}
\begin{Description}\relax
Evaluate a set of basis functions or their derivatives at
a set of argument values.
\end{Description}
\begin{Usage}
\begin{verbatim}
getbasismatrix(evalarg, basisobj, nderiv=0)
\end{verbatim}
\end{Usage}
\begin{Arguments}
\begin{ldescription}
\item[\code{evalarg}] a vector of arguments values.

\item[\code{basisobj}] a basis object.

\item[\code{nderiv}] a nonnegative integer specifying the derivative to be evaluated.

\end{ldescription}
\end{Arguments}
\begin{Value}
a matrix of basis function or derivative values.  Rows correspond
to argument values and columns to basis functions.
\end{Value}
\begin{SeeAlso}\relax
\code{\LinkA{eval.fd}{eval.fd}}
\end{SeeAlso}
\begin{Examples}
\begin{ExampleCode}
# Minimal example:  a B-spline of order 1, i.e., a step function
# with 0 interior knots:
bspl1.1 <- create.bspline.basis(norder=1, breaks=0:1)
getbasismatrix(seq(0, 1, .2), bspl1.1)

\end{ExampleCode}
\end{Examples}

