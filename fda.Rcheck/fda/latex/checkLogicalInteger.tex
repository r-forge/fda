\HeaderA{checkLogicalInteger}{Does an argument satisfy required conditions?}{checkLogicalInteger}
\aliasA{checkLogical}{checkLogicalInteger}{checkLogical}
\aliasA{checkNumeric}{checkLogicalInteger}{checkNumeric}
\keyword{attribute}{checkLogicalInteger}
\keyword{utilities}{checkLogicalInteger}
\begin{Description}\relax
Check whether an argument is a logical vector of a certain length or a
numeric vector in a certain range and issue an appropriate error or
warning if not:

\code{checkLogical} throws an error or returns FALSE with a warning
unless \code{x} is a  logical vector of exactly the required
\code{length}.

\code{checkNumeric} throws an error or returns FALSE with a warning
unless \code{x} is either NULL or a \code{numeric} vector of at most
\code{length} with \code{x} in the desired range.  

\code{checkLogicalInteger} returns a logical vector of exactly
\code{length} unless \code{x} is neither NULL nor \code{logical} of
the required \code{length} nor \code{numeric} with \code{x} in the
desired range.
\end{Description}
\begin{Usage}
\begin{verbatim}
checkLogical(x, length., warnOnly=FALSE)
checkNumeric(x, lower, upper, length., integer=TRUE, unique=TRUE,
             inclusion=c(TRUE,TRUE), warnOnly=FALSE)
checkLogicalInteger(x, length., warnOnly=FALSE)
\end{verbatim}
\end{Usage}
\begin{Arguments}
\begin{ldescription}
\item[\code{x}] an object to be checked 
\item[\code{length.}] The required length for \code{x} if \code{logical} and not NULL or
the maximum length if \code{numeric}.  

\item[\code{lower, upper}] lower and upper limits for \code{x}.  

\item[\code{integer}] logical:  If true, a \code{numeric} \code{x} must be
\code{integer}.  

\item[\code{unique}] logical:  TRUE if duplicates are NOT allowed in \code{x}.  

\item[\code{inclusion}] logical vector of length 2, similar to
\code{link[ifultools]\{checkRange\}}:  

if(inclusion[1]) (lower <= x) else (lower < x)

if(inclusion[2]) (x <= upper) else (x < upper)

\item[\code{warnOnly}] logical:  If TRUE, violations are reported as warnings, not as
errors.  

\end{ldescription}
\end{Arguments}
\begin{Details}\relax
1.  xName <- deparse(substitute(x)) to use in any required error or
warning.  

2.  if(is.null(x)) handle appropriately:  Return FALSE for
\code{checkLogical}, TRUE for \code{checkNumeric} and rep(TRUE,
length.) for \code{checkLogicalInteger}.  

3.  Check class(x).

4.  Check other conditions.
\end{Details}
\begin{Value}
\code{checkLogical} returns a logical vector of the required
\code{length.}, unless it issues an error message.

\code{checkNumeric} returns a numeric vector of at most \code{length.}
with all elements between \code{lower} and \code{upper}, and
optionally \code{unique}, unless it issues an error message.

\code{checkLogicalInteger} returns a logical vector of the required
\code{length.}, unless it issues an error message.
\end{Value}
\begin{Author}\relax
Spencer Graves
\end{Author}
\begin{SeeAlso}\relax
\code{\LinkA{checkVectorType}{checkVectorType}},
\code{\LinkA{checkRange}{checkRange}}
\code{\LinkA{checkScalarType}{checkScalarType}}
\code{\LinkA{isVectorAtomic}{isVectorAtomic}}
\end{SeeAlso}
\begin{Examples}
\begin{ExampleCode}
##
## checkLogical
##
checkLogical(NULL, length=3, warnOnly=TRUE)
checkLogical(c(FALSE, TRUE, TRUE), length=4, warnOnly=TRUE)
checkLogical(c(FALSE, TRUE, TRUE), length=3)

##
## checkNumeric
##
checkNumeric(NULL, lower=1, upper=3)
checkNumeric(1:3, 1, 3)
checkNumeric(1:3, 1, 3, inclusion=FALSE, warnOnly=TRUE)
checkNumeric(pi, 1, 4, integer=TRUE, warnOnly=TRUE)
checkNumeric(c(1, 1), 1, 4, warnOnly=TRUE)
checkNumeric(c(1, 1), 1, 4, unique=FALSE, warnOnly=TRUE)

##
## checkLogicalInteger
##
checkLogicalInteger(NULL, 3)
checkLogicalInteger(c(FALSE, TRUE), warnOnly=TRUE) 
checkLogicalInteger(1:2, 3) 
checkLogicalInteger(2, warnOnly=TRUE) 
checkLogicalInteger(c(2, 4), 3, warnOnly=TRUE)

##
## checkLogicalInteger names its calling function 
## rather than itself as the location of error detection
## if possible
##
tstFun <- function(x, length., warnOnly=FALSE){
   checkLogicalInteger(x, length., warnOnly) 
}
tstFun(NULL, 3)
tstFun(4, 3, warnOnly=TRUE)

tstFun2 <- function(x, length., warnOnly=FALSE){
   tstFun(x, length., warnOnly)
}
tstFun2(4, 3, warnOnly=TRUE)

\end{ExampleCode}
\end{Examples}

