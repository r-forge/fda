\HeaderA{pinch}{pinch force data}{pinch}
\aliasA{pinchtime}{pinch}{pinchtime}
\keyword{datasets}{pinch}
\begin{Description}\relax
151 measurements of pinch force during 20 replications, registered, with
time from start of measurement.
\end{Description}
\begin{Usage}
\begin{verbatim}
pinch
pinchtime
\end{verbatim}
\end{Usage}
\begin{Format}\relax
\describe{
\item[pinch] Matrix of dimension c(151, 20) = 20 replications of measuring
pinch force every 2 milliseconds for 300 milliseconds.  

\item[pinchtime] time in seconds from the start = seq(0, 0.3, 151) = every 2
milliseconds.

}
\end{Format}
\begin{Details}\relax
Measurements every 2 milliseconds.
\end{Details}
\begin{Source}\relax
Ramsay, James O., and Silverman, Bernard W. (2006), \emph{Functional
Data Analysis, 2nd ed.}, Springer, New York, p. 13, Figure 1.11,
pp. 22-23, Figure 2.2, and p. 144, Figure 7.13.
\end{Source}
\begin{Examples}
\begin{ExampleCode}
  plot(pinchtime, pinch[, 1], type="b")
\end{ExampleCode}
\end{Examples}

