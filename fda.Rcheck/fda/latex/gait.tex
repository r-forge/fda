\HeaderA{gait}{Hip and knee angle while walking}{gait}
\keyword{datasets}{gait}
\begin{Description}\relax
Hip and knee angle in degrees through a 20 point movement cycle for 39
boys
\end{Description}
\begin{Format}\relax
An array of dim c(20, 39, 2) giving the "Hip Angle" and "Knee Angle"
for 39 repetitions of a 20 point gait cycle.
\end{Format}
\begin{Details}\relax
The components of dimnames(gait) are as follows:

[[1]] standardized gait time = seq(from=0.025, to=0.975, by=0.05) 

[[2]] subject ID = "boy1", "boy2", ..., "boy39"  

[[3]] gait variable = "Hip Angle" or "Knee Angle"
\end{Details}
\begin{Source}\relax
Ramsay, James O., and Silverman, Bernard W. (2006), \emph{Functional
Data Analysis, 2nd ed.}, Springer, New York.

Ramsay, James O., and Silverman, Bernard W. (2002), \emph{Applied
Functional Data Analysis}, Springer, New York.
\end{Source}
\begin{Examples}
\begin{ExampleCode}
plot(gait[,1, 1], gait[, 1, 2], type="b")
\end{ExampleCode}
\end{Examples}

