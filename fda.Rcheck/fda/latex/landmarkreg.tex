\HeaderA{landmarkreg}{Landmark Registration of Functional Observations}{landmarkreg}
\keyword{smooth}{landmarkreg}
\begin{Description}\relax
It is common to see that among a set of functions certain prominent
features such peaks and valleys, called $landmarks$, do not occur
at the same times, or other
argument values.  This is called $phase variation$, and it can be
essential to align these features before proceeding with further
functional data analyses.  This function uses the timings of these
features to align or register the curves.  The registration involves
estimating a nonlinear transformation of the argument continuum for each
functional observation.  This transformation is called a warping
function. It must be strictly increasing and smooth.
\end{Description}
\begin{Usage}
\begin{verbatim}
landmarkreg(fdobj, ximarks, x0marks=xmeanmarks,
            WfdPar, monwrd=FALSE)
\end{verbatim}
\end{Usage}
\begin{Arguments}
\begin{ldescription}
\item[\code{fdobj}] a functional data object containing the curves to be registered.

\item[\code{ximarks}] a matrix containing the timings or argument values associated with
the landmarks for the observations in \code{fd} to be registered.
The number of rows N equals the number of observations, and the
number of columns NL equals the number of landmarks. These landmark
times must be in the interior of the interval over which the
functions are defined.

\item[\code{x0marks}] a vector of length NL of times of landmarks for target curve.  If
not supplied, the mean of the landmark times in \code{ximarks} is
used.

\item[\code{WfdPar}] a functional parameter object defining the warping functions that
transform time in order to register the curves.

\item[\code{monwrd}] A logical value:  if \code{TRUE}, the warping function is estimated
using a monotone smoothing methhod; otherwise, a regular smoothing
method is used, which is not guaranteed to give strictly monotonic
warping functions.

\end{ldescription}
\end{Arguments}
\begin{Details}\relax
It is essential that the location of every landmark be clearly defined
in each of the curves as well as the template function.  If this is
not the case, consider using the continuous registration function
\code{register.fd}.  Although requiring that a monotone smoother be
used to estimate the warping functions is safer, it adds considerably
to the computatation time since monotone smoothing is itself an
iterative process.  It is usually better to try an initial
registration with this feature to see if there are any failures of
monotonicity.  Moreover, monotonicity failures can usually be cured
by increasing the smoothing parameter defining \code{WfdPar}.  Not
much curvature is usually required in the warping functions, so a
rather low power basis, usually B-splines, is suitable for defining
the functional paramter argument \code{WfdPar}.  A registration with
a few prominent landmarks is often a good preliminary to using the
more sophisticated but more lengthy process in \code{register.fd}.
\end{Details}
\begin{Value}
a named list of length 2 with components:

\begin{ldescription}
\item[\code{fdreg}] a functional data object for the registered curves.

\item[\code{warpfd}] a functional data object for the warping functions.

\end{ldescription}
\end{Value}
\begin{SeeAlso}\relax
\code{\LinkA{register.fd}{register.fd}},
\code{\LinkA{smooth.morph}{smooth.morph}}
\end{SeeAlso}
\begin{Examples}
\begin{ExampleCode}
#See the analysis for the lip data in the examples.
\end{ExampleCode}
\end{Examples}

