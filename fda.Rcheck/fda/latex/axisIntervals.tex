\HeaderA{axisIntervals}{Mark Intervals on a Plot Axis}{axisIntervals}
\keyword{smooth}{axisIntervals}
\keyword{hplot}{axisIntervals}
\begin{Description}\relax
Adds an axis to the current plot, with tick marks delimiting interval
described by labels
\end{Description}
\begin{Usage}
\begin{verbatim}
axisIntervals(side, atTick1=monthBegin.5, atTick2=monthEnd.5,
      atLabels=monthMid, labels=month.abb, cex.axis=0.9, ...)
\end{verbatim}
\end{Usage}
\begin{Arguments}
\begin{ldescription}
\item[\code{side}] an integer specifying which side of the plot the axis is to
be drawn on.  The axis is placed as follows: 1=below, 2=left,
3=above and 4=right.

\item[\code{atTick1}] the points at which tick-marks marking the starting points of the
intervals are to be drawn.  This defaults to 'monthBegin.5' to mark
monthly periods for an annual cycle.  These are constructed by
calling axis(side, at=atTick1, labels=FALSE, ...).  For more detail
on this, see 'axis'.  

\item[\code{atTick2}] the points at which tick-marks marking the ends of the
intervals are to be drawn.  This defaults to 'monthBegin.5' to mark
monthly periods for an annual cycle.  These are constructed by
calling axis(side, at=atTick2, labels=FALSE, ...).  Use atTick2=NA
to rely only on atTick1.  For more detail
on this, see 'axis'.  

\item[\code{atLabels}] the points at which 'labels' should be typed.  These are constructed
by calling axis(side, at=atLabels, tick=FALSE, ...).  For more detail
on this, see 'axis'.  

\item[\code{labels}] Labels to be typed at locations 'atLabels'.  This is accomplished by
calling axis(side, at=atLabels, labels=labels, tick=FALSE, ...).
For more detail on this, see 'axis'.   

\item[\code{cex.axis}] Character expansion (magnification) used for axis annotations
('labels' in this function call) relative
to the current setting of 'cex'.  For more detail, see 'par'.    

\item[\code{... }] additional arguments passed to 
\code{axis}. 

\end{ldescription}
\end{Arguments}
\begin{Value}
The value from the third (labels) call to 'axis'.  This function is
usually invoked for its side effect, which is to add an axis to an
already existing plot.
\end{Value}
\begin{Section}{Side Effects}
An axis is added to the current plot.
\end{Section}
\begin{Author}\relax
Spencer Graves
\end{Author}
\begin{SeeAlso}\relax
\code{\LinkA{axis}{axis}}, 
\code{\LinkA{par}{par}}
\code{\LinkA{monthBegin.5}{monthBegin.5}}
\code{\LinkA{monthEnd.5}{monthEnd.5}}
\code{\LinkA{monthMid}{monthMid}}
\code{\LinkA{month.abb}{month.abb}}
\code{\LinkA{monthLetters}{monthLetters}}
\end{SeeAlso}
\begin{Examples}
\begin{ExampleCode}
daybasis65 <- create.fourier.basis(c(0, 365), 65)

daytempfd <- with(CanadianWeather, data2fd(
       dailyAv[,,"Temperature.C"], day.5,
       daybasis65, argnames=list("Day", "Station", "Deg C")) )
 
with(CanadianWeather, plotfit.fd(
      dailyAv[,,"Temperature.C"], argvals=day.5,
          daytempfd, index=1, titles=place, axes=FALSE) )
# Label the horizontal axis with the month names
axisIntervals(1) 
axis(2)
# Depending on the physical size of the plot,
# axis labels may not all print.
# In that case, there are 2 options:
# (1) reduce 'cex.lab'.
# (2) Use different labels as illustrated by adding
#     such an axis to the top of this plot 

axisIntervals(3, labels=monthLetters, cex.lab=1.2, line=-0.5) 
# 'line' argument here is passed to 'axis' via '...' 

\end{ExampleCode}
\end{Examples}

