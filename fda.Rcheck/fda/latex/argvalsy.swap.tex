\HeaderA{argvalsy.swap}{Swap argvals with y if the latter is simpler.}{argvalsy.swap}
\keyword{smooth}{argvalsy.swap}
\begin{Description}\relax
Preprocess \code{argvals}, \code{y}, and \code{basisobj}.  If only one
of \code{argvals} and \code{y} is provided, use  it as \code{y} and
take \code{argvals} as a vector spanning basisobj[['rangreval']].  If
both are provided, the simpler becomes \code{argvals}.  If both have
the same dimensions but only one lies in basisobj[['rangreval']], that
becomes \code{argvals}.
\end{Description}
\begin{Usage}
\begin{verbatim}
argvalsy.swap(argvals=NULL, y=NULL, basisobj=NULL) 
\end{verbatim}
\end{Usage}
\begin{Arguments}
\begin{ldescription}
\item[\code{argvals}] a vector or array of argument values.

\item[\code{y}] an array containing sampled values of curves.  

\item[\code{basisobj}] One of the following:

\Itemize{
\item[basisfd] a functional basis object (class \code{basisfd}. 

\item[fd] a functional data object (class \code{fd}), from which its
\code{basis} component is extracted.  

\item[fdPar] a functional parameter object (class \code{fdPar}), from which
its \code{basis} component is extracted.  

\item[integer] an integer giving the order of a B-spline basis,
create.bspline.basis(argvals, norder=basisobj) 

\item[numeric vector] specifying the knots for a B-spline basis,
create.bspline.basis(basisobj) 
         
\item[NULL] Defaults to create.bspline.basis(argvals).

}

\end{ldescription}
\end{Arguments}
\begin{Details}\relax
1.  If \code{y} is NULL, replace by \code{argvals}.

2.  If \code{argvals} is NULL, replace by
seq(basisobj[['rangeval']][1], basisobj[['rangeval']][2], dim(y)[1])
with a warning.  

3.  If the dimensions of \code{argvals} and \code{y} match and only
one is contained in basisobj[['rangeval']], use that as \code{argvals}
and the other as \code{y}.

4.  if \code{y} has fewer dimensions than \code{argvals}, swap them.
\end{Details}
\begin{Value}
a list with components \code{argvals}, \code{y}, and \code{basisobj}.
\end{Value}
\begin{SeeAlso}\relax
\code{\LinkA{Data2fd}{Data2fd}} 
\code{\LinkA{smooth.basis}{smooth.basis}}, 
\code{\LinkA{smooth.basisPar}{smooth.basisPar}}
\end{SeeAlso}
\begin{Examples}
\begin{ExampleCode}
##
## one argument:  y
##
argvalsy.swap(1:5)
# warning ... 

##
## (argvals, y), same dimensions:  retain order 
##
argy1 <- argvalsy.swap(seq(0, 1, .2), 1:6)
argy1a <- argvalsy.swap(1:6, seq(0, 1, .2))


all.equal(argy1[[1]], argy1a[[2]]) &&
all.equal(argy1[[2]], argy1a[[1]])
# TRUE;  basisobj different 


# lengths do not match 
## Not run: 
argvalsy.swap(1:4, 1:5)
## End(Not run) 

##
## two numeric arguments, different dimensions:  put simplest first 
##
argy2 <- argvalsy.swap(seq(0, 1, .2), matrix(1:12, 6))


all.equal(argy2,
argvalsy.swap(matrix(1:12, 6), seq(0, 1, .2)) )
# TRUE with a warning ... 


## Not run: 
argvalsy.swap(seq(0, 1, .2), matrix(1:12, 2))
# ERROR:  first dimension does not match 
## End(Not run)

##
## one numeric, one basisobj
##
argy3 <- argvalsy.swap(1:6, b=4)
# warning:  argvals assumed seq(0, 1, .2) 

argy3. <- argvalsy.swap(1:6, b=create.bspline.basis(breaks=0:1))
# warning:  argvals assumed seq(0, 1, .2) 

argy3.6 <- argvalsy.swap(seq(0, 1, .2), b=create.bspline.basis(breaks=1:3))
# warning:  argvals assumed seq(1, 3 length=6)

##
## two numeric, one basisobj:  first matches basisobj
##
#  OK 
argy3a <- argvalsy.swap(1:6, seq(0, 1, .2),
              create.bspline.basis(breaks=c(1, 4, 8))) 

#  Swap (argvals, y) 

all.equal(argy3a,
argvalsy.swap(seq(0, 1, .2), 1:6, 
              create.bspline.basis(breaks=c(1, 4, 8))) )
# TRUE with a warning 


## Not run: 
# neither match basisobj:  error  
argvalsy.swap(seq(0, 1, .2), 1:6, 
              create.bspline.basis(breaks=1:3) ) 
## End(Not run)

\end{ExampleCode}
\end{Examples}

