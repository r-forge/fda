\HeaderA{mean.fd}{Mean of Functional Data}{mean.fd}
\keyword{smooth}{mean.fd}
\begin{Description}\relax
Evaluate the mean of a set of functions in a functional data object.
\end{Description}
\begin{Usage}
\begin{verbatim}
mean.fd(x, ...)
\end{verbatim}
\end{Usage}
\begin{Arguments}
\begin{ldescription}
\item[\code{x}] a functional data object.

\item[\code{...}] Other arguments to match the generic function for 'mean'
\end{ldescription}
\end{Arguments}
\begin{Value}
a functional data object with a single replication
that contains the mean of the functions in the object \code{fd}.
\end{Value}
\begin{SeeAlso}\relax
\code{\LinkA{stddev.fd}{stddev.fd}}, 
\code{\LinkA{var.fd}{var.fd}}, 
\code{\LinkA{sum.fd}{sum.fd}}, 
\code{\LinkA{center.fd}{center.fd}}
\code{\LinkA{mean}{mean}}
\end{SeeAlso}
\begin{Examples}
\begin{ExampleCode}
##
## 1.  univeriate:  lip motion
##
liptime  <- seq(0,1,.02)
liprange <- c(0,1)

#  -------------  create the fd object -----------------
#       use 31 order 6 splines so we can look at acceleration

nbasis <- 51
norder <- 6
lipbasis <- create.bspline.basis(liprange, nbasis, norder)

#  ------------  apply some light smoothing to this object  -------

lipLfdobj   <- int2Lfd(4)
lipLambda   <- 1e-12
lipfdPar <- fdPar(lipbasis, lipLfdobj, lipLambda)

lipfd <- smooth.basis(liptime, lip, lipfdPar)$fd
names(lipfd$fdnames) = c("Normalized time", "Replications", "mm")

lipmeanfd <- mean.fd(lipfd)
plot(lipmeanfd)

##
## 2.  Trivariate:  CanadianWeather
##
dayrng <- c(0, 365) 

nbasis <- 51
norder <- 6 

weatherBasis <- create.fourier.basis(dayrng, nbasis)

weather.fd <- smooth.basis(day.5, CanadianWeather$dailyAv,
            weatherBasis)

str(weather.fd.mean <- mean.fd(weather.fd$fd))

\end{ExampleCode}
\end{Examples}

