\HeaderA{coef.fd}{Extract functional coefficients}{coef.fd}
\aliasA{coef.fdPar}{coef.fd}{coef.fdPar}
\aliasA{coef.fdSmooth}{coef.fd}{coef.fdSmooth}
\aliasA{coef.Taylor}{coef.fd}{coef.Taylor}
\aliasA{coefficients.fd}{coef.fd}{coefficients.fd}
\aliasA{coefficients.fdPar}{coef.fd}{coefficients.fdPar}
\aliasA{coefficients.fdSmooth}{coef.fd}{coefficients.fdSmooth}
\aliasA{coefficients.Taylor}{coef.fd}{coefficients.Taylor}
\keyword{utilities}{coef.fd}
\begin{Description}\relax
Obtain the coefficients component from a functional object (functional
data, class \code{fd}, functional parameter, class \code{fdPar}, a
functional smooth, class \code{fdSmooth}, or a Taylor spline
representation, class \code{Taylor}.
\end{Description}
\begin{Usage}
\begin{verbatim}
## S3 method for class 'fd':
coef(object, ...)
## S3 method for class 'fdPar':
coef(object, ...)
## S3 method for class 'fdSmooth':
coef(object, ...)
## S3 method for class 'Taylor':
coef(object, ...)
## S3 method for class 'fd':
coefficients(object, ...)
## S3 method for class 'fdPar':
coefficients(object, ...)
## S3 method for class 'fdSmooth':
coefficients(object, ...)
## S3 method for class 'Taylor':
coefficients(object, ...)
\end{verbatim}
\end{Usage}
\begin{Arguments}
\begin{ldescription}
\item[\code{object}] An object whose functional coefficients are desired 

\item[\code{... }] other arguments 

\end{ldescription}
\end{Arguments}
\begin{Details}\relax
Functional representations are evaluated by multiplying a basis
function matrix times a coefficient vector, matrix or 3-dimensional
array. (The basis function matrix contains the basis functions as
columns evaluated at the \code{evalarg} values as rows.)
\end{Details}
\begin{Value}
A numeric vector or array of the coefficients.
\end{Value}
\begin{SeeAlso}\relax
\code{\LinkA{coef}{coef}}
\code{\LinkA{fd}{fd}}
\code{\LinkA{fdPar}{fdPar}}
\code{\LinkA{smooth.basisPar}{smooth.basisPar}}
\code{\LinkA{smooth.basis}{smooth.basis}}
\end{SeeAlso}
\begin{Examples}
\begin{ExampleCode}
##
## coef.fd
##
bspl1.1 <- create.bspline.basis(norder=1, breaks=0:1)
fd.bspl1.1 <- fd(0, basisobj=bspl1.1)
coef(fd.bspl1.1)


##
## coef.fdPar 
##
rangeval <- c(-3,3)
#  set up some standard normal data
x <- rnorm(50)
#  make sure values within the range
x[x < -3] <- -2.99
x[x >  3] <-  2.99
#  set up basis for W(x)
basisobj <- create.bspline.basis(rangeval, 11)
#  set up initial value for Wfdobj
Wfd0 <- fd(matrix(0,11,1), basisobj)
WfdParobj <- fdPar(Wfd0)

coef(WfdParobj)


##
## coef.fdSmooth
##

girlGrowthSm <- with(growth, smooth.basisPar(argvals=age, y=hgtf))
coef(girlGrowthSm)


##
## coef.Taylor 
##
# coming soon.

\end{ExampleCode}
\end{Examples}

