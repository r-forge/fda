\HeaderA{bifd}{Create a bivariate functional data object}{bifd}
\keyword{attribute}{bifd}
\begin{Description}\relax
This function creates a bivariate functional data object, which
consists of two bases for expanding a functional data object of two 
variables, s and t, and a set of coefficients defining this expansion.
The bases are contained in "basisfd" objects.
\end{Description}
\begin{Usage}
\begin{verbatim}
bifd (coef=matrix(0,2,1), sbasisobj=create.bspline.basis(),
      tbasisobj=create.bspline.basis(), fdnames=defaultnames)
\end{verbatim}
\end{Usage}
\begin{Arguments}
\begin{ldescription}
\item[\code{coef}] a two-, three-, or four-dimensional array containing
coefficient values for the expansion of each set of bivariate
function values=terms of a set of basis function values

If 'coef' is two dimensional, this implies that there is only
one variable and only one replication.  In that case, 
the first and second dimensions correspond to
the basis functions for the first and second argument,
respectively.

If 'coef' is three dimensional, this implies that there are multiple
replicates on only one variable.  In that case, 
the first and second dimensions correspond to
the basis functions for the first and second argument,
respectively, and the third dimension corresponds to
replications.

If 'coef' has four dimensions, the fourth dimension
corresponds to variables.

\item[\code{sbasisobj}] a functional data basis object
for the first argument s of the bivariate function.  

\item[\code{tbasisobj}] a functional data basis object
for the second argument t of the bivariate function.  

\item[\code{fdnames}] A list of length 4 containing dimnames for 'coefs' if it is a
4-dimensional array.  If it is only 2- or 3-dimensional, the later
components of fdnames are not applied to 'coefs'.  In any event, the
components of fdnames describe the following:

(1) The row of 'coefs' corresponding to the bases in sbasisobj.
Defaults to sbasisobj[["names"]] if non-null and of the proper
length, or to existing dimnames(coefs)[[1]] if non-null and of the
proper length, and to 's1', 's2', ...,
otherwise.  

(2) The columns of 'coefs' corresponding to the bases in tbasisobj.
Defaults to tbasisobj[["names"]] if non-null and of the proper
length, or to existing dimnames(coefs)[[2]] if non-null and of the
proper length, and to 't1', 't2', ...,  
otherwise.  

(3) The replicates.  Defaults to dimnames(coefs)[[3]] if non-null
and of the proper length, and to 'rep1', ..., otherwise.

(4) Variable names.  Defaults to dimnames(coefs)[[4]] if non-null
and of the proper length, and to 'var1', ..., otherwise.  

\end{ldescription}
\end{Arguments}
\begin{Value}
A bivariate functional data object = a list of class 'bifd' 
with the following components:

\begin{ldescription}
\item[\code{coefs}] the input 'coefs' possible with dimnames from dfnames if provided or
from sbasisobj$names and tbasisobsj$names

\item[\code{sbasisobj}] a functional data basis object
for the first argument s of the bivariate function.  

\item[\code{tbasisobj}] a functional data basis object
for the second argument t of the bivariate function.  

\item[\code{bifdnames}] a list of length 4 giving names for the dimensions of coefs, with
one or two unused lists of names if length(dim(coefs)) is only two
or one, respectively.  

\end{ldescription}
\end{Value}
\begin{Author}\relax
Spencer Graves
\end{Author}
\begin{SeeAlso}\relax
\code{\LinkA{basisfd}{basisfd}}
\code{\LinkA{data2fd}{data2fd}} 
\code{\LinkA{objAndNames}{objAndNames}}
\end{SeeAlso}
\begin{Examples}
\begin{ExampleCode}
Bspl2 <- create.bspline.basis(nbasis=2, norder=1)
Bspl3 <- create.bspline.basis(nbasis=3, norder=2)

(bBspl2.3 <- bifd(array(1:6, dim=2:3), Bspl2, Bspl3))
str(bBspl2.3)

\end{ExampleCode}
\end{Examples}

