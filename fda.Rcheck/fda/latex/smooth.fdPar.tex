\HeaderA{smooth.fdPar}{Smooth a functional data object using a directly specified roughness
penalty}{smooth.fdPar}
\keyword{smooth}{smooth.fdPar}
\begin{Description}\relax
Smooth data already converted to a functional data object, fdobj,
using directly specified criteria.
\end{Description}
\begin{Usage}
\begin{verbatim}
smooth.fdPar(fdobj, Lfdobj=int2Lfd(0), lambda=0,
             estimate=TRUE, penmat=NULL) 
\end{verbatim}
\end{Usage}
\begin{Arguments}
\begin{ldescription}
\item[\code{fdobj}] a functional data object to be smoothed.    

\item[\code{Lfdobj}] either a nonnegative integer or a linear differential operator
object 

\item[\code{lambda}] a nonnegative real number specifying the amount of smoothing
to be applied to the estimated functional parameter.

\item[\code{estimate}] a logical value:  if \code{TRUE}, the functional parameter is
estimated, otherwise, it is held fixed.

\item[\code{penmat}] a roughness penalty matrix.  Including this can eliminate the need
to compute this matrix over and over again in some types of
calculations.

\end{ldescription}
\end{Arguments}
\begin{Details}\relax
1.  fdPar

2.  smooth.fd
\end{Details}
\begin{Value}
a functional data object.
\end{Value}
\begin{References}\relax
Ramsay, James O., and Silverman, Bernard W. (2005), \emph{Functional 
Data Analysis, 2nd ed.}, Springer, New York. 

Ramsay, James O., and Silverman, Bernard W. (2002), \emph{Applied
Functional Data Analysis}, Springer, New York.
\end{References}
\begin{SeeAlso}\relax
\code{\LinkA{smooth.fd}{smooth.fd}}, 
\code{\LinkA{fdPar}{fdPar}}, 
\code{\LinkA{smooth.basis}{smooth.basis}}, 
\code{\LinkA{smooth.pos}{smooth.pos}}, 
\code{\LinkA{smooth.morph}{smooth.morph}}
\end{SeeAlso}
\begin{Examples}
\begin{ExampleCode}
#  Shows the effects of two levels of smoothing
#  where the size of the third derivative is penalized.
#  The null space contains quadratic functions.
x <- seq(-1,1,0.02)
y <- x + 3*exp(-6*x^2) + rnorm(rep(1,101))*0.2
#  set up a saturated B-spline basis
basisobj <- create.bspline.basis(c(-1,1),81)
#  convert to a functional data object that interpolates the data.
result <- smooth.basis(x, y, basisobj)
yfd  <- result$fd
#  set up a functional parameter object with smoothing
#  parameter 1e-6 and a penalty on the 2nd derivative.
yfdPar <- fdPar(yfd, 2, 1e-6)
yfd1 <- smooth.fd(yfd, yfdPar)

yfd1. <- smooth.fdPar(yfd, 2, 1e-6)
all.equal(yfd1, yfd1.)
# TRUE

#  set up a functional parameter object with smoothing
#  parameter 1 and a penalty on the 2nd derivative.
yfd2 <- smooth.fdPar(yfd, 2, 1)

#  plot the data and smooth
plot(x,y)           # plot the data
lines(yfd1, lty=1)  #  add moderately penalized smooth
lines(yfd2, lty=3)  #  add heavily  penalized smooth
legend(-1,3,c("0.000001","1"),lty=c(1,3))
#  plot the data and smoothing using function plotfit.fd
plotfit.fd(y, x, yfd1)  # plot data and smooth

\end{ExampleCode}
\end{Examples}

