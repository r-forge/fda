\HeaderA{df2lambda}{Convert Degrees of Freedom to a Smoothing Parameter Value}{df2lambda}
\keyword{smooth}{df2lambda}
\begin{Description}\relax
The degree of roughness of an estimated function is controlled by a
smoothing parameter $lambda$ that directly multiplies the penalty.
However, it can be difficult to interpret or choose this value, and it
is often easier to determine the roughness by choosing a value that is
equivalent of the degrees of freedom used by the smoothing procedure.
This function converts a degrees of freedom value into a multipler
$lambda$.
\end{Description}
\begin{Usage}
\begin{verbatim}
df2lambda(argvals, basisobj, wtvec=rep(1, n), Lfdobj=0,
          df=nbasis)
\end{verbatim}
\end{Usage}
\begin{Arguments}
\begin{ldescription}
\item[\code{argvals}] a vector containing rgument values associated with the values to
be smoothed.

\item[\code{basisobj}] a basis function object.

\item[\code{wtvec}] a vector of weights for the data to be smoothed.

\item[\code{Lfdobj}] either a nonnegative integer or a linear differential operator object.

\item[\code{df}] the degrees of freedom to be converted.

\end{ldescription}
\end{Arguments}
\begin{Details}\relax
The conversion requires a one-dimensional optimization and may be
therefore computationally intensive.
\end{Details}
\begin{Value}
a positive smoothing parameter value $lambda$
\end{Value}
\begin{SeeAlso}\relax
\code{\LinkA{lambda2df}{lambda2df}}, 
\code{\LinkA{lambda2gcv}{lambda2gcv}}
\end{SeeAlso}
\begin{Examples}
\begin{ExampleCode}

#  Smooth growth curves using a specified value of
#  degrees of freedom.
#  Set up the ages of height measurements for Berkeley data
age <- c( seq(1, 2, 0.25), seq(3, 8, 1), seq(8.5, 18, 0.5))
#  Range of observations
rng <- c(1,18)
#  Set up a B-spline basis of order 6 with knots at ages
knots  <- age
norder <- 6
nbasis <- length(knots) + norder - 2
hgtbasis <- create.bspline.basis(rng, nbasis, norder, knots)
#  Find the smoothing parameter equivalent to 12
#  degrees of freedom
lambda <- df2lambda(age, hgtbasis, df=12)
#  Set up a functional parameter object for estimating
#  growth curves.  The 4th derivative is penalyzed to
#  ensure a smooth 2nd derivative or acceleration.
Lfdobj <- 4
growfdPar <- fdPar(hgtbasis, Lfdobj, lambda)
#  Smooth the data.  The data for the girls are in matrix
#  hgtf.
hgtffd <- smooth.basis(age, growth$hgtf, growfdPar)$fd
#  Plot the curves
plot(hgtffd)

\end{ExampleCode}
\end{Examples}

