\HeaderA{Fstat.fd}{F-statistic for functional linear regression.}{Fstat.fd}
\keyword{smooth}{Fstat.fd}
\begin{Description}\relax
Fstat.fd calculates a pointwise F-statistic for functional linear regression.
\end{Description}
\begin{Usage}
\begin{verbatim} Fstat.fd(y,yhat,argvals=NULL)\end{verbatim}
\end{Usage}
\begin{Arguments}
\begin{ldescription}
\item[\code{y}] the dependent variable object.  It may be:
\Itemize{
\item a vector if the dependent variable is scalar.
\item a functional data object if the dependent variable is functional.
}

\item[\code{yhat}] The predicted values corresponding to \code{y}. It must be of the same class.

\item[\code{argvals}] If \code{yfdPar} is a functional data object, the points at which to evaluate
the pointwise F-statistic.

\end{ldescription}
\end{Arguments}
\begin{Details}\relax
An F-statistic is calculated as the ratio of residual variance to predicted
variance.

If \code{argvals} is not specified and \code{yfdPar} is a \code{fd} object,
it defaults to 101 equally-spaced points on the range of \code{yfdPar}.
\end{Details}
\begin{Value}
A list with components
\begin{ldescription}
\item[\code{F}] the calculated pointwise F-statistics.
\item[\code{argvals}] argument values for evaluating the F-statistic if \code{yfdPar} is
a functional data object.

\end{ldescription}

normal-bracket21bracket-normal
\end{Value}
\begin{Source}\relax
Ramsay, James O., and Silverman, Bernard W. (2006), \emph{Functional
Data Analysis, 2nd ed.}, Springer, New York.
\end{Source}
\begin{SeeAlso}\relax
\code{\LinkA{fRegress}{fRegress}}
\code{\LinkA{Fstat.fd}{Fstat.fd}}
\end{SeeAlso}

