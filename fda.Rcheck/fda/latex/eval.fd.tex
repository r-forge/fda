\HeaderA{eval.fd}{Values of a Functional Data Object}{eval.fd}
\keyword{smooth}{eval.fd}
\begin{Description}\relax
Evaluate a functional data object at specified argument values, or
evaluate a derivative or the result of applying a linear differential 
operator to the functional object.
\end{Description}
\begin{Usage}
\begin{verbatim}
eval.fd(evalarg, fdobj, Lfdobj=0)
\end{verbatim}
\end{Usage}
\begin{Arguments}
\begin{ldescription}
\item[\code{evalarg}] a vector of argument values at which the functional data object is
to be evaluated.

\item[\code{fdobj}] a functional data object to be evaluated.

\item[\code{Lfdobj}] either a nonnegative integer or a linear differential operator
object.  If present, the derivative or the value of applying the
operator is evaluated rather than the functions themselves.

\end{ldescription}
\end{Arguments}
\begin{Value}
an array of 2 or 3 dimensions containing the function
values.  The first dimension corresponds to the argument values in
\code{evalarg},
the second to replications, and the third if present to functions.
\end{Value}
\begin{SeeAlso}\relax
\code{\LinkA{getbasismatrix}{getbasismatrix}}, 
\code{\LinkA{eval.bifd}{eval.bifd}}, 
\code{\LinkA{eval.penalty}{eval.penalty}}, 
\code{\LinkA{eval.monfd}{eval.monfd}}, 
\code{\LinkA{eval.posfd}{eval.posfd}}
\end{SeeAlso}
\begin{Examples}
\begin{ExampleCode}

#    set up the fourier basis
daybasis <- create.fourier.basis(c(0, 365), nbasis=65)
#  Make temperature fd object
#  Temperature data are in 12 by 365 matrix tempav
#  See analyses of weather data.
#  Set up sampling points at mid days
#  Convert the data to a functional data object
tempfd <- data2fd(CanadianWeather$dailyAv[,,"Temperature.C"],
                   day.5, daybasis)
#   set up the harmonic acceleration operator
Lbasis  <- create.constant.basis(c(0, 365))
Lcoef   <- matrix(c(0,(2*pi/365)^2,0),1,3)
bfdobj  <- fd(Lcoef,Lbasis)
bwtlist <- fd2list(bfdobj)
harmaccelLfd <- Lfd(3, bwtlist)
#   evaluate the value of the harmonic acceleration
#   operator at the sampling points
Ltempmat <- eval.fd(day.5, tempfd, harmaccelLfd)
#  Plot the values of this operator
matplot(day.5, Ltempmat, type="l")

\end{ExampleCode}
\end{Examples}

