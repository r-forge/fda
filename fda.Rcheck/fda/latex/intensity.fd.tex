\HeaderA{intensity.fd}{Intensity Function for Point Process}{intensity.fd}
\keyword{smooth}{intensity.fd}
\begin{Description}\relax
The intensity $mu$ of a series of event times that obey a
homogeneous Poisson process is the mean number of events per unit time.
When this event rate varies over time, the process is said to be
nonhomogeneous, and $mu(t)$, and is estimated by this function
\code{intensity.fd}.
\end{Description}
\begin{Usage}
\begin{verbatim}
intensity.fd(x, WfdParobj, conv=0.0001, iterlim=20,
             dbglev=1)
\end{verbatim}
\end{Usage}
\begin{Arguments}
\begin{ldescription}
\item[\code{x}] a vector containing a strictly increasing series of event times.
These event times assume that the the events begin to be observed
at time 0, and therefore are times since the beginning of
observation.

\item[\code{WfdParobj}] a functional parameter object estimating the log-intensity function
$W(t) = log[mu(t)]$ .
Because the intensity function $mu(t)$ is necessarily positive,
it is represented by \code{mu(x) = exp[W(x)]}.

\item[\code{conv}] a convergence criterion, required because the estimation
process is iterative.

\item[\code{iterlim}] maximum number of iterations that are allowed.

\item[\code{dbglev}] either 0, 1, or 2.  This controls the amount information printed out on
each iteration, with 0 implying no output, 1 intermediate output level,
and 2 full output.  If levels 1 and 2 are used, turn off the output
buffering option.

\end{ldescription}
\end{Arguments}
\begin{Details}\relax
The intensity function $I(t)$ is almost the same thing as a
probability density function $p(t)$ estimated by function
\code{densify.fd}.  The only difference is the absence of
the normalizing constant $C$ that a density function requires
in order to have a unit integral.
The goal of the function is provide a smooth intensity function
estimate that approaches some target intensity by an amount that is
controlled by the linear differential operator \code{Lfdobj} and
the penalty parameter in argument \code{WfdPar}.
For example, if the first derivative of
$W(t)$ is penalized heavily, this will force the function to
approach a constant, which in turn will force the estimated Poisson
process itself to be nearly homogeneous.
To plot the intensity function or to evaluate it,
evaluate \code{Wfdobj}, exponentiate the resulting vector.
\end{Details}
\begin{Value}
a named list of length 4 containing:

\begin{ldescription}
\item[\code{Wfdobj}] a functional data object defining function $W(x)$ that that
optimizes the fit to the data of the monotone function that it defines.

\item[\code{Flist}] a named list containing three results for the final converged solution:
(1)
\bold{f}: the optimal function value being minimized,
(2)
\bold{grad}: the gradient vector at the optimal solution,   and
(3)
\bold{norm}: the norm of the gradient vector at the optimal solution.

\item[\code{iternum}] the number of iterations.

\item[\code{iterhist}] a \code{iternum+1} by 5 matrix containing the iteration
history.

\end{ldescription}
\end{Value}
\begin{SeeAlso}\relax
\code{\LinkA{density.fd}{density.fd}}
\end{SeeAlso}
\begin{Examples}
\begin{ExampleCode}

#  Generate 101 Poisson-distributed event times with
#  intensity or rate two events per unit time
N  <- 101
mu <- 2
#  generate 101 uniform deviates
uvec <- runif(rep(0,N))
#  convert to 101 exponential waiting times
wvec <- -log(1-uvec)/mu
#  accumulate to get event times
tvec <- cumsum(wvec)
tmax <- max(tvec)
#  set up an order 4 B-spline basis over [0,tmax] with
#  21 equally spaced knots
tbasis <- create.bspline.basis(c(0,tmax), 23)
#  set up a functional parameter object for W(t),
#  the log intensity function.  The first derivative
#  is penalized in order to smooth toward a constant
lambda <- 10
WfdParobj <- fdPar(tbasis, 1, lambda)
#  estimate the intensity function
Wfdobj <- intensity.fd(tvec, WfdParobj)$Wfdobj
#  get intensity function values at 0 and event times
events <- c(0,tvec)
intenvec <- exp(eval.fd(events,Wfdobj))
#  plot intensity function
plot(events, intenvec, type="b")
lines(c(0,tmax),c(mu,mu),lty=4)

\end{ExampleCode}
\end{Examples}

