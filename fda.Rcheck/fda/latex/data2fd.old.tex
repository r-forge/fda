\HeaderA{data2fd.old}{Depricated:  use 'Data2fd'}{data2fd.old}
\aliasA{data2fd}{data2fd.old}{data2fd}
\keyword{smooth}{data2fd.old}
\begin{Description}\relax
This function converts an array \code{y} of function values
plus an array \code{argvals} of argument values into a
functional data object.  This a function that tries to
do as much for the user as possible.  A basis function
expansion is used to represent the curve, but no roughness
penalty is used.  The data are fit using the least squares
fitting criterion.  NOTE:  Interpolation with data2fd(...) can be 
shockingly bad, as illustrated in one of the examples.
\end{Description}
\begin{Usage}
\begin{verbatim}
data2fd(y, argvals=seq(0, 1, len = n), basisobj,
        fdnames=defaultnames,
        argnames=c("time", "reps", "values"))
\end{verbatim}
\end{Usage}
\begin{Arguments}
\begin{ldescription}
\item[\code{y}] an array containing sampled values of curves.

If \code{y} is a vector, only one replicate and variable are
assumed.

If \code{y} is a matrix, rows must correspond to argument values and
columns to replications or cases, and it will be assumed that there
is only one variable per observation.

If \code{y} is a three-dimensional array, the first dimension (rows)
corresponds to argument values, the second (columns) to
replications, and the third layers) to variables within
replications.  Missing values are permitted, and the number of
values may vary from one replication to another.  If this is the
case, the number of rows must equal the maximum number of argument
values, and columns of \code{y} having fewer values must be padded
out with NA's. 

\item[\code{argvals}] a set of argument values.

If this is a vector, the same set of argument values is used for all
columns of \code{y}.  If \code{argvals} is a matrix, the columns
correspond to the columns of \code{y}, and contain the argument
values for that replicate or case.    

\item[\code{basisobj}] either:  A \code{basisfd} object created by function
create.basis.fd(), 
or the value NULL, in which case a \code{basisfd} object is set up
by the function, using the values of the next three arguments.

\item[\code{fdnames}] A list of length 3, each member being a string vector containing
labels for the levels of the corresponding dimension of the discrete
data.  The first dimension is for argument values, and is given the
default name "time", the second is for replications, and is given
the default name "reps", and the third is for functions, and is
given the default name "values".  These default names are
assigned in function \{tt data2fd\}, which also assigns default
string vectors by using the dimnames attribute of the discrete data
array. 

\item[\code{argnames}] a character vector of length 3 containing:

\Itemize{
\item the name of the argument, e.g. "time" or "age"
\item a description of the cases, e.g. "weather stations"
\item the name of the observed function value, e.g. "temperature" 
}

These strings are used as names for the members of list \code{fdnames}.

\end{ldescription}
\end{Arguments}
\begin{Details}\relax
This function tends to be used in rather simple applications where
there is no need to control the roughness of the resulting curve
with any great finesse.  The roughness is essentially controlled
by how many basis functions are used.  In more sophisticated
applications, it would be better to use the function \code{\LinkA{smooth.basis}{smooth.basis}}
\end{Details}
\begin{Value}
an object of the \code{fd} class containing:

\begin{ldescription}
\item[\code{coefs}] the coefficient array

\item[\code{basis}] a basis object and

\item[\code{fdnames}] a list containing names for the arguments, function values
and variables

\end{ldescription}
\end{Value}
\begin{SeeAlso}\relax
\code{\bsl{}line\{Data2fd\}}
\code{\LinkA{smooth.basis}{smooth.basis}}, 
\code{\LinkA{smooth.basisPar}{smooth.basisPar}}, 
\code{\LinkA{project.basis}{project.basis}}, 
\code{\LinkA{smooth.fd}{smooth.fd}}, 
\code{\LinkA{smooth.monotone}{smooth.monotone}}, 
\code{\LinkA{smooth.pos}{smooth.pos}}
\code{\LinkA{day.5}{day.5}}
\end{SeeAlso}
\begin{Examples}
\begin{ExampleCode}
# Simplest possible example
b1.2 <- create.bspline.basis(norder=1, breaks=c(0, .5, 1))
# 2 bases, order 1 = degree 0 = step functions

str(fd1.2 <- data2fd(0:1, basisobj=b1.2))
plot(fd1.2)
# A step function:  0 to time=0.5, then 1 after 

b2.3 <- create.bspline.basis(norder=2, breaks=c(0, .5, 1))
# 3 bases, order 2 = degree 1 =
# continuous, bounded, locally linear

str(fd2.3 <- data2fd(0:1, basisobj=b2.3))
round(fd2.3$coefs, 4)
# 0, -.25, 1 
plot(fd2.3)
# Officially acceptable but crazy:
# Initial negative slope from (0,0) to (0.5, -0.25),
# then positive slope to (1,1).  

b3.4 <- create.bspline.basis(norder=3, breaks=c(0, .5, 1))
# 4 bases, order 3 = degree 2 =
# continuous, bounded, locally quadratic 

str(fd3.4 <- data2fd(0:1, basisobj=b3.4))
round(fd3.4$coefs, 4)
# 0, .25, -.5, 1 
plot(fd3.4)
# Officially acceptable but crazy:
# Initial positive then swings negative
# between 0.4 and ~0.75 before becoming positive again
# with a steep slope running to (1,1).  


#  Simple example 
gaitbasis3 <- create.fourier.basis(nbasis=3)
str(gaitbasis3) # note:  'names' for 3 bases
gaitfd3 <- data2fd(gait, basisobj=gaitbasis3)
str(gaitfd3)
# Note: dimanes for 'coefs' + basis[['names']]
# + 'fdnames'

#    set up the fourier basis
daybasis <- create.fourier.basis(c(0, 365), nbasis=65)
#  Make temperature fd object
#  Temperature data are in 12 by 365 matrix tempav
#    See analyses of weather data.

#  Convert the data to a functional data object
tempfd <- data2fd(CanadianWeather$dailyAv[,,"Temperature.C"],
                  day.5, daybasis)
#  plot the temperature curves
plot(tempfd)

# Terrifying interpolation
hgtbasis <- with(growth, create.bspline.basis(range(age), 
                                              breaks=age, norder=6))
girl.data2fd <- with(growth, data2fd(hgtf, age, hgtbasis))
age2 <- with(growth, sort(c(age, (age[-1]+age[-length(age)])/2)))
girlPred <- eval.fd(age2, girl.data2fd)
range(growth$hgtf)
range(growth$hgtf-girlPred[seq(1, by=2, length=31),])
# 5.5e-6 0.028 <
# The predictions are consistently too small
# but by less than 0.05 percent 

matplot(age2, girlPred, type="l")
with(growth, matpoints(age, hgtf))
# girl.data2fd fits the data fine but goes berzerk
# between points 

\end{ExampleCode}
\end{Examples}

