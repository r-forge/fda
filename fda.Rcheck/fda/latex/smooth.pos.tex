\HeaderA{smooth.pos}{Smooth Data with a Positive Function}{smooth.pos}
\keyword{smooth}{smooth.pos}
\begin{Description}\relax
A set of data is smoothed with a functional data object that only
takes positive values.  For example, this function can be used
to estimate a smooth variance function from a set of squared residuals.
A function $W(t)$ is estimated such that that the smoothing
function is $exp[W(t)]$.
\end{Description}
\begin{Usage}
\begin{verbatim}
smooth.pos(argvals, y, WfdParobj, wt=rep(1,nobs),
           conv=.0001, iterlim=20, dbglev=1)
\end{verbatim}
\end{Usage}
\begin{Arguments}
\begin{ldescription}
\item[\code{argvals}] a vector of argument values.

\item[\code{y}] a vector of data values.  This function can only smooth
one set of data at a time.

\item[\code{WfdParobj}] a functional parameter object that provides an initial
value for the coefficients defining function $W(t)$,
and a roughness penalty on this function.

\item[\code{wt}] a vector of weights to be used in the smoothing.

\item[\code{conv}] a convergence criterion.

\item[\code{iterlim}] the maximum number of iterations allowed in the minimization
of error sum of squares.

\item[\code{dbglev}] either 0, 1, or 2.  This controls the amount information printed out on
each iteration, with 0 implying no output, 1 intermediate output level,
and 2 full output.  If either level 1 or 2 is specified, it can be
helpful to turn off the output buffering feature of S-PLUS.

\end{ldescription}
\end{Arguments}
\begin{Value}
a named list of length 4 containing:

\begin{ldescription}
\item[\code{Wfdobj}] a functional data object defining function $W(x)$ that that
optimizes the fit to the data of the monotone function that it defines.

\item[\code{Flist}] a named list containing three results for the final converged solution:
(1)
\bold{f}: the optimal function value being minimized,
(2)
\bold{grad}: the gradient vector at the optimal solution,   and
(3)
\bold{norm}: the norm of the gradient vector at the optimal solution.

\item[\code{iternum}] the number of iterations.

\item[\code{iternum}] the number of iterations.

\item[\code{iterhist}] a \code{iternum+1} by 5 matrix containing the iteration
history.

\end{ldescription}
\end{Value}
\begin{SeeAlso}\relax
\code{\LinkA{smooth.monotone}{smooth.monotone}}, 
\code{\LinkA{smooth.morph}{smooth.morph}}
\end{SeeAlso}
\begin{Examples}
\begin{ExampleCode}
#See the analyses of the daily weather data for examples.
\end{ExampleCode}
\end{Examples}

