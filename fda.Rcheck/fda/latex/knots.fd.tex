\HeaderA{knots.fd}{Extract the knots from a function basis or data object}{knots.fd}
\aliasA{knots.basisfd}{knots.fd}{knots.basisfd}
\aliasA{knots.fdSmooth}{knots.fd}{knots.fdSmooth}
\keyword{smooth}{knots.fd}
\keyword{optimize}{knots.fd}
\begin{Description}\relax
Extract either all or only the interior knots from an object of class
\code{basisfd}, \code{fd}, or \code{fdSmooth}.
\end{Description}
\begin{Usage}
\begin{verbatim}
## S3 method for class 'fd':
knots(Fn, interior=TRUE, ...)
## S3 method for class 'fdSmooth':
knots(Fn, interior=TRUE, ...)
## S3 method for class 'basisfd':
knots(Fn, interior=TRUE, ...)
\end{verbatim}
\end{Usage}
\begin{Arguments}
\begin{ldescription}
\item[\code{Fn}] an object of class \code{basisfd} or containing such an object 

\item[\code{interior}] logical:

if TRUE, the first Fn[["k"]]+1 of Fn[["knots"]] are dropped, and the
next Fn[["g"]] are returned.

Otherwise, the first Fn[["n"]] of Fn[["knots"]] are returned.  

\item[\code{...}] ignored
\end{ldescription}
\end{Arguments}
\begin{Details}\relax
The interior knots of a \code{bspline} basis are stored in the
\code{params} component.  The remaining knots are in the
\code{rangeval} component, with mulltiplicity norder(Fn).
\end{Details}
\begin{Value}
Numeric vector.  If 'interior' is TRUE, this is the \code{params}
component of the \code{bspline} basis.  Otherwise, \code{params} is
bracketed by rep(rangeval, norder(basisfd).
\end{Value}
\begin{Author}\relax
Spencer Graves
\end{Author}
\begin{References}\relax
Dierckx, P. (1991) \emph{Curve and Surface Fitting with Splines},
Oxford Science Publications.
\end{References}
\begin{SeeAlso}\relax
\code{\LinkA{fd}{fd}},
\code{\LinkA{create.bspline.basis}{create.bspline.basis}},
\code{\LinkA{knots.dierckx}{knots.dierckx}}
\end{SeeAlso}
\begin{Examples}
\begin{ExampleCode}
x <- 0:24
y <- c(1.0,1.0,1.4,1.1,1.0,1.0,4.0,9.0,13.0,
       13.4,12.8,13.1,13.0,14.0,13.0,13.5,
       10.0,2.0,3.0,2.5,2.5,2.5,3.0,4.0,3.5)
if(require(DierckxSpline)){
   z1 <- curfit(x, y, method = "ss", s = 0, k = 3)
   knots1 <- knots(z1)
   knots1All <- knots(z1, interior=FALSE) # to see all knots
#
   fda1 <- dierckx2fd(z1)
   fdaKnots <- knots(fda1)
   fdaKnotsA <- knots(fda1, interior=FALSE)
   stopifnot(all.equal(knots1, fdaKnots))
   stopifnot(all.equal(knots1All, fdaKnotsA))
}

# knots.fdSmooth 
girlGrowthSm <- with(growth, smooth.basisPar(argvals=age, y=hgtf))

girlKnots.fdSm <- knots(girlGrowthSm) 
girlKnots.fdSmA <- knots(girlGrowthSm, interior=FALSE)
stopifnot(all.equal(girlKnots.fdSm, girlKnots.fdSmA[5:33]))

girlKnots.fd <- knots(girlGrowthSm$fd) 
girlKnots.fdA <- knots(girlGrowthSm$fd, interior=FALSE)

stopifnot(all.equal(girlKnots.fdSm, girlKnots.fd))
stopifnot(all.equal(girlKnots.fdSmA, girlKnots.fdA))

\end{ExampleCode}
\end{Examples}

