\HeaderA{plot.basisfd}{Plot a Basis Object}{plot.basisfd}
\keyword{smooth}{plot.basisfd}
\begin{Description}\relax
Plots all the basis functions.
\end{Description}
\begin{Usage}
\begin{verbatim}
plot.basisfd(x, knots=TRUE, ...)
\end{verbatim}
\end{Usage}
\begin{Arguments}
\begin{ldescription}
\item[\code{x}] a basis object

\item[\code{knots}] logical:  If TRUE and x[['type']] == 'bslpine', the knot locations
are plotted using vertical dotted, red lines.  Ignored otherwise.  

\item[\code{... }] additional plotting parameters passed to \code{matplot}.  

\end{ldescription}
\end{Arguments}
\begin{Value}
none
\end{Value}
\begin{Section}{Side Effects}
a plot of the basis functions
\end{Section}
\begin{SeeAlso}\relax
\code{\LinkA{plot.fd}{plot.fd}}
\end{SeeAlso}
\begin{Examples}
\begin{ExampleCode}

# set up the b-spline basis for the lip data, using 23 basis functions,
#   order 4 (cubic), and equally spaced knots.
#  There will be 23 - 4 = 19 interior knots at 0.05, ..., 0.95
lipbasis <- create.bspline.basis(c(0,1), 23)
# plot the basis functions
plot(lipbasis)

\end{ExampleCode}
\end{Examples}

