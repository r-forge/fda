\HeaderA{CSTR}{Continuously Stirred Temperature Reactor}{CSTR}
\aliasA{CSTR2}{CSTR}{CSTR2}
\aliasA{CSTR2in}{CSTR}{CSTR2in}
\aliasA{CSTRfitLS}{CSTR}{CSTRfitLS}
\aliasA{CSTRfn}{CSTR}{CSTRfn}
\aliasA{CSTRres}{CSTR}{CSTRres}
\aliasA{CSTRres0}{CSTR}{CSTRres0}
\aliasA{CSTRsse}{CSTR}{CSTRsse}
\keyword{smooth}{CSTR}
\begin{Description}\relax
Functions for solving the Continuously Stirred Temperature Reactor
(CSTR) Ordinary Differential Equations (ODEs).  A solution for
observations where metrology error is assumed to be negligible can be 
obtained via lsoda(y, Time, CSTR2, parms);  CSTR2 calls CSTR2in.  When
metrology error can not be ignored, use CSTRfn (which calls
CSTRfitLS).  To estimate parameters in the CSTR differential equation
system (kref, EoverR, a, and / or b), pass either CSTRres or CSTRres0
to nls.  If nls fails to converge, first use optim or nlminb with
CSTRsse, then pass the estimates to nls.
\end{Description}
\begin{Usage}
\begin{verbatim}
CSTR2in(Time, condition =
   c('all.cool.step', 'all.hot.step', 'all.hot.ramp', 'all.cool.ramp',
     'Tc.hot.exponential', 'Tc.cool.exponential', 'Tc.hot.ramp',
     'Tc.cool.ramp', 'Tc.hot.step', 'Tc.cool.step'),
   tau=1)
CSTR2(Time, y, parms)  

CSTRfitLS(coef, datstruct, fitstruct, lambda, gradwrd=FALSE)
CSTRfn(parvec, datstruct, fitstruct, CSTRbasis, lambda, gradwrd=TRUE)
CSTRres(kref=NULL, EoverR=NULL, a=NULL, b=NULL,
        datstruct, fitstruct, CSTRbasis, lambda, gradwrd=FALSE)
CSTRres0(kref=NULL, EoverR=NULL, a=NULL, b=NULL, gradwrd=FALSE)
CSTRsse(par, datstruct, fitstruct, CSTRbasis, lambda)

\end{verbatim}
\end{Usage}
\begin{Arguments}
\begin{ldescription}
\item[\code{Time}] The time(s) for which computation(s) are desired

\item[\code{condition}] a character string with the name of one of ten preprogrammed input
scenarios.  

\item[\code{ tau }] time for exponential decay of exp(-1) under condition =
'Tc.hot.exponential' or 'Tc.cool.exponential';  ignored for other
values of 'condition'. 

\item[\code{y}] Either a vector of length 2 or a matrix with 2 columns giving the
observation(s) on Concentration and Temperature for which
computation(s) are desired  

\item[\code{parms}] a list of CSTR model parameters passed via the lsoda 'parms'
argument.  This list consists of the following 3 components:  

\Itemize{
\item[fitstruct] a list with 12 components describing the structure for fitting.
This is the same as the 'fitstruct' argument of 'CSTRfitLS' and
'CSTRfn' without the 'fit' component;  see below.


\item[condition] a character string identifying the inputs to the simulation.
Currently, any of the following are accepted:  'all.cool.step',
'all.hot.step', 'all.hot.ramp', 'all.cool.ramp',
'Tc.hot.exponential', 'Tc.cool.exponential', 'Tc.hot.ramp',
'Tc.cool.ramp', 'Tc.hot.step', or 'Tc.cool.step'.  

\item[Tlim] end time for the computations.

}

\item[\code{coef}] a matrix with one row for each basis function in fitstruct and
columns c("Conc", "Temp") or a vector form of such a matrix.  

\item[\code{datstruct}] a list describing the structure of the data.  CSTRfitLS uses the
following components:

\Itemize{
\item[basismat, Dbasismat] basis coefficent matrices with one row for each observation and
one column for each basis vector.  These are typically produced
by code something like the following:

basismat <- eval.basis(Time, CSTRbasis)

Dbasismat <- eval.basis(Time, CSTRbasis, 1)
    
\item[Cwt, Twt] scalar variances of 'fd' functional data objects for
Concentration and Temperature used to place the two series on
comparable scales.   


\item[y] a matrix with 2 columns for the observed 'Conc' and 'Temp'.

\item[quadbasismat, Dquadbasismat] basis coefficient matrices with one row for each quadrature
point and one column for each basis vector.  These are typically
produced by code something like the following:  

quadbasismat <- eval.basis(quadpts, CSTRbasis) 

Dquadbasismat <- eval.basis(quadpts, CSTRbasis, 1)


\item[Fc, F., CA0, T0, Tc] input series for CSTRfitLS and CSTRfn as the output list
produced by CSTR2in.


\item[quadpts] Quadrature points created by 'quadset' and stored in
CSTRbasis[["quadvals"]][, "quadpts"].   

\item[quadwts] Quadrature weights created by 'quadset' and stored in
CSTRbasis[["quadvals"]][, "quadpts"].  

}

\item[\code{fitstruct}] a list with 14 components:  

\Itemize{
\item[V] volume in cubic meters

\item[Cp] concentration in cal/(g.K) for computing betaTC and betaTT;  see
details below. 

\item[rho] density in grams per cubic meter 

\item[delH] cal/kmol 

\item[Cpc] concentration in cal/(g.K) used for computing alpha;  see
details below.
    
\item[Tref] reference temperature. 

\item[kref] reference value 

\item[EoverR] E/R in units of K/1e4 

\item[a] scale factor for Fco in alpha;  see details below.

\item[b] power of Fco in alpha;  see details below.

\item[Tcin] Tc input temperature vector.

\item[fit] logical vector of length 2 indicating whether Contentration or
Temperature or both are considered to be observed and used for
parameter estimation. 

\item[coef0] data.frame(Conc = Cfdsmth[["coef"]], Temp = Tfdsmth[["coef"]]),
where Cfdsmth and Tfdsmth are the objects returned by
smooth.basis when applied to the observations on Conc and Temp,
respectively.  

\item[estimate] logical vector of length 4 indicating which of kref, EoverR, a
and b are taken from 'parvec';  all others are taken from
'fitstruct'.  

}

\item[\code{lambda}] a 2-vector of rate parameters 'lambdaC' and 'lambdaT'.  

\item[\code{gradwrd}] a logical scalar TRUE if the gradient is to be returned as well as
the residuals matrix.      

\item[\code{parvec, par}] initial values for the parameters specified by fitstruct[[
"estimate"]] to be estimated. 

\item[\code{CSTRbasis}] Quadrature basis returned by 'quadset'.  

\item[\code{kref, EoverR, a, b}] the kref, EoverR, a, and b coefficients of the CSTR model as
individual arguments of CSTRres to support using 'nls' with the CSTR
model.  Those actually provided by name will be estimated;  the
others will be taken from '.fitstruct';  see details.   

\end{ldescription}
\end{Arguments}
\begin{Details}\relax
Ramsay et al. (2007) considers the following differential equation 
system for a continuously stirred temperature reactor (CSTR):

dC/dt = (-betaCC(T, F.in)*C + F.in*C.in)

dT/dt = (-betaTT(Fcvec, F.in)*T + betaTC(T, F.in)*C +
alpha(Fcvec)*T.co)

where

betaCC(T, F.in) = kref*exp(-1e4*EoverR*(1/T - 1/Tref)) + F.in

betaTT(Fcvec, F.in) = alpha(Fcvec) + F.in

betaTC(T, F.in) = (-delH/(rho*Cp))*betaCC(T, F.in)

\deqn{
alpha(Fcvec) = (a*Fcvec^(b+1) / (K1*(Fcvec + K2*Fcvec^b)))
}{}

K1 = V*rho*Cp

K2 = 1/(2*rhoc*Cpc)  

The four functions CSTR2in, CSTR2, CSTRfitLS, and CSTRfn compute
coefficients of basis vectors for two different solutions to this set
of differential equations.  Functions CSTR2in and CSTR2 work with
'lsoda' to provide a solution to this system of equations.  Functions 
CSTSRitLS and CSTRfn are used to estimate parameters to fit this
differential equation system to noisy data.  These solutions are
conditioned on specified values for kref, EoverR, a, and b.  The other
two functions CSTRres and CSTRres0 support estimation of these
parameters using 'nls'.    

CSTR2in translates a character string 'condition' into a data.frame
containing system inputs for which the reaction of the system is
desired.  CSTR2 calls CSTR2in and then computes the corresponding
predicted first derivatives of CSTR system outputs according to the
right hand side of the system equations.  CSTR2 can be called by
'lsoda' in the 'odesolve' package to actually solve the system of
equations.  To solve the CSTR equations for another set of inputs, the
easiest modification might be to change CSTR2in to return the desired
inputs.  Another alternative would be to add an argument
'input.data.frame' that would be used in place of CSTR2in when
present.

CSTRfitLS computes standardized residuals for systems outputs Conc, 
Temp or both as specified by fitstruct[["fit"]], a logical vector of 
length 2.  The standardization is sqrt(datstruct[["Cwt"]]) and / or
sqrt(datstruct[["Twt"]]) for Conc and Temp, respectively.  CSTRfitLS
also returns standardized deviations from the predicted first
derivatives for Conc and Temp.

CSTRfn uses a Gauss-Newton optimization to estimates the coefficients
of CSTRbasis to minimize the weighted sum of squares of residuals
returned by CSTRfitLS.  

CSTRres and CSTRres0 provide alternative interfaces between 'nls' and
'CSTRfn'.  Both get the parameters to be estimated via their official
function arguments, kref, EoverR, a, and / or b.  The subset of these
paramters to estimate must be specified both directly in the function
call to 'nls' and indirectly via fitstruct[["estimate"]].  CSTRres
gets the other CSTRfn arguments (datstruct, fitstruct, CSTRbasis, and
lambda) via the 'data' argument of 'nls'.  (The version of 'nls' in
the 'stats' package in R 2.5.0 required 'data' to be a data.frame, not
merely a list.  The version of 'nls' in 'fda' does not require this.)
By contrast, CSTRres0 uses 'get' to obtain these other arguments as
.datstruct, .fitstruct, .CSTRbasis and .lambda.  Thus, before calling
nls(~CSTRres0(...), ...), the values of these arguments must be
assigned to .datstruct, .fitstruct, .CSTRbasis and .lambda.  If 'nls'
is called from the global environment, it will look for these objects
in the global environment.

CSTRres0 has one feature absent from CSTRres:  If a variable
.CSTRres0.trace is available, it is assumed to be a matrix with columns
kref, EoverR, a, b, SSE, plus all residuals.  These numbers are
rbinded as an additional row of this matrix.  This is provided to help
diagnose a problem were 'nls' was terminating with "step factor
... reduced below 'minFactor'", facilitating the comparison between R
and Matlab for the precise sets of parameter values tested by 'nls'.    

CSTRsse computes sum of squares of residuals for use with optim or
nlminb.
\end{Details}
\begin{Value}
\begin{ldescription}
\item[\code{CSTR2in}] a matrix with number of rows = length(Time) and columns for F., CA0,
T0, Tcin, and Fc.  This gives the inputs to the CSTR simulation for
the chosen 'condition'. 

\item[\code{CSTR2}] a list with one component being a matrix with number of rows =
length(tobs) and 2 columns giving the first derivatives of Conc and
Temp according to the right hand side of the differential equation.
CSTR2 calls CSTR2in to get its inputs. 

\item[\code{CSTRfitLS}] a list with one or two components as follows:  

\Itemize{
\item[res] a list with two components

Sres = a matrix giving the residuals between observed and
predicted datstruct[["y"]] divided by sqrt(datstruct[[c("Cwt",
"Twt")]]) so the result is dimensionless.  dim(Sres) =  
dim(datstruct[["y"]]).  Thus, if datstruct[["y"]] has only one
column, 'Sres' has only one column. 

Lres = a matrix with two columns giving the difference between 
left and right hand sides of the CSTR differential equation at
all the quadrature points.  dim(Lres) = c(nquad, 2). 
   
\item[Dres] If gradwrd=TRUE, a list with two components:

DSres = a matrix with one row for each element of res[["Sres"]]
and two columns for each basis function.

DLres = a matrix with two rows for each quadrature point and two
columns for each basis function.

If gradwrd=FALSE, this component is not present.  

}

\item[\code{CSTRfn}] a list with five components:

\Itemize{
\item[res] the 'res' component of the final 'CSTRfitLS' object reformatted
with its component Sres first followed by Lres, using
with(CSTRfitLS(...)[["res"]], c(Sres, Lres)). 

\item[Dres] one of two very different gradient matrices depending on the
value of 'gradwrd'.

If gradwrd = TRUE, Dres is a matrix with one row for each
observation value to match and one column for each parameter
taken from 'parvec' per fitstruct[["estimate"]].  Also, if
fitstruct[["fit"]] = c(1,1), CSTRfn tries to  match both
Concentration and Temperature, and rows corresponding to
Concentration come first following by rows corresponding to
Temperature. 

If gradwrd = FALSE, this is the 'Dres' component of the final
'CSTRfitLS' object reformatted as follows:  

Dres <- with(CSTRfitLS(...)[["Dres"]], rbind(DSres, DLres))

\item[fitstruct] a list components matching the 'fitstruct' input, with
coefficients estimated replaced by their initial values from
parvec and with coef0 replace by its final estimate. 

\item[df] estimated degrees of freedom as the trace of the appropriate
matrix. 

\item[gcv] the Generalized cross validation estimate of the mean square
error, as discussed in Ramsay and Silverman (2006, sec. 5.4).   

} 

\item[\code{CSTRres, CSTRres0}] the 'res' component of CSTRfd(...) as a column vector.  This allows
us to use 'nls' with the CSTR model.  This can be especially useful
as 'nls' has several helper functions to facilitate evaluating
goodness of fit and and uncertainty in parameter estimates.  

\item[\code{CSTRsse}] sum(res*res) from CSTRfd(...).  This allows us to use 'optim' or
'nlminb' with the CSTR model.  This can also be used to obtain
starting values for 'nls' in cases where 'nls' fails to converge
from the initiall provided starting values.  Apart from 'par', the
other arguments 'datstruct', 'fitstruct', 'CSTRbasis', and 'lambda',
must be passed via '...' in 'optim' or 'nlminb'.  

\end{ldescription}
\end{Value}
\begin{References}\relax
Ramsay, J. O., Hooker, G., Cao, J. and Campbell, D. (2007) Parameter
estimation for differential equations: A generalized smoothing
approach (with discussion). Journal of the Royal Statistical Society,
Series B. To appear.

Ramsay, J. O., and Silverman, B. W. (2006) Functional Data Analysis,
2nd ed. (Springer)

Ramsay, James O., and Silverman, Bernard W. (2002), \emph{Applied
Functional Data Analysis}, Springer, New York
\end{References}
\begin{SeeAlso}\relax
\code{\LinkA{lsoda}{lsoda}}
\code{\LinkA{nls}{nls}}
\end{SeeAlso}
\begin{Examples}
\begin{ExampleCode}
###
###
### 1.  lsoda(y, times, func=CSTR2, parms=...)
###
###
#  The system of two nonlinear equations has five forcing or  
#  input functions.
#  These equations are taken from
#  Marlin, T. E. (2000) Process Control, 2nd Edition, McGraw Hill,
#  pages 899-902.
##
##  Set up the problem 
##
fitstruct <- list(V    = 1.0,#  volume in cubic meters 
                  Cp   = 1.0,#  concentration in cal/(g.K)
                  rho  = 1.0,#  density in grams per cubic meter 
                  delH = -130.0,# cal/kmol
                  Cpc  = 1.0,#  concentration in cal/(g.K)
                  rhoc = 1.0,#  cal/kmol
                  Tref = 350)#  reference temperature
#  store true values of known parameters 
EoverRtru = 0.83301#   E/R in units K/1e4
kreftru   = 0.4610 #   reference value
atru      = 1.678#     a in units (cal/min)/K/1e6
btru      = 0.5#       dimensionless exponent

#

fitstruct[["kref"]]   = kreftru#      
fitstruct[["EoverR"]] = EoverRtru#  kref = 0.4610
fitstruct[["a"]]      = atru#       a in units (cal/min)/K/1e6
fitstruct[["b"]]      = btru#       dimensionless exponent
 
Tlim  = 64#    reaction observed over interval [0, Tlim]
delta = 1/12#  observe every five seconds
tspan = seq(0, Tlim, delta)#

coolStepInput <- CSTR2in(tspan, 'all.cool.step')

#  set constants for ODE solver

#  cool condition solution
#  initial conditions

Cinit.cool = 1.5965#  initial concentration in kmol per cubic meter
Tinit.cool = 341.3754# initial temperature in deg K
yinit = c(Conc = Cinit.cool, Temp=Tinit.cool)

#  load cool input into fitstruct

fitstruct[["Tcin"]] = coolStepInput[, "Tcin"];

#  solve  differential equation with true parameter values

if (require(odesolve)) {
coolStepSoln <- lsoda(y=yinit, times=tspan, func=CSTR2,
  parms=list(fitstruct=fitstruct, condition='all.cool.step', Tlim=Tlim) )
}
###
###
### 2.  CSTRfn 
###
###

# See the script in '~R\library\fda\scripts\CSTR\CSTR_demo.R'
#  for more examples.  

\end{ExampleCode}
\end{Examples}

