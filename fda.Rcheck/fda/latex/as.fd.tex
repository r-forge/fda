\HeaderA{as.fd}{Convert a spline object to class 'fd'}{as.fd}
\methaliasA{as.fd.dierckx}{as.fd}{as.fd.dierckx}
\methaliasA{as.fd.fdSmooth}{as.fd}{as.fd.fdSmooth}
\methaliasA{as.fd.function}{as.fd}{as.fd.function}
\aliasA{as.fd.smooth.spline}{as.fd}{as.fd.smooth.spline}
\keyword{smooth}{as.fd}
\keyword{manip}{as.fd}
\begin{Description}\relax
Translate a spline object of another class into the Functional Data
(class \code{fd}) format.
\end{Description}
\begin{Usage}
\begin{verbatim}
as.fd(x, ...)
## S3 method for class 'fdSmooth':
as.fd(x, ...)
## S3 method for class 'dierckx':
as.fd(x, ...) 
## S3 method for class 'function':
as.fd(x, ...)
## S3 method for class 'smooth.spline':
as.fd(x, ...) 
\end{verbatim}
\end{Usage}
\begin{Arguments}
\begin{ldescription}
\item[\code{x}] an object to be converted to class \code{fd}.  

\item[\code{...}] optional arguments passed to specific methods, currently unused.  

\end{ldescription}
\end{Arguments}
\begin{Details}\relax
The behavior depends on the \code{class} and nature of \code{x}.

\Itemize{
\item[as.fd.fdSmooth] extract the \code{fd} component

\item[as.fd.dierckx] The 'fda' package (as of version 2.0.0) supports B-splines with
coincident boundary knots.  For periodic phenomena, the
\code{DierckxSpline} packages uses periodic spines, while
\code{fda} recommends finite Fourier series.  Accordingly,
\code{as.fd.dierckx} if x[["periodic"]] is TRUE.  

The following describes how the components of a \code{dierckx}
object are handled by as.dierckx(as.fd(x)):   

\Itemize{
\item[x] lost.  Restored from the knots.
\item[y] lost.  Restored from spline predictions at the restored values
of 'x'.  

\item[w] lost.  Restored as rep(1, length(x)).
\item[from, to] fd[["basis"]][["rangeval"]] 
\item[k] coded indirectly as fd[["basis"]][["nbasis"]] -
length(fd[["basis"]][["params"]]) - 1.  

\item[s] lost, restored as 0.
\item[nest] lost, restored as length(x) + k + 1
\item[n] coded indirectly as 2*fd[["basis"]][["nbasis"]] -
length(fd[["basis"]][["params"]]).

\item[knots] The end knots are stored (unreplicated) in
fd[["basis"]][["rangeval"]], while the interior knots are
stored in fd[["basis"]][["params"]].

\item[fp] lost.  Restored as 0.
\item[wrk, lwrk, iwrk] lost.  Restore by refitting to the knots.  

\item[ier] lost.  Restored as 0.
\item[message] lost.  Restored as character(0).
\item[g] stored indirectly as
length(fd[["basis"]][["params"]]). 
\item[method] lost.  Restored as "ss".
\item[periodic] 'dierckx2fd' only translates 'dierckx' objects
with coincident boundary knots.  Therefore, 'periodic' is
restored as FALSE.
 
\item[routine] lost.  Restored as 'curfit.default'.
\item[xlab] fd[["fdnames"]][["args"]]
\item[ylab] fd[["fdnames"]][["funs"]]
}


\item[as.fd.function] Create an \code{fd} object from a function of the form created by
\code{splinefun}.  This will translate method = 'fmn' and
'natural' but not 'periodic':  'fmn' splines are isomorphic to
standard B-splines with coincident boundary knots, which is the
basis produced by \code{create.bspline.basis}.  'natural' splines
occupy a subspace of this space, with the restriction that the
second derivative at the end points is zero (as noted in the
Wikipedia \code{spline} article).  'periodic' splines do not use
coindicent boundary knots and are not currently supported in
\code{fda};  instead, \code{fda} uses finite Fourier bases for
periodic phenomena.  


\item[as.fd.smooth.spline] Create an \code{fd} object from a \code{smooth.spline} object.

}
\end{Details}
\begin{Value}
\code{as.fd.dierckx} converts an object of class 'dierckx' into one of
class \code{fd}.
\end{Value}
\begin{Author}\relax
Spencer Graves
\end{Author}
\begin{References}\relax
Dierckx, P. (1991) \emph{Curve and Surface Fitting with Splines},
Oxford Science Publications.

Ramsay, James O., and Silverman, Bernard W. (2005), \emph{Functional 
Data Analysis, 2nd ed.}, Springer, New York. 

Ramsay, James O., and Silverman, Bernard W. (2002), \emph{Applied
Functional Data Analysis}, Springer, New York.

\code{spline} entry in \emph{Wikipedia}
\url{http://en.wikipedia.org/wiki/Spline_(mathematics)}
\end{References}
\begin{SeeAlso}\relax
\code{\LinkA{as.dierckx}{as.dierckx}}
\code{\LinkA{curfit}{curfit}}
\code{\LinkA{fd}{fd}}
\code{\LinkA{splinefun}{splinefun}}
\end{SeeAlso}
\begin{Examples}
\begin{ExampleCode}
##
## as.fd.fdSmooth
##
girlGrowthSm <- with(growth, smooth.basisPar(argvals=age, y=hgtf))
girlGrowth.fd <- as.fd(girlGrowthSm)

##
## as.fd.dierckx
##
x <- 0:24
y <- c(1.0,1.0,1.4,1.1,1.0,1.0,4.0,9.0,13.0,
       13.4,12.8,13.1,13.0,14.0,13.0,13.5,
       10.0,2.0,3.0,2.5,2.5,2.5,3.0,4.0,3.5)
library(DierckxSpline) 
curfit.xy <- curfit(x, y, s=0)

curfit.fd <- as.fd(curfit.xy)
plot(curfit.fd) # as an 'fd' object 
points(x, y) # Curve goes through the points.  

x. <- seq(0, 24, length=241)
pred.y <- predict(curfit.xy, x.) 
lines(x., pred.y, lty="dashed", lwd=3, col="blue")
# dierckx and fd objects match.


all.equal(knots(curfit.xy, FALSE), knots(curfit.fd, FALSE))


all.equal(coef(curfit.xy), as.vector(coef(curfit.fd)))




##
## as.fd.function(splinefun(...), ...) 
## 
x2 <- 1:7
y2 <- sin((x2-0.5)*pi)
f <- splinefun(x2, y2)
fd. <- as.fd(f)
x. <- seq(1, 7, .02)
fx. <- f(x.)
fdx. <- eval.fd(x., fd.) 
plot(range(x2), range(y2, fx., fdx.), type='n')
points(x2, y2)
lines(x., sin((x.-0.5)*pi), lty='dashed') 
lines(x., f(x.), col='blue')
lines(x., eval.fd(x., fd.), col='red', lwd=3, lty='dashed')
# splinefun and as.fd(splineful(...)) are close
# but quite different from the actual function
# apart from the actual 7 points fitted,
# which are fitted exactly
# ... and there is no information in the data
# to support a better fit!

# Translate also a natural spline 
fn <- splinefun(x2, y2, method='natural')
fn. <- as.fd(fn)
lines(x., fn(x.), lty='dotted', col='blue')
lines(x., eval.fd(x., fn.), col='green', lty='dotted', lwd=3)

## Not run: 
# Will NOT translate a periodic spline
fp <- splinefun(x, y, method='periodic')
as.fd(fp)
#Error in as.fd.function(fp) : 
#  x (fp)  uses periodic B-splines, and as.fd is programmed
#   to translate only B-splines with coincident boundary knots.

## End(Not run)

##
## as.fd.smooth.spline
##
cars.spl <- with(cars, smooth.spline(speed, dist))
cars.fd <- as.fd(cars.spl)

plot(dist~speed, cars)
lines(cars.spl)
sp. <- with(cars, seq(min(speed), max(speed), len=101))
d. <- eval.fd(sp., cars.fd)
lines(sp., d., lty=2, col='red', lwd=3)
\end{ExampleCode}
\end{Examples}

