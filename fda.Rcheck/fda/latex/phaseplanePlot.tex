\HeaderA{phaseplanePlot}{Phase-plane plot}{phaseplanePlot}
\keyword{smooth}{phaseplanePlot}
\keyword{hplot}{phaseplanePlot}
\begin{Description}\relax
Plot acceleration (or Ldfobj2) vs. velocity (or Lfdobj1) of a function
data object.
\end{Description}
\begin{Usage}
\begin{verbatim}
phaseplanePlot(evalarg, fdobj, Lfdobj1=1, Lfdobj2=2,
        lty=c("longdash", "solid"),  
      labels=list(evalarg=seq(evalarg[1], max(evalarg), length=13),
             labels=monthLetters),
      abline=list(h=0, v=0, lty=2), xlab="Velocity",
      ylab="Acceleration", ...)
\end{verbatim}
\end{Usage}
\begin{Arguments}
\begin{ldescription}
\item[\code{evalarg}] a vector of argument values at which the functional data object is
to be evaluated.

Defaults to a sequence of 181 points in the range
specified by fdobj[["basis"]][["rangeval"]].      

If(length(evalarg) == 1)it is replaced by seq(evalarg[1],
evalarg[1]+1, length=181).  

If(length(evalarg) == 2)it is replaced by seq(evalarg[1],
evalarg[2], length=181).      

\item[\code{fdobj}] a functional data object to be evaluated.

\item[\code{Lfdobj1}] either a nonnegative integer or a linear differential operator
object.  The points plotted on the horizontal axis are
eval.fd(evalarg, fdobj, Lfdobj1).  By default, this is the
velocity.  

\item[\code{Lfdobj2}] either a nonnegative integer or a linear differential operator
object.  The points plotted on the vertical axis are
eval.fd(evalarg, fdobj, Lfdobj2).  By default, this is the
acceleration.  

\item[\code{lty}] line types for the first and second halves of the plot.  

\item[\code{labels}] a list of length two:

evalarg = a numeric vector of 'evalarg' values to be labeled.

labels = a character vector of labels, replicated to the same length
as labels[["evalarg"]] in case it's not of the same length.  

\item[\code{abline}] arguments to a call to abline.  

\item[\code{xlab}] x axis label 

\item[\code{ylab}] y axis label 

\item[\code{...}] optional arguments passed to plot.  

\end{ldescription}
\end{Arguments}
\begin{Value}
Invisibly returns a matrix with two columns containg the points
plotted.
\end{Value}
\begin{SeeAlso}\relax
\code{\LinkA{plot}{plot}}, 
\code{\LinkA{eval.fd}{eval.fd}}
\code{\LinkA{plot.fd}{plot.fd}}
\end{SeeAlso}
\begin{Examples}
\begin{ExampleCode}
goodsbasis <- create.bspline.basis(rangeval=c(1919,2000),
                                   nbasis=979, norder=8)
LfdobjNonDur <- int2Lfd(4) 

library(zoo)
logNondurSm <- smooth.basisPar(argvals=index(nondurables),
                y=log10(coredata(nondurables)), fdobj=goodsbasis,
                Lfdobj=LfdobjNonDur, lambda=1e-11)
phaseplanePlot(1964, logNondurSm$fd)

\end{ExampleCode}
\end{Examples}

