\HeaderA{expect.phi}{Expectatation of basis functions}{expect.phi}
\aliasA{expectden.phi}{expect.phi}{expectden.phi}
\aliasA{expectden.phiphit}{expect.phi}{expectden.phiphit}
\aliasA{normden.phi}{expect.phi}{normden.phi}
\aliasA{normint.phi}{expect.phi}{normint.phi}
\keyword{smooth}{expect.phi}
\keyword{smooth}{expect.phi}
\begin{Description}\relax
Computes expectations of basis functions with respect to a density
by numerical integration using Romberg integration
\end{Description}
\begin{Usage}
\begin{verbatim}
normint.phi(basisobj, cvec, JMAX=15, EPS=1e-7) 
normden.phi(basisobj, cvec, JMAX=15, EPS=1e-7) 
expect.phi(basisobj, cvec, nderiv=0, rng=rangeval,
                     JMAX=15, EPS=1e-7) 
expectden.phi(basisobj, cvec, Cval=1, nderiv=0, rng=rangeval,
                     JMAX=15, EPS=1e-7)
expectden.phiphit(basisobj, cvec, Cval=1, nderiv1=0,
                 nderiv2=0, rng=rangeval, JMAX=15, EPS=1e-7) 
\end{verbatim}
\end{Usage}
\begin{Arguments}
\begin{ldescription}
\item[\code{basisobj}] a basis function object 

\item[\code{cvec}] coefficient vector defining density, of length NBASIS 

\item[\code{Cval}] normalizing constant defining density 

\item[\code{nderiv, nderiv1, nderiv2}] order of derivative required for basis function expectation


\item[\code{rng}] a vector of length 2 giving the interval over which the integration is
to take place

\item[\code{JMAX}] maximum number of allowable iterations 

\item[\code{EPS}] convergence criterion for relative stop 

\end{ldescription}
\end{Arguments}
\begin{Details}\relax
normint.phi computes integrals of  
p(x) = exp phi'(x) 

normdel.phi computes integrals of
p(x) = exp phi"(x) 

expect.phi computes expectations of basis functions with respect to
intensity
p(x) <- exp t(c)*phi(x)


expectden.phi computes expectations of basis functions with respect
to density

p(x) <- exp(t(c)*phi(x))/Cval

expectden.phiphit computes expectations of cross product of basis
functions with respect to density

p(x) <- exp(t(c)*phi(x))/Cval
\end{Details}
\begin{Value}
A vector SS of length NBASIS of integrals of functions.
\end{Value}
\begin{SeeAlso}\relax
\code{\LinkA{plot.basisfd}{plot.basisfd}},
\end{SeeAlso}

