\HeaderA{dateAccessories}{Numeric and character vectors to facilitate working with dates}{dateAccessories}
\aliasA{day.5}{dateAccessories}{day.5}
\aliasA{dayOfYear}{dateAccessories}{dayOfYear}
\aliasA{daysPerMonth}{dateAccessories}{daysPerMonth}
\aliasA{monthAccessories}{dateAccessories}{monthAccessories}
\aliasA{monthBegin.5}{dateAccessories}{monthBegin.5}
\aliasA{monthEnd}{dateAccessories}{monthEnd}
\methaliasA{monthEnd.5}{dateAccessories}{monthEnd.5}
\aliasA{monthLetters}{dateAccessories}{monthLetters}
\aliasA{monthMid}{dateAccessories}{monthMid}
\aliasA{weeks}{dateAccessories}{weeks}
\keyword{datasets}{dateAccessories}
\begin{Description}\relax
Numeric and character vectors to simplify functional data computations
and plotting involving dates.
\end{Description}
\begin{Format}\relax
\item[dayOfYear] a numeric vector = 1:365 

\item[day.5 ] a numeric vector = dayOfYear-0.5 = 0.5, 1.5, ..., 364.5 

\item[daysPerMonth] a numeric vector of the days in each month (ignoring leap years)
with names = month.abb

\item[monthEnd] a numeric vector of cumsum(daysPerMonth) with names = month.abb

\item[monthEnd.5] a numeric vector of the middle of the last day of each month with
names = month.abb = c(Jan=30.5, Feb=58.5, ..., Dec=364.5)

\item[monthBegin.5] a numeric vector of the middle of the first day of each month with
names - month.abb = c(Jan=0.5, Feb=31.5, ..., Dec=334.5) 

\item[monthMid] a numeric vector of the middle of the month = (monthBegin.5 +
monthEnd.5)/2  

\item[monthLetters] A character vector of c("j", "F", "m", "A", "M", "J", "J", "A",
"S", "O", "N", "D"), with 'month.abb' as the names.  

\item[weeks] a numeric vector of length 53 marking 52 periods of approximately 7
days each throughout the year = c(0, 365/52, ..., 365)
\end{Format}
\begin{Details}\relax
Miscellaneous vectors often used in 'fda' scripts.
\end{Details}
\begin{Source}\relax
Ramsay, James O., and Silverman, Bernard W. (2006), \emph{Functional
Data Analysis, 2nd ed.}, Springer, New York, pp. 5, 47-53.

Ramsay, James O., and Silverman, Bernard W. (2002), \emph{Applied
Functional Data Analysis}, Springer, New York
\end{Source}
\begin{SeeAlso}\relax
\code{\LinkA{axisIntervals}{axisIntervals}} 
\code{\LinkA{month.abb}{month.abb}}
\end{SeeAlso}
\begin{Examples}
\begin{ExampleCode}
daybasis65 <- create.fourier.basis(c(0, 365), 65)
daytempfd <- with(CanadianWeather, smooth.basisPar(day.5, 
    dailyAv[,,"Temperature.C"]) )
plot(daytempfd, axes=FALSE)
axisIntervals(1) 
# axisIntervals by default uses
# monthBegin.5, monthEnd.5, monthMid, and month.abb
axis(2)  
\end{ExampleCode}
\end{Examples}

