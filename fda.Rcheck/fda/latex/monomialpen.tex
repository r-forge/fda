\HeaderA{monomialpen}{Evaluate Monomial Roughness Penalty Matrix}{monomialpen}
\keyword{smooth}{monomialpen}
\begin{Description}\relax
The roughness penalty matrix is the set of
inner products of all pairs of a derivative of integer powers of the
argument.
\end{Description}
\begin{Usage}
\begin{verbatim}
monomialpen(basisobj, Lfdobj=int2Lfd(2),
            rng=basisobj$rangeval)
\end{verbatim}
\end{Usage}
\begin{Arguments}
\begin{ldescription}
\item[\code{basisobj}] a monomial basis object.

\item[\code{Lfdobj}] either a nonnegative integer specifying an order of derivative
or a linear differential operator object.

\item[\code{rng}] the inner product may be computed over a range that is contained
within the range defined in the basis object.  This is a vector
or length two defining the range.

\end{ldescription}
\end{Arguments}
\begin{Value}
a symmetric matrix of order equal to the number of
monomial basis functions.
\end{Value}
\begin{SeeAlso}\relax
\code{\LinkA{polynompen}{polynompen}}, 
\code{\LinkA{exponpen}{exponpen}}, 
\code{\LinkA{fourierpen}{fourierpen}}, 
\code{\LinkA{bsplinepen}{bsplinepen}}, 
\code{\LinkA{polygpen}{polygpen}}
\end{SeeAlso}
\begin{Examples}
\begin{ExampleCode}

# set up a monomial basis for the first five powers
nbasis   <- 5
basisobj <- create.monomial.basis(c(-1,1),nbasis)
#  evaluate the rougness penalty matrix for the
#  second derivative.
penmat <- monomialpen(basisobj, 2)

\end{ExampleCode}
\end{Examples}

