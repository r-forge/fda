\HeaderA{smooth.monotone}{Monotone Smoothing of Data}{smooth.monotone}
\keyword{smooth}{smooth.monotone}
\begin{Description}\relax
When the discrete data that are observed reflect a smooth strictly
increasing or strictly decreasing function, it is often desirable to
smooth the data with a strictly monotone function, even though the
data themselves may not be monotone due to observational error.  An
example is when data are collected on the size of a growing organism
over time.  This function computes such a smoothing function, but,
unlike other smoothing functions, for only for one curve at a time.
The smoothing function minimizes a weighted error sum of squares
criterion.  This minimization requires iteration, and therefore is
more computationally intensive than normal smoothing.

The monotone smooth is beta[1]+beta[2]*integral(exp(Wfdobj)), where
Wfdobj is a functional data object.  Since exp(Wfdobj)>0, its integral
is monotonically increasing.
\end{Description}
\begin{Usage}
\begin{verbatim}
smooth.monotone(x, y, WfdParobj, wt=rep(1,nobs),
                zmat=matrix(1,nobs,1), conv=.0001, iterlim=20,
                active=c(FALSE,rep(TRUE,ncvec-1)), dbglev=1)
\end{verbatim}
\end{Usage}
\begin{Arguments}
\begin{ldescription}
\item[\code{x}] a vector of argument values.

\item[\code{y}] a vector of data values.  This function can only smooth
one set of data at a time.

\item[\code{WfdParobj}] a functional parameter object that provides an initial
value for the coefficients defining function $W(t)$,
and a roughness penalty on this function.

\item[\code{wt}] a vector of weights to be used in the smoothing.

\item[\code{zmat}] a design matrix or a matrix of covariate values that also
define the smooth of the data.

\item[\code{conv}] a convergence criterion.

\item[\code{iterlim}] the maximum number of iterations allowed in the minimization
of error sum of squares.

\item[\code{active}] a logical vector specifying which coefficients defining
$W(t)$ are estimated.  Normally, the first coefficient
is fixed.

\item[\code{dbglev}] either 0, 1, or 2.  This controls the amount information printed out on
each iteration, with 0 implying no output, 1 intermediate output level,
and 2 full output.  If either level 1 or 2 is specified, it can be
helpful to turn off the output buffering feature of S-PLUS.

\end{ldescription}
\end{Arguments}
\begin{Details}\relax
The smoothing function  $f(x)$ is determined by
three objects that need to be estimated from the data:

\Itemize{
\item $W(x)$, a functional data object that is first
exponentiated and then the result integrated.  This is the heart
of the monotone smooth.  The closer $W(x)$ is to zero, the
closer the monotone smooth becomes a straight line.  The closer
$W(x)$ becomes a constant, the more the monotone smoother
becomes an exponential function.  It is assumed that $W(0) = 0.$

\item $b0$, an intercept term that determines the value of the
smoothing function at $x = 0$. 

\item $b1$, a regression coefficient that determines the slope
of the smoothing function at $x = 0$. 
}
In addition, it is possible to have the intercept $b0$
depend in turn on the values of one or more covariates through the
design matrix \code{Zmat} as follows:
$b0 = Z c$. In this case, the single
intercept coefficient is replaced by the regression coefficients
in vector $c$ multipying the design matrix.
\end{Details}
\begin{Value}
a named list of length 5 containing:

\begin{ldescription}
\item[\code{Wfdobj}] a functional data object defining function $W(x)$ that optimizes the fit
to the data of the monotone function that it defines. 

\item[\code{beta}] the optimal regression coefficient values.

\item[\code{Flist}] a named list containing three results for the final converged solution:
(1)
\bold{f}: the optimal function value being minimized,
(2)
\bold{grad}: the gradient vector at the optimal solution,   and
(3)
\bold{norm}: the norm of the gradient vector at the optimal solution.

\item[\code{iternum}] the number of iterations.

\item[\code{iterhist}] \code{iternum+1} by 5 matrix containing the iteration
history.

\end{ldescription}
\end{Value}
\begin{References}\relax
Ramsay, James O., and Silverman, Bernard W. (2005), \emph{Functional 
Data Analysis, 2nd ed.}, Springer, New York. 

Ramsay, James O., and Silverman, Bernard W. (2002), \emph{Applied
Functional Data Analysis}, Springer, New York.
\end{References}
\begin{SeeAlso}\relax
\code{\LinkA{smooth.basis}{smooth.basis}}, 
\code{\LinkA{smooth.pos}{smooth.pos}}, 
\code{\LinkA{smooth.morph}{smooth.morph}}
\end{SeeAlso}
\begin{Examples}
\begin{ExampleCode}

#  Estimate the acceleration functions for growth curves
#  See the analyses of the growth data.
#  Set up the ages of height measurements for Berkeley data
age <- c( seq(1, 2, 0.25), seq(3, 8, 1), seq(8.5, 18, 0.5))
#  Range of observations
rng <- c(1,18)
#  First set up a basis for monotone smooth
#  We use b-spline basis functions of order 6
#  Knots are positioned at the ages of observation.
norder <- 6
nage   <- 31
nbasis <- nage + norder - 2
wbasis <- create.bspline.basis(rng, nbasis, norder, age)
#  starting values for coefficient
cvec0 <- matrix(0,nbasis,1)
Wfd0  <- fd(cvec0, wbasis)
#  set up functional parameter object
Lfdobj    <- 3          #  penalize curvature of acceleration
lambda    <- 10^(-0.5)  #  smoothing parameter
growfdPar <- fdPar(Wfd0, Lfdobj, lambda)
#  Set up wgt vector
wgt   <- rep(1,nage)
#  Smooth the data for the first girl
hgt1 = growth$hgtf[,1]
result <- smooth.monotone(age, hgt1, growfdPar, wgt)
#  Extract the functional data object and regression
#  coefficients
Wfd  <- result$Wfdobj
beta <- result$beta
#  Evaluate the fitted height curve over a fine mesh
agefine <- seq(1,18,len=101)
hgtfine <- beta[1] + beta[2]*eval.monfd(agefine, Wfd)
#  Plot the data and the curve
plot(age, hgt1, type="p")
lines(agefine, hgtfine)
#  Evaluate the acceleration curve
accfine <- beta[2]*eval.monfd(agefine, Wfd, 2)
#  Plot the acceleration curve
plot(agefine, accfine, type="l")
lines(c(1,18),c(0,0),lty=4)

\end{ExampleCode}
\end{Examples}

