\HeaderA{lambda2df}{Convert Smoothing Parameter to Degrees of Freedom}{lambda2df}
\keyword{smooth}{lambda2df}
\begin{Description}\relax
The degree of roughness of an estimated function is controlled by a
smoothing parameter $lambda$ that directly multiplies the penalty.
However, it can be difficult to interpret or choose this value, and it
is often easier to determine the roughness by choosing a value that is
equivalent of the degrees of freedom used by the smoothing procedure.
This function converts a multipler $lambda$ into a degrees of freedom value.
\end{Description}
\begin{Usage}
\begin{verbatim}
lambda2df(argvals, basisobj, wtvec=rep(1, n),
          Lfdobj=NULL, lambda=0)
\end{verbatim}
\end{Usage}
\begin{Arguments}
\begin{ldescription}
\item[\code{argvals}] a vector containing the argument values used in the
smooth of the data.

\item[\code{basisobj}] the basis object used in the smoothing of the data.

\item[\code{wtvec}] the weight vector, if any, that was used in the smoothing
of the data.

\item[\code{Lfdobj}] the linear differential operator object used to defining
the roughness penalty employed in smoothing the data.

\item[\code{lambda}] the smoothing parameter to be converted.

\end{ldescription}
\end{Arguments}
\begin{Value}
the equivalent degrees of freedom value.
\end{Value}
\begin{SeeAlso}\relax
\code{\LinkA{df2lambda}{df2lambda}}
\end{SeeAlso}

