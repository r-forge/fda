\HeaderA{as.array3}{Reshape a vector or array to have 3 dimensions.}{as.array3}
\keyword{utilities}{as.array3}
\begin{Description}\relax
Coerce a vector or array to have 3 dimensions, preserving dimnames if
feasible.  Throw an error if length(dim(x)) > 3.
\end{Description}
\begin{Usage}
\begin{verbatim}
as.array3(x) 
\end{verbatim}
\end{Usage}
\begin{Arguments}
\begin{ldescription}
\item[\code{x}] A vector or array.  

\end{ldescription}
\end{Arguments}
\begin{Details}\relax
1.  dimx <- dim(x);  ndim <- length(dimx) 

2.  if(ndim==3)return(x).

3.  if(ndim>3)stop.

4.  x2 <- as.matrix(x)

5.  dim(x2) <- c(dim(x2), 1)

6.  xnames <- dimnames(x)

7.  if(is.list(xnames))dimnames(x2) <- list(xnames[[1]], xnames[[2]],
NULL)
\end{Details}
\begin{Value}
A 3-dimensional array with names matching \code{x}
\end{Value}
\begin{Author}\relax
Spencer Graves
\end{Author}
\begin{SeeAlso}\relax
\code{\LinkA{dim}{dim}},
\code{\LinkA{dimnames}{dimnames}}
\code{\LinkA{checkDims3}{checkDims3}}
\end{SeeAlso}
\begin{Examples}
\begin{ExampleCode}
##
## vector -> array 
##
as.array3(c(a=1, b=2)) 

##
## matrix -> array 
##
as.array3(matrix(1:6, 2))
as.array3(matrix(1:6, 2, dimnames=list(letters[1:2], LETTERS[3:5]))) 

##
## array -> array 
##
as.array3(array(1:6, 1:3)) 

##
## 4-d array 
##
## Not run: 
as.array3(array(1:24, 1:4)) 
Error in as.array3(array(1:24, 1:4)) : 
  length(dim(array(1:24, 1:4)) = 4 > 3
## End(Not run)
\end{ExampleCode}
\end{Examples}

