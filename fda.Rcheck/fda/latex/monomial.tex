\HeaderA{monomial}{Evaluate Monomial Basis}{monomial}
\keyword{smooth}{monomial}
\begin{Description}\relax
Computes the values of the powers of argument t.
\end{Description}
\begin{Usage}
\begin{verbatim}
monomial(evalarg, exponents, nderiv=0)
\end{verbatim}
\end{Usage}
\begin{Arguments}
\begin{ldescription}
\item[\code{evalarg}] a vector of argument values.

\item[\code{exponents}] a vector of nonnegative integer values specifying the
powers to be computed.

\item[\code{nderiv}] a nonnegative integer specifying the order of derivative to be
evaluated.

\end{ldescription}
\end{Arguments}
\begin{Value}
a matrix of values of basis functions.  Rows correspond to
argument values and columns to basis functions.
\end{Value}
\begin{SeeAlso}\relax
\code{\LinkA{polynom}{polynom}}, 
\code{\LinkA{power}{power}}, 
\code{\LinkA{expon}{expon}}, 
\code{\LinkA{fourier}{fourier}}, 
\code{\LinkA{polyg}{polyg}}, 
\code{\LinkA{bsplineS}{bsplineS}}
\end{SeeAlso}
\begin{Examples}
\begin{ExampleCode}

# set up a monomial basis for the first five powers
nbasis   <- 5
basisobj <- create.monomial.basis(c(-1,1),nbasis)
#  evaluate the basis
tval <- seq(-1,1,0.1)
basismat <- monomial(tval, 1:basisobj$nbasis)

\end{ExampleCode}
\end{Examples}

